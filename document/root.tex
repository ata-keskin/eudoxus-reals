\documentclass[11pt,a4paper]{article}
\usepackage[T1]{fontenc}
\usepackage{isabelle,isabellesym}

% further packages required for unusual symbols (see also
% isabellesym.sty), use only when needed

\usepackage{amssymb}
  %for \<leadsto>, \<box>, \<diamond>, \<sqsupset>, \<mho>, \<Join>,
  %\<lhd>, \<lesssim>, \<greatersim>, \<lessapprox>, \<greaterapprox>,
  %\<triangleq>, \<yen>, \<lozenge>

%\usepackage{eurosym}
  %for \<euro>

%\usepackage[only,bigsqcap,bigparallel,fatsemi,interleave,sslash]{stmaryrd}
  %for \<Sqinter>, \<Parallel>, \<Zsemi>, \<Parallel>, \<sslash>

%\usepackage{eufrak}
  %for \<AA> ... \<ZZ>, \<aa> ... \<zz> (also included in amssymb)

%\usepackage{textcomp}
  %for \<onequarter>, \<onehalf>, \<threequarters>, \<degree>, \<cent>,
  %\<currency>

% this should be the last package used
\usepackage{pdfsetup}

% urls in roman style, theory text in math-similar italics
\urlstyle{rm}
\isabellestyle{it}

% for uniform font size
%\renewcommand{\isastyle}{\isastyleminor}


\begin{document}

\title{Eudoxus Reals}
\author{Ata Keskin}

\maketitle

\begin{abstract}
In this project, we present a peculiar construction of the real numbers, called ``Eudoxus reals'', using Isabelle/HOL. Similar to the classical method of Dedekind cuts, our approach starts from first principles. However, unlike Dedekind cuts, Eudoxus reals directly derive real numbers from integers, bypassing the intermediate step of constructing rational numbers.

This construction of the real numbers was first discovered by Stephen Schanuel. Schanuel named his construction after the ancient Greek philosopher Eudoxus, who developed a theory of magnitude and proportion to explain the relations between the discrete and the continuous. Our formalization is based on R.D. Arthan's paper detailing the construction \cite{arthan2004eudoxus}. For establishing the existence of multiplicative inverses for positive slopes, we used the idea of finding a suitable representative from S\l awomir Ko\l ody\'naski's construction on IsarMathLib which is based on Zermelo--Fraenkel set theory.
\end{abstract}

\tableofcontents
\pagebreak
% sane default for proof documents
\parindent 0pt\parskip 0.5ex

% generated text of all theories
%
\begin{isabellebody}%
\setisabellecontext{Slope}%
%
\isadelimtheory
\isanewline
\isanewline
%
\endisadelimtheory
%
\isatagtheory
\isacommand{theory}\isamarkupfalse%
\ Slope\isanewline
\isakeyword{imports}\ {\isachardoublequoteopen}HOL{\isachardot}{\kern0pt}Archimedean{\isacharunderscore}{\kern0pt}Field{\isachardoublequoteclose}\isanewline
\isakeyword{begin}%
\endisatagtheory
{\isafoldtheory}%
%
\isadelimtheory
%
\endisadelimtheory
%
\isadelimdocument
%
\endisadelimdocument
%
\isatagdocument
%
\isamarkupsection{Slopes%
}
\isamarkuptrue%
%
\isamarkupsubsection{Bounded Functions%
}
\isamarkuptrue%
%
\endisatagdocument
{\isafolddocument}%
%
\isadelimdocument
%
\endisadelimdocument
\isacommand{definition}\isamarkupfalse%
\ bounded\ {\isacharcolon}{\kern0pt}{\isacharcolon}{\kern0pt}\ {\isachardoublequoteopen}{\isacharparenleft}{\kern0pt}{\isacharprime}{\kern0pt}a\ {\isasymRightarrow}\ int{\isacharparenright}{\kern0pt}\ {\isasymRightarrow}\ bool{\isachardoublequoteclose}\ \isakeyword{where}\isanewline
\ \ {\isachardoublequoteopen}bounded\ f\ {\isasymlongleftrightarrow}\ bdd{\isacharunderscore}{\kern0pt}above\ {\isacharparenleft}{\kern0pt}{\isacharparenleft}{\kern0pt}{\isasymlambda}z{\isachardot}{\kern0pt}\ {\isasymbar}f\ z{\isasymbar}{\isacharparenright}{\kern0pt}\ {\isacharbackquote}{\kern0pt}\ UNIV{\isacharparenright}{\kern0pt}{\isachardoublequoteclose}\isanewline
\isanewline
\isacommand{lemma}\isamarkupfalse%
\ boundedI{\isacharcolon}{\kern0pt}\isanewline
\ \ \isakeyword{assumes}\ {\isachardoublequoteopen}{\isasymAnd}z{\isachardot}{\kern0pt}\ {\isasymbar}f\ z{\isasymbar}\ {\isasymle}\ C{\isachardoublequoteclose}\isanewline
\ \ \isakeyword{shows}\ {\isachardoublequoteopen}bounded\ f{\isachardoublequoteclose}\isanewline
%
\isadelimproof
\ \ %
\endisadelimproof
%
\isatagproof
\isacommand{unfolding}\isamarkupfalse%
\ bounded{\isacharunderscore}{\kern0pt}def\ \isacommand{by}\isamarkupfalse%
\ {\isacharparenleft}{\kern0pt}rule\ bdd{\isacharunderscore}{\kern0pt}aboveI{\isadigit{2}}{\isacharcomma}{\kern0pt}\ force\ intro{\isacharcolon}{\kern0pt}\ assms{\isacharparenright}{\kern0pt}%
\endisatagproof
{\isafoldproof}%
%
\isadelimproof
\isanewline
%
\endisadelimproof
\isanewline
\isacommand{lemma}\isamarkupfalse%
\ boundedE{\isacharbrackleft}{\kern0pt}elim{\isacharbrackright}{\kern0pt}{\isacharcolon}{\kern0pt}\isanewline
\ \ \isakeyword{assumes}\ {\isachardoublequoteopen}bounded\ f{\isachardoublequoteclose}\ {\isachardoublequoteopen}{\isasymexists}C{\isachardot}{\kern0pt}\ {\isacharparenleft}{\kern0pt}{\isasymforall}z{\isachardot}{\kern0pt}\ {\isasymbar}f\ z{\isasymbar}\ {\isasymle}\ C{\isacharparenright}{\kern0pt}\ {\isasymand}\ {\isadigit{0}}\ {\isasymle}\ C\ {\isasymLongrightarrow}\ P{\isachardoublequoteclose}\isanewline
\ \ \isakeyword{shows}\ P\isanewline
%
\isadelimproof
\ \ %
\endisadelimproof
%
\isatagproof
\isacommand{using}\isamarkupfalse%
\ assms\ \isacommand{unfolding}\isamarkupfalse%
\ bounded{\isacharunderscore}{\kern0pt}def\ bdd{\isacharunderscore}{\kern0pt}above{\isacharunderscore}{\kern0pt}def\ \isacommand{by}\isamarkupfalse%
\ fastforce%
\endisatagproof
{\isafoldproof}%
%
\isadelimproof
\isanewline
%
\endisadelimproof
\isanewline
\isacommand{lemma}\isamarkupfalse%
\ boundedE{\isacharunderscore}{\kern0pt}strict{\isacharcolon}{\kern0pt}\isanewline
\ \ \isakeyword{assumes}\ {\isachardoublequoteopen}bounded\ f{\isachardoublequoteclose}\ {\isachardoublequoteopen}{\isasymexists}C{\isachardot}{\kern0pt}\ {\isacharparenleft}{\kern0pt}{\isasymforall}z{\isachardot}{\kern0pt}\ {\isasymbar}f\ z{\isasymbar}\ {\isacharless}{\kern0pt}\ C{\isacharparenright}{\kern0pt}\ {\isasymand}\ {\isadigit{0}}\ {\isacharless}{\kern0pt}\ C\ {\isasymLongrightarrow}\ P{\isachardoublequoteclose}\isanewline
\ \ \isakeyword{shows}\ P\isanewline
%
\isadelimproof
\ \ %
\endisadelimproof
%
\isatagproof
\isacommand{by}\isamarkupfalse%
\ {\isacharparenleft}{\kern0pt}meson\ bounded{\isacharunderscore}{\kern0pt}def\ bdd{\isacharunderscore}{\kern0pt}above{\isacharunderscore}{\kern0pt}def\ assms\ boundedE\ gt{\isacharunderscore}{\kern0pt}ex\ order{\isachardot}{\kern0pt}strict{\isacharunderscore}{\kern0pt}trans{\isadigit{1}}{\isacharparenright}{\kern0pt}%
\endisatagproof
{\isafoldproof}%
%
\isadelimproof
\isanewline
%
\endisadelimproof
\isanewline
\isacommand{lemma}\isamarkupfalse%
\ bounded{\isacharunderscore}{\kern0pt}alt{\isacharunderscore}{\kern0pt}def{\isacharcolon}{\kern0pt}\ {\isachardoublequoteopen}bounded\ f\ {\isasymlongleftrightarrow}\ {\isacharparenleft}{\kern0pt}{\isasymexists}C{\isachardot}{\kern0pt}\ {\isasymforall}z{\isachardot}{\kern0pt}\ {\isasymbar}f\ z{\isasymbar}\ {\isasymle}\ C{\isacharparenright}{\kern0pt}{\isachardoublequoteclose}%
\isadelimproof
\ %
\endisadelimproof
%
\isatagproof
\isacommand{using}\isamarkupfalse%
\ boundedI\ boundedE\ \isacommand{by}\isamarkupfalse%
\ meson%
\endisatagproof
{\isafoldproof}%
%
\isadelimproof
%
\endisadelimproof
\isanewline
\isanewline
\isacommand{lemma}\isamarkupfalse%
\ bounded{\isacharunderscore}{\kern0pt}iff{\isacharunderscore}{\kern0pt}finite{\isacharunderscore}{\kern0pt}range{\isacharcolon}{\kern0pt}\ {\isachardoublequoteopen}bounded\ f\ {\isasymlongleftrightarrow}\ finite\ {\isacharparenleft}{\kern0pt}range\ f{\isacharparenright}{\kern0pt}{\isachardoublequoteclose}\isanewline
%
\isadelimproof
%
\endisadelimproof
%
\isatagproof
\isacommand{proof}\isamarkupfalse%
\isanewline
\ \ \isacommand{assume}\isamarkupfalse%
\ {\isachardoublequoteopen}bounded\ f{\isachardoublequoteclose}\isanewline
\ \ \isacommand{then}\isamarkupfalse%
\ \isacommand{obtain}\isamarkupfalse%
\ C\ \isakeyword{where}\ bound{\isacharcolon}{\kern0pt}\ {\isachardoublequoteopen}{\isasymbar}z{\isasymbar}\ {\isasymle}\ C{\isachardoublequoteclose}\ \isakeyword{if}\ {\isachardoublequoteopen}z\ {\isasymin}\ range\ f{\isachardoublequoteclose}\ \isakeyword{for}\ z\ \isacommand{by}\isamarkupfalse%
\ blast\isanewline
\ \ \isacommand{have}\isamarkupfalse%
\ {\isachardoublequoteopen}range\ f\ {\isasymsubseteq}\ {\isacharbraceleft}{\kern0pt}z{\isachardot}{\kern0pt}\ z\ {\isasymle}\ C\ {\isasymand}\ {\isacharminus}{\kern0pt}z\ {\isasymle}\ C{\isacharbraceright}{\kern0pt}{\isachardoublequoteclose}\ \isacommand{using}\isamarkupfalse%
\ abs{\isacharunderscore}{\kern0pt}le{\isacharunderscore}{\kern0pt}D{\isadigit{1}}{\isacharbrackleft}{\kern0pt}OF\ bound{\isacharbrackright}{\kern0pt}\ abs{\isacharunderscore}{\kern0pt}le{\isacharunderscore}{\kern0pt}D{\isadigit{2}}{\isacharbrackleft}{\kern0pt}OF\ bound{\isacharbrackright}{\kern0pt}\ \isacommand{by}\isamarkupfalse%
\ blast\isanewline
\ \ \isacommand{also}\isamarkupfalse%
\ \isacommand{have}\isamarkupfalse%
\ {\isachardoublequoteopen}{\isachardot}{\kern0pt}{\isachardot}{\kern0pt}{\isachardot}{\kern0pt}\ {\isacharequal}{\kern0pt}\ {\isacharbraceleft}{\kern0pt}{\isacharparenleft}{\kern0pt}{\isacharminus}{\kern0pt}C{\isacharparenright}{\kern0pt}{\isachardot}{\kern0pt}{\isachardot}{\kern0pt}C{\isacharbraceright}{\kern0pt}{\isachardoublequoteclose}\ \isacommand{by}\isamarkupfalse%
\ force\isanewline
\ \ \isacommand{finally}\isamarkupfalse%
\ \isacommand{show}\isamarkupfalse%
\ {\isachardoublequoteopen}finite\ {\isacharparenleft}{\kern0pt}range\ f{\isacharparenright}{\kern0pt}{\isachardoublequoteclose}\ \isacommand{using}\isamarkupfalse%
\ finite{\isacharunderscore}{\kern0pt}subset\ finite{\isacharunderscore}{\kern0pt}atLeastAtMost{\isacharunderscore}{\kern0pt}int\ \isacommand{by}\isamarkupfalse%
\ blast\isanewline
\isacommand{next}\isamarkupfalse%
\isanewline
\ \ \isacommand{assume}\isamarkupfalse%
\ {\isachardoublequoteopen}finite\ {\isacharparenleft}{\kern0pt}range\ f{\isacharparenright}{\kern0pt}{\isachardoublequoteclose}\isanewline
\ \ \isacommand{hence}\isamarkupfalse%
\ {\isachardoublequoteopen}{\isasymbar}f\ z{\isasymbar}\ {\isasymle}\ max\ {\isacharparenleft}{\kern0pt}abs\ {\isacharparenleft}{\kern0pt}Sup\ {\isacharparenleft}{\kern0pt}range\ f{\isacharparenright}{\kern0pt}{\isacharparenright}{\kern0pt}{\isacharparenright}{\kern0pt}\ {\isacharparenleft}{\kern0pt}abs\ {\isacharparenleft}{\kern0pt}Inf\ {\isacharparenleft}{\kern0pt}range\ f{\isacharparenright}{\kern0pt}{\isacharparenright}{\kern0pt}{\isacharparenright}{\kern0pt}{\isachardoublequoteclose}\ \isakeyword{for}\ z\ \isanewline
\ \ \ \ \isacommand{using}\isamarkupfalse%
\ cInf{\isacharunderscore}{\kern0pt}lower{\isacharbrackleft}{\kern0pt}OF\ {\isacharunderscore}{\kern0pt}\ bdd{\isacharunderscore}{\kern0pt}below{\isacharunderscore}{\kern0pt}finite{\isacharcomma}{\kern0pt}\ of\ {\isachardoublequoteopen}f\ z{\isachardoublequoteclose}\ {\isachardoublequoteopen}range\ f{\isachardoublequoteclose}{\isacharbrackright}{\kern0pt}\ cSup{\isacharunderscore}{\kern0pt}upper{\isacharbrackleft}{\kern0pt}OF\ {\isacharunderscore}{\kern0pt}\ bdd{\isacharunderscore}{\kern0pt}above{\isacharunderscore}{\kern0pt}finite{\isacharcomma}{\kern0pt}\ of\ {\isachardoublequoteopen}f\ z{\isachardoublequoteclose}\ {\isachardoublequoteopen}range\ f{\isachardoublequoteclose}{\isacharbrackright}{\kern0pt}\ \isacommand{by}\isamarkupfalse%
\ force\isanewline
\ \ \isacommand{thus}\isamarkupfalse%
\ {\isachardoublequoteopen}bounded\ f{\isachardoublequoteclose}\ \isacommand{by}\isamarkupfalse%
\ {\isacharparenleft}{\kern0pt}rule\ boundedI{\isacharparenright}{\kern0pt}\isanewline
\isacommand{qed}\isamarkupfalse%
%
\endisatagproof
{\isafoldproof}%
%
\isadelimproof
\isanewline
%
\endisadelimproof
\isanewline
\isacommand{lemma}\isamarkupfalse%
\ bounded{\isacharunderscore}{\kern0pt}constant{\isacharcolon}{\kern0pt}\isanewline
\ \ \isakeyword{shows}\ {\isachardoublequoteopen}bounded\ {\isacharparenleft}{\kern0pt}{\isasymlambda}{\isacharunderscore}{\kern0pt}{\isachardot}{\kern0pt}\ c{\isacharparenright}{\kern0pt}{\isachardoublequoteclose}\isanewline
%
\isadelimproof
\ \ %
\endisadelimproof
%
\isatagproof
\isacommand{by}\isamarkupfalse%
\ {\isacharparenleft}{\kern0pt}rule\ boundedI{\isacharbrackleft}{\kern0pt}of\ {\isacharunderscore}{\kern0pt}\ {\isachardoublequoteopen}{\isasymbar}c{\isasymbar}{\isachardoublequoteclose}{\isacharbrackright}{\kern0pt}{\isacharcomma}{\kern0pt}\ blast{\isacharparenright}{\kern0pt}%
\endisatagproof
{\isafoldproof}%
%
\isadelimproof
\isanewline
%
\endisadelimproof
\isanewline
\isacommand{lemma}\isamarkupfalse%
\ bounded{\isacharunderscore}{\kern0pt}add{\isacharcolon}{\kern0pt}\isanewline
\ \ \isakeyword{assumes}\ {\isachardoublequoteopen}bounded\ f{\isachardoublequoteclose}\ {\isachardoublequoteopen}bounded\ g{\isachardoublequoteclose}\isanewline
\ \ \isakeyword{shows}\ {\isachardoublequoteopen}bounded\ {\isacharparenleft}{\kern0pt}{\isasymlambda}z{\isachardot}{\kern0pt}\ f\ z\ {\isacharplus}{\kern0pt}\ g\ z{\isacharparenright}{\kern0pt}{\isachardoublequoteclose}\isanewline
%
\isadelimproof
%
\endisadelimproof
%
\isatagproof
\isacommand{proof}\isamarkupfalse%
\ {\isacharminus}{\kern0pt}\isanewline
\ \ \isacommand{obtain}\isamarkupfalse%
\ C{\isacharunderscore}{\kern0pt}f\ C{\isacharunderscore}{\kern0pt}g\ \isakeyword{where}\ {\isachardoublequoteopen}{\isasymbar}f\ z{\isasymbar}\ {\isasymle}\ C{\isacharunderscore}{\kern0pt}f{\isachardoublequoteclose}\ {\isachardoublequoteopen}{\isasymbar}g\ z{\isasymbar}\ {\isasymle}\ C{\isacharunderscore}{\kern0pt}g{\isachardoublequoteclose}\ \isakeyword{for}\ z\ \isacommand{using}\isamarkupfalse%
\ assms\ \isacommand{by}\isamarkupfalse%
\ blast\isanewline
\ \ \isacommand{hence}\isamarkupfalse%
\ {\isachardoublequoteopen}{\isasymbar}f\ z\ {\isacharplus}{\kern0pt}\ g\ z{\isasymbar}\ {\isasymle}\ C{\isacharunderscore}{\kern0pt}f\ {\isacharplus}{\kern0pt}\ C{\isacharunderscore}{\kern0pt}g{\isachardoublequoteclose}\ \isakeyword{for}\ z\ \isacommand{by}\isamarkupfalse%
\ {\isacharparenleft}{\kern0pt}meson\ abs{\isacharunderscore}{\kern0pt}triangle{\isacharunderscore}{\kern0pt}ineq\ add{\isacharunderscore}{\kern0pt}mono\ dual{\isacharunderscore}{\kern0pt}order{\isachardot}{\kern0pt}trans{\isacharparenright}{\kern0pt}\isanewline
\ \ \isacommand{thus}\isamarkupfalse%
\ {\isacharquery}{\kern0pt}thesis\ \isacommand{by}\isamarkupfalse%
\ {\isacharparenleft}{\kern0pt}blast\ intro{\isacharcolon}{\kern0pt}\ boundedI{\isacharparenright}{\kern0pt}\isanewline
\isacommand{qed}\isamarkupfalse%
%
\endisatagproof
{\isafoldproof}%
%
\isadelimproof
\isanewline
%
\endisadelimproof
\isanewline
\isacommand{lemma}\isamarkupfalse%
\ bounded{\isacharunderscore}{\kern0pt}mult{\isacharcolon}{\kern0pt}\isanewline
\ \ \isakeyword{assumes}\ {\isachardoublequoteopen}bounded\ f{\isachardoublequoteclose}\ {\isachardoublequoteopen}bounded\ g{\isachardoublequoteclose}\isanewline
\ \ \isakeyword{shows}\ {\isachardoublequoteopen}bounded\ {\isacharparenleft}{\kern0pt}{\isasymlambda}z{\isachardot}{\kern0pt}\ f\ z\ {\isacharasterisk}{\kern0pt}\ g\ z{\isacharparenright}{\kern0pt}{\isachardoublequoteclose}\isanewline
%
\isadelimproof
%
\endisadelimproof
%
\isatagproof
\isacommand{proof}\isamarkupfalse%
\ {\isacharminus}{\kern0pt}\isanewline
\ \ \isacommand{obtain}\isamarkupfalse%
\ C\ \isakeyword{where}\ bound{\isacharcolon}{\kern0pt}\ {\isachardoublequoteopen}{\isasymbar}f\ z{\isasymbar}\ {\isasymle}\ C{\isachardoublequoteclose}\ \isakeyword{and}\ C{\isacharunderscore}{\kern0pt}nonneg{\isacharcolon}{\kern0pt}\ {\isachardoublequoteopen}{\isadigit{0}}\ {\isasymle}\ C{\isachardoublequoteclose}\ \isakeyword{for}\ z\ \isacommand{using}\isamarkupfalse%
\ assms\ \isacommand{by}\isamarkupfalse%
\ blast\isanewline
\ \ \isacommand{obtain}\isamarkupfalse%
\ C{\isacharprime}{\kern0pt}\ \isakeyword{where}\ bound{\isacharprime}{\kern0pt}{\isacharcolon}{\kern0pt}\ {\isachardoublequoteopen}{\isasymbar}g\ z{\isasymbar}\ {\isasymle}\ C{\isacharprime}{\kern0pt}{\isachardoublequoteclose}\ \isakeyword{for}\ z\ \isacommand{using}\isamarkupfalse%
\ assms\ \isacommand{by}\isamarkupfalse%
\ blast\isanewline
\ \ \isacommand{show}\isamarkupfalse%
\ {\isacharquery}{\kern0pt}thesis\ \isacommand{using}\isamarkupfalse%
\ mult{\isacharunderscore}{\kern0pt}mono{\isacharbrackleft}{\kern0pt}OF\ bound\ bound{\isacharprime}{\kern0pt}\ C{\isacharunderscore}{\kern0pt}nonneg\ abs{\isacharunderscore}{\kern0pt}ge{\isacharunderscore}{\kern0pt}zero{\isacharbrackright}{\kern0pt}\ \isacommand{by}\isamarkupfalse%
\ {\isacharparenleft}{\kern0pt}simp\ only{\isacharcolon}{\kern0pt}\ boundedI{\isacharbrackleft}{\kern0pt}of\ {\isachardoublequoteopen}{\isasymlambda}z{\isachardot}{\kern0pt}\ f\ z\ {\isacharasterisk}{\kern0pt}\ g\ z{\isachardoublequoteclose}\ {\isachardoublequoteopen}C\ {\isacharasterisk}{\kern0pt}\ C{\isacharprime}{\kern0pt}{\isachardoublequoteclose}{\isacharbrackright}{\kern0pt}\ abs{\isacharunderscore}{\kern0pt}mult{\isacharparenright}{\kern0pt}\isanewline
\isacommand{qed}\isamarkupfalse%
%
\endisatagproof
{\isafoldproof}%
%
\isadelimproof
\isanewline
%
\endisadelimproof
\isanewline
\isacommand{lemma}\isamarkupfalse%
\ bounded{\isacharunderscore}{\kern0pt}mult{\isacharunderscore}{\kern0pt}const{\isacharcolon}{\kern0pt}\isanewline
\ \ \isakeyword{assumes}\ {\isachardoublequoteopen}bounded\ f{\isachardoublequoteclose}\isanewline
\ \ \isakeyword{shows}\ {\isachardoublequoteopen}bounded\ {\isacharparenleft}{\kern0pt}{\isasymlambda}z{\isachardot}{\kern0pt}\ c\ {\isacharasterisk}{\kern0pt}\ f\ z{\isacharparenright}{\kern0pt}{\isachardoublequoteclose}\isanewline
%
\isadelimproof
\ \ %
\endisadelimproof
%
\isatagproof
\isacommand{by}\isamarkupfalse%
\ {\isacharparenleft}{\kern0pt}rule\ bounded{\isacharunderscore}{\kern0pt}mult{\isacharbrackleft}{\kern0pt}OF\ bounded{\isacharunderscore}{\kern0pt}constant{\isacharbrackleft}{\kern0pt}of\ c{\isacharbrackright}{\kern0pt}\ assms{\isacharbrackright}{\kern0pt}{\isacharparenright}{\kern0pt}%
\endisatagproof
{\isafoldproof}%
%
\isadelimproof
\isanewline
%
\endisadelimproof
\isanewline
\isacommand{lemma}\isamarkupfalse%
\ bounded{\isacharunderscore}{\kern0pt}uminus{\isacharcolon}{\kern0pt}\isanewline
\ \ \isakeyword{assumes}\ {\isachardoublequoteopen}bounded\ f{\isachardoublequoteclose}\isanewline
\ \ \isakeyword{shows}\ {\isachardoublequoteopen}bounded\ {\isacharparenleft}{\kern0pt}{\isasymlambda}x{\isachardot}{\kern0pt}\ {\isacharminus}{\kern0pt}\ f\ x{\isacharparenright}{\kern0pt}{\isachardoublequoteclose}\isanewline
%
\isadelimproof
\ \ %
\endisadelimproof
%
\isatagproof
\isacommand{using}\isamarkupfalse%
\ bounded{\isacharunderscore}{\kern0pt}mult{\isacharunderscore}{\kern0pt}const{\isacharbrackleft}{\kern0pt}OF\ assms{\isacharcomma}{\kern0pt}\ of\ {\isachardoublequoteopen}{\isacharminus}{\kern0pt}\ {\isadigit{1}}{\isachardoublequoteclose}{\isacharbrackright}{\kern0pt}\ \isacommand{by}\isamarkupfalse%
\ simp%
\endisatagproof
{\isafoldproof}%
%
\isadelimproof
\isanewline
%
\endisadelimproof
\isanewline
\isacommand{lemma}\isamarkupfalse%
\ bounded{\isacharunderscore}{\kern0pt}comp{\isacharcolon}{\kern0pt}\isanewline
\ \ \isakeyword{assumes}\ {\isachardoublequoteopen}bounded\ f{\isachardoublequoteclose}\isanewline
\ \ \isakeyword{shows}\ {\isachardoublequoteopen}bounded\ {\isacharparenleft}{\kern0pt}f\ o\ g{\isacharparenright}{\kern0pt}{\isachardoublequoteclose}\ \isakeyword{and}\ {\isachardoublequoteopen}bounded\ {\isacharparenleft}{\kern0pt}g\ o\ f{\isacharparenright}{\kern0pt}{\isachardoublequoteclose}\isanewline
%
\isadelimproof
%
\endisadelimproof
%
\isatagproof
\isacommand{proof}\isamarkupfalse%
\ {\isacharminus}{\kern0pt}\isanewline
\ \ \isacommand{show}\isamarkupfalse%
\ {\isachardoublequoteopen}bounded\ {\isacharparenleft}{\kern0pt}f\ o\ g{\isacharparenright}{\kern0pt}{\isachardoublequoteclose}\ \isacommand{using}\isamarkupfalse%
\ assms\ boundedI\ comp{\isacharunderscore}{\kern0pt}def\ boundedE\ \isacommand{by}\isamarkupfalse%
\ metis\isanewline
\isacommand{next}\isamarkupfalse%
\isanewline
\ \ \isacommand{have}\isamarkupfalse%
\ {\isachardoublequoteopen}range\ {\isacharparenleft}{\kern0pt}g\ o\ f{\isacharparenright}{\kern0pt}\ {\isacharequal}{\kern0pt}\ g\ {\isacharbackquote}{\kern0pt}\ range\ f{\isachardoublequoteclose}\ \isacommand{by}\isamarkupfalse%
\ fastforce\isanewline
\ \ \isacommand{thus}\isamarkupfalse%
\ {\isachardoublequoteopen}bounded\ {\isacharparenleft}{\kern0pt}g\ o\ f{\isacharparenright}{\kern0pt}{\isachardoublequoteclose}\ \isacommand{using}\isamarkupfalse%
\ assms\ \isacommand{by}\isamarkupfalse%
\ {\isacharparenleft}{\kern0pt}fastforce\ simp{\isacharcolon}{\kern0pt}\ bounded{\isacharunderscore}{\kern0pt}iff{\isacharunderscore}{\kern0pt}finite{\isacharunderscore}{\kern0pt}range{\isacharparenright}{\kern0pt}\isanewline
\isacommand{qed}\isamarkupfalse%
%
\endisatagproof
{\isafoldproof}%
%
\isadelimproof
%
\endisadelimproof
%
\isadelimdocument
%
\endisadelimdocument
%
\isatagdocument
%
\isamarkupsubsection{Properties of Slopes%
}
\isamarkuptrue%
%
\endisatagdocument
{\isafolddocument}%
%
\isadelimdocument
%
\endisadelimdocument
\isacommand{definition}\isamarkupfalse%
\ slope\ {\isacharcolon}{\kern0pt}{\isacharcolon}{\kern0pt}\ {\isachardoublequoteopen}{\isacharparenleft}{\kern0pt}int\ {\isasymRightarrow}\ int{\isacharparenright}{\kern0pt}\ {\isasymRightarrow}\ bool{\isachardoublequoteclose}\ \isakeyword{where}\isanewline
\ \ {\isachardoublequoteopen}slope\ f\ {\isasymlongleftrightarrow}\ bounded\ {\isacharparenleft}{\kern0pt}{\isasymlambda}{\isacharparenleft}{\kern0pt}m{\isacharcomma}{\kern0pt}\ n{\isacharparenright}{\kern0pt}{\isachardot}{\kern0pt}\ f\ {\isacharparenleft}{\kern0pt}m\ {\isacharplus}{\kern0pt}\ n{\isacharparenright}{\kern0pt}\ {\isacharminus}{\kern0pt}\ {\isacharparenleft}{\kern0pt}f\ m\ {\isacharplus}{\kern0pt}\ f\ n{\isacharparenright}{\kern0pt}{\isacharparenright}{\kern0pt}{\isachardoublequoteclose}\isanewline
\isanewline
\isacommand{lemma}\isamarkupfalse%
\ bounded{\isacharunderscore}{\kern0pt}slopeI{\isacharcolon}{\kern0pt}\isanewline
\ \ \isakeyword{assumes}\ {\isachardoublequoteopen}bounded\ f{\isachardoublequoteclose}\isanewline
\ \ \isakeyword{shows}\ {\isachardoublequoteopen}slope\ f{\isachardoublequoteclose}\isanewline
%
\isadelimproof
%
\endisadelimproof
%
\isatagproof
\isacommand{proof}\isamarkupfalse%
\ {\isacharminus}{\kern0pt}\isanewline
\ \ \isacommand{obtain}\isamarkupfalse%
\ C\ \isakeyword{where}\ {\isachardoublequoteopen}{\isasymbar}f\ x{\isasymbar}\ {\isasymle}\ C{\isachardoublequoteclose}\ \isakeyword{for}\ x\ \isacommand{using}\isamarkupfalse%
\ assms\ \isacommand{by}\isamarkupfalse%
\ blast\isanewline
\ \ \isacommand{hence}\isamarkupfalse%
\ {\isachardoublequoteopen}{\isasymbar}f\ {\isacharparenleft}{\kern0pt}m\ {\isacharplus}{\kern0pt}\ n{\isacharparenright}{\kern0pt}\ {\isacharminus}{\kern0pt}\ {\isacharparenleft}{\kern0pt}f\ m\ {\isacharplus}{\kern0pt}\ f\ n{\isacharparenright}{\kern0pt}{\isasymbar}\ {\isasymle}\ C\ {\isacharplus}{\kern0pt}\ {\isacharparenleft}{\kern0pt}C\ {\isacharplus}{\kern0pt}\ C{\isacharparenright}{\kern0pt}{\isachardoublequoteclose}\ \isakeyword{for}\ m\ n\isanewline
\ \ \ \ \isacommand{using}\isamarkupfalse%
\ abs{\isacharunderscore}{\kern0pt}triangle{\isacharunderscore}{\kern0pt}ineq{\isadigit{4}}{\isacharbrackleft}{\kern0pt}of\ {\isachardoublequoteopen}f\ {\isacharparenleft}{\kern0pt}m\ {\isacharplus}{\kern0pt}\ n{\isacharparenright}{\kern0pt}{\isachardoublequoteclose}\ {\isachardoublequoteopen}f\ m\ {\isacharplus}{\kern0pt}\ f\ n{\isachardoublequoteclose}{\isacharbrackright}{\kern0pt}\ abs{\isacharunderscore}{\kern0pt}triangle{\isacharunderscore}{\kern0pt}ineq{\isacharbrackleft}{\kern0pt}of\ {\isachardoublequoteopen}f\ m{\isachardoublequoteclose}\ {\isachardoublequoteopen}f\ n{\isachardoublequoteclose}{\isacharbrackright}{\kern0pt}\ \isacommand{by}\isamarkupfalse%
\ {\isacharparenleft}{\kern0pt}meson\ add{\isacharunderscore}{\kern0pt}mono\ order{\isacharunderscore}{\kern0pt}trans{\isacharparenright}{\kern0pt}\isanewline
\ \ \isacommand{thus}\isamarkupfalse%
\ {\isacharquery}{\kern0pt}thesis\ \isacommand{unfolding}\isamarkupfalse%
\ slope{\isacharunderscore}{\kern0pt}def\ \isacommand{by}\isamarkupfalse%
\ {\isacharparenleft}{\kern0pt}fast\ intro{\isacharcolon}{\kern0pt}\ boundedI{\isacharparenright}{\kern0pt}\isanewline
\isacommand{qed}\isamarkupfalse%
%
\endisatagproof
{\isafoldproof}%
%
\isadelimproof
\isanewline
%
\endisadelimproof
\isanewline
\isacommand{lemma}\isamarkupfalse%
\ slopeE{\isacharbrackleft}{\kern0pt}elim{\isacharbrackright}{\kern0pt}{\isacharcolon}{\kern0pt}\isanewline
\ \ \isakeyword{assumes}\ {\isachardoublequoteopen}slope\ f{\isachardoublequoteclose}\isanewline
\ \ \isakeyword{obtains}\ C\ \isakeyword{where}\ {\isachardoublequoteopen}{\isasymAnd}m\ n{\isachardot}{\kern0pt}\ {\isasymbar}f\ {\isacharparenleft}{\kern0pt}m\ {\isacharplus}{\kern0pt}\ n{\isacharparenright}{\kern0pt}\ {\isacharminus}{\kern0pt}\ {\isacharparenleft}{\kern0pt}f\ m\ {\isacharplus}{\kern0pt}\ f\ n{\isacharparenright}{\kern0pt}{\isasymbar}\ {\isasymle}\ C{\isachardoublequoteclose}\ {\isachardoublequoteopen}{\isadigit{0}}\ {\isasymle}\ C{\isachardoublequoteclose}%
\isadelimproof
\ %
\endisadelimproof
%
\isatagproof
\isacommand{using}\isamarkupfalse%
\ assms\ \isacommand{unfolding}\isamarkupfalse%
\ slope{\isacharunderscore}{\kern0pt}def\ \isacommand{by}\isamarkupfalse%
\ fastforce%
\endisatagproof
{\isafoldproof}%
%
\isadelimproof
%
\endisadelimproof
\isanewline
\isanewline
\isacommand{lemma}\isamarkupfalse%
\ slope{\isacharunderscore}{\kern0pt}add{\isacharcolon}{\kern0pt}\isanewline
\ \ \isakeyword{assumes}\ {\isachardoublequoteopen}slope\ f{\isachardoublequoteclose}\ {\isachardoublequoteopen}slope\ g{\isachardoublequoteclose}\isanewline
\ \ \isakeyword{shows}\ {\isachardoublequoteopen}slope\ {\isacharparenleft}{\kern0pt}{\isasymlambda}z{\isachardot}{\kern0pt}\ f\ z\ {\isacharplus}{\kern0pt}\ g\ z{\isacharparenright}{\kern0pt}{\isachardoublequoteclose}\isanewline
%
\isadelimproof
%
\endisadelimproof
%
\isatagproof
\isacommand{proof}\isamarkupfalse%
\ {\isacharminus}{\kern0pt}\isanewline
\ \ \isacommand{obtain}\isamarkupfalse%
\ C\ \isakeyword{where}\ bound{\isacharcolon}{\kern0pt}\ {\isachardoublequoteopen}{\isasymbar}f\ {\isacharparenleft}{\kern0pt}m\ {\isacharplus}{\kern0pt}\ n{\isacharparenright}{\kern0pt}\ {\isacharminus}{\kern0pt}\ {\isacharparenleft}{\kern0pt}f\ m\ {\isacharplus}{\kern0pt}\ f\ n{\isacharparenright}{\kern0pt}{\isasymbar}\ {\isasymle}\ C{\isachardoublequoteclose}\ \isakeyword{for}\ m\ n\ \isacommand{using}\isamarkupfalse%
\ assms\ \isacommand{by}\isamarkupfalse%
\ fast\isanewline
\ \ \isacommand{obtain}\isamarkupfalse%
\ C{\isacharprime}{\kern0pt}\ \isakeyword{where}\ bound{\isacharprime}{\kern0pt}{\isacharcolon}{\kern0pt}\ {\isachardoublequoteopen}{\isasymbar}g\ {\isacharparenleft}{\kern0pt}m\ {\isacharplus}{\kern0pt}\ n{\isacharparenright}{\kern0pt}\ {\isacharminus}{\kern0pt}\ {\isacharparenleft}{\kern0pt}g\ m\ {\isacharplus}{\kern0pt}\ g\ n{\isacharparenright}{\kern0pt}{\isasymbar}\ {\isasymle}\ C{\isacharprime}{\kern0pt}{\isachardoublequoteclose}\ \isakeyword{for}\ m\ n\ \isacommand{using}\isamarkupfalse%
\ assms\ \isacommand{by}\isamarkupfalse%
\ fast\isanewline
\ \ \isacommand{have}\isamarkupfalse%
\ {\isachardoublequoteopen}{\isasymbar}f\ {\isacharparenleft}{\kern0pt}m\ {\isacharplus}{\kern0pt}\ n{\isacharparenright}{\kern0pt}\ {\isacharminus}{\kern0pt}\ {\isacharparenleft}{\kern0pt}f\ m\ {\isacharplus}{\kern0pt}\ f\ n{\isacharparenright}{\kern0pt}{\isasymbar}\ {\isacharplus}{\kern0pt}\ {\isasymbar}g\ {\isacharparenleft}{\kern0pt}m\ {\isacharplus}{\kern0pt}\ n{\isacharparenright}{\kern0pt}\ {\isacharminus}{\kern0pt}\ {\isacharparenleft}{\kern0pt}g\ m\ {\isacharplus}{\kern0pt}\ g\ n{\isacharparenright}{\kern0pt}{\isasymbar}\ {\isasymle}\ C\ {\isacharplus}{\kern0pt}\ C{\isacharprime}{\kern0pt}{\isachardoublequoteclose}\ \isakeyword{for}\ m\ n\ \isacommand{using}\isamarkupfalse%
\ add{\isacharunderscore}{\kern0pt}mono{\isacharunderscore}{\kern0pt}thms{\isacharunderscore}{\kern0pt}linordered{\isacharunderscore}{\kern0pt}semiring{\isacharparenleft}{\kern0pt}{\isadigit{1}}{\isacharparenright}{\kern0pt}\ bound\ bound{\isacharprime}{\kern0pt}\ \isacommand{by}\isamarkupfalse%
\ fast\isanewline
\ \ \isacommand{moreover}\isamarkupfalse%
\ \isacommand{have}\isamarkupfalse%
\ {\isachardoublequoteopen}{\isasymbar}{\isacharparenleft}{\kern0pt}{\isasymlambda}z{\isachardot}{\kern0pt}\ f\ z\ {\isacharplus}{\kern0pt}\ g\ z{\isacharparenright}{\kern0pt}\ {\isacharparenleft}{\kern0pt}m\ {\isacharplus}{\kern0pt}\ n{\isacharparenright}{\kern0pt}\ {\isacharminus}{\kern0pt}\ {\isacharparenleft}{\kern0pt}{\isacharparenleft}{\kern0pt}{\isasymlambda}z{\isachardot}{\kern0pt}\ f\ z\ {\isacharplus}{\kern0pt}\ g\ z{\isacharparenright}{\kern0pt}\ m\ {\isacharplus}{\kern0pt}\ {\isacharparenleft}{\kern0pt}{\isasymlambda}z{\isachardot}{\kern0pt}\ f\ z\ {\isacharplus}{\kern0pt}\ g\ z{\isacharparenright}{\kern0pt}\ n{\isacharparenright}{\kern0pt}{\isasymbar}\ {\isasymle}\ {\isasymbar}f\ {\isacharparenleft}{\kern0pt}m\ {\isacharplus}{\kern0pt}\ n{\isacharparenright}{\kern0pt}\ {\isacharminus}{\kern0pt}\ {\isacharparenleft}{\kern0pt}f\ m\ {\isacharplus}{\kern0pt}\ f\ n{\isacharparenright}{\kern0pt}{\isasymbar}\ {\isacharplus}{\kern0pt}\ {\isasymbar}g\ {\isacharparenleft}{\kern0pt}m\ {\isacharplus}{\kern0pt}\ n{\isacharparenright}{\kern0pt}\ {\isacharminus}{\kern0pt}\ {\isacharparenleft}{\kern0pt}g\ m\ {\isacharplus}{\kern0pt}\ g\ n{\isacharparenright}{\kern0pt}{\isasymbar}{\isachardoublequoteclose}\ \isakeyword{for}\ m\ n\ \isacommand{by}\isamarkupfalse%
\ linarith\isanewline
\ \ \isacommand{ultimately}\isamarkupfalse%
\ \isacommand{have}\isamarkupfalse%
\ {\isachardoublequoteopen}{\isasymbar}{\isacharparenleft}{\kern0pt}{\isasymlambda}z{\isachardot}{\kern0pt}\ f\ z\ {\isacharplus}{\kern0pt}\ g\ z{\isacharparenright}{\kern0pt}\ {\isacharparenleft}{\kern0pt}m\ {\isacharplus}{\kern0pt}\ n{\isacharparenright}{\kern0pt}\ {\isacharminus}{\kern0pt}\ {\isacharparenleft}{\kern0pt}{\isacharparenleft}{\kern0pt}{\isasymlambda}z{\isachardot}{\kern0pt}\ f\ z\ {\isacharplus}{\kern0pt}\ g\ z{\isacharparenright}{\kern0pt}\ m\ {\isacharplus}{\kern0pt}\ {\isacharparenleft}{\kern0pt}{\isasymlambda}z{\isachardot}{\kern0pt}\ f\ z\ {\isacharplus}{\kern0pt}\ g\ z{\isacharparenright}{\kern0pt}\ n{\isacharparenright}{\kern0pt}{\isasymbar}\ {\isasymle}\ C\ {\isacharplus}{\kern0pt}\ C{\isacharprime}{\kern0pt}{\isachardoublequoteclose}\ \isakeyword{for}\ m\ n\ \isacommand{using}\isamarkupfalse%
\ order{\isacharunderscore}{\kern0pt}trans\ \isacommand{by}\isamarkupfalse%
\ fast\isanewline
\ \ \isacommand{thus}\isamarkupfalse%
\ {\isachardoublequoteopen}slope\ {\isacharparenleft}{\kern0pt}{\isasymlambda}z{\isachardot}{\kern0pt}\ f\ z\ {\isacharplus}{\kern0pt}\ g\ z{\isacharparenright}{\kern0pt}{\isachardoublequoteclose}\ \isacommand{unfolding}\isamarkupfalse%
\ slope{\isacharunderscore}{\kern0pt}def\ \isacommand{by}\isamarkupfalse%
\ {\isacharparenleft}{\kern0pt}fast\ intro{\isacharcolon}{\kern0pt}\ boundedI{\isacharparenright}{\kern0pt}\isanewline
\isacommand{qed}\isamarkupfalse%
%
\endisatagproof
{\isafoldproof}%
%
\isadelimproof
\isanewline
%
\endisadelimproof
\isanewline
\isacommand{lemma}\isamarkupfalse%
\ slope{\isacharunderscore}{\kern0pt}symmetric{\isacharunderscore}{\kern0pt}bound{\isacharcolon}{\kern0pt}\isanewline
\ \ \isakeyword{assumes}\ {\isachardoublequoteopen}slope\ f{\isachardoublequoteclose}\isanewline
\ \ \isakeyword{obtains}\ C\ \isakeyword{where}\ {\isachardoublequoteopen}{\isasymAnd}p\ q{\isachardot}{\kern0pt}\ {\isasymbar}p\ {\isacharasterisk}{\kern0pt}\ f\ q\ {\isacharminus}{\kern0pt}\ q\ {\isacharasterisk}{\kern0pt}\ f\ p{\isasymbar}\ {\isasymle}\ {\isacharparenleft}{\kern0pt}{\isasymbar}p{\isasymbar}\ {\isacharplus}{\kern0pt}\ {\isasymbar}q{\isasymbar}\ {\isacharplus}{\kern0pt}\ {\isadigit{2}}{\isacharparenright}{\kern0pt}\ {\isacharasterisk}{\kern0pt}\ C{\isachardoublequoteclose}\ {\isachardoublequoteopen}{\isadigit{0}}\ {\isasymle}\ C{\isachardoublequoteclose}\isanewline
%
\isadelimproof
%
\endisadelimproof
%
\isatagproof
\isacommand{proof}\isamarkupfalse%
\ {\isacharminus}{\kern0pt}\isanewline
\ \ \isacommand{obtain}\isamarkupfalse%
\ C\ \isakeyword{where}\ bound{\isacharcolon}{\kern0pt}\ {\isachardoublequoteopen}{\isasymbar}f\ {\isacharparenleft}{\kern0pt}m\ {\isacharplus}{\kern0pt}\ n{\isacharparenright}{\kern0pt}\ {\isacharminus}{\kern0pt}\ {\isacharparenleft}{\kern0pt}f\ m\ {\isacharplus}{\kern0pt}\ f\ n{\isacharparenright}{\kern0pt}{\isasymbar}\ {\isasymle}\ C{\isachardoublequoteclose}\ \isakeyword{and}\ C{\isacharunderscore}{\kern0pt}nonneg{\isacharcolon}{\kern0pt}\ {\isachardoublequoteopen}{\isadigit{0}}\ {\isasymle}\ C{\isachardoublequoteclose}\ \isakeyword{for}\ m\ n\ \isacommand{using}\isamarkupfalse%
\ assms\ \isacommand{by}\isamarkupfalse%
\ fast\isanewline
\ \ \isanewline
\ \ \isacommand{have}\isamarkupfalse%
\ {\isacharasterisk}{\kern0pt}{\isacharcolon}{\kern0pt}\ {\isachardoublequoteopen}{\isasymbar}f\ {\isacharparenleft}{\kern0pt}p\ {\isacharasterisk}{\kern0pt}\ q{\isacharparenright}{\kern0pt}\ {\isacharminus}{\kern0pt}\ p\ {\isacharasterisk}{\kern0pt}\ f\ q{\isasymbar}\ {\isasymle}\ {\isacharparenleft}{\kern0pt}{\isasymbar}p{\isasymbar}\ {\isacharplus}{\kern0pt}\ {\isadigit{1}}{\isacharparenright}{\kern0pt}\ {\isacharasterisk}{\kern0pt}\ C{\isachardoublequoteclose}\ \isakeyword{for}\ p\ q\isanewline
\ \ \isacommand{proof}\isamarkupfalse%
\ {\isacharparenleft}{\kern0pt}induction\ p\ rule{\isacharcolon}{\kern0pt}\ int{\isacharunderscore}{\kern0pt}induct{\isacharbrackleft}{\kern0pt}\isakeyword{where}\ {\isacharquery}{\kern0pt}k{\isacharequal}{\kern0pt}{\isadigit{0}}{\isacharbrackright}{\kern0pt}{\isacharparenright}{\kern0pt}\isanewline
\ \ \ \ \isacommand{case}\isamarkupfalse%
\ base\isanewline
\ \ \ \ \isacommand{then}\isamarkupfalse%
\ \isacommand{show}\isamarkupfalse%
\ {\isacharquery}{\kern0pt}case\ \isacommand{using}\isamarkupfalse%
\ bound{\isacharbrackleft}{\kern0pt}of\ {\isadigit{0}}\ {\isadigit{0}}{\isacharbrackright}{\kern0pt}\ \isacommand{by}\isamarkupfalse%
\ force\isanewline
\ \ \isacommand{next}\isamarkupfalse%
\isanewline
\ \ \ \ \isacommand{case}\isamarkupfalse%
\ {\isacharparenleft}{\kern0pt}step{\isadigit{1}}\ p{\isacharparenright}{\kern0pt}\isanewline
\ \ \ \ \isacommand{have}\isamarkupfalse%
\ {\isachardoublequoteopen}{\isasymbar}f\ {\isacharparenleft}{\kern0pt}{\isacharparenleft}{\kern0pt}p\ {\isacharplus}{\kern0pt}\ {\isadigit{1}}{\isacharparenright}{\kern0pt}\ {\isacharasterisk}{\kern0pt}\ q{\isacharparenright}{\kern0pt}\ {\isacharminus}{\kern0pt}\ f\ {\isacharparenleft}{\kern0pt}p\ {\isacharasterisk}{\kern0pt}\ q{\isacharparenright}{\kern0pt}\ {\isacharminus}{\kern0pt}\ f\ q{\isasymbar}\ {\isasymle}\ C{\isachardoublequoteclose}\ \isacommand{using}\isamarkupfalse%
\ bound{\isacharbrackleft}{\kern0pt}of\ {\isachardoublequoteopen}p\ {\isacharasterisk}{\kern0pt}\ q{\isachardoublequoteclose}\ q{\isacharbrackright}{\kern0pt}\ \ \isacommand{by}\isamarkupfalse%
\ {\isacharparenleft}{\kern0pt}auto\ simp{\isacharcolon}{\kern0pt}\ distrib{\isacharunderscore}{\kern0pt}left\ mult{\isachardot}{\kern0pt}commute{\isacharparenright}{\kern0pt}\isanewline
\ \ \ \ \isacommand{hence}\isamarkupfalse%
\ {\isachardoublequoteopen}{\isasymbar}f\ {\isacharparenleft}{\kern0pt}{\isacharparenleft}{\kern0pt}p\ {\isacharplus}{\kern0pt}\ {\isadigit{1}}{\isacharparenright}{\kern0pt}\ {\isacharasterisk}{\kern0pt}\ q{\isacharparenright}{\kern0pt}\ {\isacharminus}{\kern0pt}\ f\ q\ {\isacharminus}{\kern0pt}\ p\ {\isacharasterisk}{\kern0pt}\ f\ q{\isasymbar}\ {\isasymle}\ C\ {\isacharplus}{\kern0pt}\ {\isacharparenleft}{\kern0pt}{\isasymbar}p{\isasymbar}\ {\isacharplus}{\kern0pt}\ {\isadigit{1}}{\isacharparenright}{\kern0pt}\ {\isacharasterisk}{\kern0pt}\ C{\isachardoublequoteclose}\ \isacommand{using}\isamarkupfalse%
\ step{\isadigit{1}}\ \isacommand{by}\isamarkupfalse%
\ fastforce\isanewline
\ \ \ \ \isacommand{thus}\isamarkupfalse%
\ {\isacharquery}{\kern0pt}case\ \isacommand{using}\isamarkupfalse%
\ step{\isadigit{1}}\ \isacommand{by}\isamarkupfalse%
\ {\isacharparenleft}{\kern0pt}auto\ simp\ add{\isacharcolon}{\kern0pt}\ distrib{\isacharunderscore}{\kern0pt}left\ mult{\isachardot}{\kern0pt}commute{\isacharparenright}{\kern0pt}\isanewline
\ \ \isacommand{next}\isamarkupfalse%
\isanewline
\ \ \ \ \isacommand{case}\isamarkupfalse%
\ {\isacharparenleft}{\kern0pt}step{\isadigit{2}}\ p{\isacharparenright}{\kern0pt}\isanewline
\ \ \ \ \isacommand{have}\isamarkupfalse%
\ {\isachardoublequoteopen}{\isasymbar}f\ {\isacharparenleft}{\kern0pt}{\isacharparenleft}{\kern0pt}p\ {\isacharminus}{\kern0pt}\ {\isadigit{1}}{\isacharparenright}{\kern0pt}\ {\isacharasterisk}{\kern0pt}\ q{\isacharparenright}{\kern0pt}\ {\isacharplus}{\kern0pt}\ f\ q\ {\isacharminus}{\kern0pt}\ f\ {\isacharparenleft}{\kern0pt}p\ {\isacharasterisk}{\kern0pt}\ q{\isacharparenright}{\kern0pt}{\isasymbar}\ {\isasymle}\ C{\isachardoublequoteclose}\ \isacommand{using}\isamarkupfalse%
\ bound{\isacharbrackleft}{\kern0pt}of\ {\isachardoublequoteopen}p\ {\isacharasterisk}{\kern0pt}\ q\ {\isacharminus}{\kern0pt}\ q{\isachardoublequoteclose}\ {\isachardoublequoteopen}q{\isachardoublequoteclose}{\isacharbrackright}{\kern0pt}\ \isacommand{by}\isamarkupfalse%
\ {\isacharparenleft}{\kern0pt}auto\ simp{\isacharcolon}{\kern0pt}\ mult{\isachardot}{\kern0pt}commute\ right{\isacharunderscore}{\kern0pt}diff{\isacharunderscore}{\kern0pt}distrib{\isacharprime}{\kern0pt}{\isacharparenright}{\kern0pt}\isanewline
\ \ \ \ \isacommand{hence}\isamarkupfalse%
\ {\isachardoublequoteopen}{\isasymbar}f\ {\isacharparenleft}{\kern0pt}{\isacharparenleft}{\kern0pt}p\ {\isacharminus}{\kern0pt}\ {\isadigit{1}}{\isacharparenright}{\kern0pt}\ {\isacharasterisk}{\kern0pt}\ q{\isacharparenright}{\kern0pt}\ {\isacharplus}{\kern0pt}\ f\ q\ {\isacharminus}{\kern0pt}\ p\ {\isacharasterisk}{\kern0pt}\ f\ q{\isasymbar}\ {\isasymle}\ C\ {\isacharplus}{\kern0pt}\ {\isacharparenleft}{\kern0pt}{\isasymbar}p{\isasymbar}\ {\isacharplus}{\kern0pt}\ {\isadigit{1}}{\isacharparenright}{\kern0pt}\ {\isacharasterisk}{\kern0pt}\ C{\isachardoublequoteclose}\ \isacommand{using}\isamarkupfalse%
\ step{\isadigit{2}}\ \isacommand{by}\isamarkupfalse%
\ force\isanewline
\ \ \ \ \isacommand{hence}\isamarkupfalse%
\ {\isachardoublequoteopen}{\isasymbar}f\ {\isacharparenleft}{\kern0pt}{\isacharparenleft}{\kern0pt}p\ {\isacharminus}{\kern0pt}\ {\isadigit{1}}{\isacharparenright}{\kern0pt}\ {\isacharasterisk}{\kern0pt}\ q{\isacharparenright}{\kern0pt}\ {\isacharminus}{\kern0pt}\ {\isacharparenleft}{\kern0pt}p\ {\isacharminus}{\kern0pt}\ {\isadigit{1}}{\isacharparenright}{\kern0pt}\ {\isacharasterisk}{\kern0pt}\ f\ q{\isasymbar}\ {\isasymle}\ C\ {\isacharplus}{\kern0pt}\ {\isacharparenleft}{\kern0pt}{\isasymbar}p\ {\isacharminus}{\kern0pt}\ {\isadigit{1}}{\isasymbar}{\isacharparenright}{\kern0pt}\ {\isacharasterisk}{\kern0pt}\ C{\isachardoublequoteclose}\ \isacommand{using}\isamarkupfalse%
\ step{\isadigit{2}}\ \isacommand{by}\isamarkupfalse%
\ {\isacharparenleft}{\kern0pt}auto\ simp{\isacharcolon}{\kern0pt}\ mult{\isachardot}{\kern0pt}commute\ right{\isacharunderscore}{\kern0pt}diff{\isacharunderscore}{\kern0pt}distrib{\isacharprime}{\kern0pt}{\isacharparenright}{\kern0pt}\isanewline
\ \ \ \ \isacommand{thus}\isamarkupfalse%
\ {\isacharquery}{\kern0pt}case\ \isacommand{by}\isamarkupfalse%
\ {\isacharparenleft}{\kern0pt}auto\ simp\ add{\isacharcolon}{\kern0pt}\ distrib{\isacharunderscore}{\kern0pt}left\ mult{\isachardot}{\kern0pt}commute{\isacharparenright}{\kern0pt}\isanewline
\ \ \isacommand{qed}\isamarkupfalse%
\isanewline
\isanewline
\ \ \isacommand{have}\isamarkupfalse%
\ {\isachardoublequoteopen}{\isasymbar}p\ {\isacharasterisk}{\kern0pt}\ f\ q\ {\isacharminus}{\kern0pt}\ q\ {\isacharasterisk}{\kern0pt}\ f\ p{\isasymbar}\ {\isasymle}\ {\isacharparenleft}{\kern0pt}{\isasymbar}p{\isasymbar}\ {\isacharplus}{\kern0pt}\ {\isasymbar}q{\isasymbar}\ {\isacharplus}{\kern0pt}\ {\isadigit{2}}{\isacharparenright}{\kern0pt}\ {\isacharasterisk}{\kern0pt}\ C{\isachardoublequoteclose}\ \isakeyword{for}\ p\ q\isanewline
\ \ \isacommand{proof}\isamarkupfalse%
\ {\isacharminus}{\kern0pt}\isanewline
\ \ \ \ \isacommand{have}\isamarkupfalse%
\ {\isachardoublequoteopen}{\isasymbar}p\ {\isacharasterisk}{\kern0pt}\ f\ q\ {\isacharminus}{\kern0pt}\ q\ {\isacharasterisk}{\kern0pt}\ f\ p{\isasymbar}\ {\isasymle}\ {\isasymbar}f\ {\isacharparenleft}{\kern0pt}p\ {\isacharasterisk}{\kern0pt}\ q{\isacharparenright}{\kern0pt}\ {\isacharminus}{\kern0pt}\ p\ {\isacharasterisk}{\kern0pt}\ f\ q{\isasymbar}\ {\isacharplus}{\kern0pt}\ {\isasymbar}f\ {\isacharparenleft}{\kern0pt}q\ {\isacharasterisk}{\kern0pt}\ p{\isacharparenright}{\kern0pt}\ {\isacharminus}{\kern0pt}\ q\ {\isacharasterisk}{\kern0pt}\ f\ p{\isasymbar}{\isachardoublequoteclose}\ \isacommand{by}\isamarkupfalse%
\ {\isacharparenleft}{\kern0pt}fastforce\ simp{\isacharcolon}{\kern0pt}\ mult{\isachardot}{\kern0pt}commute{\isacharparenright}{\kern0pt}\isanewline
\ \ \ \ \isacommand{also}\isamarkupfalse%
\ \isacommand{have}\isamarkupfalse%
\ {\isachardoublequoteopen}{\isachardot}{\kern0pt}{\isachardot}{\kern0pt}{\isachardot}{\kern0pt}\ {\isasymle}\ {\isacharparenleft}{\kern0pt}{\isasymbar}p{\isasymbar}\ {\isacharplus}{\kern0pt}\ {\isadigit{1}}{\isacharparenright}{\kern0pt}\ {\isacharasterisk}{\kern0pt}\ C\ {\isacharplus}{\kern0pt}\ {\isacharparenleft}{\kern0pt}{\isasymbar}q{\isasymbar}\ {\isacharplus}{\kern0pt}\ {\isadigit{1}}{\isacharparenright}{\kern0pt}\ {\isacharasterisk}{\kern0pt}\ C{\isachardoublequoteclose}\ \isacommand{using}\isamarkupfalse%
\ {\isacharasterisk}{\kern0pt}{\isacharbrackleft}{\kern0pt}of\ p\ q{\isacharbrackright}{\kern0pt}\ {\isacharasterisk}{\kern0pt}{\isacharbrackleft}{\kern0pt}of\ q\ p{\isacharbrackright}{\kern0pt}\ \isacommand{by}\isamarkupfalse%
\ fastforce\isanewline
\ \ \ \ \isacommand{also}\isamarkupfalse%
\ \isacommand{have}\isamarkupfalse%
\ {\isachardoublequoteopen}{\isachardot}{\kern0pt}{\isachardot}{\kern0pt}{\isachardot}{\kern0pt}\ {\isacharequal}{\kern0pt}\ {\isacharparenleft}{\kern0pt}{\isasymbar}p{\isasymbar}\ {\isacharplus}{\kern0pt}\ {\isasymbar}q{\isasymbar}\ {\isacharplus}{\kern0pt}\ {\isadigit{2}}{\isacharparenright}{\kern0pt}\ {\isacharasterisk}{\kern0pt}\ C{\isachardoublequoteclose}\ \isacommand{by}\isamarkupfalse%
\ algebra\isanewline
\ \ \ \ \isacommand{finally}\isamarkupfalse%
\ \isacommand{show}\isamarkupfalse%
\ {\isacharquery}{\kern0pt}thesis\ \isacommand{{\isachardot}{\kern0pt}}\isamarkupfalse%
\isanewline
\ \ \isacommand{qed}\isamarkupfalse%
\isanewline
\ \ \isacommand{thus}\isamarkupfalse%
\ {\isacharquery}{\kern0pt}thesis\ \isacommand{using}\isamarkupfalse%
\ that\ C{\isacharunderscore}{\kern0pt}nonneg\ \isacommand{by}\isamarkupfalse%
\ blast\isanewline
\isacommand{qed}\isamarkupfalse%
%
\endisatagproof
{\isafoldproof}%
%
\isadelimproof
\isanewline
%
\endisadelimproof
\isanewline
\isacommand{lemma}\isamarkupfalse%
\ slope{\isacharunderscore}{\kern0pt}linear{\isacharunderscore}{\kern0pt}bound{\isacharcolon}{\kern0pt}\isanewline
\ \ \isakeyword{assumes}\ {\isachardoublequoteopen}slope\ f{\isachardoublequoteclose}\isanewline
\ \ \isakeyword{obtains}\ A\ B\ \isakeyword{where}\ {\isachardoublequoteopen}{\isasymforall}n{\isachardot}{\kern0pt}\ {\isasymbar}f\ n{\isasymbar}\ {\isasymle}\ A\ {\isacharasterisk}{\kern0pt}\ {\isasymbar}n{\isasymbar}\ {\isacharplus}{\kern0pt}\ B{\isachardoublequoteclose}\ {\isachardoublequoteopen}{\isadigit{0}}\ {\isasymle}\ A{\isachardoublequoteclose}\ {\isachardoublequoteopen}{\isadigit{0}}\ {\isasymle}\ B{\isachardoublequoteclose}\isanewline
%
\isadelimproof
%
\endisadelimproof
%
\isatagproof
\isacommand{proof}\isamarkupfalse%
\ {\isacharminus}{\kern0pt}\isanewline
\ \ \isacommand{obtain}\isamarkupfalse%
\ C\ \isakeyword{where}\ bound{\isacharcolon}{\kern0pt}\ {\isachardoublequoteopen}{\isasymbar}p\ {\isacharasterisk}{\kern0pt}\ f\ q\ {\isacharminus}{\kern0pt}\ q\ {\isacharasterisk}{\kern0pt}\ f\ p{\isasymbar}\ {\isasymle}\ {\isacharparenleft}{\kern0pt}{\isasymbar}p{\isasymbar}\ {\isacharplus}{\kern0pt}\ {\isasymbar}q{\isasymbar}\ {\isacharplus}{\kern0pt}\ {\isadigit{2}}{\isacharparenright}{\kern0pt}\ {\isacharasterisk}{\kern0pt}\ C{\isachardoublequoteclose}\ {\isachardoublequoteopen}{\isadigit{0}}\ {\isasymle}\ C{\isachardoublequoteclose}\ \isakeyword{for}\ p\ q\ \isacommand{using}\isamarkupfalse%
\ assms\ slope{\isacharunderscore}{\kern0pt}symmetric{\isacharunderscore}{\kern0pt}bound\ \isacommand{by}\isamarkupfalse%
\ blast\isanewline
\isanewline
\ \ \isacommand{have}\isamarkupfalse%
\ {\isachardoublequoteopen}{\isasymbar}f\ p{\isasymbar}\ {\isasymle}\ {\isacharparenleft}{\kern0pt}C\ {\isacharplus}{\kern0pt}\ {\isasymbar}f\ {\isadigit{1}}{\isasymbar}{\isacharparenright}{\kern0pt}\ {\isacharasterisk}{\kern0pt}\ {\isasymbar}p{\isasymbar}\ {\isacharplus}{\kern0pt}\ {\isadigit{3}}\ {\isacharasterisk}{\kern0pt}\ C{\isachardoublequoteclose}\ \isakeyword{for}\ p\isanewline
\ \ \isacommand{proof}\isamarkupfalse%
\ {\isacharminus}{\kern0pt}\isanewline
\ \ \ \ \isacommand{have}\isamarkupfalse%
\ {\isachardoublequoteopen}{\isasymbar}p\ {\isacharasterisk}{\kern0pt}\ f\ {\isadigit{1}}\ {\isacharminus}{\kern0pt}\ f\ p{\isasymbar}\ {\isasymle}\ {\isacharparenleft}{\kern0pt}{\isasymbar}p{\isasymbar}\ {\isacharplus}{\kern0pt}\ {\isadigit{3}}{\isacharparenright}{\kern0pt}\ {\isacharasterisk}{\kern0pt}\ C{\isachardoublequoteclose}\ \isacommand{using}\isamarkupfalse%
\ bound{\isacharparenleft}{\kern0pt}{\isadigit{1}}{\isacharparenright}{\kern0pt}{\isacharbrackleft}{\kern0pt}of\ {\isacharunderscore}{\kern0pt}\ {\isadigit{1}}{\isacharbrackright}{\kern0pt}\ \isacommand{by}\isamarkupfalse%
\ {\isacharparenleft}{\kern0pt}simp\ add{\isacharcolon}{\kern0pt}\ add{\isachardot}{\kern0pt}commute{\isacharparenright}{\kern0pt}\isanewline
\ \ \ \ \isacommand{hence}\isamarkupfalse%
\ {\isachardoublequoteopen}{\isasymbar}f\ p\ {\isacharminus}{\kern0pt}\ p\ {\isacharasterisk}{\kern0pt}\ f\ {\isadigit{1}}{\isasymbar}\ {\isasymle}\ {\isacharparenleft}{\kern0pt}{\isasymbar}p{\isasymbar}\ {\isacharplus}{\kern0pt}\ {\isadigit{3}}{\isacharparenright}{\kern0pt}\ {\isacharasterisk}{\kern0pt}\ C{\isachardoublequoteclose}\ \isacommand{by}\isamarkupfalse%
\ {\isacharparenleft}{\kern0pt}subst\ abs{\isacharunderscore}{\kern0pt}minus{\isacharbrackleft}{\kern0pt}of\ {\isachardoublequoteopen}f\ p\ {\isacharminus}{\kern0pt}\ p\ {\isacharasterisk}{\kern0pt}\ f\ {\isadigit{1}}{\isachardoublequoteclose}{\isacharcomma}{\kern0pt}\ symmetric{\isacharbrackright}{\kern0pt}{\isacharcomma}{\kern0pt}\ simp{\isacharparenright}{\kern0pt}\isanewline
\ \ \ \ \isacommand{hence}\isamarkupfalse%
\ {\isachardoublequoteopen}{\isasymbar}f\ p{\isasymbar}\ {\isasymle}\ {\isacharparenleft}{\kern0pt}{\isasymbar}p{\isasymbar}\ {\isacharplus}{\kern0pt}\ {\isadigit{3}}{\isacharparenright}{\kern0pt}\ {\isacharasterisk}{\kern0pt}\ C\ {\isacharplus}{\kern0pt}\ {\isasymbar}p\ {\isacharasterisk}{\kern0pt}\ f\ {\isadigit{1}}{\isasymbar}{\isachardoublequoteclose}\ \isacommand{using}\isamarkupfalse%
\ dual{\isacharunderscore}{\kern0pt}order{\isachardot}{\kern0pt}trans\ abs{\isacharunderscore}{\kern0pt}triangle{\isacharunderscore}{\kern0pt}ineq{\isadigit{2}}\ diff{\isacharunderscore}{\kern0pt}le{\isacharunderscore}{\kern0pt}eq\ \isacommand{by}\isamarkupfalse%
\ fast\isanewline
\ \ \ \ \isacommand{hence}\isamarkupfalse%
\ {\isachardoublequoteopen}{\isasymbar}f\ p{\isasymbar}\ {\isasymle}\ {\isasymbar}p{\isasymbar}\ {\isacharasterisk}{\kern0pt}\ C\ {\isacharplus}{\kern0pt}\ {\isadigit{3}}\ {\isacharasterisk}{\kern0pt}\ C\ {\isacharplus}{\kern0pt}\ {\isasymbar}p{\isasymbar}\ {\isacharasterisk}{\kern0pt}\ {\isasymbar}f\ {\isadigit{1}}{\isasymbar}{\isachardoublequoteclose}\ \isacommand{by}\isamarkupfalse%
\ {\isacharparenleft}{\kern0pt}simp\ add{\isacharcolon}{\kern0pt}\ abs{\isacharunderscore}{\kern0pt}mult\ int{\isacharunderscore}{\kern0pt}distrib{\isacharparenleft}{\kern0pt}{\isadigit{2}}{\isacharparenright}{\kern0pt}\ mult{\isachardot}{\kern0pt}commute{\isacharparenright}{\kern0pt}\isanewline
\ \ \ \ \isacommand{hence}\isamarkupfalse%
\ {\isachardoublequoteopen}{\isasymbar}f\ p{\isasymbar}\ {\isasymle}\ {\isasymbar}p{\isasymbar}\ {\isacharasterisk}{\kern0pt}\ {\isacharparenleft}{\kern0pt}C\ {\isacharplus}{\kern0pt}\ {\isasymbar}f\ {\isadigit{1}}{\isasymbar}{\isacharparenright}{\kern0pt}\ {\isacharplus}{\kern0pt}\ {\isadigit{3}}\ {\isacharasterisk}{\kern0pt}\ C{\isachardoublequoteclose}\ \isacommand{by}\isamarkupfalse%
\ {\isacharparenleft}{\kern0pt}simp\ add{\isacharcolon}{\kern0pt}\ ring{\isacharunderscore}{\kern0pt}class{\isachardot}{\kern0pt}ring{\isacharunderscore}{\kern0pt}distribs{\isacharparenleft}{\kern0pt}{\isadigit{1}}{\isacharparenright}{\kern0pt}{\isacharparenright}{\kern0pt}\isanewline
\ \ \ \ \isacommand{thus}\isamarkupfalse%
\ {\isacharquery}{\kern0pt}thesis\ \isacommand{using}\isamarkupfalse%
\ mult{\isachardot}{\kern0pt}commute\ \isacommand{by}\isamarkupfalse%
\ metis\isanewline
\ \ \isacommand{qed}\isamarkupfalse%
\isanewline
\ \ \isacommand{thus}\isamarkupfalse%
\ {\isacharquery}{\kern0pt}thesis\ \isacommand{using}\isamarkupfalse%
\ that\ bound{\isacharparenleft}{\kern0pt}{\isadigit{2}}{\isacharparenright}{\kern0pt}\ \isacommand{by}\isamarkupfalse%
\ fastforce\isanewline
\isacommand{qed}\isamarkupfalse%
%
\endisatagproof
{\isafoldproof}%
%
\isadelimproof
\isanewline
%
\endisadelimproof
\isanewline
\isacommand{lemma}\isamarkupfalse%
\ slope{\isacharunderscore}{\kern0pt}comp{\isacharcolon}{\kern0pt}\isanewline
\ \ \isakeyword{assumes}\ {\isachardoublequoteopen}slope\ f{\isachardoublequoteclose}\ {\isachardoublequoteopen}slope\ g{\isachardoublequoteclose}\isanewline
\ \ \isakeyword{shows}\ {\isachardoublequoteopen}slope\ {\isacharparenleft}{\kern0pt}f\ o\ g{\isacharparenright}{\kern0pt}{\isachardoublequoteclose}\isanewline
%
\isadelimproof
%
\endisadelimproof
%
\isatagproof
\isacommand{proof}\isamarkupfalse%
{\isacharminus}{\kern0pt}\isanewline
\ \ \isacommand{obtain}\isamarkupfalse%
\ C\ \isakeyword{where}\ bound{\isacharcolon}{\kern0pt}\ {\isachardoublequoteopen}{\isasymbar}f\ {\isacharparenleft}{\kern0pt}m\ {\isacharplus}{\kern0pt}\ n{\isacharparenright}{\kern0pt}\ {\isacharminus}{\kern0pt}\ {\isacharparenleft}{\kern0pt}f\ m\ {\isacharplus}{\kern0pt}\ f\ n{\isacharparenright}{\kern0pt}{\isasymbar}\ {\isasymle}\ C{\isachardoublequoteclose}\ \isakeyword{for}\ m\ n\ \isacommand{using}\isamarkupfalse%
\ assms\ \isacommand{by}\isamarkupfalse%
\ fast\isanewline
\ \ \isacommand{obtain}\isamarkupfalse%
\ C{\isacharprime}{\kern0pt}\ \isakeyword{where}\ bound{\isacharprime}{\kern0pt}{\isacharcolon}{\kern0pt}\ {\isachardoublequoteopen}{\isasymbar}g\ {\isacharparenleft}{\kern0pt}m\ {\isacharplus}{\kern0pt}\ n{\isacharparenright}{\kern0pt}\ {\isacharminus}{\kern0pt}\ {\isacharparenleft}{\kern0pt}g\ m\ {\isacharplus}{\kern0pt}\ g\ n{\isacharparenright}{\kern0pt}{\isasymbar}\ {\isasymle}\ C{\isacharprime}{\kern0pt}{\isachardoublequoteclose}\ \isakeyword{for}\ m\ n\ \isacommand{using}\isamarkupfalse%
\ assms\ \isacommand{by}\isamarkupfalse%
\ fast\isanewline
\ \ \isacommand{obtain}\isamarkupfalse%
\ A\ B\ \isakeyword{where}\ f{\isacharunderscore}{\kern0pt}linear{\isacharunderscore}{\kern0pt}bound{\isacharcolon}{\kern0pt}\ {\isachardoublequoteopen}{\isasymbar}f\ n{\isasymbar}\ {\isasymle}\ A\ {\isacharasterisk}{\kern0pt}\ {\isasymbar}n{\isasymbar}\ {\isacharplus}{\kern0pt}\ B{\isachardoublequoteclose}\ {\isachardoublequoteopen}{\isadigit{0}}\ {\isasymle}\ A{\isachardoublequoteclose}\ {\isachardoublequoteopen}{\isadigit{0}}\ {\isasymle}\ B{\isachardoublequoteclose}\ \isakeyword{for}\ n\ \isacommand{using}\isamarkupfalse%
\ slope{\isacharunderscore}{\kern0pt}linear{\isacharunderscore}{\kern0pt}bound{\isacharbrackleft}{\kern0pt}OF\ assms{\isacharparenleft}{\kern0pt}{\isadigit{1}}{\isacharparenright}{\kern0pt}{\isacharbrackright}{\kern0pt}\ \isacommand{by}\isamarkupfalse%
\ blast\isanewline
\ \ \isacommand{{\isacharbraceleft}{\kern0pt}}\isamarkupfalse%
\isanewline
\ \ \ \ \isacommand{fix}\isamarkupfalse%
\ m\ n\isanewline
\ \ \ \ \isacommand{have}\isamarkupfalse%
\ {\isachardoublequoteopen}{\isasymbar}f\ {\isacharparenleft}{\kern0pt}g\ {\isacharparenleft}{\kern0pt}m\ {\isacharplus}{\kern0pt}\ n{\isacharparenright}{\kern0pt}{\isacharparenright}{\kern0pt}\ {\isacharminus}{\kern0pt}\ {\isacharparenleft}{\kern0pt}f\ {\isacharparenleft}{\kern0pt}g\ m{\isacharparenright}{\kern0pt}\ {\isacharplus}{\kern0pt}\ f\ {\isacharparenleft}{\kern0pt}g\ n{\isacharparenright}{\kern0pt}{\isacharparenright}{\kern0pt}{\isasymbar}\ {\isasymle}\ {\isacharparenleft}{\kern0pt}{\isasymbar}f\ {\isacharparenleft}{\kern0pt}g\ {\isacharparenleft}{\kern0pt}m\ {\isacharplus}{\kern0pt}\ n{\isacharparenright}{\kern0pt}{\isacharparenright}{\kern0pt}\ {\isacharminus}{\kern0pt}\ f\ {\isacharparenleft}{\kern0pt}g\ m\ {\isacharplus}{\kern0pt}\ g\ n{\isacharparenright}{\kern0pt}{\isasymbar}\ {\isacharplus}{\kern0pt}\ {\isasymbar}f\ {\isacharparenleft}{\kern0pt}g\ m\ {\isacharplus}{\kern0pt}\ g\ n{\isacharparenright}{\kern0pt}\ {\isacharminus}{\kern0pt}\ {\isacharparenleft}{\kern0pt}f\ {\isacharparenleft}{\kern0pt}g\ m{\isacharparenright}{\kern0pt}\ {\isacharplus}{\kern0pt}\ f\ {\isacharparenleft}{\kern0pt}g\ n{\isacharparenright}{\kern0pt}{\isacharparenright}{\kern0pt}{\isasymbar}\ {\isacharcolon}{\kern0pt}{\isacharcolon}{\kern0pt}\ int{\isacharparenright}{\kern0pt}{\isachardoublequoteclose}\ \isacommand{by}\isamarkupfalse%
\ linarith\isanewline
\ \ \ \ \isacommand{also}\isamarkupfalse%
\ \isacommand{have}\isamarkupfalse%
\ {\isachardoublequoteopen}{\isachardot}{\kern0pt}{\isachardot}{\kern0pt}{\isachardot}{\kern0pt}\ {\isasymle}\ {\isasymbar}f\ {\isacharparenleft}{\kern0pt}g\ {\isacharparenleft}{\kern0pt}m\ {\isacharplus}{\kern0pt}\ n{\isacharparenright}{\kern0pt}{\isacharparenright}{\kern0pt}\ {\isacharminus}{\kern0pt}\ f\ {\isacharparenleft}{\kern0pt}g\ m\ {\isacharplus}{\kern0pt}\ g\ n{\isacharparenright}{\kern0pt}{\isasymbar}\ {\isacharplus}{\kern0pt}\ C{\isachardoublequoteclose}\ \isacommand{using}\isamarkupfalse%
\ bound{\isacharbrackleft}{\kern0pt}of\ {\isachardoublequoteopen}g\ m{\isachardoublequoteclose}\ {\isachardoublequoteopen}g\ n{\isachardoublequoteclose}{\isacharbrackright}{\kern0pt}\ \isacommand{by}\isamarkupfalse%
\ auto\isanewline
\ \ \ \ \isacommand{also}\isamarkupfalse%
\ \isacommand{have}\isamarkupfalse%
\ {\isachardoublequoteopen}{\isachardot}{\kern0pt}{\isachardot}{\kern0pt}{\isachardot}{\kern0pt}\ {\isasymle}\ {\isasymbar}f\ {\isacharparenleft}{\kern0pt}g\ {\isacharparenleft}{\kern0pt}m\ {\isacharplus}{\kern0pt}\ n{\isacharparenright}{\kern0pt}{\isacharparenright}{\kern0pt}\ {\isacharminus}{\kern0pt}\ f\ {\isacharparenleft}{\kern0pt}g\ m\ {\isacharplus}{\kern0pt}\ g\ n{\isacharparenright}{\kern0pt}\ {\isacharminus}{\kern0pt}\ f{\isacharparenleft}{\kern0pt}g\ {\isacharparenleft}{\kern0pt}m\ {\isacharplus}{\kern0pt}\ n{\isacharparenright}{\kern0pt}\ {\isacharminus}{\kern0pt}\ {\isacharparenleft}{\kern0pt}g\ m\ {\isacharplus}{\kern0pt}\ g\ n{\isacharparenright}{\kern0pt}{\isacharparenright}{\kern0pt}{\isasymbar}\ {\isacharplus}{\kern0pt}\ {\isasymbar}f\ {\isacharparenleft}{\kern0pt}g\ {\isacharparenleft}{\kern0pt}m\ {\isacharplus}{\kern0pt}\ n{\isacharparenright}{\kern0pt}\ {\isacharminus}{\kern0pt}\ {\isacharparenleft}{\kern0pt}g\ m\ {\isacharplus}{\kern0pt}\ g\ n{\isacharparenright}{\kern0pt}{\isacharparenright}{\kern0pt}{\isasymbar}\ {\isacharplus}{\kern0pt}\ C{\isachardoublequoteclose}\ \isacommand{by}\isamarkupfalse%
\ fastforce\isanewline
\ \ \ \ \isacommand{also}\isamarkupfalse%
\ \isacommand{have}\isamarkupfalse%
\ {\isachardoublequoteopen}{\isachardot}{\kern0pt}{\isachardot}{\kern0pt}{\isachardot}{\kern0pt}\ {\isasymle}\ {\isasymbar}f\ {\isacharparenleft}{\kern0pt}g\ {\isacharparenleft}{\kern0pt}m\ {\isacharplus}{\kern0pt}\ n{\isacharparenright}{\kern0pt}\ {\isacharminus}{\kern0pt}\ {\isacharparenleft}{\kern0pt}g\ m\ {\isacharplus}{\kern0pt}\ g\ n{\isacharparenright}{\kern0pt}{\isacharparenright}{\kern0pt}{\isasymbar}\ {\isacharplus}{\kern0pt}\ {\isadigit{2}}\ {\isacharasterisk}{\kern0pt}\ C{\isachardoublequoteclose}\ \isacommand{using}\isamarkupfalse%
\ bound{\isacharbrackleft}{\kern0pt}of\ {\isachardoublequoteopen}g\ {\isacharparenleft}{\kern0pt}m\ {\isacharplus}{\kern0pt}\ n{\isacharparenright}{\kern0pt}\ {\isacharminus}{\kern0pt}\ {\isacharparenleft}{\kern0pt}g\ m\ {\isacharplus}{\kern0pt}\ g\ n{\isacharparenright}{\kern0pt}{\isachardoublequoteclose}\ {\isachardoublequoteopen}{\isacharparenleft}{\kern0pt}g\ m\ {\isacharplus}{\kern0pt}\ g\ n{\isacharparenright}{\kern0pt}{\isachardoublequoteclose}{\isacharbrackright}{\kern0pt}\ \isacommand{by}\isamarkupfalse%
\ fastforce\isanewline
\ \ \ \ \isacommand{also}\isamarkupfalse%
\ \isacommand{have}\isamarkupfalse%
\ {\isachardoublequoteopen}{\isachardot}{\kern0pt}{\isachardot}{\kern0pt}{\isachardot}{\kern0pt}\ {\isasymle}\ A\ {\isacharasterisk}{\kern0pt}\ {\isasymbar}g\ {\isacharparenleft}{\kern0pt}m\ {\isacharplus}{\kern0pt}\ n{\isacharparenright}{\kern0pt}\ {\isacharminus}{\kern0pt}\ {\isacharparenleft}{\kern0pt}g\ m\ {\isacharplus}{\kern0pt}\ g\ n{\isacharparenright}{\kern0pt}{\isasymbar}\ {\isacharplus}{\kern0pt}\ B\ {\isacharplus}{\kern0pt}\ {\isadigit{2}}\ {\isacharasterisk}{\kern0pt}\ C{\isachardoublequoteclose}\ \isacommand{using}\isamarkupfalse%
\ f{\isacharunderscore}{\kern0pt}linear{\isacharunderscore}{\kern0pt}bound{\isacharparenleft}{\kern0pt}{\isadigit{1}}{\isacharparenright}{\kern0pt}{\isacharbrackleft}{\kern0pt}of\ {\isachardoublequoteopen}g\ {\isacharparenleft}{\kern0pt}m\ {\isacharplus}{\kern0pt}\ n{\isacharparenright}{\kern0pt}\ {\isacharminus}{\kern0pt}\ {\isacharparenleft}{\kern0pt}g\ m\ {\isacharplus}{\kern0pt}\ g\ n{\isacharparenright}{\kern0pt}{\isachardoublequoteclose}{\isacharbrackright}{\kern0pt}\ \isacommand{by}\isamarkupfalse%
\ linarith\isanewline
\ \ \ \ \isacommand{also}\isamarkupfalse%
\ \isacommand{have}\isamarkupfalse%
\ {\isachardoublequoteopen}{\isachardot}{\kern0pt}{\isachardot}{\kern0pt}{\isachardot}{\kern0pt}\ {\isasymle}\ A\ {\isacharasterisk}{\kern0pt}\ C{\isacharprime}{\kern0pt}\ {\isacharplus}{\kern0pt}\ B\ {\isacharplus}{\kern0pt}\ {\isadigit{2}}\ {\isacharasterisk}{\kern0pt}\ C{\isachardoublequoteclose}\ \isacommand{using}\isamarkupfalse%
\ mult{\isacharunderscore}{\kern0pt}left{\isacharunderscore}{\kern0pt}mono{\isacharbrackleft}{\kern0pt}OF\ bound{\isacharprime}{\kern0pt}{\isacharbrackleft}{\kern0pt}of\ m\ n{\isacharbrackright}{\kern0pt}{\isacharcomma}{\kern0pt}\ OF\ f{\isacharunderscore}{\kern0pt}linear{\isacharunderscore}{\kern0pt}bound{\isacharparenleft}{\kern0pt}{\isadigit{2}}{\isacharparenright}{\kern0pt}{\isacharbrackright}{\kern0pt}\ \isacommand{by}\isamarkupfalse%
\ presburger\isanewline
\ \ \ \ \isacommand{finally}\isamarkupfalse%
\ \isacommand{have}\isamarkupfalse%
\ {\isachardoublequoteopen}{\isasymbar}f\ {\isacharparenleft}{\kern0pt}g\ {\isacharparenleft}{\kern0pt}m\ {\isacharplus}{\kern0pt}\ n{\isacharparenright}{\kern0pt}{\isacharparenright}{\kern0pt}\ {\isacharminus}{\kern0pt}\ {\isacharparenleft}{\kern0pt}f\ {\isacharparenleft}{\kern0pt}g\ m{\isacharparenright}{\kern0pt}\ {\isacharplus}{\kern0pt}\ f\ {\isacharparenleft}{\kern0pt}g\ n{\isacharparenright}{\kern0pt}{\isacharparenright}{\kern0pt}{\isasymbar}\ {\isasymle}\ A\ {\isacharasterisk}{\kern0pt}\ C{\isacharprime}{\kern0pt}\ {\isacharplus}{\kern0pt}\ B\ {\isacharplus}{\kern0pt}\ {\isadigit{2}}\ {\isacharasterisk}{\kern0pt}\ C{\isachardoublequoteclose}\ \isacommand{by}\isamarkupfalse%
\ blast\isanewline
\ \ \isacommand{{\isacharbraceright}{\kern0pt}}\isamarkupfalse%
\isanewline
\ \ \isacommand{thus}\isamarkupfalse%
\ {\isachardoublequoteopen}slope\ {\isacharparenleft}{\kern0pt}f\ o\ g{\isacharparenright}{\kern0pt}{\isachardoublequoteclose}\ \isacommand{unfolding}\isamarkupfalse%
\ comp{\isacharunderscore}{\kern0pt}def\ slope{\isacharunderscore}{\kern0pt}def\ \isacommand{by}\isamarkupfalse%
\ {\isacharparenleft}{\kern0pt}fast\ intro{\isacharcolon}{\kern0pt}\ boundedI{\isacharparenright}{\kern0pt}\isanewline
\isacommand{qed}\isamarkupfalse%
%
\endisatagproof
{\isafoldproof}%
%
\isadelimproof
\isanewline
%
\endisadelimproof
\isanewline
\isacommand{lemma}\isamarkupfalse%
\ slope{\isacharunderscore}{\kern0pt}scale{\isacharcolon}{\kern0pt}\ {\isachardoublequoteopen}slope\ {\isacharparenleft}{\kern0pt}{\isacharparenleft}{\kern0pt}{\isacharasterisk}{\kern0pt}{\isacharparenright}{\kern0pt}\ a{\isacharparenright}{\kern0pt}{\isachardoublequoteclose}%
\isadelimproof
\ %
\endisadelimproof
%
\isatagproof
\isacommand{by}\isamarkupfalse%
\ {\isacharparenleft}{\kern0pt}auto\ simp\ add{\isacharcolon}{\kern0pt}\ slope{\isacharunderscore}{\kern0pt}def\ distrib{\isacharunderscore}{\kern0pt}left\ intro{\isacharcolon}{\kern0pt}\ boundedI{\isacharparenright}{\kern0pt}%
\endisatagproof
{\isafoldproof}%
%
\isadelimproof
%
\endisadelimproof
\isanewline
\isanewline
\isacommand{lemma}\isamarkupfalse%
\ slope{\isacharunderscore}{\kern0pt}zero{\isacharcolon}{\kern0pt}\ {\isachardoublequoteopen}slope\ {\isacharparenleft}{\kern0pt}{\isasymlambda}{\isacharunderscore}{\kern0pt}{\isachardot}{\kern0pt}\ {\isadigit{0}}{\isacharparenright}{\kern0pt}{\isachardoublequoteclose}%
\isadelimproof
\ %
\endisadelimproof
%
\isatagproof
\isacommand{using}\isamarkupfalse%
\ slope{\isacharunderscore}{\kern0pt}scale{\isacharbrackleft}{\kern0pt}of\ {\isadigit{0}}{\isacharbrackright}{\kern0pt}\ \isacommand{by}\isamarkupfalse%
\ {\isacharparenleft}{\kern0pt}simp\ add{\isacharcolon}{\kern0pt}\ lambda{\isacharunderscore}{\kern0pt}zero{\isacharparenright}{\kern0pt}%
\endisatagproof
{\isafoldproof}%
%
\isadelimproof
%
\endisadelimproof
\isanewline
\isanewline
\isacommand{lemma}\isamarkupfalse%
\ slope{\isacharunderscore}{\kern0pt}one{\isacharcolon}{\kern0pt}\ {\isachardoublequoteopen}slope\ id{\isachardoublequoteclose}%
\isadelimproof
\ %
\endisadelimproof
%
\isatagproof
\isacommand{using}\isamarkupfalse%
\ slope{\isacharunderscore}{\kern0pt}scale{\isacharbrackleft}{\kern0pt}of\ {\isadigit{1}}{\isacharbrackright}{\kern0pt}\ \isacommand{by}\isamarkupfalse%
\ {\isacharparenleft}{\kern0pt}simp\ add{\isacharcolon}{\kern0pt}\ slope{\isacharunderscore}{\kern0pt}def{\isacharparenright}{\kern0pt}%
\endisatagproof
{\isafoldproof}%
%
\isadelimproof
%
\endisadelimproof
\isanewline
\isanewline
\isacommand{lemma}\isamarkupfalse%
\ slope{\isacharunderscore}{\kern0pt}uminus{\isacharcolon}{\kern0pt}\ {\isachardoublequoteopen}slope\ uminus{\isachardoublequoteclose}%
\isadelimproof
\ %
\endisadelimproof
%
\isatagproof
\isacommand{using}\isamarkupfalse%
\ slope{\isacharunderscore}{\kern0pt}scale{\isacharbrackleft}{\kern0pt}of\ {\isachardoublequoteopen}{\isacharminus}{\kern0pt}{\isadigit{1}}{\isachardoublequoteclose}{\isacharbrackright}{\kern0pt}\ \isacommand{by}\isamarkupfalse%
\ {\isacharparenleft}{\kern0pt}simp\ add{\isacharcolon}{\kern0pt}\ slope{\isacharunderscore}{\kern0pt}def{\isacharparenright}{\kern0pt}%
\endisatagproof
{\isafoldproof}%
%
\isadelimproof
%
\endisadelimproof
\isanewline
\isanewline
\isacommand{lemma}\isamarkupfalse%
\ slope{\isacharunderscore}{\kern0pt}uminus{\isacharprime}{\kern0pt}{\isacharcolon}{\kern0pt}\isanewline
\ \ \isakeyword{assumes}\ {\isachardoublequoteopen}slope\ f{\isachardoublequoteclose}\isanewline
\ \ \isakeyword{shows}\ {\isachardoublequoteopen}slope\ {\isacharparenleft}{\kern0pt}{\isasymlambda}x{\isachardot}{\kern0pt}\ {\isacharminus}{\kern0pt}\ f\ x{\isacharparenright}{\kern0pt}{\isachardoublequoteclose}\isanewline
%
\isadelimproof
\ \ %
\endisadelimproof
%
\isatagproof
\isacommand{using}\isamarkupfalse%
\ slope{\isacharunderscore}{\kern0pt}comp{\isacharbrackleft}{\kern0pt}OF\ slope{\isacharunderscore}{\kern0pt}uminus\ assms{\isacharbrackright}{\kern0pt}\ \isacommand{by}\isamarkupfalse%
\ {\isacharparenleft}{\kern0pt}simp\ add{\isacharcolon}{\kern0pt}\ slope{\isacharunderscore}{\kern0pt}def{\isacharparenright}{\kern0pt}%
\endisatagproof
{\isafoldproof}%
%
\isadelimproof
\isanewline
%
\endisadelimproof
\isanewline
\isacommand{lemma}\isamarkupfalse%
\ slope{\isacharunderscore}{\kern0pt}minus{\isacharcolon}{\kern0pt}\isanewline
\ \ \isakeyword{assumes}\ {\isachardoublequoteopen}slope\ f{\isachardoublequoteclose}\ {\isachardoublequoteopen}slope\ g{\isachardoublequoteclose}\isanewline
\ \ \isakeyword{shows}\ {\isachardoublequoteopen}slope\ {\isacharparenleft}{\kern0pt}{\isasymlambda}x{\isachardot}{\kern0pt}\ f\ x\ {\isacharminus}{\kern0pt}\ g\ x{\isacharparenright}{\kern0pt}{\isachardoublequoteclose}\isanewline
%
\isadelimproof
\ \ %
\endisadelimproof
%
\isatagproof
\isacommand{using}\isamarkupfalse%
\ slope{\isacharunderscore}{\kern0pt}add{\isacharbrackleft}{\kern0pt}OF\ assms{\isacharparenleft}{\kern0pt}{\isadigit{1}}{\isacharparenright}{\kern0pt}\ slope{\isacharunderscore}{\kern0pt}uminus{\isacharprime}{\kern0pt}{\isacharcomma}{\kern0pt}\ OF\ assms{\isacharparenleft}{\kern0pt}{\isadigit{2}}{\isacharparenright}{\kern0pt}{\isacharbrackright}{\kern0pt}\ \isacommand{by}\isamarkupfalse%
\ simp%
\endisatagproof
{\isafoldproof}%
%
\isadelimproof
\isanewline
%
\endisadelimproof
\isanewline
\isacommand{lemma}\isamarkupfalse%
\ slope{\isacharunderscore}{\kern0pt}comp{\isacharunderscore}{\kern0pt}commute{\isacharcolon}{\kern0pt}\isanewline
\ \ \isakeyword{assumes}\ {\isachardoublequoteopen}slope\ f{\isachardoublequoteclose}\ {\isachardoublequoteopen}slope\ g{\isachardoublequoteclose}\isanewline
\ \ \isakeyword{shows}\ {\isachardoublequoteopen}bounded\ {\isacharparenleft}{\kern0pt}{\isasymlambda}z{\isachardot}{\kern0pt}\ {\isacharparenleft}{\kern0pt}f\ o\ g{\isacharparenright}{\kern0pt}\ z\ {\isacharminus}{\kern0pt}\ {\isacharparenleft}{\kern0pt}g\ o\ f{\isacharparenright}{\kern0pt}\ z{\isacharparenright}{\kern0pt}{\isachardoublequoteclose}\isanewline
%
\isadelimproof
%
\endisadelimproof
%
\isatagproof
\isacommand{proof}\isamarkupfalse%
\ {\isacharminus}{\kern0pt}\isanewline
\ \ \isacommand{obtain}\isamarkupfalse%
\ C\ \isakeyword{where}\ bound{\isacharcolon}{\kern0pt}\ {\isachardoublequoteopen}{\isasymbar}z\ {\isacharasterisk}{\kern0pt}\ f\ {\isacharparenleft}{\kern0pt}g\ z{\isacharparenright}{\kern0pt}\ {\isacharminus}{\kern0pt}\ {\isacharparenleft}{\kern0pt}g\ z{\isacharparenright}{\kern0pt}\ {\isacharasterisk}{\kern0pt}\ {\isacharparenleft}{\kern0pt}f\ z{\isacharparenright}{\kern0pt}{\isasymbar}\ {\isasymle}\ {\isacharparenleft}{\kern0pt}{\isasymbar}z{\isasymbar}\ {\isacharplus}{\kern0pt}\ {\isasymbar}g\ z{\isasymbar}\ {\isacharplus}{\kern0pt}\ {\isadigit{2}}{\isacharparenright}{\kern0pt}\ {\isacharasterisk}{\kern0pt}\ C{\isachardoublequoteclose}\ {\isachardoublequoteopen}{\isadigit{0}}\ {\isasymle}\ C{\isachardoublequoteclose}\ \isakeyword{for}\ z\ \isacommand{using}\isamarkupfalse%
\ slope{\isacharunderscore}{\kern0pt}symmetric{\isacharunderscore}{\kern0pt}bound{\isacharbrackleft}{\kern0pt}OF\ assms{\isacharparenleft}{\kern0pt}{\isadigit{1}}{\isacharparenright}{\kern0pt}{\isacharbrackright}{\kern0pt}\ \isacommand{by}\isamarkupfalse%
\ metis\isanewline
\ \ \isacommand{obtain}\isamarkupfalse%
\ C{\isacharprime}{\kern0pt}\ \isakeyword{where}\ bound{\isacharprime}{\kern0pt}{\isacharcolon}{\kern0pt}\ {\isachardoublequoteopen}{\isasymbar}{\isacharparenleft}{\kern0pt}f\ z{\isacharparenright}{\kern0pt}\ {\isacharasterisk}{\kern0pt}\ {\isacharparenleft}{\kern0pt}g\ z{\isacharparenright}{\kern0pt}\ {\isacharminus}{\kern0pt}\ z\ {\isacharasterisk}{\kern0pt}\ g\ {\isacharparenleft}{\kern0pt}f\ z{\isacharparenright}{\kern0pt}{\isasymbar}\ {\isasymle}\ {\isacharparenleft}{\kern0pt}{\isasymbar}f\ z{\isasymbar}\ {\isacharplus}{\kern0pt}\ {\isasymbar}z{\isasymbar}\ {\isacharplus}{\kern0pt}\ {\isadigit{2}}{\isacharparenright}{\kern0pt}\ {\isacharasterisk}{\kern0pt}\ C{\isacharprime}{\kern0pt}{\isachardoublequoteclose}\ {\isachardoublequoteopen}{\isadigit{0}}\ {\isasymle}\ C{\isacharprime}{\kern0pt}{\isachardoublequoteclose}\ \isakeyword{for}\ z\ \isacommand{using}\isamarkupfalse%
\ slope{\isacharunderscore}{\kern0pt}symmetric{\isacharunderscore}{\kern0pt}bound{\isacharbrackleft}{\kern0pt}OF\ assms{\isacharparenleft}{\kern0pt}{\isadigit{2}}{\isacharparenright}{\kern0pt}{\isacharbrackright}{\kern0pt}\ \isacommand{by}\isamarkupfalse%
\ metis\isanewline
\isanewline
\ \ \isacommand{obtain}\isamarkupfalse%
\ A\ B\ \isakeyword{where}\ f{\isacharunderscore}{\kern0pt}lbound{\isacharcolon}{\kern0pt}\ {\isachardoublequoteopen}{\isasymbar}f\ z{\isasymbar}\ {\isasymle}\ A\ {\isacharasterisk}{\kern0pt}\ {\isasymbar}z{\isasymbar}\ {\isacharplus}{\kern0pt}\ B{\isachardoublequoteclose}\ {\isachardoublequoteopen}{\isadigit{0}}\ {\isasymle}\ A{\isachardoublequoteclose}\ {\isachardoublequoteopen}{\isadigit{0}}\ {\isasymle}\ B{\isachardoublequoteclose}\ \isakeyword{for}\ z\ \isacommand{using}\isamarkupfalse%
\ slope{\isacharunderscore}{\kern0pt}linear{\isacharunderscore}{\kern0pt}bound{\isacharbrackleft}{\kern0pt}OF\ assms{\isacharparenleft}{\kern0pt}{\isadigit{1}}{\isacharparenright}{\kern0pt}{\isacharbrackright}{\kern0pt}\ \isacommand{by}\isamarkupfalse%
\ blast\isanewline
\ \ \isacommand{obtain}\isamarkupfalse%
\ A{\isacharprime}{\kern0pt}\ B{\isacharprime}{\kern0pt}\ \isakeyword{where}\ g{\isacharunderscore}{\kern0pt}lbound{\isacharcolon}{\kern0pt}\ {\isachardoublequoteopen}{\isasymbar}g\ z{\isasymbar}\ {\isasymle}\ A{\isacharprime}{\kern0pt}\ {\isacharasterisk}{\kern0pt}\ {\isasymbar}z{\isasymbar}\ {\isacharplus}{\kern0pt}\ B{\isacharprime}{\kern0pt}{\isachardoublequoteclose}\ {\isachardoublequoteopen}{\isadigit{0}}\ {\isasymle}\ A{\isacharprime}{\kern0pt}{\isachardoublequoteclose}\ {\isachardoublequoteopen}{\isadigit{0}}\ {\isasymle}\ B{\isacharprime}{\kern0pt}{\isachardoublequoteclose}\ \isakeyword{for}\ z\ \isacommand{using}\isamarkupfalse%
\ slope{\isacharunderscore}{\kern0pt}linear{\isacharunderscore}{\kern0pt}bound{\isacharbrackleft}{\kern0pt}OF\ assms{\isacharparenleft}{\kern0pt}{\isadigit{2}}{\isacharparenright}{\kern0pt}{\isacharbrackright}{\kern0pt}\ \isacommand{by}\isamarkupfalse%
\ blast\isanewline
\isanewline
\ \ \isacommand{have}\isamarkupfalse%
\ combined{\isacharunderscore}{\kern0pt}bound{\isacharcolon}{\kern0pt}\ {\isachardoublequoteopen}{\isasymbar}z\ {\isacharasterisk}{\kern0pt}\ f\ {\isacharparenleft}{\kern0pt}g\ z{\isacharparenright}{\kern0pt}\ {\isacharminus}{\kern0pt}\ z\ {\isacharasterisk}{\kern0pt}\ g\ {\isacharparenleft}{\kern0pt}f\ z{\isacharparenright}{\kern0pt}{\isasymbar}\ {\isasymle}\ {\isacharparenleft}{\kern0pt}{\isasymbar}z{\isasymbar}\ {\isacharplus}{\kern0pt}\ {\isasymbar}g\ z{\isasymbar}\ {\isacharplus}{\kern0pt}\ {\isadigit{2}}{\isacharparenright}{\kern0pt}\ {\isacharasterisk}{\kern0pt}\ C\ {\isacharplus}{\kern0pt}\ {\isacharparenleft}{\kern0pt}{\isasymbar}f\ z{\isasymbar}\ {\isacharplus}{\kern0pt}\ {\isasymbar}z{\isasymbar}\ {\isacharplus}{\kern0pt}\ {\isadigit{2}}{\isacharparenright}{\kern0pt}\ {\isacharasterisk}{\kern0pt}\ C{\isacharprime}{\kern0pt}{\isachardoublequoteclose}\ \isakeyword{for}\ z\ \isanewline
\ \ \ \ \isacommand{by}\isamarkupfalse%
\ {\isacharparenleft}{\kern0pt}intro\ order{\isacharunderscore}{\kern0pt}trans{\isacharbrackleft}{\kern0pt}OF\ {\isacharunderscore}{\kern0pt}\ add{\isacharunderscore}{\kern0pt}mono{\isacharbrackleft}{\kern0pt}OF\ bound{\isacharparenleft}{\kern0pt}{\isadigit{1}}{\isacharparenright}{\kern0pt}\ bound{\isacharprime}{\kern0pt}{\isacharparenleft}{\kern0pt}{\isadigit{1}}{\isacharparenright}{\kern0pt}{\isacharbrackright}{\kern0pt}{\isacharbrackright}{\kern0pt}{\isacharparenright}{\kern0pt}\ {\isacharparenleft}{\kern0pt}fastforce\ simp\ add{\isacharcolon}{\kern0pt}\ mult{\isachardot}{\kern0pt}commute{\isacharbrackleft}{\kern0pt}of\ {\isachardoublequoteopen}f\ z{\isachardoublequoteclose}\ {\isachardoublequoteopen}g\ z{\isachardoublequoteclose}{\isacharbrackright}{\kern0pt}{\isacharparenright}{\kern0pt}\isanewline
\isanewline
\ \ \isacommand{{\isacharbraceleft}{\kern0pt}}\isamarkupfalse%
\isanewline
\ \ \ \ \isacommand{fix}\isamarkupfalse%
\ z\isanewline
\ \ \ \ \isacommand{define}\isamarkupfalse%
\ D\ E\ \isakeyword{where}\ {\isachardoublequoteopen}D\ {\isacharequal}{\kern0pt}\ {\isacharparenleft}{\kern0pt}C\ {\isacharplus}{\kern0pt}\ C{\isacharprime}{\kern0pt}\ {\isacharplus}{\kern0pt}\ A{\isacharprime}{\kern0pt}\ {\isacharasterisk}{\kern0pt}\ C\ {\isacharplus}{\kern0pt}\ A\ {\isacharasterisk}{\kern0pt}\ C{\isacharprime}{\kern0pt}{\isacharparenright}{\kern0pt}{\isachardoublequoteclose}\ \isakeyword{and}\ {\isachardoublequoteopen}E\ {\isacharequal}{\kern0pt}\ {\isacharparenleft}{\kern0pt}{\isadigit{2}}\ {\isacharplus}{\kern0pt}\ B{\isacharprime}{\kern0pt}{\isacharparenright}{\kern0pt}\ {\isacharasterisk}{\kern0pt}\ C\ {\isacharplus}{\kern0pt}\ {\isacharparenleft}{\kern0pt}{\isadigit{2}}\ {\isacharplus}{\kern0pt}\ B{\isacharparenright}{\kern0pt}\ {\isacharasterisk}{\kern0pt}\ C{\isacharprime}{\kern0pt}{\isachardoublequoteclose}\isanewline
\ \ \ \ \isacommand{have}\isamarkupfalse%
\ E{\isacharunderscore}{\kern0pt}nonneg{\isacharcolon}{\kern0pt}\ {\isachardoublequoteopen}{\isadigit{0}}\ {\isasymle}\ E{\isachardoublequoteclose}\ \isacommand{unfolding}\isamarkupfalse%
\ E{\isacharunderscore}{\kern0pt}def\ \isacommand{using}\isamarkupfalse%
\ g{\isacharunderscore}{\kern0pt}lbound\ bound\ f{\isacharunderscore}{\kern0pt}lbound\ bound{\isacharprime}{\kern0pt}\ \isacommand{by}\isamarkupfalse%
\ simp\isanewline
\ \ \ \ \isacommand{have}\isamarkupfalse%
\ D{\isacharunderscore}{\kern0pt}nonneg{\isacharcolon}{\kern0pt}\ {\isachardoublequoteopen}{\isadigit{0}}\ {\isasymle}\ D{\isachardoublequoteclose}\ \isacommand{unfolding}\isamarkupfalse%
\ D{\isacharunderscore}{\kern0pt}def\ \isacommand{using}\isamarkupfalse%
\ g{\isacharunderscore}{\kern0pt}lbound\ bound\ f{\isacharunderscore}{\kern0pt}lbound\ bound{\isacharprime}{\kern0pt}\ \isacommand{by}\isamarkupfalse%
\ simp\isanewline
\isanewline
\ \ \ \ \isacommand{have}\isamarkupfalse%
\ {\isachardoublequoteopen}{\isacharparenleft}{\kern0pt}{\isasymbar}z{\isasymbar}\ {\isacharplus}{\kern0pt}\ {\isasymbar}g\ z{\isasymbar}\ {\isacharplus}{\kern0pt}\ {\isadigit{2}}{\isacharparenright}{\kern0pt}\ {\isacharasterisk}{\kern0pt}\ C\ {\isacharplus}{\kern0pt}\ {\isacharparenleft}{\kern0pt}{\isasymbar}f\ z{\isasymbar}\ {\isacharplus}{\kern0pt}\ {\isasymbar}z{\isasymbar}\ {\isacharplus}{\kern0pt}\ {\isadigit{2}}{\isacharparenright}{\kern0pt}\ {\isacharasterisk}{\kern0pt}\ C{\isacharprime}{\kern0pt}\ {\isacharequal}{\kern0pt}\ {\isasymbar}z{\isasymbar}\ {\isacharasterisk}{\kern0pt}\ {\isacharparenleft}{\kern0pt}C\ {\isacharplus}{\kern0pt}\ C{\isacharprime}{\kern0pt}{\isacharparenright}{\kern0pt}\ {\isacharplus}{\kern0pt}\ {\isasymbar}g\ z{\isasymbar}\ {\isacharasterisk}{\kern0pt}\ C\ {\isacharplus}{\kern0pt}\ {\isasymbar}f\ z{\isasymbar}\ {\isacharasterisk}{\kern0pt}\ C{\isacharprime}{\kern0pt}\ {\isacharplus}{\kern0pt}\ {\isadigit{2}}\ {\isacharasterisk}{\kern0pt}\ C\ {\isacharplus}{\kern0pt}\ {\isadigit{2}}\ {\isacharasterisk}{\kern0pt}\ C{\isacharprime}{\kern0pt}{\isachardoublequoteclose}\ \isacommand{by}\isamarkupfalse%
\ algebra\isanewline
\ \ \ \ \isacommand{hence}\isamarkupfalse%
\ {\isachardoublequoteopen}{\isasymbar}z{\isasymbar}\ {\isacharasterisk}{\kern0pt}\ {\isasymbar}f\ {\isacharparenleft}{\kern0pt}g\ z{\isacharparenright}{\kern0pt}\ {\isacharminus}{\kern0pt}\ g\ {\isacharparenleft}{\kern0pt}f\ z{\isacharparenright}{\kern0pt}{\isasymbar}\ {\isasymle}\ {\isasymbar}z{\isasymbar}\ {\isacharasterisk}{\kern0pt}\ {\isacharparenleft}{\kern0pt}C\ {\isacharplus}{\kern0pt}\ C{\isacharprime}{\kern0pt}{\isacharparenright}{\kern0pt}\ {\isacharplus}{\kern0pt}\ {\isasymbar}g\ z{\isasymbar}\ {\isacharasterisk}{\kern0pt}\ C\ {\isacharplus}{\kern0pt}\ {\isasymbar}f\ z{\isasymbar}\ {\isacharasterisk}{\kern0pt}\ C{\isacharprime}{\kern0pt}\ {\isacharplus}{\kern0pt}\ {\isadigit{2}}\ {\isacharasterisk}{\kern0pt}\ C\ {\isacharplus}{\kern0pt}\ {\isadigit{2}}\ {\isacharasterisk}{\kern0pt}\ C{\isacharprime}{\kern0pt}{\isachardoublequoteclose}\ \isacommand{using}\isamarkupfalse%
\ combined{\isacharunderscore}{\kern0pt}bound\ right{\isacharunderscore}{\kern0pt}diff{\isacharunderscore}{\kern0pt}distrib\ abs{\isacharunderscore}{\kern0pt}mult\ \isacommand{by}\isamarkupfalse%
\ metis\isanewline
\ \ \ \ \isacommand{also}\isamarkupfalse%
\ \isacommand{have}\isamarkupfalse%
\ {\isachardoublequoteopen}{\isachardot}{\kern0pt}{\isachardot}{\kern0pt}{\isachardot}{\kern0pt}\ {\isasymle}\ {\isasymbar}z{\isasymbar}\ {\isacharasterisk}{\kern0pt}\ {\isacharparenleft}{\kern0pt}C\ {\isacharplus}{\kern0pt}\ C{\isacharprime}{\kern0pt}{\isacharparenright}{\kern0pt}\ {\isacharplus}{\kern0pt}\ {\isacharparenleft}{\kern0pt}A{\isacharprime}{\kern0pt}\ {\isacharasterisk}{\kern0pt}\ {\isasymbar}z{\isasymbar}\ {\isacharplus}{\kern0pt}\ B{\isacharprime}{\kern0pt}{\isacharparenright}{\kern0pt}\ {\isacharasterisk}{\kern0pt}\ C\ {\isacharplus}{\kern0pt}\ {\isasymbar}f\ z{\isasymbar}\ {\isacharasterisk}{\kern0pt}\ C{\isacharprime}{\kern0pt}\ {\isacharplus}{\kern0pt}\ {\isadigit{2}}\ {\isacharasterisk}{\kern0pt}\ C\ {\isacharplus}{\kern0pt}\ {\isadigit{2}}\ {\isacharasterisk}{\kern0pt}\ C{\isacharprime}{\kern0pt}{\isachardoublequoteclose}\ \isacommand{using}\isamarkupfalse%
\ mult{\isacharunderscore}{\kern0pt}right{\isacharunderscore}{\kern0pt}mono{\isacharbrackleft}{\kern0pt}OF\ g{\isacharunderscore}{\kern0pt}lbound{\isacharparenleft}{\kern0pt}{\isadigit{1}}{\isacharparenright}{\kern0pt}{\isacharbrackleft}{\kern0pt}of\ z{\isacharbrackright}{\kern0pt}\ bound{\isacharparenleft}{\kern0pt}{\isadigit{2}}{\isacharparenright}{\kern0pt}{\isacharbrackright}{\kern0pt}\ \isacommand{by}\isamarkupfalse%
\ presburger\isanewline
\ \ \ \ \isacommand{also}\isamarkupfalse%
\ \isacommand{have}\isamarkupfalse%
\ {\isachardoublequoteopen}{\isachardot}{\kern0pt}{\isachardot}{\kern0pt}{\isachardot}{\kern0pt}\ {\isasymle}\ {\isasymbar}z{\isasymbar}\ {\isacharasterisk}{\kern0pt}\ {\isacharparenleft}{\kern0pt}C\ {\isacharplus}{\kern0pt}\ C{\isacharprime}{\kern0pt}{\isacharparenright}{\kern0pt}\ {\isacharplus}{\kern0pt}\ {\isacharparenleft}{\kern0pt}A{\isacharprime}{\kern0pt}\ {\isacharasterisk}{\kern0pt}\ {\isasymbar}z{\isasymbar}\ {\isacharplus}{\kern0pt}\ B{\isacharprime}{\kern0pt}{\isacharparenright}{\kern0pt}\ {\isacharasterisk}{\kern0pt}\ C\ {\isacharplus}{\kern0pt}\ {\isacharparenleft}{\kern0pt}A\ {\isacharasterisk}{\kern0pt}\ {\isasymbar}z{\isasymbar}\ {\isacharplus}{\kern0pt}\ B{\isacharparenright}{\kern0pt}\ {\isacharasterisk}{\kern0pt}\ C{\isacharprime}{\kern0pt}\ {\isacharplus}{\kern0pt}\ {\isadigit{2}}\ {\isacharasterisk}{\kern0pt}\ C\ {\isacharplus}{\kern0pt}\ {\isadigit{2}}\ {\isacharasterisk}{\kern0pt}\ C{\isacharprime}{\kern0pt}{\isachardoublequoteclose}\ \isacommand{using}\isamarkupfalse%
\ mult{\isacharunderscore}{\kern0pt}right{\isacharunderscore}{\kern0pt}mono{\isacharbrackleft}{\kern0pt}OF\ f{\isacharunderscore}{\kern0pt}lbound{\isacharparenleft}{\kern0pt}{\isadigit{1}}{\isacharparenright}{\kern0pt}{\isacharbrackleft}{\kern0pt}of\ z{\isacharbrackright}{\kern0pt}\ bound{\isacharprime}{\kern0pt}{\isacharparenleft}{\kern0pt}{\isadigit{2}}{\isacharparenright}{\kern0pt}{\isacharbrackright}{\kern0pt}\ \isacommand{by}\isamarkupfalse%
\ presburger\isanewline
\ \ \ \ \isacommand{also}\isamarkupfalse%
\ \isacommand{have}\isamarkupfalse%
\ {\isachardoublequoteopen}{\isachardot}{\kern0pt}{\isachardot}{\kern0pt}{\isachardot}{\kern0pt}\ {\isacharequal}{\kern0pt}\ {\isasymbar}z{\isasymbar}\ {\isacharasterisk}{\kern0pt}\ {\isacharparenleft}{\kern0pt}C\ {\isacharplus}{\kern0pt}\ C{\isacharprime}{\kern0pt}\ {\isacharplus}{\kern0pt}\ A{\isacharprime}{\kern0pt}\ {\isacharasterisk}{\kern0pt}\ C\ {\isacharplus}{\kern0pt}\ A\ {\isacharasterisk}{\kern0pt}\ C{\isacharprime}{\kern0pt}{\isacharparenright}{\kern0pt}\ {\isacharplus}{\kern0pt}\ {\isacharparenleft}{\kern0pt}{\isadigit{2}}\ {\isacharplus}{\kern0pt}\ B{\isacharprime}{\kern0pt}{\isacharparenright}{\kern0pt}\ {\isacharasterisk}{\kern0pt}\ C\ {\isacharplus}{\kern0pt}\ {\isacharparenleft}{\kern0pt}{\isadigit{2}}\ {\isacharplus}{\kern0pt}\ B{\isacharparenright}{\kern0pt}\ {\isacharasterisk}{\kern0pt}\ C{\isacharprime}{\kern0pt}{\isachardoublequoteclose}\ \isacommand{by}\isamarkupfalse%
\ algebra\isanewline
\ \ \ \ \isacommand{finally}\isamarkupfalse%
\ \isacommand{have}\isamarkupfalse%
\ {\isacharasterisk}{\kern0pt}{\isacharcolon}{\kern0pt}\ {\isachardoublequoteopen}{\isasymbar}z{\isasymbar}\ {\isacharasterisk}{\kern0pt}\ {\isasymbar}f\ {\isacharparenleft}{\kern0pt}g\ z{\isacharparenright}{\kern0pt}\ {\isacharminus}{\kern0pt}\ g\ {\isacharparenleft}{\kern0pt}f\ z{\isacharparenright}{\kern0pt}{\isasymbar}\ {\isasymle}\ {\isasymbar}z{\isasymbar}\ {\isacharasterisk}{\kern0pt}\ D\ {\isacharplus}{\kern0pt}\ E{\isachardoublequoteclose}\ \isacommand{unfolding}\isamarkupfalse%
\ D{\isacharunderscore}{\kern0pt}def\ E{\isacharunderscore}{\kern0pt}def\ \isacommand{by}\isamarkupfalse%
\ presburger\isanewline
\ \ \ \ \isacommand{have}\isamarkupfalse%
\ {\isachardoublequoteopen}{\isasymbar}f\ {\isacharparenleft}{\kern0pt}g\ z{\isacharparenright}{\kern0pt}\ {\isacharminus}{\kern0pt}\ g\ {\isacharparenleft}{\kern0pt}f\ z{\isacharparenright}{\kern0pt}{\isasymbar}\ {\isasymle}\ D\ {\isacharplus}{\kern0pt}\ E\ {\isacharplus}{\kern0pt}\ {\isasymbar}f\ {\isacharparenleft}{\kern0pt}g\ {\isadigit{0}}{\isacharparenright}{\kern0pt}\ {\isacharminus}{\kern0pt}\ g\ {\isacharparenleft}{\kern0pt}f\ {\isadigit{0}}{\isacharparenright}{\kern0pt}{\isasymbar}{\isachardoublequoteclose}\isanewline
\ \ \ \ \isacommand{proof}\isamarkupfalse%
\ {\isacharparenleft}{\kern0pt}cases\ {\isachardoublequoteopen}z\ {\isacharequal}{\kern0pt}\ {\isadigit{0}}{\isachardoublequoteclose}{\isacharparenright}{\kern0pt}\isanewline
\ \ \ \ \ \ \isacommand{case}\isamarkupfalse%
\ True\isanewline
\ \ \ \ \ \ \isacommand{then}\isamarkupfalse%
\ \isacommand{show}\isamarkupfalse%
\ {\isacharquery}{\kern0pt}thesis\ \isacommand{using}\isamarkupfalse%
\ D{\isacharunderscore}{\kern0pt}nonneg\ E{\isacharunderscore}{\kern0pt}nonneg\ \isacommand{by}\isamarkupfalse%
\ fastforce\isanewline
\ \ \ \ \isacommand{next}\isamarkupfalse%
\isanewline
\ \ \ \ \ \ \isacommand{case}\isamarkupfalse%
\ False\isanewline
\ \ \ \ \ \ \isacommand{have}\isamarkupfalse%
\ {\isachardoublequoteopen}{\isasymbar}z{\isasymbar}\ {\isacharasterisk}{\kern0pt}\ {\isasymbar}f\ {\isacharparenleft}{\kern0pt}g\ z{\isacharparenright}{\kern0pt}\ {\isacharminus}{\kern0pt}\ g\ {\isacharparenleft}{\kern0pt}f\ z{\isacharparenright}{\kern0pt}{\isasymbar}\ {\isasymle}\ {\isasymbar}z{\isasymbar}\ {\isacharasterisk}{\kern0pt}\ {\isacharparenleft}{\kern0pt}D\ {\isacharplus}{\kern0pt}\ E{\isacharparenright}{\kern0pt}{\isachardoublequoteclose}\ \isanewline
\ \ \ \ \ \ \ \ \isacommand{using}\isamarkupfalse%
\ mult{\isacharunderscore}{\kern0pt}right{\isacharunderscore}{\kern0pt}mono{\isacharbrackleft}{\kern0pt}OF\ Ints{\isacharunderscore}{\kern0pt}nonzero{\isacharunderscore}{\kern0pt}abs{\isacharunderscore}{\kern0pt}ge{\isadigit{1}}{\isacharbrackleft}{\kern0pt}OF\ {\isacharunderscore}{\kern0pt}\ False{\isacharbrackright}{\kern0pt}\ E{\isacharunderscore}{\kern0pt}nonneg{\isacharbrackright}{\kern0pt}\ distrib{\isacharunderscore}{\kern0pt}left{\isacharbrackleft}{\kern0pt}of\ {\isachardoublequoteopen}{\isasymbar}z{\isasymbar}{\isachardoublequoteclose}\ D\ E{\isacharbrackright}{\kern0pt}\ {\isacharasterisk}{\kern0pt}\isanewline
\ \ \ \ \ \ \ \ \isacommand{by}\isamarkupfalse%
\ {\isacharparenleft}{\kern0pt}simp\ add{\isacharcolon}{\kern0pt}\ Ints{\isacharunderscore}{\kern0pt}def{\isacharparenright}{\kern0pt}\isanewline
\ \ \ \ \ \ \isacommand{hence}\isamarkupfalse%
\ {\isachardoublequoteopen}{\isasymbar}f\ {\isacharparenleft}{\kern0pt}g\ z{\isacharparenright}{\kern0pt}\ {\isacharminus}{\kern0pt}\ g\ {\isacharparenleft}{\kern0pt}f\ z{\isacharparenright}{\kern0pt}{\isasymbar}\ {\isasymle}\ D\ {\isacharplus}{\kern0pt}\ E{\isachardoublequoteclose}\ \isacommand{using}\isamarkupfalse%
\ False\ \isacommand{by}\isamarkupfalse%
\ simp\isanewline
\ \ \ \ \ \ \isacommand{thus}\isamarkupfalse%
\ {\isacharquery}{\kern0pt}thesis\ \isacommand{by}\isamarkupfalse%
\ linarith\isanewline
\ \ \ \ \isacommand{qed}\isamarkupfalse%
\isanewline
\ \ \isacommand{{\isacharbraceright}{\kern0pt}}\isamarkupfalse%
\isanewline
\ \ \isacommand{thus}\isamarkupfalse%
\ {\isacharquery}{\kern0pt}thesis\ \isacommand{by}\isamarkupfalse%
\ {\isacharparenleft}{\kern0pt}fastforce\ intro{\isacharcolon}{\kern0pt}\ boundedI{\isacharparenright}{\kern0pt}\isanewline
\isacommand{qed}\isamarkupfalse%
%
\endisatagproof
{\isafoldproof}%
%
\isadelimproof
\isanewline
%
\endisadelimproof
\isanewline
\isacommand{lemma}\isamarkupfalse%
\ int{\isacharunderscore}{\kern0pt}set{\isacharunderscore}{\kern0pt}infiniteI{\isacharcolon}{\kern0pt}\isanewline
\ \ \isakeyword{assumes}\ {\isachardoublequoteopen}{\isasymAnd}C{\isachardot}{\kern0pt}\ C\ {\isasymge}\ {\isadigit{0}}\ {\isasymLongrightarrow}\ {\isasymexists}N{\isasymge}C{\isachardot}{\kern0pt}\ N\ {\isasymin}\ {\isacharparenleft}{\kern0pt}A\ {\isacharcolon}{\kern0pt}{\isacharcolon}{\kern0pt}\ int\ set{\isacharparenright}{\kern0pt}{\isachardoublequoteclose}\ \isanewline
\ \ \isakeyword{shows}\ {\isachardoublequoteopen}infinite\ A{\isachardoublequoteclose}\isanewline
%
\isadelimproof
\ \ %
\endisadelimproof
%
\isatagproof
\isacommand{by}\isamarkupfalse%
\ {\isacharparenleft}{\kern0pt}meson\ assms\ abs{\isacharunderscore}{\kern0pt}ge{\isacharunderscore}{\kern0pt}zero\ abs{\isacharunderscore}{\kern0pt}le{\isacharunderscore}{\kern0pt}iff\ gt{\isacharunderscore}{\kern0pt}ex\ le{\isacharunderscore}{\kern0pt}cSup{\isacharunderscore}{\kern0pt}finite\ linorder{\isacharunderscore}{\kern0pt}not{\isacharunderscore}{\kern0pt}less\ order{\isacharunderscore}{\kern0pt}less{\isacharunderscore}{\kern0pt}le{\isacharunderscore}{\kern0pt}trans{\isacharparenright}{\kern0pt}%
\endisatagproof
{\isafoldproof}%
%
\isadelimproof
\isanewline
%
\endisadelimproof
\isanewline
\isacommand{lemma}\isamarkupfalse%
\ int{\isacharunderscore}{\kern0pt}set{\isacharunderscore}{\kern0pt}infiniteD{\isacharcolon}{\kern0pt}\isanewline
\ \ \isakeyword{assumes}\ {\isachardoublequoteopen}infinite\ {\isacharparenleft}{\kern0pt}A\ {\isacharcolon}{\kern0pt}{\isacharcolon}{\kern0pt}\ int\ set{\isacharparenright}{\kern0pt}{\isachardoublequoteclose}\ {\isachardoublequoteopen}C\ {\isasymge}\ {\isadigit{0}}{\isachardoublequoteclose}\isanewline
\ \ \isakeyword{obtains}\ z\ \isakeyword{where}\ {\isachardoublequoteopen}z\ {\isasymin}\ A{\isachardoublequoteclose}\ {\isachardoublequoteopen}C\ {\isasymle}\ {\isasymbar}z{\isasymbar}{\isachardoublequoteclose}\isanewline
%
\isadelimproof
%
\endisadelimproof
%
\isatagproof
\isacommand{proof}\isamarkupfalse%
\ {\isacharminus}{\kern0pt}\isanewline
\ \ \isacommand{{\isacharbraceleft}{\kern0pt}}\isamarkupfalse%
\isanewline
\ \ \ \ \isacommand{assume}\isamarkupfalse%
\ asm{\isacharcolon}{\kern0pt}\ {\isachardoublequoteopen}{\isasymforall}z\ {\isasymin}\ A{\isachardot}{\kern0pt}\ C\ {\isachargreater}{\kern0pt}\ {\isasymbar}z{\isasymbar}{\isachardoublequoteclose}\isanewline
\ \ \ \ \isacommand{let}\isamarkupfalse%
\ {\isacharquery}{\kern0pt}f\ {\isacharequal}{\kern0pt}\ {\isachardoublequoteopen}{\isasymlambda}z{\isachardot}{\kern0pt}\ {\isacharparenleft}{\kern0pt}if\ z\ {\isasymin}\ A\ then\ z\ else\ {\isacharparenleft}{\kern0pt}{\isadigit{0}}{\isacharcolon}{\kern0pt}{\isacharcolon}{\kern0pt}int{\isacharparenright}{\kern0pt}{\isacharparenright}{\kern0pt}{\isachardoublequoteclose}\isanewline
\ \ \ \ \isacommand{have}\isamarkupfalse%
\ bounded{\isacharcolon}{\kern0pt}\ {\isachardoublequoteopen}{\isasymforall}z\ {\isasymin}\ A{\isachardot}{\kern0pt}\ {\isasymbar}{\isacharquery}{\kern0pt}f\ z{\isasymbar}\ {\isasymle}\ C{\isachardoublequoteclose}\ \isacommand{using}\isamarkupfalse%
\ asm\ \isacommand{by}\isamarkupfalse%
\ fastforce\isanewline
\ \ \ \ \isacommand{moreover}\isamarkupfalse%
\ \isacommand{have}\isamarkupfalse%
\ {\isachardoublequoteopen}{\isasymforall}z\ {\isasymin}\ UNIV\ {\isacharminus}{\kern0pt}\ A{\isachardot}{\kern0pt}\ {\isasymbar}{\isacharquery}{\kern0pt}f\ z{\isasymbar}\ {\isasymle}\ C{\isachardoublequoteclose}\ \isacommand{using}\isamarkupfalse%
\ assms\ \isacommand{by}\isamarkupfalse%
\ fastforce\isanewline
\ \ \ \ \isacommand{ultimately}\isamarkupfalse%
\ \isacommand{have}\isamarkupfalse%
\ {\isachardoublequoteopen}bounded\ {\isacharquery}{\kern0pt}f{\isachardoublequoteclose}\ \isacommand{by}\isamarkupfalse%
\ {\isacharparenleft}{\kern0pt}blast\ intro{\isacharcolon}{\kern0pt}\ boundedI{\isacharparenright}{\kern0pt}\isanewline
\ \ \ \ \isacommand{hence}\isamarkupfalse%
\ False\ \isacommand{using}\isamarkupfalse%
\ bounded{\isacharunderscore}{\kern0pt}iff{\isacharunderscore}{\kern0pt}finite{\isacharunderscore}{\kern0pt}range\ assms\ \isacommand{by}\isamarkupfalse%
\ force\isanewline
\ \ \isacommand{{\isacharbraceright}{\kern0pt}}\isamarkupfalse%
\isanewline
\ \ \isacommand{thus}\isamarkupfalse%
\ {\isacharquery}{\kern0pt}thesis\ \isacommand{using}\isamarkupfalse%
\ that\ \isacommand{by}\isamarkupfalse%
\ fastforce\isanewline
\isacommand{qed}\isamarkupfalse%
%
\endisatagproof
{\isafoldproof}%
%
\isadelimproof
\isanewline
%
\endisadelimproof
\isanewline
\isacommand{lemma}\isamarkupfalse%
\ bounded{\isacharunderscore}{\kern0pt}odd{\isacharcolon}{\kern0pt}\isanewline
\ \ \isakeyword{fixes}\ f\ {\isacharcolon}{\kern0pt}{\isacharcolon}{\kern0pt}\ {\isachardoublequoteopen}int\ {\isasymRightarrow}\ int{\isachardoublequoteclose}\isanewline
\ \ \isakeyword{assumes}\ {\isachardoublequoteopen}{\isasymAnd}z{\isachardot}{\kern0pt}\ z\ {\isacharless}{\kern0pt}\ {\isadigit{0}}\ {\isasymLongrightarrow}\ f\ z\ {\isacharequal}{\kern0pt}\ {\isacharminus}{\kern0pt}f\ {\isacharparenleft}{\kern0pt}{\isacharminus}{\kern0pt}z{\isacharparenright}{\kern0pt}{\isachardoublequoteclose}\ {\isachardoublequoteopen}{\isasymAnd}n{\isachardot}{\kern0pt}\ n\ {\isachargreater}{\kern0pt}\ {\isadigit{0}}\ {\isasymLongrightarrow}\ {\isasymbar}f\ n{\isasymbar}\ {\isasymle}\ C{\isachardoublequoteclose}\isanewline
\ \ \isakeyword{shows}\ {\isachardoublequoteopen}bounded\ f{\isachardoublequoteclose}\isanewline
%
\isadelimproof
%
\endisadelimproof
%
\isatagproof
\isacommand{proof}\isamarkupfalse%
\ {\isacharminus}{\kern0pt}\isanewline
\ \ \isacommand{have}\isamarkupfalse%
\ {\isachardoublequoteopen}{\isasymbar}f\ n{\isasymbar}\ {\isasymle}\ C\ {\isacharplus}{\kern0pt}\ {\isasymbar}f\ {\isadigit{0}}{\isasymbar}{\isachardoublequoteclose}\ \isakeyword{if}\ {\isachardoublequoteopen}n\ {\isasymge}\ {\isadigit{0}}{\isachardoublequoteclose}\ \isakeyword{for}\ n\ \isacommand{using}\isamarkupfalse%
\ assms\ \isacommand{by}\isamarkupfalse%
\ {\isacharparenleft}{\kern0pt}metis\ abs{\isacharunderscore}{\kern0pt}ge{\isacharunderscore}{\kern0pt}zero\ abs{\isacharunderscore}{\kern0pt}of{\isacharunderscore}{\kern0pt}nonneg\ add{\isacharunderscore}{\kern0pt}increasing{\isadigit{2}}\ le{\isacharunderscore}{\kern0pt}add{\isacharunderscore}{\kern0pt}same{\isacharunderscore}{\kern0pt}cancel{\isadigit{2}}\ that\ zero{\isacharunderscore}{\kern0pt}less{\isacharunderscore}{\kern0pt}abs{\isacharunderscore}{\kern0pt}iff{\isacharparenright}{\kern0pt}\isanewline
\ \ \isacommand{hence}\isamarkupfalse%
\ {\isachardoublequoteopen}{\isasymbar}f\ n{\isasymbar}\ {\isasymle}\ C\ {\isacharplus}{\kern0pt}\ {\isasymbar}f\ {\isadigit{0}}{\isasymbar}{\isachardoublequoteclose}\ \isakeyword{for}\ n\ \isacommand{using}\isamarkupfalse%
\ assms\ \isacommand{by}\isamarkupfalse%
\ {\isacharparenleft}{\kern0pt}cases\ {\isachardoublequoteopen}{\isadigit{0}}\ {\isasymle}\ n{\isachardoublequoteclose}{\isacharparenright}{\kern0pt}\ fastforce{\isacharplus}{\kern0pt}\isanewline
\ \ \isacommand{thus}\isamarkupfalse%
\ {\isacharquery}{\kern0pt}thesis\ \isacommand{by}\isamarkupfalse%
\ {\isacharparenleft}{\kern0pt}rule\ boundedI{\isacharparenright}{\kern0pt}\isanewline
\isacommand{qed}\isamarkupfalse%
%
\endisatagproof
{\isafoldproof}%
%
\isadelimproof
\isanewline
%
\endisadelimproof
\isanewline
\isacommand{lemma}\isamarkupfalse%
\ slope{\isacharunderscore}{\kern0pt}odd{\isacharcolon}{\kern0pt}\ \isanewline
\ \ \isakeyword{assumes}\ {\isachardoublequoteopen}{\isasymAnd}z{\isachardot}{\kern0pt}\ z\ {\isacharless}{\kern0pt}\ {\isadigit{0}}\ {\isasymLongrightarrow}\ f\ z\ {\isacharequal}{\kern0pt}\ {\isacharminus}{\kern0pt}\ f\ {\isacharparenleft}{\kern0pt}{\isacharminus}{\kern0pt}\ z{\isacharparenright}{\kern0pt}{\isachardoublequoteclose}\ \isanewline
\ \ \ \ \ \ \ \ \ \ {\isachardoublequoteopen}{\isasymAnd}m\ n{\isachardot}{\kern0pt}\ {\isasymlbrakk}m\ {\isachargreater}{\kern0pt}\ {\isadigit{0}}{\isacharsemicolon}{\kern0pt}\ n\ {\isachargreater}{\kern0pt}\ {\isadigit{0}}{\isasymrbrakk}\ {\isasymLongrightarrow}\ {\isasymbar}f\ {\isacharparenleft}{\kern0pt}m\ {\isacharplus}{\kern0pt}\ n{\isacharparenright}{\kern0pt}\ {\isacharminus}{\kern0pt}\ {\isacharparenleft}{\kern0pt}f\ m\ {\isacharplus}{\kern0pt}\ f\ n{\isacharparenright}{\kern0pt}{\isasymbar}\ {\isasymle}\ C{\isachardoublequoteclose}\isanewline
\ \ \isakeyword{shows}\ {\isachardoublequoteopen}slope\ f{\isachardoublequoteclose}\isanewline
%
\isadelimproof
%
\endisadelimproof
%
\isatagproof
\isacommand{proof}\isamarkupfalse%
\ {\isacharminus}{\kern0pt}\isanewline
\ \ \isacommand{define}\isamarkupfalse%
\ C{\isacharprime}{\kern0pt}\ \isakeyword{where}\ {\isachardoublequoteopen}C{\isacharprime}{\kern0pt}\ {\isacharequal}{\kern0pt}\ C\ {\isacharplus}{\kern0pt}\ {\isasymbar}f\ {\isadigit{0}}{\isasymbar}{\isachardoublequoteclose}\isanewline
\ \ \isacommand{have}\isamarkupfalse%
\ {\isachardoublequoteopen}C\ {\isasymge}\ {\isadigit{0}}{\isachardoublequoteclose}\ \isacommand{using}\isamarkupfalse%
\ assms{\isacharparenleft}{\kern0pt}{\isadigit{2}}{\isacharparenright}{\kern0pt}{\isacharbrackleft}{\kern0pt}of\ {\isadigit{1}}\ {\isadigit{1}}{\isacharbrackright}{\kern0pt}\ \isacommand{by}\isamarkupfalse%
\ simp\isanewline
\ \ \isacommand{hence}\isamarkupfalse%
\ bound{\isacharcolon}{\kern0pt}\ {\isachardoublequoteopen}{\isasymbar}f\ {\isacharparenleft}{\kern0pt}m\ {\isacharplus}{\kern0pt}\ n{\isacharparenright}{\kern0pt}\ {\isacharminus}{\kern0pt}\ {\isacharparenleft}{\kern0pt}f\ m\ {\isacharplus}{\kern0pt}\ f\ n{\isacharparenright}{\kern0pt}{\isasymbar}\ {\isasymle}\ C{\isacharprime}{\kern0pt}{\isachardoublequoteclose}\ \isakeyword{if}\ {\isachardoublequoteopen}m\ {\isasymge}\ {\isadigit{0}}{\isachardoublequoteclose}\ {\isachardoublequoteopen}n\ {\isasymge}\ {\isadigit{0}}{\isachardoublequoteclose}\ \isakeyword{for}\ m\ n\ \isanewline
\ \ \ \ \isacommand{unfolding}\isamarkupfalse%
\ C{\isacharprime}{\kern0pt}{\isacharunderscore}{\kern0pt}def\ \isacommand{using}\isamarkupfalse%
\ assms{\isacharparenleft}{\kern0pt}{\isadigit{2}}{\isacharparenright}{\kern0pt}\ that\ \isanewline
\ \ \ \ \isacommand{by}\isamarkupfalse%
\ {\isacharparenleft}{\kern0pt}cases\ {\isachardoublequoteopen}m\ {\isacharequal}{\kern0pt}\ {\isadigit{0}}\ {\isasymor}\ n\ {\isacharequal}{\kern0pt}\ {\isadigit{0}}{\isachardoublequoteclose}{\isacharparenright}{\kern0pt}\ {\isacharparenleft}{\kern0pt}force{\isacharcomma}{\kern0pt}\ metis\ abs{\isacharunderscore}{\kern0pt}ge{\isacharunderscore}{\kern0pt}zero\ add{\isacharunderscore}{\kern0pt}increasing{\isadigit{2}}\ order{\isacharunderscore}{\kern0pt}le{\isacharunderscore}{\kern0pt}less{\isacharparenright}{\kern0pt}\isanewline
\ \ \isacommand{{\isacharbraceleft}{\kern0pt}}\isamarkupfalse%
\isanewline
\ \ \ \ \isacommand{fix}\isamarkupfalse%
\ m\ n\isanewline
\ \ \ \ \isacommand{have}\isamarkupfalse%
\ {\isachardoublequoteopen}{\isasymbar}f\ {\isacharparenleft}{\kern0pt}m\ {\isacharplus}{\kern0pt}\ n{\isacharparenright}{\kern0pt}\ {\isacharminus}{\kern0pt}\ {\isacharparenleft}{\kern0pt}f\ m\ {\isacharplus}{\kern0pt}\ f\ n{\isacharparenright}{\kern0pt}{\isasymbar}\ {\isasymle}\ C{\isacharprime}{\kern0pt}{\isachardoublequoteclose}\isanewline
\ \ \ \ \isacommand{proof}\isamarkupfalse%
\ {\isacharparenleft}{\kern0pt}cases\ {\isachardoublequoteopen}m\ {\isasymge}\ {\isadigit{0}}{\isachardoublequoteclose}{\isacharparenright}{\kern0pt}\isanewline
\ \ \ \ \ \ \isacommand{case}\isamarkupfalse%
\ m{\isacharunderscore}{\kern0pt}nonneg{\isacharcolon}{\kern0pt}\ True\isanewline
\ \ \ \ \ \ \isacommand{show}\isamarkupfalse%
\ {\isacharquery}{\kern0pt}thesis\isanewline
\ \ \ \ \ \ \isacommand{proof}\isamarkupfalse%
\ {\isacharparenleft}{\kern0pt}cases\ {\isachardoublequoteopen}n\ {\isasymge}\ {\isadigit{0}}{\isachardoublequoteclose}{\isacharparenright}{\kern0pt}\isanewline
\ \ \ \ \ \ \ \ \isacommand{case}\isamarkupfalse%
\ True\isanewline
\ \ \ \ \ \ \ \ \isacommand{thus}\isamarkupfalse%
\ {\isacharquery}{\kern0pt}thesis\ \isacommand{using}\isamarkupfalse%
\ bound\ m{\isacharunderscore}{\kern0pt}nonneg\ \isacommand{by}\isamarkupfalse%
\ fast\isanewline
\ \ \ \ \ \ \isacommand{next}\isamarkupfalse%
\isanewline
\ \ \ \ \ \ \ \ \isacommand{case}\isamarkupfalse%
\ False\isanewline
\ \ \ \ \ \ \ \ \isacommand{hence}\isamarkupfalse%
\ f{\isacharunderscore}{\kern0pt}n{\isacharcolon}{\kern0pt}\ {\isachardoublequoteopen}f\ n\ {\isacharequal}{\kern0pt}\ {\isacharminus}{\kern0pt}\ f\ {\isacharparenleft}{\kern0pt}{\isacharminus}{\kern0pt}\ n{\isacharparenright}{\kern0pt}{\isachardoublequoteclose}\ \isacommand{using}\isamarkupfalse%
\ assms\ \isacommand{by}\isamarkupfalse%
\ simp\isanewline
\ \ \ \ \ \ \ \ \isacommand{show}\isamarkupfalse%
\ {\isacharquery}{\kern0pt}thesis\isanewline
\ \ \ \ \ \ \ \ \isacommand{proof}\isamarkupfalse%
\ {\isacharparenleft}{\kern0pt}cases\ {\isachardoublequoteopen}m\ {\isacharplus}{\kern0pt}\ n\ {\isasymge}\ {\isadigit{0}}{\isachardoublequoteclose}{\isacharparenright}{\kern0pt}\isanewline
\ \ \ \ \ \ \ \ \ \ \isacommand{case}\isamarkupfalse%
\ True\isanewline
\ \ \ \ \ \ \ \ \ \ \isacommand{have}\isamarkupfalse%
\ {\isachardoublequoteopen}{\isasymbar}f\ {\isacharparenleft}{\kern0pt}m\ {\isacharplus}{\kern0pt}\ n{\isacharparenright}{\kern0pt}\ {\isacharminus}{\kern0pt}\ {\isacharparenleft}{\kern0pt}f\ m\ {\isacharplus}{\kern0pt}\ f\ n{\isacharparenright}{\kern0pt}{\isasymbar}\ {\isacharequal}{\kern0pt}\ {\isasymbar}f\ {\isacharparenleft}{\kern0pt}m\ {\isacharplus}{\kern0pt}\ n\ {\isacharplus}{\kern0pt}\ {\isacharminus}{\kern0pt}n{\isacharparenright}{\kern0pt}\ {\isacharminus}{\kern0pt}\ {\isacharparenleft}{\kern0pt}f\ {\isacharparenleft}{\kern0pt}m\ {\isacharplus}{\kern0pt}\ n{\isacharparenright}{\kern0pt}\ {\isacharplus}{\kern0pt}\ f\ {\isacharparenleft}{\kern0pt}{\isacharminus}{\kern0pt}n{\isacharparenright}{\kern0pt}{\isacharparenright}{\kern0pt}{\isasymbar}{\isachardoublequoteclose}\ \isacommand{using}\isamarkupfalse%
\ f{\isacharunderscore}{\kern0pt}n\ \isacommand{by}\isamarkupfalse%
\ auto\isanewline
\ \ \ \ \ \ \ \ \ \ \isacommand{thus}\isamarkupfalse%
\ {\isacharquery}{\kern0pt}thesis\ \isacommand{using}\isamarkupfalse%
\ bound{\isacharbrackleft}{\kern0pt}OF\ True{\isacharbrackright}{\kern0pt}\ \isacommand{by}\isamarkupfalse%
\ {\isacharparenleft}{\kern0pt}metis\ False\ neg{\isacharunderscore}{\kern0pt}{\isadigit{0}}{\isacharunderscore}{\kern0pt}le{\isacharunderscore}{\kern0pt}iff{\isacharunderscore}{\kern0pt}le\ nle{\isacharunderscore}{\kern0pt}le{\isacharparenright}{\kern0pt}\isanewline
\ \ \ \ \ \ \ \ \isacommand{next}\isamarkupfalse%
\isanewline
\ \ \ \ \ \ \ \ \ \ \isacommand{case}\isamarkupfalse%
\ False\isanewline
\ \ \ \ \ \ \ \ \ \ \isacommand{hence}\isamarkupfalse%
\ {\isachardoublequoteopen}f\ {\isacharparenleft}{\kern0pt}m\ {\isacharplus}{\kern0pt}\ n{\isacharparenright}{\kern0pt}\ {\isacharequal}{\kern0pt}\ {\isacharminus}{\kern0pt}\ f\ {\isacharparenleft}{\kern0pt}{\isacharminus}{\kern0pt}\ {\isacharparenleft}{\kern0pt}m\ {\isacharplus}{\kern0pt}\ n{\isacharparenright}{\kern0pt}{\isacharparenright}{\kern0pt}{\isachardoublequoteclose}\ \isacommand{using}\isamarkupfalse%
\ assms\ \isacommand{by}\isamarkupfalse%
\ force\isanewline
\ \ \ \ \ \ \ \ \ \ \isacommand{hence}\isamarkupfalse%
\ {\isachardoublequoteopen}{\isasymbar}f\ {\isacharparenleft}{\kern0pt}m\ {\isacharplus}{\kern0pt}\ n{\isacharparenright}{\kern0pt}\ {\isacharminus}{\kern0pt}\ {\isacharparenleft}{\kern0pt}f\ m\ {\isacharplus}{\kern0pt}\ f\ n{\isacharparenright}{\kern0pt}{\isasymbar}\ {\isacharequal}{\kern0pt}\ {\isasymbar}f\ {\isacharparenleft}{\kern0pt}{\isacharminus}{\kern0pt}\ {\isacharparenleft}{\kern0pt}m\ {\isacharplus}{\kern0pt}\ n{\isacharparenright}{\kern0pt}\ {\isacharplus}{\kern0pt}\ m{\isacharparenright}{\kern0pt}\ {\isacharminus}{\kern0pt}\ {\isacharparenleft}{\kern0pt}f\ {\isacharparenleft}{\kern0pt}{\isacharminus}{\kern0pt}\ {\isacharparenleft}{\kern0pt}m\ {\isacharplus}{\kern0pt}\ n{\isacharparenright}{\kern0pt}{\isacharparenright}{\kern0pt}\ {\isacharplus}{\kern0pt}\ f\ m{\isacharparenright}{\kern0pt}{\isasymbar}{\isachardoublequoteclose}\ \isacommand{using}\isamarkupfalse%
\ f{\isacharunderscore}{\kern0pt}n\ \isacommand{by}\isamarkupfalse%
\ force\isanewline
\ \ \ \ \ \ \ \ \ \ \isacommand{thus}\isamarkupfalse%
\ {\isacharquery}{\kern0pt}thesis\ \isacommand{using}\isamarkupfalse%
\ m{\isacharunderscore}{\kern0pt}nonneg\ bound{\isacharbrackleft}{\kern0pt}of\ {\isachardoublequoteopen}{\isacharminus}{\kern0pt}\ {\isacharparenleft}{\kern0pt}m\ {\isacharplus}{\kern0pt}\ n{\isacharparenright}{\kern0pt}{\isachardoublequoteclose}\ m{\isacharbrackright}{\kern0pt}\ False\ \isacommand{by}\isamarkupfalse%
\ simp\isanewline
\ \ \ \ \ \ \ \ \isacommand{qed}\isamarkupfalse%
\isanewline
\ \ \ \ \ \ \isacommand{qed}\isamarkupfalse%
\isanewline
\ \ \ \ \isacommand{next}\isamarkupfalse%
\isanewline
\ \ \ \ \ \ \isacommand{case}\isamarkupfalse%
\ m{\isacharunderscore}{\kern0pt}neg{\isacharcolon}{\kern0pt}\ False\isanewline
\ \ \ \ \ \ \isacommand{hence}\isamarkupfalse%
\ f{\isacharunderscore}{\kern0pt}m{\isacharcolon}{\kern0pt}\ {\isachardoublequoteopen}f\ m\ {\isacharequal}{\kern0pt}\ {\isacharminus}{\kern0pt}\ f\ {\isacharparenleft}{\kern0pt}{\isacharminus}{\kern0pt}\ m{\isacharparenright}{\kern0pt}{\isachardoublequoteclose}\ \isacommand{using}\isamarkupfalse%
\ assms\ \isacommand{by}\isamarkupfalse%
\ simp\isanewline
\ \ \ \ \ \ \isacommand{show}\isamarkupfalse%
\ {\isacharquery}{\kern0pt}thesis\isanewline
\ \ \ \ \ \ \isacommand{proof}\isamarkupfalse%
\ {\isacharparenleft}{\kern0pt}cases\ {\isachardoublequoteopen}n\ {\isasymge}\ {\isadigit{0}}{\isachardoublequoteclose}{\isacharparenright}{\kern0pt}\isanewline
\ \ \ \ \ \ \ \ \isacommand{case}\isamarkupfalse%
\ True\isanewline
\ \ \ \ \ \ \ \ \isacommand{show}\isamarkupfalse%
\ {\isacharquery}{\kern0pt}thesis\isanewline
\ \ \ \ \ \ \ \ \isacommand{proof}\isamarkupfalse%
\ {\isacharparenleft}{\kern0pt}cases\ {\isachardoublequoteopen}m\ {\isacharplus}{\kern0pt}\ n\ {\isasymge}\ {\isadigit{0}}{\isachardoublequoteclose}{\isacharparenright}{\kern0pt}\isanewline
\ \ \ \ \ \ \ \ \ \ \isacommand{case}\isamarkupfalse%
\ True\isanewline
\ \ \ \ \ \ \ \ \ \ \isacommand{have}\isamarkupfalse%
\ {\isachardoublequoteopen}{\isasymbar}f\ {\isacharparenleft}{\kern0pt}m\ {\isacharplus}{\kern0pt}\ n{\isacharparenright}{\kern0pt}\ {\isacharminus}{\kern0pt}\ {\isacharparenleft}{\kern0pt}f\ m\ {\isacharplus}{\kern0pt}\ f\ n{\isacharparenright}{\kern0pt}{\isasymbar}\ {\isacharequal}{\kern0pt}\ {\isasymbar}f\ {\isacharparenleft}{\kern0pt}m\ {\isacharplus}{\kern0pt}\ n\ {\isacharplus}{\kern0pt}\ {\isacharminus}{\kern0pt}m{\isacharparenright}{\kern0pt}\ {\isacharminus}{\kern0pt}\ {\isacharparenleft}{\kern0pt}f\ {\isacharparenleft}{\kern0pt}m\ {\isacharplus}{\kern0pt}\ n{\isacharparenright}{\kern0pt}\ {\isacharplus}{\kern0pt}\ f\ {\isacharparenleft}{\kern0pt}{\isacharminus}{\kern0pt}m{\isacharparenright}{\kern0pt}{\isacharparenright}{\kern0pt}{\isasymbar}{\isachardoublequoteclose}\ \isacommand{using}\isamarkupfalse%
\ f{\isacharunderscore}{\kern0pt}m\ \isacommand{by}\isamarkupfalse%
\ force\isanewline
\ \ \ \ \ \ \ \ \ \ \isacommand{thus}\isamarkupfalse%
\ {\isacharquery}{\kern0pt}thesis\ \isacommand{using}\isamarkupfalse%
\ bound{\isacharbrackleft}{\kern0pt}OF\ True{\isacharcomma}{\kern0pt}\ of\ {\isachardoublequoteopen}{\isacharminus}{\kern0pt}\ m{\isachardoublequoteclose}{\isacharbrackright}{\kern0pt}\ m{\isacharunderscore}{\kern0pt}neg\ \isacommand{by}\isamarkupfalse%
\ simp\isanewline
\ \ \ \ \ \ \ \ \isacommand{next}\isamarkupfalse%
\isanewline
\ \ \ \ \ \ \ \ \ \ \isacommand{case}\isamarkupfalse%
\ False\isanewline
\ \ \ \ \ \ \ \ \ \ \isacommand{hence}\isamarkupfalse%
\ {\isachardoublequoteopen}f\ {\isacharparenleft}{\kern0pt}m\ {\isacharplus}{\kern0pt}\ n{\isacharparenright}{\kern0pt}\ {\isacharequal}{\kern0pt}\ {\isacharminus}{\kern0pt}\ f\ {\isacharparenleft}{\kern0pt}{\isacharminus}{\kern0pt}\ {\isacharparenleft}{\kern0pt}m\ {\isacharplus}{\kern0pt}\ n{\isacharparenright}{\kern0pt}{\isacharparenright}{\kern0pt}{\isachardoublequoteclose}\ \isacommand{using}\isamarkupfalse%
\ assms\ \isacommand{by}\isamarkupfalse%
\ force\isanewline
\ \ \ \ \ \ \ \ \ \ \isacommand{hence}\isamarkupfalse%
{\isachardoublequoteopen}{\isasymbar}f\ {\isacharparenleft}{\kern0pt}m\ {\isacharplus}{\kern0pt}\ n{\isacharparenright}{\kern0pt}\ {\isacharminus}{\kern0pt}\ {\isacharparenleft}{\kern0pt}f\ m\ {\isacharplus}{\kern0pt}\ f\ n{\isacharparenright}{\kern0pt}{\isasymbar}\ {\isacharequal}{\kern0pt}\ {\isasymbar}f\ {\isacharparenleft}{\kern0pt}{\isacharminus}{\kern0pt}\ {\isacharparenleft}{\kern0pt}m\ {\isacharplus}{\kern0pt}\ n{\isacharparenright}{\kern0pt}\ {\isacharplus}{\kern0pt}\ n{\isacharparenright}{\kern0pt}\ {\isacharminus}{\kern0pt}\ {\isacharparenleft}{\kern0pt}f\ {\isacharparenleft}{\kern0pt}{\isacharminus}{\kern0pt}\ {\isacharparenleft}{\kern0pt}m\ {\isacharplus}{\kern0pt}\ n{\isacharparenright}{\kern0pt}{\isacharparenright}{\kern0pt}\ {\isacharplus}{\kern0pt}\ f\ n{\isacharparenright}{\kern0pt}{\isasymbar}{\isachardoublequoteclose}\ \isacommand{using}\isamarkupfalse%
\ f{\isacharunderscore}{\kern0pt}m\ \isacommand{by}\isamarkupfalse%
\ force\isanewline
\ \ \ \ \ \ \ \ \ \ \isacommand{thus}\isamarkupfalse%
\ {\isacharquery}{\kern0pt}thesis\ \isacommand{using}\isamarkupfalse%
\ bound{\isacharbrackleft}{\kern0pt}of\ {\isachardoublequoteopen}{\isacharminus}{\kern0pt}\ {\isacharparenleft}{\kern0pt}m\ {\isacharplus}{\kern0pt}\ n{\isacharparenright}{\kern0pt}{\isachardoublequoteclose}\ n{\isacharbrackright}{\kern0pt}\ True\ False\ \isacommand{by}\isamarkupfalse%
\ simp\isanewline
\ \ \ \ \ \ \ \ \isacommand{qed}\isamarkupfalse%
\isanewline
\ \ \ \ \ \ \isacommand{next}\isamarkupfalse%
\isanewline
\ \ \ \ \ \ \ \ \isacommand{case}\isamarkupfalse%
\ False\isanewline
\ \ \ \ \ \ \ \ \isacommand{hence}\isamarkupfalse%
\ f{\isacharunderscore}{\kern0pt}n{\isacharcolon}{\kern0pt}\ {\isachardoublequoteopen}f\ n\ {\isacharequal}{\kern0pt}\ {\isacharminus}{\kern0pt}\ f\ {\isacharparenleft}{\kern0pt}{\isacharminus}{\kern0pt}\ n{\isacharparenright}{\kern0pt}{\isachardoublequoteclose}\ \isacommand{using}\isamarkupfalse%
\ assms\ \isacommand{by}\isamarkupfalse%
\ simp\isanewline
\ \ \ \ \ \ \ \ \isacommand{have}\isamarkupfalse%
\ {\isachardoublequoteopen}f\ {\isacharparenleft}{\kern0pt}m\ {\isacharplus}{\kern0pt}\ n{\isacharparenright}{\kern0pt}\ {\isacharequal}{\kern0pt}\ {\isacharminus}{\kern0pt}\ f\ {\isacharparenleft}{\kern0pt}{\isacharminus}{\kern0pt}\ m\ {\isacharplus}{\kern0pt}\ {\isacharminus}{\kern0pt}\ n{\isacharparenright}{\kern0pt}{\isachardoublequoteclose}\ \isacommand{using}\isamarkupfalse%
\ m{\isacharunderscore}{\kern0pt}neg\ False\ assms\ \isacommand{by}\isamarkupfalse%
\ fastforce\isanewline
\ \ \ \ \ \ \ \ \isacommand{hence}\isamarkupfalse%
\ {\isachardoublequoteopen}{\isasymbar}f\ {\isacharparenleft}{\kern0pt}m\ {\isacharplus}{\kern0pt}\ n{\isacharparenright}{\kern0pt}\ {\isacharminus}{\kern0pt}\ {\isacharparenleft}{\kern0pt}f\ m\ {\isacharplus}{\kern0pt}\ f\ n{\isacharparenright}{\kern0pt}{\isasymbar}\ {\isacharequal}{\kern0pt}\ {\isasymbar}{\isacharminus}{\kern0pt}\ f\ {\isacharparenleft}{\kern0pt}{\isacharminus}{\kern0pt}\ m\ {\isacharplus}{\kern0pt}\ {\isacharminus}{\kern0pt}n{\isacharparenright}{\kern0pt}\ {\isacharminus}{\kern0pt}\ {\isacharparenleft}{\kern0pt}{\isacharminus}{\kern0pt}\ f\ {\isacharparenleft}{\kern0pt}{\isacharminus}{\kern0pt}m{\isacharparenright}{\kern0pt}\ {\isacharplus}{\kern0pt}\ {\isacharminus}{\kern0pt}\ f{\isacharparenleft}{\kern0pt}{\isacharminus}{\kern0pt}\ n{\isacharparenright}{\kern0pt}{\isacharparenright}{\kern0pt}{\isasymbar}{\isachardoublequoteclose}\ \isacommand{using}\isamarkupfalse%
\ f{\isacharunderscore}{\kern0pt}m\ f{\isacharunderscore}{\kern0pt}n\ \isacommand{by}\isamarkupfalse%
\ argo\isanewline
\ \ \ \ \ \ \ \ \isacommand{also}\isamarkupfalse%
\ \isacommand{have}\isamarkupfalse%
\ {\isachardoublequoteopen}{\isachardot}{\kern0pt}{\isachardot}{\kern0pt}{\isachardot}{\kern0pt}\ {\isacharequal}{\kern0pt}\ {\isasymbar}f\ {\isacharparenleft}{\kern0pt}{\isacharminus}{\kern0pt}m\ {\isacharplus}{\kern0pt}\ {\isacharminus}{\kern0pt}n{\isacharparenright}{\kern0pt}\ {\isacharminus}{\kern0pt}\ {\isacharparenleft}{\kern0pt}f\ {\isacharparenleft}{\kern0pt}{\isacharminus}{\kern0pt}m{\isacharparenright}{\kern0pt}\ {\isacharplus}{\kern0pt}\ f{\isacharparenleft}{\kern0pt}{\isacharminus}{\kern0pt}n{\isacharparenright}{\kern0pt}{\isacharparenright}{\kern0pt}{\isasymbar}{\isachardoublequoteclose}\ \isacommand{by}\isamarkupfalse%
\ linarith\isanewline
\ \ \ \ \ \ \ \ \isacommand{finally}\isamarkupfalse%
\ \isacommand{show}\isamarkupfalse%
\ {\isacharquery}{\kern0pt}thesis\ \isacommand{using}\isamarkupfalse%
\ bound{\isacharbrackleft}{\kern0pt}of\ {\isachardoublequoteopen}{\isacharminus}{\kern0pt}\ m{\isachardoublequoteclose}\ {\isachardoublequoteopen}{\isacharminus}{\kern0pt}\ n{\isachardoublequoteclose}{\isacharbrackright}{\kern0pt}\ False\ m{\isacharunderscore}{\kern0pt}neg\ \isacommand{by}\isamarkupfalse%
\ simp\isanewline
\ \ \ \ \ \ \isacommand{qed}\isamarkupfalse%
\isanewline
\ \ \ \ \isacommand{qed}\isamarkupfalse%
\isanewline
\ \ \isacommand{{\isacharbraceright}{\kern0pt}}\isamarkupfalse%
\isanewline
\ \ \isacommand{thus}\isamarkupfalse%
\ {\isacharquery}{\kern0pt}thesis\ \isacommand{unfolding}\isamarkupfalse%
\ slope{\isacharunderscore}{\kern0pt}def\ \isacommand{by}\isamarkupfalse%
\ {\isacharparenleft}{\kern0pt}fast\ intro{\isacharcolon}{\kern0pt}\ boundedI{\isacharparenright}{\kern0pt}\isanewline
\isacommand{qed}\isamarkupfalse%
%
\endisatagproof
{\isafoldproof}%
%
\isadelimproof
\isanewline
%
\endisadelimproof
\isanewline
\isacommand{lemma}\isamarkupfalse%
\ slope{\isacharunderscore}{\kern0pt}bounded{\isacharunderscore}{\kern0pt}comp{\isacharunderscore}{\kern0pt}right{\isacharunderscore}{\kern0pt}abs{\isacharcolon}{\kern0pt}\isanewline
\ \ \isakeyword{assumes}\ {\isachardoublequoteopen}slope\ f{\isachardoublequoteclose}\ {\isachardoublequoteopen}bounded\ {\isacharparenleft}{\kern0pt}f\ o\ abs{\isacharparenright}{\kern0pt}{\isachardoublequoteclose}\isanewline
\ \ \isakeyword{shows}\ {\isachardoublequoteopen}bounded\ f{\isachardoublequoteclose}\isanewline
%
\isadelimproof
%
\endisadelimproof
%
\isatagproof
\isacommand{proof}\isamarkupfalse%
\ {\isacharminus}{\kern0pt}\isanewline
\ \ \isacommand{obtain}\isamarkupfalse%
\ B\ \isakeyword{where}\ {\isachardoublequoteopen}{\isasymforall}z{\isachardot}{\kern0pt}\ {\isasymbar}f\ {\isasymbar}z{\isasymbar}{\isasymbar}\ {\isasymle}\ B{\isachardoublequoteclose}\ \isakeyword{and}\ B{\isacharunderscore}{\kern0pt}nonneg{\isacharcolon}{\kern0pt}\ {\isachardoublequoteopen}{\isadigit{0}}\ {\isasymle}\ B{\isachardoublequoteclose}\ \isacommand{using}\isamarkupfalse%
\ assms\ \isacommand{by}\isamarkupfalse%
\ fastforce\isanewline
\ \ \isacommand{hence}\isamarkupfalse%
\ B{\isacharunderscore}{\kern0pt}bound{\isacharcolon}{\kern0pt}\ {\isachardoublequoteopen}{\isasymforall}z\ {\isasymge}\ {\isadigit{0}}{\isachardot}{\kern0pt}\ {\isasymbar}f\ z{\isasymbar}\ {\isasymle}\ B{\isachardoublequoteclose}\ \isacommand{by}\isamarkupfalse%
\ {\isacharparenleft}{\kern0pt}metis\ abs{\isacharunderscore}{\kern0pt}of{\isacharunderscore}{\kern0pt}nonneg{\isacharparenright}{\kern0pt}\isanewline
\isanewline
\ \ \isacommand{obtain}\isamarkupfalse%
\ D\ \isakeyword{where}\ D{\isacharunderscore}{\kern0pt}bound{\isacharcolon}{\kern0pt}\ {\isachardoublequoteopen}{\isasymbar}f\ {\isacharparenleft}{\kern0pt}m\ {\isacharplus}{\kern0pt}\ n{\isacharparenright}{\kern0pt}\ {\isacharminus}{\kern0pt}\ {\isacharparenleft}{\kern0pt}f\ m\ {\isacharplus}{\kern0pt}\ f\ n{\isacharparenright}{\kern0pt}{\isasymbar}\ {\isasymle}\ D{\isachardoublequoteclose}\ \isakeyword{and}\ D{\isacharunderscore}{\kern0pt}nonneg{\isacharcolon}{\kern0pt}\ {\isachardoublequoteopen}{\isadigit{0}}\ {\isasymle}\ D{\isachardoublequoteclose}\ \isakeyword{for}\ m\ n\ \isacommand{using}\isamarkupfalse%
\ assms\ \isacommand{by}\isamarkupfalse%
\ fast\isanewline
\isanewline
\ \ \isacommand{have}\isamarkupfalse%
\ bound{\isacharcolon}{\kern0pt}\ {\isachardoublequoteopen}{\isasymbar}f\ {\isacharparenleft}{\kern0pt}{\isacharminus}{\kern0pt}m{\isacharparenright}{\kern0pt}{\isasymbar}\ {\isasymle}\ {\isasymbar}f\ {\isadigit{0}}{\isasymbar}\ {\isacharplus}{\kern0pt}\ B\ {\isacharplus}{\kern0pt}\ D{\isachardoublequoteclose}\ \isakeyword{if}\ {\isachardoublequoteopen}m\ {\isasymge}\ {\isadigit{0}}{\isachardoublequoteclose}\ \isakeyword{for}\ m\ \isacommand{using}\isamarkupfalse%
\ D{\isacharunderscore}{\kern0pt}bound{\isacharbrackleft}{\kern0pt}of\ {\isachardoublequoteopen}{\isacharminus}{\kern0pt}m{\isachardoublequoteclose}\ m{\isacharbrackright}{\kern0pt}\ B{\isacharunderscore}{\kern0pt}bound\ that\ \isacommand{by}\isamarkupfalse%
\ auto\isanewline
\isanewline
\ \ \isacommand{have}\isamarkupfalse%
\ {\isachardoublequoteopen}{\isasymbar}f\ z{\isasymbar}\ {\isasymle}\ {\isasymbar}f\ {\isadigit{0}}{\isasymbar}\ {\isacharplus}{\kern0pt}\ B\ {\isacharplus}{\kern0pt}\ D{\isachardoublequoteclose}\ \isakeyword{for}\ z\ \isacommand{using}\isamarkupfalse%
\ B{\isacharunderscore}{\kern0pt}bound\ B{\isacharunderscore}{\kern0pt}nonneg\ D{\isacharunderscore}{\kern0pt}nonneg\ bound{\isacharbrackleft}{\kern0pt}of\ {\isachardoublequoteopen}{\isacharminus}{\kern0pt}z{\isachardoublequoteclose}{\isacharbrackright}{\kern0pt}\ \isacommand{by}\isamarkupfalse%
\ {\isacharparenleft}{\kern0pt}cases\ {\isachardoublequoteopen}z\ {\isasymge}\ {\isadigit{0}}{\isachardoublequoteclose}{\isacharparenright}{\kern0pt}\ fastforce{\isacharplus}{\kern0pt}\isanewline
\ \ \isacommand{thus}\isamarkupfalse%
\ {\isachardoublequoteopen}bounded\ f{\isachardoublequoteclose}\ \isacommand{by}\isamarkupfalse%
\ {\isacharparenleft}{\kern0pt}rule\ boundedI{\isacharparenright}{\kern0pt}\isanewline
\isacommand{qed}\isamarkupfalse%
%
\endisatagproof
{\isafoldproof}%
%
\isadelimproof
\isanewline
%
\endisadelimproof
\isanewline
\isacommand{corollary}\isamarkupfalse%
\ slope{\isacharunderscore}{\kern0pt}finite{\isacharunderscore}{\kern0pt}range{\isacharunderscore}{\kern0pt}iff{\isacharcolon}{\kern0pt}\isanewline
\ \ \isakeyword{assumes}\ {\isachardoublequoteopen}slope\ f{\isachardoublequoteclose}\isanewline
\ \ \isakeyword{shows}\ {\isachardoublequoteopen}finite\ {\isacharparenleft}{\kern0pt}range\ f{\isacharparenright}{\kern0pt}\ {\isasymlongleftrightarrow}\ finite\ {\isacharparenleft}{\kern0pt}f\ {\isacharbackquote}{\kern0pt}\ {\isacharbraceleft}{\kern0pt}{\isadigit{0}}{\isachardot}{\kern0pt}{\isachardot}{\kern0pt}{\isacharbraceright}{\kern0pt}{\isacharparenright}{\kern0pt}{\isachardoublequoteclose}\ {\isacharparenleft}{\kern0pt}\isakeyword{is}\ {\isachardoublequoteopen}{\isacharquery}{\kern0pt}lhs\ {\isasymlongleftrightarrow}\ {\isacharquery}{\kern0pt}rhs{\isachardoublequoteclose}{\isacharparenright}{\kern0pt}\isanewline
%
\isadelimproof
%
\endisadelimproof
%
\isatagproof
\isacommand{proof}\isamarkupfalse%
\ {\isacharparenleft}{\kern0pt}rule\ iffI{\isacharparenright}{\kern0pt}\isanewline
\ \ \isacommand{assume}\isamarkupfalse%
\ asm{\isacharcolon}{\kern0pt}\ {\isacharquery}{\kern0pt}rhs\isanewline
\ \ \isacommand{have}\isamarkupfalse%
\ {\isachardoublequoteopen}range\ {\isacharparenleft}{\kern0pt}f\ o\ abs{\isacharparenright}{\kern0pt}\ {\isacharequal}{\kern0pt}\ f\ {\isacharbackquote}{\kern0pt}\ {\isacharbraceleft}{\kern0pt}{\isadigit{0}}{\isachardot}{\kern0pt}{\isachardot}{\kern0pt}{\isacharbraceright}{\kern0pt}{\isachardoublequoteclose}\ \isacommand{unfolding}\isamarkupfalse%
\ comp{\isacharunderscore}{\kern0pt}def\ atLeast{\isacharunderscore}{\kern0pt}def\ image{\isacharunderscore}{\kern0pt}def\ \isacommand{by}\isamarkupfalse%
\ {\isacharparenleft}{\kern0pt}metis\ UNIV{\isacharunderscore}{\kern0pt}I\ abs{\isacharunderscore}{\kern0pt}ge{\isacharunderscore}{\kern0pt}zero\ abs{\isacharunderscore}{\kern0pt}of{\isacharunderscore}{\kern0pt}nonneg\ mem{\isacharunderscore}{\kern0pt}Collect{\isacharunderscore}{\kern0pt}eq{\isacharparenright}{\kern0pt}\isanewline
\ \ \isacommand{thus}\isamarkupfalse%
\ {\isacharquery}{\kern0pt}lhs\ \isacommand{using}\isamarkupfalse%
\ slope{\isacharunderscore}{\kern0pt}bounded{\isacharunderscore}{\kern0pt}comp{\isacharunderscore}{\kern0pt}right{\isacharunderscore}{\kern0pt}abs{\isacharbrackleft}{\kern0pt}OF\ assms{\isacharbrackright}{\kern0pt}\ asm\ \isacommand{by}\isamarkupfalse%
\ {\isacharparenleft}{\kern0pt}fastforce\ simp\ add{\isacharcolon}{\kern0pt}\ bounded{\isacharunderscore}{\kern0pt}iff{\isacharunderscore}{\kern0pt}finite{\isacharunderscore}{\kern0pt}range{\isacharparenright}{\kern0pt}\isanewline
\isacommand{qed}\isamarkupfalse%
\ {\isacharparenleft}{\kern0pt}metis\ image{\isacharunderscore}{\kern0pt}subsetI\ rangeI\ finite{\isacharunderscore}{\kern0pt}subset{\isacharparenright}{\kern0pt}%
\endisatagproof
{\isafoldproof}%
%
\isadelimproof
\isanewline
%
\endisadelimproof
\isanewline
\isacommand{lemma}\isamarkupfalse%
\ slope{\isacharunderscore}{\kern0pt}positive{\isacharunderscore}{\kern0pt}lower{\isacharunderscore}{\kern0pt}bound{\isacharcolon}{\kern0pt}\isanewline
\ \ \isakeyword{assumes}\ {\isachardoublequoteopen}slope\ f{\isachardoublequoteclose}\ {\isachardoublequoteopen}infinite\ {\isacharparenleft}{\kern0pt}f\ {\isacharbackquote}{\kern0pt}\ {\isacharbraceleft}{\kern0pt}{\isadigit{0}}{\isachardot}{\kern0pt}{\isachardot}{\kern0pt}{\isacharbraceright}{\kern0pt}\ {\isasyminter}\ {\isacharbraceleft}{\kern0pt}{\isadigit{0}}{\isacharless}{\kern0pt}{\isachardot}{\kern0pt}{\isachardot}{\kern0pt}{\isacharbraceright}{\kern0pt}{\isacharparenright}{\kern0pt}{\isachardoublequoteclose}\ {\isachardoublequoteopen}D\ {\isachargreater}{\kern0pt}\ {\isadigit{0}}{\isachardoublequoteclose}\isanewline
\ \ \isakeyword{obtains}\ M\ \isakeyword{where}\ {\isachardoublequoteopen}M\ {\isachargreater}{\kern0pt}\ {\isadigit{0}}{\isachardoublequoteclose}\ {\isachardoublequoteopen}{\isasymAnd}m{\isachardot}{\kern0pt}\ m\ {\isachargreater}{\kern0pt}\ {\isadigit{0}}\ {\isasymLongrightarrow}\ {\isacharparenleft}{\kern0pt}m\ {\isacharplus}{\kern0pt}\ {\isadigit{1}}{\isacharparenright}{\kern0pt}\ {\isacharasterisk}{\kern0pt}\ D\ {\isasymle}\ f\ {\isacharparenleft}{\kern0pt}m\ {\isacharasterisk}{\kern0pt}\ M{\isacharparenright}{\kern0pt}{\isachardoublequoteclose}\isanewline
%
\isadelimproof
%
\endisadelimproof
%
\isatagproof
\isacommand{proof}\isamarkupfalse%
\ {\isacharminus}{\kern0pt}\isanewline
\ \ \isacommand{{\isacharbraceleft}{\kern0pt}}\isamarkupfalse%
\isanewline
\ \ \ \ \isacommand{have}\isamarkupfalse%
\ D{\isacharunderscore}{\kern0pt}nonneg{\isacharcolon}{\kern0pt}\ {\isachardoublequoteopen}D\ {\isasymge}\ {\isadigit{0}}{\isachardoublequoteclose}\ \isacommand{using}\isamarkupfalse%
\ assms\ \isacommand{by}\isamarkupfalse%
\ force\isanewline
\ \ \ \ \isacommand{obtain}\isamarkupfalse%
\ C\ \isakeyword{where}\ C{\isacharunderscore}{\kern0pt}bound{\isacharcolon}{\kern0pt}\ {\isachardoublequoteopen}{\isasymbar}f\ {\isacharparenleft}{\kern0pt}m\ {\isacharplus}{\kern0pt}\ n{\isacharparenright}{\kern0pt}\ {\isacharminus}{\kern0pt}\ {\isacharparenleft}{\kern0pt}f\ m\ {\isacharplus}{\kern0pt}\ f\ n{\isacharparenright}{\kern0pt}{\isasymbar}\ {\isasymle}\ C{\isachardoublequoteclose}\ \isakeyword{and}\ C{\isacharunderscore}{\kern0pt}nonneg{\isacharcolon}{\kern0pt}\ {\isachardoublequoteopen}{\isadigit{0}}\ {\isasymle}\ C{\isachardoublequoteclose}\ \isakeyword{for}\ m\ n\ \isacommand{using}\isamarkupfalse%
\ assms\ \isacommand{by}\isamarkupfalse%
\ fast\isanewline
\isanewline
\ \ \ \ \isacommand{obtain}\isamarkupfalse%
\ f{\isacharunderscore}{\kern0pt}M\ \isakeyword{where}\ {\isachardoublequoteopen}{\isadigit{2}}\ {\isacharasterisk}{\kern0pt}\ {\isacharparenleft}{\kern0pt}C\ {\isacharplus}{\kern0pt}\ D{\isacharparenright}{\kern0pt}\ {\isasymle}\ {\isasymbar}f{\isacharunderscore}{\kern0pt}M{\isasymbar}{\isachardoublequoteclose}\ {\isachardoublequoteopen}f{\isacharunderscore}{\kern0pt}M\ {\isasymin}\ {\isacharparenleft}{\kern0pt}f\ {\isacharbackquote}{\kern0pt}\ {\isacharbraceleft}{\kern0pt}{\isadigit{0}}{\isachardot}{\kern0pt}{\isachardot}{\kern0pt}{\isacharbraceright}{\kern0pt}\ {\isasyminter}\ {\isacharbraceleft}{\kern0pt}{\isadigit{0}}{\isacharless}{\kern0pt}{\isachardot}{\kern0pt}{\isachardot}{\kern0pt}{\isacharbraceright}{\kern0pt}{\isacharparenright}{\kern0pt}{\isachardoublequoteclose}\ \isacommand{using}\isamarkupfalse%
\ mult{\isacharunderscore}{\kern0pt}left{\isacharunderscore}{\kern0pt}mono{\isacharbrackleft}{\kern0pt}of\ {\isachardoublequoteopen}C\ {\isacharplus}{\kern0pt}\ D{\isachardoublequoteclose}\ {\isacharunderscore}{\kern0pt}\ {\isadigit{2}}{\isacharbrackright}{\kern0pt}\ D{\isacharunderscore}{\kern0pt}nonneg\ \isacommand{by}\isamarkupfalse%
\ {\isacharparenleft}{\kern0pt}metis\ assms{\isacharparenleft}{\kern0pt}{\isadigit{2}}{\isacharparenright}{\kern0pt}\ abs{\isacharunderscore}{\kern0pt}ge{\isacharunderscore}{\kern0pt}zero\ abs{\isacharunderscore}{\kern0pt}le{\isacharunderscore}{\kern0pt}D{\isadigit{1}}\ int{\isacharunderscore}{\kern0pt}set{\isacharunderscore}{\kern0pt}infiniteD{\isacharparenright}{\kern0pt}\isanewline
\ \ \ \ \isacommand{then}\isamarkupfalse%
\ \isacommand{obtain}\isamarkupfalse%
\ M\ \isakeyword{where}\ M{\isacharunderscore}{\kern0pt}bound{\isacharcolon}{\kern0pt}\ {\isachardoublequoteopen}{\isadigit{2}}\ {\isacharasterisk}{\kern0pt}\ {\isacharparenleft}{\kern0pt}C\ {\isacharplus}{\kern0pt}\ D{\isacharparenright}{\kern0pt}\ {\isasymle}\ {\isasymbar}f\ M{\isasymbar}{\isachardoublequoteclose}\ {\isachardoublequoteopen}{\isadigit{0}}\ {\isacharless}{\kern0pt}\ f\ M{\isachardoublequoteclose}\ \isakeyword{and}\ M{\isacharunderscore}{\kern0pt}nonneg{\isacharcolon}{\kern0pt}\ {\isachardoublequoteopen}{\isadigit{0}}\ {\isasymle}\ M{\isachardoublequoteclose}\ \isacommand{by}\isamarkupfalse%
\ blast\isanewline
\ \ \isanewline
\ \ \ \ \isacommand{have}\isamarkupfalse%
\ neg{\isacharunderscore}{\kern0pt}bound{\isacharcolon}{\kern0pt}\ {\isachardoublequoteopen}{\isacharparenleft}{\kern0pt}f\ {\isacharparenleft}{\kern0pt}m\ {\isacharasterisk}{\kern0pt}\ M\ {\isacharplus}{\kern0pt}\ M{\isacharparenright}{\kern0pt}\ {\isacharminus}{\kern0pt}\ {\isacharparenleft}{\kern0pt}f\ {\isacharparenleft}{\kern0pt}m\ {\isacharasterisk}{\kern0pt}\ M{\isacharparenright}{\kern0pt}\ {\isacharplus}{\kern0pt}\ f\ M{\isacharparenright}{\kern0pt}{\isacharparenright}{\kern0pt}\ {\isasymge}\ {\isacharminus}{\kern0pt}C{\isachardoublequoteclose}\ \isakeyword{for}\ m\ \isacommand{by}\isamarkupfalse%
\ {\isacharparenleft}{\kern0pt}metis\ C{\isacharunderscore}{\kern0pt}bound\ abs{\isacharunderscore}{\kern0pt}diff{\isacharunderscore}{\kern0pt}le{\isacharunderscore}{\kern0pt}iff\ minus{\isacharunderscore}{\kern0pt}int{\isacharunderscore}{\kern0pt}code{\isacharparenleft}{\kern0pt}{\isadigit{1}}{\isacharcomma}{\kern0pt}{\isadigit{2}}{\isacharparenright}{\kern0pt}{\isacharparenright}{\kern0pt}\isanewline
\ \ \ \ \isacommand{hence}\isamarkupfalse%
\ neg{\isacharunderscore}{\kern0pt}bound{\isacharprime}{\kern0pt}{\isacharcolon}{\kern0pt}\ {\isachardoublequoteopen}{\isacharparenleft}{\kern0pt}f\ {\isacharparenleft}{\kern0pt}m\ {\isacharasterisk}{\kern0pt}\ M\ {\isacharplus}{\kern0pt}\ M{\isacharparenright}{\kern0pt}\ {\isacharminus}{\kern0pt}\ {\isacharparenleft}{\kern0pt}f\ {\isacharparenleft}{\kern0pt}m\ {\isacharasterisk}{\kern0pt}\ M{\isacharparenright}{\kern0pt}\ {\isacharplus}{\kern0pt}\ f\ M{\isacharparenright}{\kern0pt}{\isacharparenright}{\kern0pt}\ {\isasymge}\ {\isacharminus}{\kern0pt}{\isacharparenleft}{\kern0pt}C\ {\isacharplus}{\kern0pt}\ D{\isacharparenright}{\kern0pt}{\isachardoublequoteclose}\ \isakeyword{for}\ m\ \isacommand{by}\isamarkupfalse%
\ {\isacharparenleft}{\kern0pt}meson\ D{\isacharunderscore}{\kern0pt}nonneg\ add{\isacharunderscore}{\kern0pt}increasing{\isadigit{2}}\ minus{\isacharunderscore}{\kern0pt}le{\isacharunderscore}{\kern0pt}iff{\isacharparenright}{\kern0pt}\isanewline
\ \ \isanewline
\ \ \ \ \isacommand{have}\isamarkupfalse%
\ {\isacharasterisk}{\kern0pt}{\isacharcolon}{\kern0pt}\ {\isachardoublequoteopen}m\ {\isachargreater}{\kern0pt}\ {\isadigit{0}}\ {\isasymLongrightarrow}\ f\ {\isacharparenleft}{\kern0pt}m\ {\isacharasterisk}{\kern0pt}\ M{\isacharparenright}{\kern0pt}\ {\isasymge}\ {\isacharparenleft}{\kern0pt}m\ {\isacharplus}{\kern0pt}\ {\isadigit{1}}{\isacharparenright}{\kern0pt}\ {\isacharasterisk}{\kern0pt}\ {\isacharparenleft}{\kern0pt}C\ {\isacharplus}{\kern0pt}\ D{\isacharparenright}{\kern0pt}{\isachardoublequoteclose}\ \isakeyword{for}\ m\isanewline
\ \ \ \ \isacommand{proof}\isamarkupfalse%
\ {\isacharparenleft}{\kern0pt}induction\ m\ rule{\isacharcolon}{\kern0pt}\ int{\isacharunderscore}{\kern0pt}induct{\isacharbrackleft}{\kern0pt}\isakeyword{where}\ {\isacharquery}{\kern0pt}k{\isacharequal}{\kern0pt}{\isadigit{1}}{\isacharbrackright}{\kern0pt}{\isacharparenright}{\kern0pt}\isanewline
\ \ \ \ \ \ \isacommand{case}\isamarkupfalse%
\ base\isanewline
\ \ \ \ \ \ \isacommand{show}\isamarkupfalse%
\ {\isacharquery}{\kern0pt}case\ \isacommand{using}\isamarkupfalse%
\ M{\isacharunderscore}{\kern0pt}bound\ \isacommand{by}\isamarkupfalse%
\ fastforce\isanewline
\ \ \ \ \isacommand{next}\isamarkupfalse%
\isanewline
\ \ \ \ \ \ \isacommand{case}\isamarkupfalse%
\ {\isacharparenleft}{\kern0pt}step{\isadigit{1}}\ m{\isacharparenright}{\kern0pt}\isanewline
\ \ \ \ \ \ \isacommand{have}\isamarkupfalse%
\ {\isachardoublequoteopen}{\isacharparenleft}{\kern0pt}m\ {\isacharplus}{\kern0pt}\ {\isadigit{1}}\ {\isacharplus}{\kern0pt}\ {\isadigit{1}}{\isacharparenright}{\kern0pt}\ {\isacharasterisk}{\kern0pt}\ {\isacharparenleft}{\kern0pt}C\ {\isacharplus}{\kern0pt}\ D{\isacharparenright}{\kern0pt}\ {\isacharequal}{\kern0pt}\ {\isacharparenleft}{\kern0pt}m\ {\isacharplus}{\kern0pt}\ {\isadigit{1}}{\isacharparenright}{\kern0pt}\ {\isacharasterisk}{\kern0pt}\ {\isacharparenleft}{\kern0pt}C\ {\isacharplus}{\kern0pt}\ D{\isacharparenright}{\kern0pt}\ {\isacharplus}{\kern0pt}\ {\isadigit{2}}\ {\isacharasterisk}{\kern0pt}\ {\isacharparenleft}{\kern0pt}C\ {\isacharplus}{\kern0pt}\ D{\isacharparenright}{\kern0pt}\ {\isacharminus}{\kern0pt}\ {\isacharparenleft}{\kern0pt}C\ {\isacharplus}{\kern0pt}\ D{\isacharparenright}{\kern0pt}{\isachardoublequoteclose}\ \isacommand{by}\isamarkupfalse%
\ algebra\isanewline
\ \ \ \ \ \ \isacommand{also}\isamarkupfalse%
\ \isacommand{have}\isamarkupfalse%
\ {\isachardoublequoteopen}{\isachardot}{\kern0pt}{\isachardot}{\kern0pt}{\isachardot}{\kern0pt}\ {\isasymle}\ {\isacharparenleft}{\kern0pt}m\ {\isacharplus}{\kern0pt}\ {\isadigit{1}}{\isacharparenright}{\kern0pt}\ {\isacharasterisk}{\kern0pt}\ {\isacharparenleft}{\kern0pt}C\ {\isacharplus}{\kern0pt}\ D{\isacharparenright}{\kern0pt}\ {\isacharplus}{\kern0pt}\ f\ M\ {\isacharplus}{\kern0pt}\ {\isacharminus}{\kern0pt}\ {\isacharparenleft}{\kern0pt}C\ {\isacharplus}{\kern0pt}\ D{\isacharparenright}{\kern0pt}{\isachardoublequoteclose}\ \isacommand{using}\isamarkupfalse%
\ M{\isacharunderscore}{\kern0pt}bound\ \isacommand{by}\isamarkupfalse%
\ fastforce\isanewline
\ \ \ \ \ \ \isacommand{also}\isamarkupfalse%
\ \isacommand{have}\isamarkupfalse%
\ {\isachardoublequoteopen}{\isachardot}{\kern0pt}{\isachardot}{\kern0pt}{\isachardot}{\kern0pt}\ {\isasymle}\ f\ {\isacharparenleft}{\kern0pt}m\ {\isacharasterisk}{\kern0pt}\ M{\isacharparenright}{\kern0pt}\ {\isacharplus}{\kern0pt}\ f\ M\ {\isacharplus}{\kern0pt}\ {\isacharminus}{\kern0pt}\ {\isacharparenleft}{\kern0pt}C\ {\isacharplus}{\kern0pt}\ D{\isacharparenright}{\kern0pt}{\isachardoublequoteclose}\ \isacommand{using}\isamarkupfalse%
\ step{\isadigit{1}}\ \isacommand{by}\isamarkupfalse%
\ simp\isanewline
\ \ \ \ \ \ \isacommand{also}\isamarkupfalse%
\ \isacommand{have}\isamarkupfalse%
\ {\isachardoublequoteopen}{\isachardot}{\kern0pt}{\isachardot}{\kern0pt}{\isachardot}{\kern0pt}\ {\isasymle}\ {\isacharparenleft}{\kern0pt}f\ {\isacharparenleft}{\kern0pt}m\ {\isacharasterisk}{\kern0pt}\ M{\isacharparenright}{\kern0pt}\ {\isacharplus}{\kern0pt}\ f\ M{\isacharparenright}{\kern0pt}\ {\isacharplus}{\kern0pt}\ {\isacharparenleft}{\kern0pt}f\ {\isacharparenleft}{\kern0pt}m\ {\isacharasterisk}{\kern0pt}\ M\ {\isacharplus}{\kern0pt}\ M{\isacharparenright}{\kern0pt}\ {\isacharminus}{\kern0pt}\ {\isacharparenleft}{\kern0pt}f\ {\isacharparenleft}{\kern0pt}m\ {\isacharasterisk}{\kern0pt}\ M{\isacharparenright}{\kern0pt}\ {\isacharplus}{\kern0pt}\ f\ M{\isacharparenright}{\kern0pt}{\isacharparenright}{\kern0pt}{\isachardoublequoteclose}\ \isacommand{using}\isamarkupfalse%
\ add{\isacharunderscore}{\kern0pt}left{\isacharunderscore}{\kern0pt}mono{\isacharbrackleft}{\kern0pt}OF\ neg{\isacharunderscore}{\kern0pt}bound{\isacharprime}{\kern0pt}{\isacharbrackright}{\kern0pt}\ \isacommand{by}\isamarkupfalse%
\ blast\isanewline
\ \ \ \ \ \ \isacommand{also}\isamarkupfalse%
\ \isacommand{have}\isamarkupfalse%
\ {\isachardoublequoteopen}{\isachardot}{\kern0pt}{\isachardot}{\kern0pt}{\isachardot}{\kern0pt}\ {\isacharequal}{\kern0pt}\ f\ {\isacharparenleft}{\kern0pt}{\isacharparenleft}{\kern0pt}m\ {\isacharplus}{\kern0pt}\ {\isadigit{1}}{\isacharparenright}{\kern0pt}\ {\isacharasterisk}{\kern0pt}\ M{\isacharparenright}{\kern0pt}{\isachardoublequoteclose}\ \isacommand{by}\isamarkupfalse%
\ {\isacharparenleft}{\kern0pt}simp\ add{\isacharcolon}{\kern0pt}\ distrib{\isacharunderscore}{\kern0pt}right{\isacharparenright}{\kern0pt}\isanewline
\ \ \ \ \ \ \isacommand{finally}\isamarkupfalse%
\ \isacommand{show}\isamarkupfalse%
\ {\isacharquery}{\kern0pt}case\ \isacommand{by}\isamarkupfalse%
\ blast\isanewline
\ \ \ \ \isacommand{next}\isamarkupfalse%
\isanewline
\ \ \ \ \ \ \isacommand{case}\isamarkupfalse%
\ {\isacharparenleft}{\kern0pt}step{\isadigit{2}}\ i{\isacharparenright}{\kern0pt}\isanewline
\ \ \ \ \ \ \isacommand{then}\isamarkupfalse%
\ \isacommand{show}\isamarkupfalse%
\ {\isacharquery}{\kern0pt}case\ \isacommand{by}\isamarkupfalse%
\ linarith\isanewline
\ \ \ \ \isacommand{qed}\isamarkupfalse%
\isanewline
\ \ \isanewline
\ \ \ \ \isacommand{have}\isamarkupfalse%
\ {\isacharasterisk}{\kern0pt}{\isacharcolon}{\kern0pt}\ {\isachardoublequoteopen}f\ {\isacharparenleft}{\kern0pt}m\ {\isacharasterisk}{\kern0pt}\ M{\isacharparenright}{\kern0pt}\ {\isasymge}\ {\isacharparenleft}{\kern0pt}m\ {\isacharplus}{\kern0pt}\ {\isadigit{1}}{\isacharparenright}{\kern0pt}\ {\isacharasterisk}{\kern0pt}\ D{\isachardoublequoteclose}\ \isakeyword{if}\ {\isachardoublequoteopen}m\ {\isachargreater}{\kern0pt}\ {\isadigit{0}}{\isachardoublequoteclose}\ \isakeyword{for}\ m\ \isacommand{using}\isamarkupfalse%
\ {\isacharasterisk}{\kern0pt}{\isacharbrackleft}{\kern0pt}OF\ that{\isacharbrackright}{\kern0pt}\ mult{\isacharunderscore}{\kern0pt}left{\isacharunderscore}{\kern0pt}mono{\isacharbrackleft}{\kern0pt}of\ D\ {\isachardoublequoteopen}C\ {\isacharplus}{\kern0pt}\ D{\isachardoublequoteclose}\ {\isachardoublequoteopen}m\ {\isacharplus}{\kern0pt}\ {\isadigit{1}}{\isachardoublequoteclose}{\isacharbrackright}{\kern0pt}\ that\ C{\isacharunderscore}{\kern0pt}nonneg\ D{\isacharunderscore}{\kern0pt}nonneg\ \isacommand{by}\isamarkupfalse%
\ linarith\isanewline
\ \ \ \ \isacommand{moreover}\isamarkupfalse%
\ \isacommand{have}\isamarkupfalse%
\ {\isachardoublequoteopen}M\ {\isasymnoteq}\ {\isadigit{0}}{\isachardoublequoteclose}\ \isacommand{using}\isamarkupfalse%
\ M{\isacharunderscore}{\kern0pt}bound\ add{\isadigit{1}}{\isacharunderscore}{\kern0pt}zle{\isacharunderscore}{\kern0pt}eq\ assms\ neg{\isacharunderscore}{\kern0pt}bound\ \isacommand{by}\isamarkupfalse%
\ force\isanewline
\ \ \ \ \isacommand{ultimately}\isamarkupfalse%
\ \isacommand{have}\isamarkupfalse%
\ {\isachardoublequoteopen}{\isasymexists}M{\isachargreater}{\kern0pt}{\isadigit{0}}{\isachardot}{\kern0pt}\ {\isasymforall}m{\isachargreater}{\kern0pt}{\isadigit{0}}{\isachardot}{\kern0pt}\ {\isacharparenleft}{\kern0pt}m\ {\isacharplus}{\kern0pt}\ {\isadigit{1}}{\isacharparenright}{\kern0pt}\ {\isacharasterisk}{\kern0pt}\ D\ {\isasymle}\ f\ {\isacharparenleft}{\kern0pt}m\ {\isacharasterisk}{\kern0pt}\ M{\isacharparenright}{\kern0pt}\ {\isachardoublequoteclose}\ \isacommand{using}\isamarkupfalse%
\ M{\isacharunderscore}{\kern0pt}nonneg\ \isacommand{by}\isamarkupfalse%
\ force\isanewline
\ \ \isacommand{{\isacharbraceright}{\kern0pt}}\isamarkupfalse%
\isanewline
\ \ \isacommand{thus}\isamarkupfalse%
\ {\isacharquery}{\kern0pt}thesis\ \isacommand{using}\isamarkupfalse%
\ that\ \isacommand{by}\isamarkupfalse%
\ blast\isanewline
\isacommand{qed}\isamarkupfalse%
%
\endisatagproof
{\isafoldproof}%
%
\isadelimproof
%
\endisadelimproof
%
\isadelimdocument
%
\endisadelimdocument
%
\isatagdocument
%
\isamarkupsubsection{Set Membership of \isa{Inf} and \isa{Sup} on Integers%
}
\isamarkuptrue%
%
\endisatagdocument
{\isafolddocument}%
%
\isadelimdocument
%
\endisadelimdocument
\isacommand{lemma}\isamarkupfalse%
\ int{\isacharunderscore}{\kern0pt}Inf{\isacharunderscore}{\kern0pt}mem{\isacharcolon}{\kern0pt}\isanewline
\ \ \isakeyword{fixes}\ S\ {\isacharcolon}{\kern0pt}{\isacharcolon}{\kern0pt}\ {\isachardoublequoteopen}int\ set{\isachardoublequoteclose}\isanewline
\ \ \isakeyword{assumes}\ {\isachardoublequoteopen}S\ {\isasymnoteq}\ {\isacharbraceleft}{\kern0pt}{\isacharbraceright}{\kern0pt}{\isachardoublequoteclose}\ {\isachardoublequoteopen}bdd{\isacharunderscore}{\kern0pt}below\ S{\isachardoublequoteclose}\isanewline
\ \ \isakeyword{shows}\ {\isachardoublequoteopen}Inf\ S\ {\isasymin}\ S{\isachardoublequoteclose}\isanewline
%
\isadelimproof
%
\endisadelimproof
%
\isatagproof
\isacommand{proof}\isamarkupfalse%
\ {\isacharminus}{\kern0pt}\isanewline
\ \ \isacommand{have}\isamarkupfalse%
\ nonneg{\isacharcolon}{\kern0pt}\ {\isachardoublequoteopen}Inf\ {\isacharparenleft}{\kern0pt}{\isacharbraceleft}{\kern0pt}{\isadigit{0}}{\isachardot}{\kern0pt}{\isachardot}{\kern0pt}{\isacharbraceright}{\kern0pt}\ {\isasyminter}\ A{\isacharparenright}{\kern0pt}\ {\isasymin}\ {\isacharparenleft}{\kern0pt}{\isacharbraceleft}{\kern0pt}{\isadigit{0}}{\isachardot}{\kern0pt}{\isachardot}{\kern0pt}{\isacharbraceright}{\kern0pt}\ {\isasyminter}\ A{\isacharparenright}{\kern0pt}{\isachardoublequoteclose}\ \isakeyword{if}\ asm{\isacharcolon}{\kern0pt}\ {\isachardoublequoteopen}{\isacharparenleft}{\kern0pt}{\isacharbraceleft}{\kern0pt}{\isacharparenleft}{\kern0pt}{\isadigit{0}}{\isacharcolon}{\kern0pt}{\isacharcolon}{\kern0pt}int{\isacharparenright}{\kern0pt}{\isachardot}{\kern0pt}{\isachardot}{\kern0pt}{\isacharbraceright}{\kern0pt}\ {\isasyminter}\ A{\isacharparenright}{\kern0pt}\ {\isasymnoteq}\ {\isacharbraceleft}{\kern0pt}{\isacharbraceright}{\kern0pt}{\isachardoublequoteclose}\ \isakeyword{for}\ A\isanewline
\ \ \isacommand{proof}\isamarkupfalse%
\ {\isacharminus}{\kern0pt}\isanewline
\ \ \ \ \isacommand{have}\isamarkupfalse%
\ {\isachardoublequoteopen}nat\ {\isacharbackquote}{\kern0pt}\ {\isacharparenleft}{\kern0pt}{\isacharbraceleft}{\kern0pt}{\isadigit{0}}{\isachardot}{\kern0pt}{\isachardot}{\kern0pt}{\isacharbraceright}{\kern0pt}\ {\isasyminter}\ A{\isacharparenright}{\kern0pt}\ {\isasymnoteq}\ {\isacharbraceleft}{\kern0pt}{\isacharbraceright}{\kern0pt}{\isachardoublequoteclose}\ \isacommand{using}\isamarkupfalse%
\ asm\ \isacommand{by}\isamarkupfalse%
\ blast\isanewline
\ \ \ \ \isacommand{hence}\isamarkupfalse%
\ {\isachardoublequoteopen}int\ {\isacharparenleft}{\kern0pt}Inf\ {\isacharparenleft}{\kern0pt}nat\ {\isacharbackquote}{\kern0pt}\ {\isacharparenleft}{\kern0pt}{\isacharbraceleft}{\kern0pt}{\isadigit{0}}{\isachardot}{\kern0pt}{\isachardot}{\kern0pt}{\isacharbraceright}{\kern0pt}\ {\isasyminter}\ A{\isacharparenright}{\kern0pt}{\isacharparenright}{\kern0pt}{\isacharparenright}{\kern0pt}\ {\isasymin}\ int\ {\isacharbackquote}{\kern0pt}\ nat\ {\isacharbackquote}{\kern0pt}\ {\isacharparenleft}{\kern0pt}{\isacharbraceleft}{\kern0pt}{\isadigit{0}}{\isachardot}{\kern0pt}{\isachardot}{\kern0pt}{\isacharbraceright}{\kern0pt}\ {\isasyminter}\ A{\isacharparenright}{\kern0pt}{\isachardoublequoteclose}\ \isacommand{using}\isamarkupfalse%
\ wellorder{\isacharunderscore}{\kern0pt}InfI{\isacharbrackleft}{\kern0pt}of\ {\isacharunderscore}{\kern0pt}\ {\isachardoublequoteopen}nat\ {\isacharbackquote}{\kern0pt}\ {\isacharparenleft}{\kern0pt}{\isacharbraceleft}{\kern0pt}{\isadigit{0}}{\isachardot}{\kern0pt}{\isachardot}{\kern0pt}{\isacharbraceright}{\kern0pt}\ {\isasyminter}\ A{\isacharparenright}{\kern0pt}{\isachardoublequoteclose}{\isacharbrackright}{\kern0pt}\ \isacommand{by}\isamarkupfalse%
\ fast\isanewline
\ \ \ \ \isacommand{moreover}\isamarkupfalse%
\ \isacommand{have}\isamarkupfalse%
\ {\isachardoublequoteopen}int\ {\isacharbackquote}{\kern0pt}\ nat\ {\isacharbackquote}{\kern0pt}\ {\isacharparenleft}{\kern0pt}{\isacharbraceleft}{\kern0pt}{\isadigit{0}}{\isachardot}{\kern0pt}{\isachardot}{\kern0pt}{\isacharbraceright}{\kern0pt}\ {\isasyminter}\ A{\isacharparenright}{\kern0pt}\ {\isacharequal}{\kern0pt}\ {\isacharbraceleft}{\kern0pt}{\isadigit{0}}{\isachardot}{\kern0pt}{\isachardot}{\kern0pt}{\isacharbraceright}{\kern0pt}\ {\isasyminter}\ A{\isachardoublequoteclose}\ \isacommand{by}\isamarkupfalse%
\ force\isanewline
\ \ \ \ \isacommand{moreover}\isamarkupfalse%
\ \isacommand{have}\isamarkupfalse%
\ {\isachardoublequoteopen}Inf\ {\isacharparenleft}{\kern0pt}{\isacharbraceleft}{\kern0pt}{\isadigit{0}}{\isachardot}{\kern0pt}{\isachardot}{\kern0pt}{\isacharbraceright}{\kern0pt}\ {\isasyminter}\ A{\isacharparenright}{\kern0pt}\ {\isacharequal}{\kern0pt}\ int\ {\isacharparenleft}{\kern0pt}Inf\ {\isacharparenleft}{\kern0pt}nat\ {\isacharbackquote}{\kern0pt}\ {\isacharparenleft}{\kern0pt}{\isacharbraceleft}{\kern0pt}{\isadigit{0}}{\isachardot}{\kern0pt}{\isachardot}{\kern0pt}{\isacharbraceright}{\kern0pt}\ {\isasyminter}\ A{\isacharparenright}{\kern0pt}{\isacharparenright}{\kern0pt}{\isacharparenright}{\kern0pt}{\isachardoublequoteclose}\ \isanewline
\ \ \ \ \ \ \isacommand{using}\isamarkupfalse%
\ calculation\ \isacommand{by}\isamarkupfalse%
\ {\isacharparenleft}{\kern0pt}intro\ cInf{\isacharunderscore}{\kern0pt}eq{\isacharunderscore}{\kern0pt}minimum{\isacharparenright}{\kern0pt}\ {\isacharparenleft}{\kern0pt}argo{\isacharcomma}{\kern0pt}\ metis\ IntD{\isadigit{2}}\ Int{\isacharunderscore}{\kern0pt}commute\ atLeast{\isacharunderscore}{\kern0pt}iff\ imageI\ le{\isacharunderscore}{\kern0pt}nat{\isacharunderscore}{\kern0pt}iff\ wellorder{\isacharunderscore}{\kern0pt}Inf{\isacharunderscore}{\kern0pt}le{\isadigit{1}}{\isacharparenright}{\kern0pt}\isanewline
\ \ \ \ \isacommand{ultimately}\isamarkupfalse%
\ \isacommand{show}\isamarkupfalse%
\ {\isacharquery}{\kern0pt}thesis\ \isacommand{by}\isamarkupfalse%
\ argo\isanewline
\ \ \isacommand{qed}\isamarkupfalse%
\isanewline
\ \ \isacommand{have}\isamarkupfalse%
\ {\isacharasterisk}{\kern0pt}{\isacharasterisk}{\kern0pt}{\isacharcolon}{\kern0pt}\ {\isachardoublequoteopen}Inf\ {\isacharparenleft}{\kern0pt}{\isacharbraceleft}{\kern0pt}b{\isachardot}{\kern0pt}{\isachardot}{\kern0pt}{\isacharbraceright}{\kern0pt}\ {\isasyminter}\ A{\isacharparenright}{\kern0pt}\ {\isasymin}\ {\isacharparenleft}{\kern0pt}{\isacharbraceleft}{\kern0pt}b{\isachardot}{\kern0pt}{\isachardot}{\kern0pt}{\isacharbraceright}{\kern0pt}\ {\isasyminter}\ A{\isacharparenright}{\kern0pt}{\isachardoublequoteclose}\ \isakeyword{if}\ asm{\isacharcolon}{\kern0pt}\ {\isachardoublequoteopen}{\isacharparenleft}{\kern0pt}{\isacharbraceleft}{\kern0pt}{\isacharparenleft}{\kern0pt}b{\isacharcolon}{\kern0pt}{\isacharcolon}{\kern0pt}int{\isacharparenright}{\kern0pt}{\isachardot}{\kern0pt}{\isachardot}{\kern0pt}{\isacharbraceright}{\kern0pt}\ {\isasyminter}\ A{\isacharparenright}{\kern0pt}\ {\isasymnoteq}\ {\isacharbraceleft}{\kern0pt}{\isacharbraceright}{\kern0pt}{\isachardoublequoteclose}\ \isakeyword{for}\ A\ b\isanewline
\ \ \isacommand{proof}\isamarkupfalse%
\ {\isacharparenleft}{\kern0pt}cases\ {\isachardoublequoteopen}b\ {\isasymge}\ {\isadigit{0}}{\isachardoublequoteclose}{\isacharparenright}{\kern0pt}\isanewline
\ \ \ \ \isacommand{case}\isamarkupfalse%
\ True\isanewline
\ \ \ \ \isacommand{hence}\isamarkupfalse%
\ {\isachardoublequoteopen}{\isacharparenleft}{\kern0pt}{\isacharbraceleft}{\kern0pt}b{\isachardot}{\kern0pt}{\isachardot}{\kern0pt}{\isacharbraceright}{\kern0pt}\ {\isasyminter}\ A{\isacharparenright}{\kern0pt}\ {\isacharequal}{\kern0pt}\ {\isacharbraceleft}{\kern0pt}{\isadigit{0}}{\isachardot}{\kern0pt}{\isachardot}{\kern0pt}{\isacharbraceright}{\kern0pt}\ {\isasyminter}\ {\isacharparenleft}{\kern0pt}{\isacharbraceleft}{\kern0pt}b{\isachardot}{\kern0pt}{\isachardot}{\kern0pt}{\isacharbraceright}{\kern0pt}\ {\isasyminter}\ A{\isacharparenright}{\kern0pt}{\isachardoublequoteclose}\ \isacommand{by}\isamarkupfalse%
\ fastforce\isanewline
\ \ \ \ \isacommand{thus}\isamarkupfalse%
\ {\isacharquery}{\kern0pt}thesis\ \isacommand{using}\isamarkupfalse%
\ asm\ nonneg\ \isacommand{by}\isamarkupfalse%
\ metis\isanewline
\ \ \isacommand{next}\isamarkupfalse%
\isanewline
\ \ \ \ \isacommand{case}\isamarkupfalse%
\ False\isanewline
\ \ \ \ \isacommand{hence}\isamarkupfalse%
\ partition{\isacharcolon}{\kern0pt}\ {\isachardoublequoteopen}{\isacharparenleft}{\kern0pt}{\isacharbraceleft}{\kern0pt}b{\isachardot}{\kern0pt}{\isachardot}{\kern0pt}{\isacharbraceright}{\kern0pt}\ {\isasyminter}\ A{\isacharparenright}{\kern0pt}\ {\isacharequal}{\kern0pt}\ {\isacharparenleft}{\kern0pt}{\isacharbraceleft}{\kern0pt}{\isadigit{0}}{\isachardot}{\kern0pt}{\isachardot}{\kern0pt}{\isacharbraceright}{\kern0pt}\ {\isasyminter}\ A{\isacharparenright}{\kern0pt}\ {\isasymunion}\ {\isacharparenleft}{\kern0pt}{\isacharbraceleft}{\kern0pt}b{\isachardot}{\kern0pt}{\isachardot}{\kern0pt}{\isacharless}{\kern0pt}{\isadigit{0}}{\isacharbraceright}{\kern0pt}\ {\isasyminter}\ A{\isacharparenright}{\kern0pt}{\isachardoublequoteclose}\ \isacommand{by}\isamarkupfalse%
\ fastforce\isanewline
\ \ \ \ \isacommand{have}\isamarkupfalse%
\ bdd{\isacharunderscore}{\kern0pt}below{\isacharcolon}{\kern0pt}\ {\isachardoublequoteopen}bdd{\isacharunderscore}{\kern0pt}below\ {\isacharparenleft}{\kern0pt}{\isacharbraceleft}{\kern0pt}{\isadigit{0}}{\isachardot}{\kern0pt}{\isachardot}{\kern0pt}{\isacharbraceright}{\kern0pt}\ {\isasyminter}\ A{\isacharparenright}{\kern0pt}{\isachardoublequoteclose}\ {\isachardoublequoteopen}bdd{\isacharunderscore}{\kern0pt}below\ {\isacharparenleft}{\kern0pt}{\isacharbraceleft}{\kern0pt}b{\isachardot}{\kern0pt}{\isachardot}{\kern0pt}{\isacharless}{\kern0pt}{\isadigit{0}}{\isacharbraceright}{\kern0pt}\ {\isasyminter}\ A{\isacharparenright}{\kern0pt}{\isachardoublequoteclose}\ \isacommand{by}\isamarkupfalse%
\ simp{\isacharplus}{\kern0pt}\isanewline
\ \ \ \ \isacommand{thus}\isamarkupfalse%
\ {\isacharquery}{\kern0pt}thesis\ \isanewline
\ \ \ \ \isacommand{proof}\isamarkupfalse%
\ {\isacharparenleft}{\kern0pt}cases\ {\isachardoublequoteopen}{\isacharparenleft}{\kern0pt}{\isacharbraceleft}{\kern0pt}{\isadigit{0}}{\isachardot}{\kern0pt}{\isachardot}{\kern0pt}{\isacharbraceright}{\kern0pt}\ {\isasyminter}\ A{\isacharparenright}{\kern0pt}\ {\isasymnoteq}\ {\isacharbraceleft}{\kern0pt}{\isacharbraceright}{\kern0pt}\ {\isasymand}\ {\isacharparenleft}{\kern0pt}{\isacharbraceleft}{\kern0pt}b{\isachardot}{\kern0pt}{\isachardot}{\kern0pt}{\isacharless}{\kern0pt}{\isadigit{0}}{\isacharbraceright}{\kern0pt}\ {\isasyminter}\ A{\isacharparenright}{\kern0pt}\ {\isasymnoteq}\ {\isacharbraceleft}{\kern0pt}{\isacharbraceright}{\kern0pt}{\isachardoublequoteclose}{\isacharparenright}{\kern0pt}\isanewline
\ \ \ \ \ \ \isacommand{case}\isamarkupfalse%
\ True\isanewline
\ \ \ \ \ \ \isacommand{have}\isamarkupfalse%
\ finite{\isacharcolon}{\kern0pt}\ {\isachardoublequoteopen}finite\ {\isacharparenleft}{\kern0pt}{\isacharbraceleft}{\kern0pt}b{\isachardot}{\kern0pt}{\isachardot}{\kern0pt}{\isacharless}{\kern0pt}{\isadigit{0}}{\isacharbraceright}{\kern0pt}\ {\isasyminter}\ A{\isacharparenright}{\kern0pt}{\isachardoublequoteclose}\ \isacommand{by}\isamarkupfalse%
\ blast\isanewline
\ \ \ \ \ \ \isacommand{have}\isamarkupfalse%
\ {\isachardoublequoteopen}{\isacharparenleft}{\kern0pt}x\ {\isacharcolon}{\kern0pt}{\isacharcolon}{\kern0pt}\ int{\isacharparenright}{\kern0pt}\ {\isasymle}\ y\ {\isasymLongrightarrow}\ inf\ x\ y\ {\isacharequal}{\kern0pt}\ x{\isachardoublequoteclose}\ \isakeyword{for}\ x\ y\ \isacommand{by}\isamarkupfalse%
\ {\isacharparenleft}{\kern0pt}simp\ add{\isacharcolon}{\kern0pt}\ inf{\isachardot}{\kern0pt}order{\isacharunderscore}{\kern0pt}iff{\isacharparenright}{\kern0pt}\isanewline
\ \ \ \ \ \ \isacommand{have}\isamarkupfalse%
\ {\isachardoublequoteopen}Inf\ {\isacharparenleft}{\kern0pt}{\isacharbraceleft}{\kern0pt}b{\isachardot}{\kern0pt}{\isachardot}{\kern0pt}{\isacharbraceright}{\kern0pt}\ {\isasyminter}\ A{\isacharparenright}{\kern0pt}\ {\isacharequal}{\kern0pt}\ inf\ {\isacharparenleft}{\kern0pt}Inf\ {\isacharparenleft}{\kern0pt}{\isacharbraceleft}{\kern0pt}{\isadigit{0}}{\isachardot}{\kern0pt}{\isachardot}{\kern0pt}{\isacharbraceright}{\kern0pt}\ {\isasyminter}\ A{\isacharparenright}{\kern0pt}{\isacharparenright}{\kern0pt}\ {\isacharparenleft}{\kern0pt}Inf\ {\isacharparenleft}{\kern0pt}{\isacharbraceleft}{\kern0pt}b{\isachardot}{\kern0pt}{\isachardot}{\kern0pt}{\isacharless}{\kern0pt}{\isadigit{0}}{\isacharbraceright}{\kern0pt}\ {\isasyminter}\ A{\isacharparenright}{\kern0pt}{\isacharparenright}{\kern0pt}{\isachardoublequoteclose}\ \isacommand{by}\isamarkupfalse%
\ {\isacharparenleft}{\kern0pt}metis\ cInf{\isacharunderscore}{\kern0pt}union{\isacharunderscore}{\kern0pt}distrib\ True\ bdd{\isacharunderscore}{\kern0pt}below\ partition{\isacharparenright}{\kern0pt}\isanewline
\ \ \ \ \ \ \isacommand{moreover}\isamarkupfalse%
\ \isacommand{have}\isamarkupfalse%
\ {\isachardoublequoteopen}Inf\ {\isacharparenleft}{\kern0pt}{\isacharbraceleft}{\kern0pt}b{\isachardot}{\kern0pt}{\isachardot}{\kern0pt}{\isacharless}{\kern0pt}{\isadigit{0}}{\isacharbraceright}{\kern0pt}\ {\isasyminter}\ A{\isacharparenright}{\kern0pt}\ {\isasymin}\ {\isacharparenleft}{\kern0pt}{\isacharbraceleft}{\kern0pt}b{\isachardot}{\kern0pt}{\isachardot}{\kern0pt}{\isacharbraceright}{\kern0pt}\ {\isasyminter}\ A{\isacharparenright}{\kern0pt}{\isachardoublequoteclose}\ \isacommand{using}\isamarkupfalse%
\ Min{\isacharunderscore}{\kern0pt}in{\isacharbrackleft}{\kern0pt}OF\ finite{\isacharbrackright}{\kern0pt}\ cInf{\isacharunderscore}{\kern0pt}eq{\isacharunderscore}{\kern0pt}Min{\isacharbrackleft}{\kern0pt}OF\ finite{\isacharbrackright}{\kern0pt}\ True\ partition\ \isacommand{by}\isamarkupfalse%
\ simp\isanewline
\ \ \ \ \ \ \isacommand{moreover}\isamarkupfalse%
\ \isacommand{have}\isamarkupfalse%
\ {\isachardoublequoteopen}Inf\ {\isacharparenleft}{\kern0pt}{\isacharbraceleft}{\kern0pt}{\isadigit{0}}{\isachardot}{\kern0pt}{\isachardot}{\kern0pt}{\isacharbraceright}{\kern0pt}\ {\isasyminter}\ A{\isacharparenright}{\kern0pt}\ {\isasymin}\ {\isacharparenleft}{\kern0pt}{\isacharbraceleft}{\kern0pt}b{\isachardot}{\kern0pt}{\isachardot}{\kern0pt}{\isacharbraceright}{\kern0pt}\ {\isasyminter}\ A{\isacharparenright}{\kern0pt}{\isachardoublequoteclose}\ \isacommand{using}\isamarkupfalse%
\ nonneg\ True\ partition\ \isacommand{by}\isamarkupfalse%
\ blast\isanewline
\ \ \ \ \ \ \isacommand{moreover}\isamarkupfalse%
\ \isacommand{have}\isamarkupfalse%
\ {\isachardoublequoteopen}inf\ {\isacharparenleft}{\kern0pt}Inf\ {\isacharparenleft}{\kern0pt}{\isacharbraceleft}{\kern0pt}{\isadigit{0}}{\isachardot}{\kern0pt}{\isachardot}{\kern0pt}{\isacharbraceright}{\kern0pt}\ {\isasyminter}\ A{\isacharparenright}{\kern0pt}{\isacharparenright}{\kern0pt}\ {\isacharparenleft}{\kern0pt}Inf\ {\isacharparenleft}{\kern0pt}{\isacharbraceleft}{\kern0pt}b{\isachardot}{\kern0pt}{\isachardot}{\kern0pt}{\isacharless}{\kern0pt}{\isadigit{0}}{\isacharbraceright}{\kern0pt}\ {\isasyminter}\ A{\isacharparenright}{\kern0pt}{\isacharparenright}{\kern0pt}\ {\isasymin}\ {\isacharbraceleft}{\kern0pt}Inf\ {\isacharparenleft}{\kern0pt}{\isacharbraceleft}{\kern0pt}{\isadigit{0}}{\isachardot}{\kern0pt}{\isachardot}{\kern0pt}{\isacharbraceright}{\kern0pt}\ {\isasyminter}\ A{\isacharparenright}{\kern0pt}{\isacharcomma}{\kern0pt}\ Inf\ {\isacharparenleft}{\kern0pt}{\isacharbraceleft}{\kern0pt}b{\isachardot}{\kern0pt}{\isachardot}{\kern0pt}{\isacharless}{\kern0pt}{\isadigit{0}}{\isacharbraceright}{\kern0pt}\ {\isasyminter}\ A{\isacharparenright}{\kern0pt}{\isacharbraceright}{\kern0pt}{\isachardoublequoteclose}\ \isacommand{by}\isamarkupfalse%
\ {\isacharparenleft}{\kern0pt}metis\ inf{\isachardot}{\kern0pt}commute\ inf{\isachardot}{\kern0pt}order{\isacharunderscore}{\kern0pt}iff\ insertI{\isadigit{1}}\ insertI{\isadigit{2}}\ nle{\isacharunderscore}{\kern0pt}le{\isacharparenright}{\kern0pt}\isanewline
\ \ \ \ \ \ \isacommand{ultimately}\isamarkupfalse%
\ \isacommand{show}\isamarkupfalse%
\ {\isacharquery}{\kern0pt}thesis\ \isacommand{by}\isamarkupfalse%
\ force\isanewline
\ \ \ \ \isacommand{next}\isamarkupfalse%
\isanewline
\ \ \ \ \ \ \isacommand{case}\isamarkupfalse%
\ False\isanewline
\ \ \ \ \ \ \isacommand{hence}\isamarkupfalse%
\ {\isachardoublequoteopen}{\isacharparenleft}{\kern0pt}{\isacharbraceleft}{\kern0pt}b{\isachardot}{\kern0pt}{\isachardot}{\kern0pt}{\isacharbraceright}{\kern0pt}\ {\isasyminter}\ A{\isacharparenright}{\kern0pt}\ {\isacharequal}{\kern0pt}\ {\isacharparenleft}{\kern0pt}{\isacharbraceleft}{\kern0pt}{\isadigit{0}}{\isachardot}{\kern0pt}{\isachardot}{\kern0pt}{\isacharbraceright}{\kern0pt}\ {\isasyminter}\ A{\isacharparenright}{\kern0pt}\ {\isasymor}\ {\isacharparenleft}{\kern0pt}{\isacharbraceleft}{\kern0pt}b{\isachardot}{\kern0pt}{\isachardot}{\kern0pt}{\isacharbraceright}{\kern0pt}\ {\isasyminter}\ A{\isacharparenright}{\kern0pt}\ {\isacharequal}{\kern0pt}\ {\isacharparenleft}{\kern0pt}{\isacharbraceleft}{\kern0pt}b{\isachardot}{\kern0pt}{\isachardot}{\kern0pt}{\isacharless}{\kern0pt}{\isadigit{0}}{\isacharbraceright}{\kern0pt}\ {\isasyminter}\ A{\isacharparenright}{\kern0pt}{\isachardoublequoteclose}\ \isacommand{using}\isamarkupfalse%
\ partition\ \isacommand{by}\isamarkupfalse%
\ auto\isanewline
\ \ \ \ \ \ \isacommand{thus}\isamarkupfalse%
\ {\isacharquery}{\kern0pt}thesis\ \isacommand{using}\isamarkupfalse%
\ Min{\isacharunderscore}{\kern0pt}in{\isacharbrackleft}{\kern0pt}of\ {\isachardoublequoteopen}{\isacharbraceleft}{\kern0pt}b{\isachardot}{\kern0pt}{\isachardot}{\kern0pt}{\isacharbraceright}{\kern0pt}\ {\isasyminter}\ A{\isachardoublequoteclose}{\isacharbrackright}{\kern0pt}\ cInf{\isacharunderscore}{\kern0pt}eq{\isacharunderscore}{\kern0pt}Min{\isacharbrackleft}{\kern0pt}of\ {\isachardoublequoteopen}{\isacharbraceleft}{\kern0pt}b{\isachardot}{\kern0pt}{\isachardot}{\kern0pt}{\isacharbraceright}{\kern0pt}\ {\isasyminter}\ A{\isachardoublequoteclose}{\isacharbrackright}{\kern0pt}\ \isacommand{by}\isamarkupfalse%
\ {\isacharparenleft}{\kern0pt}metis\ asm\ nonneg\ finite{\isacharunderscore}{\kern0pt}Int\ finite{\isacharunderscore}{\kern0pt}atLeastLessThan{\isacharunderscore}{\kern0pt}int{\isacharparenright}{\kern0pt}\isanewline
\ \ \ \ \isacommand{qed}\isamarkupfalse%
\isanewline
\ \ \isacommand{qed}\isamarkupfalse%
\isanewline
\ \ \isacommand{obtain}\isamarkupfalse%
\ b\ \isakeyword{where}\ {\isachardoublequoteopen}S\ {\isacharequal}{\kern0pt}\ {\isacharbraceleft}{\kern0pt}b{\isachardot}{\kern0pt}{\isachardot}{\kern0pt}{\isacharbraceright}{\kern0pt}\ {\isasyminter}\ S{\isachardoublequoteclose}\ \isacommand{using}\isamarkupfalse%
\ assms\ \isacommand{unfolding}\isamarkupfalse%
\ bdd{\isacharunderscore}{\kern0pt}below{\isacharunderscore}{\kern0pt}def\ \isacommand{by}\isamarkupfalse%
\ blast\isanewline
\ \ \isacommand{thus}\isamarkupfalse%
\ {\isacharquery}{\kern0pt}thesis\ \isacommand{using}\isamarkupfalse%
\ {\isacharasterisk}{\kern0pt}{\isacharasterisk}{\kern0pt}\ assms\ \isacommand{by}\isamarkupfalse%
\ metis\isanewline
\isacommand{qed}\isamarkupfalse%
%
\endisatagproof
{\isafoldproof}%
%
\isadelimproof
\isanewline
%
\endisadelimproof
\isanewline
\isacommand{lemma}\isamarkupfalse%
\ int{\isacharunderscore}{\kern0pt}Sup{\isacharunderscore}{\kern0pt}mem{\isacharcolon}{\kern0pt}\isanewline
\ \ \isakeyword{fixes}\ S\ {\isacharcolon}{\kern0pt}{\isacharcolon}{\kern0pt}\ {\isachardoublequoteopen}int\ set{\isachardoublequoteclose}\isanewline
\ \ \isakeyword{assumes}\ {\isachardoublequoteopen}S\ {\isasymnoteq}\ {\isacharbraceleft}{\kern0pt}{\isacharbraceright}{\kern0pt}{\isachardoublequoteclose}\ {\isachardoublequoteopen}bdd{\isacharunderscore}{\kern0pt}above\ S{\isachardoublequoteclose}\isanewline
\ \ \isakeyword{shows}\ {\isachardoublequoteopen}Sup\ S\ {\isasymin}\ S{\isachardoublequoteclose}\isanewline
%
\isadelimproof
%
\endisadelimproof
%
\isatagproof
\isacommand{proof}\isamarkupfalse%
\ {\isacharminus}{\kern0pt}\isanewline
\ \ \isacommand{have}\isamarkupfalse%
\ {\isachardoublequoteopen}Sup\ S\ {\isacharequal}{\kern0pt}\ {\isacharparenleft}{\kern0pt}{\isacharminus}{\kern0pt}\ Inf\ {\isacharparenleft}{\kern0pt}uminus\ {\isacharbackquote}{\kern0pt}\ S{\isacharparenright}{\kern0pt}{\isacharparenright}{\kern0pt}{\isachardoublequoteclose}\ \isacommand{unfolding}\isamarkupfalse%
\ Inf{\isacharunderscore}{\kern0pt}int{\isacharunderscore}{\kern0pt}def\ image{\isacharunderscore}{\kern0pt}comp\ \isacommand{by}\isamarkupfalse%
\ simp\isanewline
\ \ \isacommand{moreover}\isamarkupfalse%
\ \isacommand{have}\isamarkupfalse%
\ {\isachardoublequoteopen}bdd{\isacharunderscore}{\kern0pt}below\ {\isacharparenleft}{\kern0pt}uminus\ {\isacharbackquote}{\kern0pt}\ S{\isacharparenright}{\kern0pt}{\isachardoublequoteclose}\ \isacommand{using}\isamarkupfalse%
\ assms\ \isacommand{unfolding}\isamarkupfalse%
\ bdd{\isacharunderscore}{\kern0pt}below{\isacharunderscore}{\kern0pt}def\ bdd{\isacharunderscore}{\kern0pt}above{\isacharunderscore}{\kern0pt}def\ \isacommand{by}\isamarkupfalse%
\ {\isacharparenleft}{\kern0pt}metis\ imageE\ neg{\isacharunderscore}{\kern0pt}le{\isacharunderscore}{\kern0pt}iff{\isacharunderscore}{\kern0pt}le{\isacharparenright}{\kern0pt}\isanewline
\ \ \isacommand{moreover}\isamarkupfalse%
\ \isacommand{have}\isamarkupfalse%
\ {\isachardoublequoteopen}Inf\ {\isacharparenleft}{\kern0pt}uminus\ {\isacharbackquote}{\kern0pt}\ S{\isacharparenright}{\kern0pt}\ {\isasymin}\ {\isacharparenleft}{\kern0pt}uminus\ {\isacharbackquote}{\kern0pt}\ S{\isacharparenright}{\kern0pt}{\isachardoublequoteclose}\ \isacommand{using}\isamarkupfalse%
\ int{\isacharunderscore}{\kern0pt}Inf{\isacharunderscore}{\kern0pt}mem\ assms\ \isacommand{by}\isamarkupfalse%
\ simp\isanewline
\ \ \isacommand{ultimately}\isamarkupfalse%
\ \isacommand{show}\isamarkupfalse%
\ {\isacharquery}{\kern0pt}thesis\ \isacommand{by}\isamarkupfalse%
\ force\isanewline
\isacommand{qed}\isamarkupfalse%
%
\endisatagproof
{\isafoldproof}%
%
\isadelimproof
\isanewline
%
\endisadelimproof
%
\isadelimtheory
\isanewline
%
\endisadelimtheory
%
\isatagtheory
\isacommand{end}\isamarkupfalse%
%
\endisatagtheory
{\isafoldtheory}%
%
\isadelimtheory
%
\endisadelimtheory
%
\end{isabellebody}%
\endinput
%:%file=Slope.tex%:%
%:%6=2%:%
%:%7=3%:%
%:%12=4%:%
%:%13=4%:%
%:%14=5%:%
%:%15=6%:%
%:%29=8%:%
%:%33=10%:%
%:%43=12%:%
%:%44=12%:%
%:%45=13%:%
%:%46=14%:%
%:%47=15%:%
%:%48=15%:%
%:%49=16%:%
%:%50=17%:%
%:%53=18%:%
%:%57=18%:%
%:%58=18%:%
%:%59=18%:%
%:%64=18%:%
%:%67=19%:%
%:%68=20%:%
%:%69=20%:%
%:%70=21%:%
%:%71=22%:%
%:%74=23%:%
%:%78=23%:%
%:%79=23%:%
%:%80=23%:%
%:%81=23%:%
%:%86=23%:%
%:%89=24%:%
%:%90=25%:%
%:%91=25%:%
%:%92=26%:%
%:%93=27%:%
%:%96=28%:%
%:%100=28%:%
%:%101=28%:%
%:%106=28%:%
%:%109=29%:%
%:%110=30%:%
%:%111=30%:%
%:%113=30%:%
%:%117=30%:%
%:%118=30%:%
%:%119=30%:%
%:%126=30%:%
%:%127=31%:%
%:%128=32%:%
%:%129=32%:%
%:%136=33%:%
%:%137=33%:%
%:%138=34%:%
%:%139=34%:%
%:%140=35%:%
%:%141=35%:%
%:%142=35%:%
%:%143=35%:%
%:%144=36%:%
%:%145=36%:%
%:%146=36%:%
%:%147=36%:%
%:%148=37%:%
%:%149=37%:%
%:%150=37%:%
%:%151=37%:%
%:%152=38%:%
%:%153=38%:%
%:%154=38%:%
%:%155=38%:%
%:%156=38%:%
%:%157=39%:%
%:%158=39%:%
%:%159=40%:%
%:%160=40%:%
%:%161=41%:%
%:%162=41%:%
%:%163=42%:%
%:%164=42%:%
%:%165=42%:%
%:%166=43%:%
%:%167=43%:%
%:%168=43%:%
%:%169=44%:%
%:%175=44%:%
%:%178=45%:%
%:%179=46%:%
%:%180=46%:%
%:%181=47%:%
%:%184=48%:%
%:%188=48%:%
%:%189=48%:%
%:%194=48%:%
%:%197=49%:%
%:%198=50%:%
%:%199=50%:%
%:%200=51%:%
%:%201=52%:%
%:%208=53%:%
%:%209=53%:%
%:%210=54%:%
%:%211=54%:%
%:%212=54%:%
%:%213=54%:%
%:%214=55%:%
%:%215=55%:%
%:%216=55%:%
%:%217=56%:%
%:%218=56%:%
%:%219=56%:%
%:%220=57%:%
%:%226=57%:%
%:%229=58%:%
%:%230=59%:%
%:%231=59%:%
%:%232=60%:%
%:%233=61%:%
%:%240=62%:%
%:%241=62%:%
%:%242=63%:%
%:%243=63%:%
%:%244=63%:%
%:%245=63%:%
%:%246=64%:%
%:%247=64%:%
%:%248=64%:%
%:%249=64%:%
%:%250=65%:%
%:%251=65%:%
%:%252=65%:%
%:%253=65%:%
%:%254=66%:%
%:%260=66%:%
%:%263=67%:%
%:%264=68%:%
%:%265=68%:%
%:%266=69%:%
%:%267=70%:%
%:%270=71%:%
%:%274=71%:%
%:%275=71%:%
%:%280=71%:%
%:%283=72%:%
%:%284=73%:%
%:%285=73%:%
%:%286=74%:%
%:%287=75%:%
%:%290=76%:%
%:%294=76%:%
%:%295=76%:%
%:%296=76%:%
%:%301=76%:%
%:%304=77%:%
%:%305=78%:%
%:%306=78%:%
%:%307=79%:%
%:%308=80%:%
%:%315=81%:%
%:%316=81%:%
%:%317=82%:%
%:%318=82%:%
%:%319=82%:%
%:%320=82%:%
%:%321=83%:%
%:%322=83%:%
%:%323=84%:%
%:%324=84%:%
%:%325=84%:%
%:%326=85%:%
%:%327=85%:%
%:%328=85%:%
%:%329=85%:%
%:%330=86%:%
%:%345=88%:%
%:%355=90%:%
%:%356=90%:%
%:%357=91%:%
%:%358=92%:%
%:%359=93%:%
%:%360=93%:%
%:%361=94%:%
%:%362=95%:%
%:%369=96%:%
%:%370=96%:%
%:%371=97%:%
%:%372=97%:%
%:%373=97%:%
%:%374=97%:%
%:%375=98%:%
%:%376=98%:%
%:%377=99%:%
%:%378=99%:%
%:%379=99%:%
%:%380=100%:%
%:%381=100%:%
%:%382=100%:%
%:%383=100%:%
%:%384=101%:%
%:%390=101%:%
%:%393=102%:%
%:%394=103%:%
%:%395=103%:%
%:%396=104%:%
%:%397=105%:%
%:%399=105%:%
%:%403=105%:%
%:%404=105%:%
%:%405=105%:%
%:%406=105%:%
%:%413=105%:%
%:%414=106%:%
%:%415=107%:%
%:%416=107%:%
%:%417=108%:%
%:%418=109%:%
%:%425=110%:%
%:%426=110%:%
%:%427=111%:%
%:%428=111%:%
%:%429=111%:%
%:%430=111%:%
%:%431=112%:%
%:%432=112%:%
%:%433=112%:%
%:%434=112%:%
%:%435=113%:%
%:%436=113%:%
%:%437=113%:%
%:%438=113%:%
%:%439=114%:%
%:%440=114%:%
%:%441=114%:%
%:%442=114%:%
%:%443=115%:%
%:%444=115%:%
%:%445=115%:%
%:%446=115%:%
%:%447=115%:%
%:%448=116%:%
%:%449=116%:%
%:%450=116%:%
%:%451=116%:%
%:%452=117%:%
%:%458=117%:%
%:%461=118%:%
%:%462=119%:%
%:%463=119%:%
%:%464=120%:%
%:%465=121%:%
%:%472=122%:%
%:%473=122%:%
%:%474=123%:%
%:%475=123%:%
%:%476=123%:%
%:%477=123%:%
%:%478=124%:%
%:%479=125%:%
%:%480=125%:%
%:%481=126%:%
%:%482=126%:%
%:%483=127%:%
%:%484=127%:%
%:%485=128%:%
%:%486=128%:%
%:%487=128%:%
%:%488=128%:%
%:%489=128%:%
%:%490=129%:%
%:%491=129%:%
%:%492=130%:%
%:%493=130%:%
%:%494=131%:%
%:%495=131%:%
%:%496=131%:%
%:%497=131%:%
%:%498=132%:%
%:%499=132%:%
%:%500=132%:%
%:%501=132%:%
%:%502=133%:%
%:%503=133%:%
%:%504=133%:%
%:%505=133%:%
%:%506=134%:%
%:%507=134%:%
%:%508=135%:%
%:%509=135%:%
%:%510=136%:%
%:%511=136%:%
%:%512=136%:%
%:%513=136%:%
%:%514=137%:%
%:%515=137%:%
%:%516=137%:%
%:%517=137%:%
%:%518=138%:%
%:%519=138%:%
%:%520=138%:%
%:%521=138%:%
%:%522=139%:%
%:%523=139%:%
%:%524=139%:%
%:%525=140%:%
%:%526=140%:%
%:%527=141%:%
%:%528=142%:%
%:%529=142%:%
%:%530=143%:%
%:%531=143%:%
%:%532=144%:%
%:%533=144%:%
%:%534=144%:%
%:%535=145%:%
%:%536=145%:%
%:%537=145%:%
%:%538=145%:%
%:%539=145%:%
%:%540=146%:%
%:%541=146%:%
%:%542=146%:%
%:%543=146%:%
%:%544=147%:%
%:%545=147%:%
%:%546=147%:%
%:%547=147%:%
%:%548=148%:%
%:%549=148%:%
%:%550=149%:%
%:%551=149%:%
%:%552=149%:%
%:%553=149%:%
%:%554=150%:%
%:%560=150%:%
%:%563=151%:%
%:%564=152%:%
%:%565=152%:%
%:%566=153%:%
%:%567=154%:%
%:%574=155%:%
%:%575=155%:%
%:%576=156%:%
%:%577=156%:%
%:%578=156%:%
%:%579=156%:%
%:%580=157%:%
%:%581=158%:%
%:%582=158%:%
%:%583=159%:%
%:%584=159%:%
%:%585=160%:%
%:%586=160%:%
%:%587=160%:%
%:%588=160%:%
%:%589=161%:%
%:%590=161%:%
%:%591=161%:%
%:%592=162%:%
%:%593=162%:%
%:%594=162%:%
%:%595=162%:%
%:%596=163%:%
%:%597=163%:%
%:%598=163%:%
%:%599=164%:%
%:%600=164%:%
%:%601=164%:%
%:%602=165%:%
%:%603=165%:%
%:%604=165%:%
%:%605=165%:%
%:%606=166%:%
%:%607=166%:%
%:%608=167%:%
%:%609=167%:%
%:%610=167%:%
%:%611=167%:%
%:%612=168%:%
%:%618=168%:%
%:%621=169%:%
%:%622=170%:%
%:%623=170%:%
%:%624=171%:%
%:%625=172%:%
%:%632=173%:%
%:%633=173%:%
%:%634=174%:%
%:%635=174%:%
%:%636=174%:%
%:%637=174%:%
%:%638=175%:%
%:%639=175%:%
%:%640=175%:%
%:%641=175%:%
%:%642=176%:%
%:%643=176%:%
%:%644=176%:%
%:%645=176%:%
%:%646=177%:%
%:%647=177%:%
%:%648=178%:%
%:%649=178%:%
%:%650=179%:%
%:%651=179%:%
%:%652=179%:%
%:%653=180%:%
%:%654=180%:%
%:%655=180%:%
%:%656=180%:%
%:%657=180%:%
%:%658=181%:%
%:%659=181%:%
%:%660=181%:%
%:%661=181%:%
%:%662=182%:%
%:%663=182%:%
%:%664=182%:%
%:%665=182%:%
%:%666=182%:%
%:%667=183%:%
%:%668=183%:%
%:%669=183%:%
%:%670=183%:%
%:%671=183%:%
%:%672=184%:%
%:%673=184%:%
%:%674=184%:%
%:%675=184%:%
%:%676=184%:%
%:%677=185%:%
%:%678=185%:%
%:%679=185%:%
%:%680=185%:%
%:%681=186%:%
%:%682=186%:%
%:%683=187%:%
%:%684=187%:%
%:%685=187%:%
%:%686=187%:%
%:%687=188%:%
%:%693=188%:%
%:%696=189%:%
%:%697=190%:%
%:%698=190%:%
%:%700=190%:%
%:%704=190%:%
%:%705=190%:%
%:%712=190%:%
%:%713=191%:%
%:%714=192%:%
%:%715=192%:%
%:%717=192%:%
%:%721=192%:%
%:%722=192%:%
%:%723=192%:%
%:%730=192%:%
%:%731=193%:%
%:%732=194%:%
%:%733=194%:%
%:%735=194%:%
%:%739=194%:%
%:%740=194%:%
%:%741=194%:%
%:%748=194%:%
%:%749=195%:%
%:%750=196%:%
%:%751=196%:%
%:%753=196%:%
%:%757=196%:%
%:%758=196%:%
%:%759=196%:%
%:%766=196%:%
%:%767=197%:%
%:%768=198%:%
%:%769=198%:%
%:%770=199%:%
%:%771=200%:%
%:%774=201%:%
%:%778=201%:%
%:%779=201%:%
%:%780=201%:%
%:%785=201%:%
%:%788=202%:%
%:%789=203%:%
%:%790=203%:%
%:%791=204%:%
%:%792=205%:%
%:%795=206%:%
%:%799=206%:%
%:%800=206%:%
%:%801=206%:%
%:%806=206%:%
%:%809=207%:%
%:%810=208%:%
%:%811=208%:%
%:%812=209%:%
%:%813=210%:%
%:%820=211%:%
%:%821=211%:%
%:%822=212%:%
%:%823=212%:%
%:%824=212%:%
%:%825=212%:%
%:%826=213%:%
%:%827=213%:%
%:%828=213%:%
%:%829=213%:%
%:%830=214%:%
%:%831=215%:%
%:%832=215%:%
%:%833=215%:%
%:%834=215%:%
%:%835=216%:%
%:%836=216%:%
%:%837=216%:%
%:%838=216%:%
%:%839=217%:%
%:%840=218%:%
%:%841=218%:%
%:%842=219%:%
%:%843=219%:%
%:%844=220%:%
%:%845=221%:%
%:%846=221%:%
%:%847=222%:%
%:%848=222%:%
%:%849=223%:%
%:%850=223%:%
%:%851=224%:%
%:%852=224%:%
%:%853=224%:%
%:%854=224%:%
%:%855=224%:%
%:%856=225%:%
%:%857=225%:%
%:%858=225%:%
%:%859=225%:%
%:%860=225%:%
%:%861=226%:%
%:%862=227%:%
%:%863=227%:%
%:%864=227%:%
%:%865=228%:%
%:%866=228%:%
%:%867=228%:%
%:%868=228%:%
%:%869=229%:%
%:%870=229%:%
%:%871=229%:%
%:%872=229%:%
%:%873=229%:%
%:%874=230%:%
%:%875=230%:%
%:%876=230%:%
%:%877=230%:%
%:%878=230%:%
%:%879=231%:%
%:%880=231%:%
%:%881=231%:%
%:%882=231%:%
%:%883=232%:%
%:%884=232%:%
%:%885=232%:%
%:%886=232%:%
%:%887=232%:%
%:%888=233%:%
%:%889=233%:%
%:%890=234%:%
%:%891=234%:%
%:%892=235%:%
%:%893=235%:%
%:%894=236%:%
%:%895=236%:%
%:%896=236%:%
%:%897=236%:%
%:%898=236%:%
%:%899=237%:%
%:%900=237%:%
%:%901=238%:%
%:%902=238%:%
%:%903=239%:%
%:%904=239%:%
%:%905=240%:%
%:%906=240%:%
%:%907=241%:%
%:%908=241%:%
%:%909=242%:%
%:%910=242%:%
%:%911=242%:%
%:%912=242%:%
%:%913=243%:%
%:%914=243%:%
%:%915=243%:%
%:%916=244%:%
%:%917=244%:%
%:%918=245%:%
%:%919=245%:%
%:%920=246%:%
%:%921=246%:%
%:%922=246%:%
%:%923=247%:%
%:%929=247%:%
%:%932=248%:%
%:%933=249%:%
%:%934=249%:%
%:%935=250%:%
%:%936=251%:%
%:%939=252%:%
%:%943=252%:%
%:%944=252%:%
%:%949=252%:%
%:%952=253%:%
%:%953=254%:%
%:%954=254%:%
%:%955=255%:%
%:%956=256%:%
%:%963=257%:%
%:%964=257%:%
%:%965=258%:%
%:%966=258%:%
%:%967=259%:%
%:%968=259%:%
%:%969=260%:%
%:%970=260%:%
%:%971=261%:%
%:%972=261%:%
%:%973=261%:%
%:%974=261%:%
%:%975=262%:%
%:%976=262%:%
%:%977=262%:%
%:%978=262%:%
%:%979=262%:%
%:%980=263%:%
%:%981=263%:%
%:%982=263%:%
%:%983=263%:%
%:%984=264%:%
%:%985=264%:%
%:%986=264%:%
%:%987=264%:%
%:%988=265%:%
%:%989=265%:%
%:%990=266%:%
%:%991=266%:%
%:%992=266%:%
%:%993=266%:%
%:%994=267%:%
%:%1000=267%:%
%:%1003=268%:%
%:%1004=269%:%
%:%1005=269%:%
%:%1006=270%:%
%:%1007=271%:%
%:%1008=272%:%
%:%1015=273%:%
%:%1016=273%:%
%:%1017=274%:%
%:%1018=274%:%
%:%1019=274%:%
%:%1020=274%:%
%:%1021=275%:%
%:%1022=275%:%
%:%1023=275%:%
%:%1024=275%:%
%:%1025=276%:%
%:%1026=276%:%
%:%1027=276%:%
%:%1028=277%:%
%:%1034=277%:%
%:%1037=278%:%
%:%1038=279%:%
%:%1039=279%:%
%:%1040=280%:%
%:%1041=281%:%
%:%1042=282%:%
%:%1049=283%:%
%:%1050=283%:%
%:%1051=284%:%
%:%1052=284%:%
%:%1053=285%:%
%:%1054=285%:%
%:%1055=285%:%
%:%1056=285%:%
%:%1057=286%:%
%:%1058=286%:%
%:%1059=287%:%
%:%1060=287%:%
%:%1061=287%:%
%:%1062=288%:%
%:%1063=288%:%
%:%1064=289%:%
%:%1065=289%:%
%:%1066=290%:%
%:%1067=290%:%
%:%1068=291%:%
%:%1069=291%:%
%:%1070=292%:%
%:%1071=292%:%
%:%1072=293%:%
%:%1073=293%:%
%:%1074=294%:%
%:%1075=294%:%
%:%1076=295%:%
%:%1077=295%:%
%:%1078=296%:%
%:%1079=296%:%
%:%1080=297%:%
%:%1081=297%:%
%:%1082=297%:%
%:%1083=297%:%
%:%1084=298%:%
%:%1085=298%:%
%:%1086=299%:%
%:%1087=299%:%
%:%1088=300%:%
%:%1089=300%:%
%:%1090=300%:%
%:%1091=300%:%
%:%1092=301%:%
%:%1093=301%:%
%:%1094=302%:%
%:%1095=302%:%
%:%1096=303%:%
%:%1097=303%:%
%:%1098=304%:%
%:%1099=304%:%
%:%1100=304%:%
%:%1101=304%:%
%:%1102=305%:%
%:%1103=305%:%
%:%1104=305%:%
%:%1105=305%:%
%:%1106=306%:%
%:%1107=306%:%
%:%1108=307%:%
%:%1109=307%:%
%:%1110=308%:%
%:%1111=308%:%
%:%1112=308%:%
%:%1113=308%:%
%:%1114=309%:%
%:%1115=309%:%
%:%1116=309%:%
%:%1117=309%:%
%:%1118=310%:%
%:%1119=310%:%
%:%1120=310%:%
%:%1121=310%:%
%:%1122=311%:%
%:%1123=311%:%
%:%1124=312%:%
%:%1125=312%:%
%:%1126=313%:%
%:%1127=313%:%
%:%1128=314%:%
%:%1129=314%:%
%:%1130=315%:%
%:%1131=315%:%
%:%1132=315%:%
%:%1133=315%:%
%:%1134=316%:%
%:%1135=316%:%
%:%1136=317%:%
%:%1137=317%:%
%:%1138=318%:%
%:%1139=318%:%
%:%1140=319%:%
%:%1141=319%:%
%:%1142=320%:%
%:%1143=320%:%
%:%1144=321%:%
%:%1145=321%:%
%:%1146=322%:%
%:%1147=322%:%
%:%1148=322%:%
%:%1149=322%:%
%:%1150=323%:%
%:%1151=323%:%
%:%1152=323%:%
%:%1153=323%:%
%:%1154=324%:%
%:%1155=324%:%
%:%1156=325%:%
%:%1157=325%:%
%:%1158=326%:%
%:%1159=326%:%
%:%1160=326%:%
%:%1161=326%:%
%:%1162=327%:%
%:%1163=327%:%
%:%1164=327%:%
%:%1165=327%:%
%:%1166=328%:%
%:%1167=328%:%
%:%1168=328%:%
%:%1169=328%:%
%:%1170=329%:%
%:%1171=329%:%
%:%1172=330%:%
%:%1173=330%:%
%:%1174=331%:%
%:%1175=331%:%
%:%1176=332%:%
%:%1177=332%:%
%:%1178=332%:%
%:%1179=332%:%
%:%1180=333%:%
%:%1181=333%:%
%:%1182=333%:%
%:%1183=333%:%
%:%1184=334%:%
%:%1185=334%:%
%:%1186=334%:%
%:%1187=334%:%
%:%1188=335%:%
%:%1189=335%:%
%:%1190=335%:%
%:%1191=335%:%
%:%1192=336%:%
%:%1193=336%:%
%:%1194=336%:%
%:%1195=336%:%
%:%1196=336%:%
%:%1197=337%:%
%:%1198=337%:%
%:%1199=338%:%
%:%1200=338%:%
%:%1201=339%:%
%:%1202=339%:%
%:%1203=340%:%
%:%1204=340%:%
%:%1205=340%:%
%:%1206=340%:%
%:%1207=341%:%
%:%1213=341%:%
%:%1216=342%:%
%:%1217=343%:%
%:%1218=343%:%
%:%1219=344%:%
%:%1220=345%:%
%:%1227=346%:%
%:%1228=346%:%
%:%1229=347%:%
%:%1230=347%:%
%:%1231=347%:%
%:%1232=347%:%
%:%1233=348%:%
%:%1234=348%:%
%:%1235=348%:%
%:%1236=349%:%
%:%1237=350%:%
%:%1238=350%:%
%:%1239=350%:%
%:%1240=350%:%
%:%1241=351%:%
%:%1242=352%:%
%:%1243=352%:%
%:%1244=352%:%
%:%1245=352%:%
%:%1246=353%:%
%:%1247=354%:%
%:%1248=354%:%
%:%1249=354%:%
%:%1250=354%:%
%:%1251=355%:%
%:%1252=355%:%
%:%1253=355%:%
%:%1254=356%:%
%:%1260=356%:%
%:%1263=357%:%
%:%1264=358%:%
%:%1265=358%:%
%:%1266=359%:%
%:%1267=360%:%
%:%1274=361%:%
%:%1275=361%:%
%:%1276=362%:%
%:%1277=362%:%
%:%1278=363%:%
%:%1279=363%:%
%:%1280=363%:%
%:%1281=363%:%
%:%1282=364%:%
%:%1283=364%:%
%:%1284=364%:%
%:%1285=364%:%
%:%1286=365%:%
%:%1287=365%:%
%:%1292=365%:%
%:%1295=366%:%
%:%1296=367%:%
%:%1297=367%:%
%:%1298=368%:%
%:%1299=369%:%
%:%1306=370%:%
%:%1307=370%:%
%:%1308=371%:%
%:%1309=371%:%
%:%1310=372%:%
%:%1311=372%:%
%:%1312=372%:%
%:%1313=372%:%
%:%1314=373%:%
%:%1315=373%:%
%:%1316=373%:%
%:%1317=373%:%
%:%1318=374%:%
%:%1319=375%:%
%:%1320=375%:%
%:%1321=375%:%
%:%1322=375%:%
%:%1323=376%:%
%:%1324=376%:%
%:%1325=376%:%
%:%1326=376%:%
%:%1327=377%:%
%:%1328=378%:%
%:%1329=378%:%
%:%1330=378%:%
%:%1331=379%:%
%:%1332=379%:%
%:%1333=379%:%
%:%1334=380%:%
%:%1335=381%:%
%:%1336=381%:%
%:%1337=382%:%
%:%1338=382%:%
%:%1339=383%:%
%:%1340=383%:%
%:%1341=384%:%
%:%1342=384%:%
%:%1343=384%:%
%:%1344=384%:%
%:%1345=385%:%
%:%1346=385%:%
%:%1347=386%:%
%:%1348=386%:%
%:%1349=387%:%
%:%1350=387%:%
%:%1351=387%:%
%:%1352=388%:%
%:%1353=388%:%
%:%1354=388%:%
%:%1355=388%:%
%:%1356=388%:%
%:%1357=389%:%
%:%1358=389%:%
%:%1359=389%:%
%:%1360=389%:%
%:%1361=389%:%
%:%1362=390%:%
%:%1363=390%:%
%:%1364=390%:%
%:%1365=390%:%
%:%1366=390%:%
%:%1367=391%:%
%:%1368=391%:%
%:%1369=391%:%
%:%1370=391%:%
%:%1371=392%:%
%:%1372=392%:%
%:%1373=392%:%
%:%1374=392%:%
%:%1375=393%:%
%:%1376=393%:%
%:%1377=394%:%
%:%1378=394%:%
%:%1379=395%:%
%:%1380=395%:%
%:%1381=395%:%
%:%1382=395%:%
%:%1383=396%:%
%:%1384=396%:%
%:%1385=397%:%
%:%1386=398%:%
%:%1387=398%:%
%:%1388=398%:%
%:%1389=398%:%
%:%1390=399%:%
%:%1391=399%:%
%:%1392=399%:%
%:%1393=399%:%
%:%1394=399%:%
%:%1395=400%:%
%:%1396=400%:%
%:%1397=400%:%
%:%1398=400%:%
%:%1399=400%:%
%:%1400=401%:%
%:%1401=401%:%
%:%1402=402%:%
%:%1403=402%:%
%:%1404=402%:%
%:%1405=402%:%
%:%1406=403%:%
%:%1421=405%:%
%:%1431=407%:%
%:%1432=407%:%
%:%1433=408%:%
%:%1434=409%:%
%:%1435=410%:%
%:%1442=411%:%
%:%1443=411%:%
%:%1444=412%:%
%:%1445=412%:%
%:%1446=413%:%
%:%1447=413%:%
%:%1448=414%:%
%:%1449=414%:%
%:%1450=414%:%
%:%1451=414%:%
%:%1452=415%:%
%:%1453=415%:%
%:%1454=415%:%
%:%1455=415%:%
%:%1456=416%:%
%:%1457=416%:%
%:%1458=416%:%
%:%1459=416%:%
%:%1460=417%:%
%:%1461=417%:%
%:%1462=417%:%
%:%1463=418%:%
%:%1464=418%:%
%:%1465=418%:%
%:%1466=419%:%
%:%1467=419%:%
%:%1468=419%:%
%:%1469=419%:%
%:%1470=420%:%
%:%1471=420%:%
%:%1472=421%:%
%:%1473=421%:%
%:%1474=422%:%
%:%1475=422%:%
%:%1476=423%:%
%:%1477=423%:%
%:%1478=424%:%
%:%1479=424%:%
%:%1480=424%:%
%:%1481=425%:%
%:%1482=425%:%
%:%1483=425%:%
%:%1484=425%:%
%:%1485=426%:%
%:%1486=426%:%
%:%1487=427%:%
%:%1488=427%:%
%:%1489=428%:%
%:%1490=428%:%
%:%1491=428%:%
%:%1492=429%:%
%:%1493=429%:%
%:%1494=429%:%
%:%1495=430%:%
%:%1496=430%:%
%:%1497=431%:%
%:%1498=431%:%
%:%1499=432%:%
%:%1500=432%:%
%:%1501=433%:%
%:%1502=433%:%
%:%1503=433%:%
%:%1504=434%:%
%:%1505=434%:%
%:%1506=434%:%
%:%1507=435%:%
%:%1508=435%:%
%:%1509=435%:%
%:%1510=436%:%
%:%1511=436%:%
%:%1512=436%:%
%:%1513=436%:%
%:%1514=436%:%
%:%1515=437%:%
%:%1516=437%:%
%:%1517=437%:%
%:%1518=437%:%
%:%1519=437%:%
%:%1520=438%:%
%:%1521=438%:%
%:%1522=438%:%
%:%1523=438%:%
%:%1524=439%:%
%:%1525=439%:%
%:%1526=439%:%
%:%1527=439%:%
%:%1528=440%:%
%:%1529=440%:%
%:%1530=441%:%
%:%1531=441%:%
%:%1532=442%:%
%:%1533=442%:%
%:%1534=442%:%
%:%1535=442%:%
%:%1536=443%:%
%:%1537=443%:%
%:%1538=443%:%
%:%1539=443%:%
%:%1540=444%:%
%:%1541=444%:%
%:%1542=445%:%
%:%1543=445%:%
%:%1544=446%:%
%:%1545=446%:%
%:%1546=446%:%
%:%1547=446%:%
%:%1548=446%:%
%:%1549=447%:%
%:%1550=447%:%
%:%1551=447%:%
%:%1552=447%:%
%:%1553=448%:%
%:%1559=448%:%
%:%1562=449%:%
%:%1563=450%:%
%:%1564=450%:%
%:%1565=451%:%
%:%1566=452%:%
%:%1567=453%:%
%:%1574=454%:%
%:%1575=454%:%
%:%1576=455%:%
%:%1577=455%:%
%:%1578=455%:%
%:%1579=455%:%
%:%1580=456%:%
%:%1581=456%:%
%:%1582=456%:%
%:%1583=456%:%
%:%1584=456%:%
%:%1585=456%:%
%:%1586=457%:%
%:%1587=457%:%
%:%1588=457%:%
%:%1589=457%:%
%:%1590=457%:%
%:%1591=458%:%
%:%1592=458%:%
%:%1593=458%:%
%:%1594=458%:%
%:%1595=459%:%
%:%1601=459%:%
%:%1606=460%:%
%:%1611=461%:%

%
\begin{isabellebody}%
\setisabellecontext{Eudoxus}%
%
\isadelimtheory
\isanewline
\isanewline
%
\endisadelimtheory
%
\isatagtheory
\isacommand{theory}\isamarkupfalse%
\ Eudoxus\isanewline
\ \ \isakeyword{imports}\ Slope\isanewline
\isakeyword{begin}%
\endisatagtheory
{\isafoldtheory}%
%
\isadelimtheory
%
\endisadelimtheory
%
\isadelimdocument
%
\endisadelimdocument
%
\isatagdocument
%
\isamarkupsection{Eudoxus Reals%
}
\isamarkuptrue%
%
\isamarkupsubsection{Type Definition%
}
\isamarkuptrue%
%
\endisatagdocument
{\isafolddocument}%
%
\isadelimdocument
%
\endisadelimdocument
%
\begin{isamarkuptext}%
Two slopes are said to be equivalent if their difference is bounded.%
\end{isamarkuptext}\isamarkuptrue%
\isacommand{definition}\isamarkupfalse%
\ eudoxus{\isacharunderscore}{\kern0pt}rel\ {\isacharcolon}{\kern0pt}{\isacharcolon}{\kern0pt}\ {\isachardoublequoteopen}{\isacharparenleft}{\kern0pt}int\ {\isasymRightarrow}\ int{\isacharparenright}{\kern0pt}\ {\isasymRightarrow}\ {\isacharparenleft}{\kern0pt}int\ {\isasymRightarrow}\ int{\isacharparenright}{\kern0pt}\ {\isasymRightarrow}\ bool{\isachardoublequoteclose}\ {\isacharparenleft}{\kern0pt}\isakeyword{infix}\ {\isachardoublequoteopen}{\isasymsim}\isactrlsub e{\isachardoublequoteclose}\ {\isadigit{5}}{\isadigit{0}}{\isacharparenright}{\kern0pt}\ \isakeyword{where}\ \isanewline
\ \ {\isachardoublequoteopen}f\ {\isasymsim}\isactrlsub e\ g\ {\isasymequiv}\ slope\ f\ {\isasymand}\ slope\ g\ {\isasymand}\ bounded\ {\isacharparenleft}{\kern0pt}{\isasymlambda}n{\isachardot}{\kern0pt}\ f\ n\ {\isacharminus}{\kern0pt}\ g\ n{\isacharparenright}{\kern0pt}{\isachardoublequoteclose}\isanewline
\isanewline
\isacommand{lemma}\isamarkupfalse%
\ eudoxus{\isacharunderscore}{\kern0pt}rel{\isacharunderscore}{\kern0pt}equivp{\isacharcolon}{\kern0pt}\isanewline
\ \ {\isachardoublequoteopen}part{\isacharunderscore}{\kern0pt}equivp\ eudoxus{\isacharunderscore}{\kern0pt}rel{\isachardoublequoteclose}\isanewline
%
\isadelimproof
%
\endisadelimproof
%
\isatagproof
\isacommand{proof}\isamarkupfalse%
\ {\isacharparenleft}{\kern0pt}auto\ intro{\isacharbang}{\kern0pt}{\isacharcolon}{\kern0pt}\ part{\isacharunderscore}{\kern0pt}equivpI{\isacharparenright}{\kern0pt}\isanewline
\ \ \isacommand{show}\isamarkupfalse%
\ {\isachardoublequoteopen}{\isasymexists}x{\isachardot}{\kern0pt}\ x\ {\isasymsim}\isactrlsub e\ x{\isachardoublequoteclose}\ \isacommand{unfolding}\isamarkupfalse%
\ eudoxus{\isacharunderscore}{\kern0pt}rel{\isacharunderscore}{\kern0pt}def\ slope{\isacharunderscore}{\kern0pt}def\ bounded{\isacharunderscore}{\kern0pt}def\ \isacommand{by}\isamarkupfalse%
\ fast\isanewline
\ \ \isacommand{show}\isamarkupfalse%
\ {\isachardoublequoteopen}symp\ {\isacharparenleft}{\kern0pt}{\isasymsim}\isactrlsub e{\isacharparenright}{\kern0pt}{\isachardoublequoteclose}\ \ \isacommand{unfolding}\isamarkupfalse%
\ eudoxus{\isacharunderscore}{\kern0pt}rel{\isacharunderscore}{\kern0pt}def\ \isacommand{by}\isamarkupfalse%
\ {\isacharparenleft}{\kern0pt}force\ intro{\isacharcolon}{\kern0pt}\ sympI\ dest{\isacharcolon}{\kern0pt}\ bounded{\isacharunderscore}{\kern0pt}uminus\ simp{\isacharcolon}{\kern0pt}\ fun{\isacharunderscore}{\kern0pt}Compl{\isacharunderscore}{\kern0pt}def{\isacharparenright}{\kern0pt}\ \isanewline
\ \ \isacommand{show}\isamarkupfalse%
\ {\isachardoublequoteopen}transp\ {\isacharparenleft}{\kern0pt}{\isasymsim}\isactrlsub e{\isacharparenright}{\kern0pt}{\isachardoublequoteclose}\ \ \isacommand{unfolding}\isamarkupfalse%
\ eudoxus{\isacharunderscore}{\kern0pt}rel{\isacharunderscore}{\kern0pt}def\ \isacommand{by}\isamarkupfalse%
\ {\isacharparenleft}{\kern0pt}force\ intro{\isacharbang}{\kern0pt}{\isacharcolon}{\kern0pt}\ transpI\ dest{\isacharcolon}{\kern0pt}\ bounded{\isacharunderscore}{\kern0pt}add{\isacharparenright}{\kern0pt}\isanewline
\isacommand{qed}\isamarkupfalse%
%
\endisatagproof
{\isafoldproof}%
%
\isadelimproof
%
\endisadelimproof
%
\begin{isamarkuptext}%
We define the reals as the set of all equivalence classes of the relation \isa{{\isacharparenleft}{\kern0pt}{\isasymsim}\isactrlsub e{\isacharparenright}{\kern0pt}}.%
\end{isamarkuptext}\isamarkuptrue%
\isacommand{quotient{\isacharunderscore}{\kern0pt}type}\isamarkupfalse%
\ real\ {\isacharequal}{\kern0pt}\ {\isachardoublequoteopen}{\isacharparenleft}{\kern0pt}int\ {\isasymRightarrow}\ int{\isacharparenright}{\kern0pt}{\isachardoublequoteclose}\ {\isacharslash}{\kern0pt}\ partial{\isacharcolon}{\kern0pt}\ eudoxus{\isacharunderscore}{\kern0pt}rel\isanewline
%
\isadelimproof
\ \ %
\endisadelimproof
%
\isatagproof
\isacommand{by}\isamarkupfalse%
\ {\isacharparenleft}{\kern0pt}rule\ eudoxus{\isacharunderscore}{\kern0pt}rel{\isacharunderscore}{\kern0pt}equivp{\isacharparenright}{\kern0pt}%
\endisatagproof
{\isafoldproof}%
%
\isadelimproof
\isanewline
%
\endisadelimproof
\isanewline
\isacommand{lemma}\isamarkupfalse%
\ real{\isacharunderscore}{\kern0pt}quot{\isacharunderscore}{\kern0pt}type{\isacharcolon}{\kern0pt}\ {\isachardoublequoteopen}quot{\isacharunderscore}{\kern0pt}type\ {\isacharparenleft}{\kern0pt}{\isasymsim}\isactrlsub e{\isacharparenright}{\kern0pt}\ Abs{\isacharunderscore}{\kern0pt}real\ Rep{\isacharunderscore}{\kern0pt}real{\isachardoublequoteclose}\isanewline
%
\isadelimproof
\ \ %
\endisadelimproof
%
\isatagproof
\isacommand{using}\isamarkupfalse%
\ Rep{\isacharunderscore}{\kern0pt}real\ Abs{\isacharunderscore}{\kern0pt}real{\isacharunderscore}{\kern0pt}inverse\ Rep{\isacharunderscore}{\kern0pt}real{\isacharunderscore}{\kern0pt}inverse\ Rep{\isacharunderscore}{\kern0pt}real{\isacharunderscore}{\kern0pt}inject\ eudoxus{\isacharunderscore}{\kern0pt}rel{\isacharunderscore}{\kern0pt}equivp\ \isacommand{by}\isamarkupfalse%
\ {\isacharparenleft}{\kern0pt}auto\ intro{\isacharbang}{\kern0pt}{\isacharcolon}{\kern0pt}\ quot{\isacharunderscore}{\kern0pt}type{\isachardot}{\kern0pt}intro{\isacharparenright}{\kern0pt}%
\endisatagproof
{\isafoldproof}%
%
\isadelimproof
\isanewline
%
\endisadelimproof
\isanewline
\isacommand{lemma}\isamarkupfalse%
\ slope{\isacharunderscore}{\kern0pt}refl{\isacharcolon}{\kern0pt}\ {\isachardoublequoteopen}slope\ f\ {\isacharequal}{\kern0pt}\ {\isacharparenleft}{\kern0pt}f\ {\isasymsim}\isactrlsub e\ f{\isacharparenright}{\kern0pt}{\isachardoublequoteclose}\isanewline
%
\isadelimproof
\ \ %
\endisadelimproof
%
\isatagproof
\isacommand{unfolding}\isamarkupfalse%
\ eudoxus{\isacharunderscore}{\kern0pt}rel{\isacharunderscore}{\kern0pt}def\ \isacommand{by}\isamarkupfalse%
\ {\isacharparenleft}{\kern0pt}fastforce\ simp\ add{\isacharcolon}{\kern0pt}\ bounded{\isacharunderscore}{\kern0pt}constant{\isacharparenright}{\kern0pt}%
\endisatagproof
{\isafoldproof}%
%
\isadelimproof
\isanewline
%
\endisadelimproof
\isanewline
\isacommand{declare}\isamarkupfalse%
\ slope{\isacharunderscore}{\kern0pt}refl{\isacharbrackleft}{\kern0pt}THEN\ iffD{\isadigit{2}}{\isacharcomma}{\kern0pt}\ simp{\isacharbrackright}{\kern0pt}\isanewline
\isanewline
\isacommand{lemmas}\isamarkupfalse%
\ slope{\isacharunderscore}{\kern0pt}reflI\ {\isacharequal}{\kern0pt}\ slope{\isacharunderscore}{\kern0pt}refl{\isacharbrackleft}{\kern0pt}THEN\ iffD{\isadigit{1}}{\isacharbrackright}{\kern0pt}\isanewline
\isanewline
\isacommand{lemma}\isamarkupfalse%
\ slope{\isacharunderscore}{\kern0pt}induct{\isacharbrackleft}{\kern0pt}consumes\ {\isadigit{0}}{\isacharcomma}{\kern0pt}\ case{\isacharunderscore}{\kern0pt}names\ slope{\isacharbrackright}{\kern0pt}{\isacharcolon}{\kern0pt}\ \isanewline
\ \ \isakeyword{assumes}\ {\isachardoublequoteopen}{\isasymAnd}f{\isachardot}{\kern0pt}\ slope\ f\ {\isasymLongrightarrow}\ P\ {\isacharparenleft}{\kern0pt}abs{\isacharunderscore}{\kern0pt}real\ f{\isacharparenright}{\kern0pt}{\isachardoublequoteclose}\isanewline
\ \ \isakeyword{shows}\ {\isachardoublequoteopen}P\ x{\isachardoublequoteclose}\isanewline
%
\isadelimproof
\ \ %
\endisadelimproof
%
\isatagproof
\isacommand{using}\isamarkupfalse%
\ assms\ \isacommand{by}\isamarkupfalse%
\ induct\ force%
\endisatagproof
{\isafoldproof}%
%
\isadelimproof
\isanewline
%
\endisadelimproof
\isanewline
\isacommand{lemma}\isamarkupfalse%
\ abs{\isacharunderscore}{\kern0pt}real{\isacharunderscore}{\kern0pt}eq{\isacharunderscore}{\kern0pt}iff{\isacharcolon}{\kern0pt}\ {\isachardoublequoteopen}f\ {\isasymsim}\isactrlsub e\ g\ {\isasymlongleftrightarrow}\ slope\ f\ {\isasymand}\ slope\ g\ {\isasymand}\ abs{\isacharunderscore}{\kern0pt}real\ f\ {\isacharequal}{\kern0pt}\ abs{\isacharunderscore}{\kern0pt}real\ g{\isachardoublequoteclose}\ \isanewline
%
\isadelimproof
\ \ %
\endisadelimproof
%
\isatagproof
\isacommand{by}\isamarkupfalse%
\ {\isacharparenleft}{\kern0pt}metis\ Quotient{\isacharunderscore}{\kern0pt}real\ Quotient{\isacharunderscore}{\kern0pt}rel\ slope{\isacharunderscore}{\kern0pt}refl{\isacharparenright}{\kern0pt}%
\endisatagproof
{\isafoldproof}%
%
\isadelimproof
\isanewline
%
\endisadelimproof
\isanewline
\isacommand{lemma}\isamarkupfalse%
\ abs{\isacharunderscore}{\kern0pt}real{\isacharunderscore}{\kern0pt}eqI{\isacharbrackleft}{\kern0pt}intro{\isacharbrackright}{\kern0pt}{\isacharcolon}{\kern0pt}\ {\isachardoublequoteopen}f\ {\isasymsim}\isactrlsub e\ g\ {\isasymLongrightarrow}\ abs{\isacharunderscore}{\kern0pt}real\ f\ {\isacharequal}{\kern0pt}\ abs{\isacharunderscore}{\kern0pt}real\ g{\isachardoublequoteclose}%
\isadelimproof
\ %
\endisadelimproof
%
\isatagproof
\isacommand{using}\isamarkupfalse%
\ abs{\isacharunderscore}{\kern0pt}real{\isacharunderscore}{\kern0pt}eq{\isacharunderscore}{\kern0pt}iff\ \isacommand{by}\isamarkupfalse%
\ blast%
\endisatagproof
{\isafoldproof}%
%
\isadelimproof
%
\endisadelimproof
\isanewline
\isanewline
\isacommand{lemmas}\isamarkupfalse%
\ eudoxus{\isacharunderscore}{\kern0pt}rel{\isacharunderscore}{\kern0pt}sym{\isacharbrackleft}{\kern0pt}sym{\isacharbrackright}{\kern0pt}\ {\isacharequal}{\kern0pt}\ Quotient{\isacharunderscore}{\kern0pt}symp{\isacharbrackleft}{\kern0pt}OF\ Quotient{\isacharunderscore}{\kern0pt}real{\isacharcomma}{\kern0pt}\ THEN\ sympD{\isacharbrackright}{\kern0pt}\isanewline
\isacommand{lemmas}\isamarkupfalse%
\ eudoxus{\isacharunderscore}{\kern0pt}rel{\isacharunderscore}{\kern0pt}trans{\isacharbrackleft}{\kern0pt}trans{\isacharbrackright}{\kern0pt}\ {\isacharequal}{\kern0pt}\ Quotient{\isacharunderscore}{\kern0pt}transp{\isacharbrackleft}{\kern0pt}OF\ Quotient{\isacharunderscore}{\kern0pt}real{\isacharcomma}{\kern0pt}\ THEN\ transpD{\isacharbrackright}{\kern0pt}\isanewline
\isanewline
\isacommand{lemmas}\isamarkupfalse%
\ rep{\isacharunderscore}{\kern0pt}real{\isacharunderscore}{\kern0pt}abs{\isacharunderscore}{\kern0pt}real{\isacharunderscore}{\kern0pt}refl\ {\isacharequal}{\kern0pt}\ Quotient{\isacharunderscore}{\kern0pt}rep{\isacharunderscore}{\kern0pt}abs{\isacharbrackleft}{\kern0pt}OF\ Quotient{\isacharunderscore}{\kern0pt}real{\isacharcomma}{\kern0pt}\ OF\ slope{\isacharunderscore}{\kern0pt}refl{\isacharbrackleft}{\kern0pt}THEN\ iffD{\isadigit{1}}{\isacharbrackright}{\kern0pt}{\isacharcomma}{\kern0pt}\ intro{\isacharbang}{\kern0pt}{\isacharbrackright}{\kern0pt}\isanewline
\isacommand{lemmas}\isamarkupfalse%
\ rep{\isacharunderscore}{\kern0pt}real{\isacharunderscore}{\kern0pt}iff\ {\isacharequal}{\kern0pt}\ Quotient{\isacharunderscore}{\kern0pt}rel{\isacharunderscore}{\kern0pt}rep{\isacharbrackleft}{\kern0pt}OF\ Quotient{\isacharunderscore}{\kern0pt}real{\isacharcomma}{\kern0pt}\ iff{\isacharbrackright}{\kern0pt}\isanewline
\isanewline
\isacommand{declare}\isamarkupfalse%
\ Quotient{\isacharunderscore}{\kern0pt}abs{\isacharunderscore}{\kern0pt}rep{\isacharbrackleft}{\kern0pt}OF\ Quotient{\isacharunderscore}{\kern0pt}real{\isacharcomma}{\kern0pt}\ simp{\isacharbrackright}{\kern0pt}\isanewline
\isanewline
\isacommand{lemma}\isamarkupfalse%
\ slope{\isacharunderscore}{\kern0pt}rep{\isacharunderscore}{\kern0pt}real{\isacharcolon}{\kern0pt}\ {\isachardoublequoteopen}slope\ {\isacharparenleft}{\kern0pt}rep{\isacharunderscore}{\kern0pt}real\ x{\isacharparenright}{\kern0pt}{\isachardoublequoteclose}%
\isadelimproof
\ %
\endisadelimproof
%
\isatagproof
\isacommand{by}\isamarkupfalse%
\ simp%
\endisatagproof
{\isafoldproof}%
%
\isadelimproof
%
\endisadelimproof
\isanewline
\isanewline
\isacommand{lemma}\isamarkupfalse%
\ eudoxus{\isacharunderscore}{\kern0pt}relI{\isacharcolon}{\kern0pt}\isanewline
\ \ \isakeyword{assumes}\ {\isachardoublequoteopen}slope\ f{\isachardoublequoteclose}\ {\isachardoublequoteopen}slope\ g{\isachardoublequoteclose}\ {\isachardoublequoteopen}{\isasymAnd}n{\isachardot}{\kern0pt}\ n\ {\isasymge}\ N\ {\isasymLongrightarrow}\ {\isasymbar}f\ n\ {\isacharminus}{\kern0pt}\ g\ n{\isasymbar}\ {\isasymle}\ C{\isachardoublequoteclose}\isanewline
\ \ \isakeyword{shows}\ {\isachardoublequoteopen}f\ {\isasymsim}\isactrlsub e\ g{\isachardoublequoteclose}\isanewline
%
\isadelimproof
%
\endisadelimproof
%
\isatagproof
\isacommand{proof}\isamarkupfalse%
\ {\isacharminus}{\kern0pt}\isanewline
\ \ \isacommand{have}\isamarkupfalse%
\ C{\isacharunderscore}{\kern0pt}nonneg{\isacharcolon}{\kern0pt}\ {\isachardoublequoteopen}C\ {\isasymge}\ {\isadigit{0}}{\isachardoublequoteclose}\ \isacommand{using}\isamarkupfalse%
\ assms\ \isacommand{by}\isamarkupfalse%
\ force\isanewline
\isanewline
\ \ \isacommand{obtain}\isamarkupfalse%
\ C{\isacharunderscore}{\kern0pt}f\ \isakeyword{where}\ C{\isacharunderscore}{\kern0pt}f{\isacharcolon}{\kern0pt}\ {\isachardoublequoteopen}{\isasymbar}f\ {\isacharparenleft}{\kern0pt}n\ {\isacharplus}{\kern0pt}\ {\isacharparenleft}{\kern0pt}{\isacharminus}{\kern0pt}\ n{\isacharparenright}{\kern0pt}{\isacharparenright}{\kern0pt}\ {\isacharminus}{\kern0pt}\ {\isacharparenleft}{\kern0pt}f\ n\ {\isacharplus}{\kern0pt}\ f\ {\isacharparenleft}{\kern0pt}{\isacharminus}{\kern0pt}\ n{\isacharparenright}{\kern0pt}{\isacharparenright}{\kern0pt}{\isasymbar}\ {\isasymle}\ C{\isacharunderscore}{\kern0pt}f{\isachardoublequoteclose}\ {\isachardoublequoteopen}{\isadigit{0}}\ {\isasymle}\ C{\isacharunderscore}{\kern0pt}f{\isachardoublequoteclose}\ \isakeyword{for}\ n\ \isacommand{using}\isamarkupfalse%
\ assms\ \isacommand{by}\isamarkupfalse%
\ fast\isanewline
\isanewline
\ \ \isacommand{obtain}\isamarkupfalse%
\ C{\isacharunderscore}{\kern0pt}g\ \isakeyword{where}\ C{\isacharunderscore}{\kern0pt}g{\isacharcolon}{\kern0pt}\ {\isachardoublequoteopen}{\isasymbar}g\ {\isacharparenleft}{\kern0pt}n\ {\isacharplus}{\kern0pt}\ {\isacharparenleft}{\kern0pt}{\isacharminus}{\kern0pt}\ n{\isacharparenright}{\kern0pt}{\isacharparenright}{\kern0pt}\ {\isacharminus}{\kern0pt}\ {\isacharparenleft}{\kern0pt}g\ n\ {\isacharplus}{\kern0pt}\ g\ {\isacharparenleft}{\kern0pt}{\isacharminus}{\kern0pt}\ n{\isacharparenright}{\kern0pt}{\isacharparenright}{\kern0pt}{\isasymbar}\ {\isasymle}\ C{\isacharunderscore}{\kern0pt}g{\isachardoublequoteclose}\ {\isachardoublequoteopen}{\isadigit{0}}\ {\isasymle}\ C{\isacharunderscore}{\kern0pt}g{\isachardoublequoteclose}\ \isakeyword{for}\ n\ \isacommand{using}\isamarkupfalse%
\ assms\ \isacommand{by}\isamarkupfalse%
\ fast\isanewline
\ \ \isanewline
\ \ \isacommand{have}\isamarkupfalse%
\ bound{\isacharcolon}{\kern0pt}\ {\isachardoublequoteopen}{\isasymbar}f\ {\isacharparenleft}{\kern0pt}{\isacharminus}{\kern0pt}\ n{\isacharparenright}{\kern0pt}\ {\isacharminus}{\kern0pt}\ g\ {\isacharparenleft}{\kern0pt}{\isacharminus}{\kern0pt}\ n{\isacharparenright}{\kern0pt}{\isasymbar}\ {\isasymle}\ {\isasymbar}f\ n\ {\isacharminus}{\kern0pt}\ g\ n{\isasymbar}\ {\isacharplus}{\kern0pt}\ {\isasymbar}f\ {\isadigit{0}}{\isasymbar}\ {\isacharplus}{\kern0pt}\ {\isasymbar}g\ {\isadigit{0}}{\isasymbar}\ {\isacharplus}{\kern0pt}\ C{\isacharunderscore}{\kern0pt}f\ {\isacharplus}{\kern0pt}\ C{\isacharunderscore}{\kern0pt}g{\isachardoublequoteclose}\ \isakeyword{for}\ n\ \isacommand{using}\isamarkupfalse%
\ C{\isacharunderscore}{\kern0pt}f{\isacharparenleft}{\kern0pt}{\isadigit{1}}{\isacharparenright}{\kern0pt}{\isacharbrackleft}{\kern0pt}of\ n{\isacharbrackright}{\kern0pt}\ C{\isacharunderscore}{\kern0pt}g{\isacharparenleft}{\kern0pt}{\isadigit{1}}{\isacharparenright}{\kern0pt}{\isacharbrackleft}{\kern0pt}of\ n{\isacharbrackright}{\kern0pt}\ \isacommand{by}\isamarkupfalse%
\ simp\isanewline
\isanewline
\ \ \isacommand{define}\isamarkupfalse%
\ C{\isacharprime}{\kern0pt}\ \isakeyword{where}\ {\isachardoublequoteopen}C{\isacharprime}{\kern0pt}\ {\isacharequal}{\kern0pt}\ Sup\ {\isacharbraceleft}{\kern0pt}{\isasymbar}f\ n\ {\isacharminus}{\kern0pt}\ g\ n{\isasymbar}\ {\isacharbar}{\kern0pt}\ n{\isachardot}{\kern0pt}\ n\ {\isasymin}\ {\isacharbraceleft}{\kern0pt}{\isadigit{0}}{\isachardot}{\kern0pt}{\isachardot}{\kern0pt}max\ {\isadigit{0}}\ N{\isacharbraceright}{\kern0pt}{\isacharbraceright}{\kern0pt}\ {\isacharplus}{\kern0pt}\ C\ {\isacharplus}{\kern0pt}\ {\isasymbar}f\ {\isadigit{0}}{\isasymbar}\ {\isacharplus}{\kern0pt}\ {\isasymbar}g\ {\isadigit{0}}{\isasymbar}\ {\isacharplus}{\kern0pt}\ C{\isacharunderscore}{\kern0pt}f\ {\isacharplus}{\kern0pt}\ C{\isacharunderscore}{\kern0pt}g{\isachardoublequoteclose}\isanewline
\ \ \isacommand{have}\isamarkupfalse%
\ {\isacharasterisk}{\kern0pt}{\isacharcolon}{\kern0pt}\ {\isachardoublequoteopen}bdd{\isacharunderscore}{\kern0pt}above\ {\isacharbraceleft}{\kern0pt}{\isasymbar}f\ n\ {\isacharminus}{\kern0pt}\ g\ n{\isasymbar}\ {\isacharbar}{\kern0pt}n{\isachardot}{\kern0pt}\ n\ {\isasymin}\ {\isacharbraceleft}{\kern0pt}{\isadigit{0}}{\isachardot}{\kern0pt}{\isachardot}{\kern0pt}max\ {\isadigit{0}}\ N{\isacharbraceright}{\kern0pt}{\isacharbraceright}{\kern0pt}{\isachardoublequoteclose}\ \isacommand{by}\isamarkupfalse%
\ simp\isanewline
\ \ \isacommand{have}\isamarkupfalse%
\ {\isachardoublequoteopen}Sup\ {\isacharbraceleft}{\kern0pt}{\isasymbar}f\ n\ {\isacharminus}{\kern0pt}\ g\ n{\isasymbar}\ {\isacharbar}{\kern0pt}n{\isachardot}{\kern0pt}\ n\ {\isasymin}\ {\isacharbraceleft}{\kern0pt}{\isadigit{0}}{\isachardot}{\kern0pt}{\isachardot}{\kern0pt}max\ {\isadigit{0}}\ N{\isacharbraceright}{\kern0pt}{\isacharbraceright}{\kern0pt}\ {\isasymin}\ {\isacharbraceleft}{\kern0pt}{\isasymbar}f\ n\ {\isacharminus}{\kern0pt}\ g\ n{\isasymbar}\ {\isacharbar}{\kern0pt}n{\isachardot}{\kern0pt}\ n\ {\isasymin}\ {\isacharbraceleft}{\kern0pt}{\isadigit{0}}{\isachardot}{\kern0pt}{\isachardot}{\kern0pt}max\ {\isadigit{0}}\ N{\isacharbraceright}{\kern0pt}{\isacharbraceright}{\kern0pt}{\isachardoublequoteclose}\ \isacommand{using}\isamarkupfalse%
\ C{\isacharunderscore}{\kern0pt}nonneg\ \isacommand{by}\isamarkupfalse%
\ {\isacharparenleft}{\kern0pt}intro\ int{\isacharunderscore}{\kern0pt}Sup{\isacharunderscore}{\kern0pt}mem{\isacharbrackleft}{\kern0pt}OF\ {\isacharunderscore}{\kern0pt}\ {\isacharasterisk}{\kern0pt}{\isacharbrackright}{\kern0pt}{\isacharparenright}{\kern0pt}\ auto\isanewline
\ \ \isacommand{hence}\isamarkupfalse%
\ Sup{\isacharunderscore}{\kern0pt}nonneg{\isacharcolon}{\kern0pt}\ {\isachardoublequoteopen}Sup\ {\isacharbraceleft}{\kern0pt}{\isasymbar}f\ n\ {\isacharminus}{\kern0pt}\ g\ n{\isasymbar}\ {\isacharbar}{\kern0pt}\ n{\isachardot}{\kern0pt}\ n\ {\isasymin}\ {\isacharbraceleft}{\kern0pt}{\isadigit{0}}{\isachardot}{\kern0pt}{\isachardot}{\kern0pt}max\ {\isadigit{0}}\ N{\isacharbraceright}{\kern0pt}{\isacharbraceright}{\kern0pt}\ {\isasymge}\ {\isadigit{0}}{\isachardoublequoteclose}\ \isacommand{by}\isamarkupfalse%
\ fastforce\isanewline
\isanewline
\ \ \isacommand{have}\isamarkupfalse%
\ {\isacharasterisk}{\kern0pt}{\isacharcolon}{\kern0pt}\ {\isachardoublequoteopen}{\isasymbar}f\ n\ {\isacharminus}{\kern0pt}\ g\ n{\isasymbar}\ {\isasymle}\ Sup\ {\isacharbraceleft}{\kern0pt}{\isasymbar}f\ n\ {\isacharminus}{\kern0pt}\ g\ n{\isasymbar}\ {\isacharbar}{\kern0pt}\ n{\isachardot}{\kern0pt}\ n\ {\isasymin}\ {\isacharbraceleft}{\kern0pt}{\isadigit{0}}{\isachardot}{\kern0pt}{\isachardot}{\kern0pt}max\ {\isadigit{0}}\ N{\isacharbraceright}{\kern0pt}{\isacharbraceright}{\kern0pt}\ {\isacharplus}{\kern0pt}\ C{\isachardoublequoteclose}\ \isakeyword{if}\ {\isachardoublequoteopen}n\ {\isasymge}\ {\isadigit{0}}{\isachardoublequoteclose}\ \isakeyword{for}\ n\ \isacommand{unfolding}\isamarkupfalse%
\ C{\isacharprime}{\kern0pt}{\isacharunderscore}{\kern0pt}def\ \isacommand{using}\isamarkupfalse%
\ cSup{\isacharunderscore}{\kern0pt}upper{\isacharbrackleft}{\kern0pt}OF\ {\isacharunderscore}{\kern0pt}\ {\isacharasterisk}{\kern0pt}{\isacharbrackright}{\kern0pt}\ that\ C{\isacharunderscore}{\kern0pt}nonneg\ Sup{\isacharunderscore}{\kern0pt}nonneg\ \isacommand{by}\isamarkupfalse%
\ {\isacharparenleft}{\kern0pt}cases\ {\isachardoublequoteopen}n\ {\isasymle}\ N{\isachardoublequoteclose}{\isacharparenright}{\kern0pt}\ {\isacharparenleft}{\kern0pt}fastforce\ simp\ add{\isacharcolon}{\kern0pt}\ add{\isachardot}{\kern0pt}commute\ add{\isacharunderscore}{\kern0pt}increasing{\isadigit{2}}\ assms{\isacharparenleft}{\kern0pt}{\isadigit{3}}{\isacharparenright}{\kern0pt}{\isacharparenright}{\kern0pt}{\isacharplus}{\kern0pt}\isanewline
\ \ \isacommand{{\isacharbraceleft}{\kern0pt}}\isamarkupfalse%
\isanewline
\ \ \ \ \isacommand{fix}\isamarkupfalse%
\ n\isanewline
\ \ \ \ \isacommand{have}\isamarkupfalse%
\ {\isachardoublequoteopen}{\isasymbar}f\ n\ {\isacharminus}{\kern0pt}\ g\ n{\isasymbar}\ {\isasymle}\ C{\isacharprime}{\kern0pt}{\isachardoublequoteclose}\isanewline
\ \ \ \ \isacommand{proof}\isamarkupfalse%
\ {\isacharparenleft}{\kern0pt}cases\ {\isachardoublequoteopen}n\ {\isasymge}\ {\isadigit{0}}{\isachardoublequoteclose}{\isacharparenright}{\kern0pt}\isanewline
\ \ \ \ \ \ \isacommand{case}\isamarkupfalse%
\ True\isanewline
\ \ \ \ \ \ \isacommand{thus}\isamarkupfalse%
\ {\isacharquery}{\kern0pt}thesis\ \isacommand{unfolding}\isamarkupfalse%
\ C{\isacharprime}{\kern0pt}{\isacharunderscore}{\kern0pt}def\ \isacommand{using}\isamarkupfalse%
\ {\isacharasterisk}{\kern0pt}\ C{\isacharunderscore}{\kern0pt}f\ C{\isacharunderscore}{\kern0pt}g\ \isacommand{by}\isamarkupfalse%
\ fastforce\isanewline
\ \ \ \ \isacommand{next}\isamarkupfalse%
\isanewline
\ \ \ \ \ \ \isacommand{case}\isamarkupfalse%
\ False\isanewline
\ \ \ \ \ \ \isacommand{hence}\isamarkupfalse%
\ {\isachardoublequoteopen}{\isacharminus}{\kern0pt}\ n\ {\isasymge}\ {\isadigit{0}}{\isachardoublequoteclose}\ \isacommand{by}\isamarkupfalse%
\ simp\isanewline
\ \ \ \ \ \ \isacommand{hence}\isamarkupfalse%
\ {\isachardoublequoteopen}{\isasymbar}f\ {\isacharparenleft}{\kern0pt}{\isacharminus}{\kern0pt}\ n{\isacharparenright}{\kern0pt}\ {\isacharminus}{\kern0pt}\ g\ {\isacharparenleft}{\kern0pt}{\isacharminus}{\kern0pt}\ n{\isacharparenright}{\kern0pt}{\isasymbar}\ {\isasymle}\ Sup\ {\isacharbraceleft}{\kern0pt}{\isasymbar}f\ n\ {\isacharminus}{\kern0pt}\ g\ n{\isasymbar}\ {\isacharbar}{\kern0pt}\ n{\isachardot}{\kern0pt}\ n\ {\isasymin}\ {\isacharbraceleft}{\kern0pt}{\isadigit{0}}{\isachardot}{\kern0pt}{\isachardot}{\kern0pt}max\ {\isadigit{0}}\ N{\isacharbraceright}{\kern0pt}{\isacharbraceright}{\kern0pt}\ {\isacharplus}{\kern0pt}\ C{\isachardoublequoteclose}\ \isacommand{using}\isamarkupfalse%
\ {\isacharasterisk}{\kern0pt}{\isacharbrackleft}{\kern0pt}of\ {\isachardoublequoteopen}{\isacharminus}{\kern0pt}\ n{\isachardoublequoteclose}{\isacharbrackright}{\kern0pt}\ \isacommand{by}\isamarkupfalse%
\ blast\isanewline
\ \ \ \ \ \ \isacommand{hence}\isamarkupfalse%
\ {\isachardoublequoteopen}{\isasymbar}f\ {\isacharparenleft}{\kern0pt}{\isacharminus}{\kern0pt}\ {\isacharparenleft}{\kern0pt}{\isacharminus}{\kern0pt}\ n{\isacharparenright}{\kern0pt}{\isacharparenright}{\kern0pt}\ {\isacharminus}{\kern0pt}\ g\ {\isacharparenleft}{\kern0pt}{\isacharminus}{\kern0pt}\ {\isacharparenleft}{\kern0pt}{\isacharminus}{\kern0pt}\ n{\isacharparenright}{\kern0pt}{\isacharparenright}{\kern0pt}{\isasymbar}\ {\isasymle}\ C{\isacharprime}{\kern0pt}{\isachardoublequoteclose}\ \isacommand{unfolding}\isamarkupfalse%
\ C{\isacharprime}{\kern0pt}{\isacharunderscore}{\kern0pt}def\ \isacommand{using}\isamarkupfalse%
\ bound{\isacharbrackleft}{\kern0pt}of\ {\isachardoublequoteopen}{\isacharminus}{\kern0pt}\ n{\isachardoublequoteclose}{\isacharbrackright}{\kern0pt}\ \isacommand{by}\isamarkupfalse%
\ linarith\isanewline
\ \ \ \ \ \ \isacommand{thus}\isamarkupfalse%
\ {\isacharquery}{\kern0pt}thesis\ \isacommand{by}\isamarkupfalse%
\ simp\isanewline
\ \ \ \ \isacommand{qed}\isamarkupfalse%
\isanewline
\ \ \isacommand{{\isacharbraceright}{\kern0pt}}\isamarkupfalse%
\isanewline
\ \ \isacommand{thus}\isamarkupfalse%
\ {\isacharquery}{\kern0pt}thesis\ \isacommand{using}\isamarkupfalse%
\ assms\ \isacommand{unfolding}\isamarkupfalse%
\ eudoxus{\isacharunderscore}{\kern0pt}rel{\isacharunderscore}{\kern0pt}def\ \isacommand{by}\isamarkupfalse%
\ {\isacharparenleft}{\kern0pt}auto\ intro{\isacharcolon}{\kern0pt}\ boundedI{\isacharparenright}{\kern0pt}\isanewline
\isacommand{qed}\isamarkupfalse%
%
\endisatagproof
{\isafoldproof}%
%
\isadelimproof
%
\endisadelimproof
%
\isadelimdocument
%
\endisadelimdocument
%
\isatagdocument
%
\isamarkupsubsection{Addition and Subtraction%
}
\isamarkuptrue%
%
\endisatagdocument
{\isafolddocument}%
%
\isadelimdocument
%
\endisadelimdocument
%
\begin{isamarkuptext}%
We define addition, subtraction and the additive identity as follows.%
\end{isamarkuptext}\isamarkuptrue%
\isacommand{instantiation}\isamarkupfalse%
\ real\ {\isacharcolon}{\kern0pt}{\isacharcolon}{\kern0pt}\ {\isachardoublequoteopen}{\isacharbraceleft}{\kern0pt}zero{\isacharcomma}{\kern0pt}\ plus{\isacharcomma}{\kern0pt}\ minus{\isacharcomma}{\kern0pt}\ uminus{\isacharbraceright}{\kern0pt}{\isachardoublequoteclose}\isanewline
\isakeyword{begin}\isanewline
\isanewline
\isacommand{quotient{\isacharunderscore}{\kern0pt}definition}\isamarkupfalse%
\isanewline
\ \ {\isachardoublequoteopen}{\isadigit{0}}\ {\isacharcolon}{\kern0pt}{\isacharcolon}{\kern0pt}\ real{\isachardoublequoteclose}\ \isakeyword{is}\ {\isachardoublequoteopen}abs{\isacharunderscore}{\kern0pt}real\ {\isacharparenleft}{\kern0pt}{\isasymlambda}{\isacharunderscore}{\kern0pt}{\isachardot}{\kern0pt}\ {\isadigit{0}}{\isacharparenright}{\kern0pt}{\isachardoublequoteclose}%
\isadelimproof
\ %
\endisadelimproof
%
\isatagproof
\isacommand{{\isachardot}{\kern0pt}}\isamarkupfalse%
%
\endisatagproof
{\isafoldproof}%
%
\isadelimproof
%
\endisadelimproof
\isanewline
\isanewline
\isacommand{declare}\isamarkupfalse%
\ slope{\isacharunderscore}{\kern0pt}zero{\isacharbrackleft}{\kern0pt}intro{\isacharbang}{\kern0pt}{\isacharcomma}{\kern0pt}\ simp{\isacharbrackright}{\kern0pt}\isanewline
\isanewline
\isacommand{lemma}\isamarkupfalse%
\ zero{\isacharunderscore}{\kern0pt}iff{\isacharunderscore}{\kern0pt}bounded{\isacharcolon}{\kern0pt}\ {\isachardoublequoteopen}f\ {\isasymsim}\isactrlsub e\ {\isacharparenleft}{\kern0pt}{\isasymlambda}{\isacharunderscore}{\kern0pt}{\isachardot}{\kern0pt}\ {\isadigit{0}}{\isacharparenright}{\kern0pt}\ {\isasymlongleftrightarrow}\ bounded\ f{\isachardoublequoteclose}%
\isadelimproof
\ %
\endisadelimproof
%
\isatagproof
\isacommand{by}\isamarkupfalse%
\ {\isacharparenleft}{\kern0pt}metis\ {\isacharparenleft}{\kern0pt}no{\isacharunderscore}{\kern0pt}types{\isacharcomma}{\kern0pt}\ lifting{\isacharparenright}{\kern0pt}\ boundedE\ boundedI\ diff{\isacharunderscore}{\kern0pt}zero\ eudoxus{\isacharunderscore}{\kern0pt}rel{\isacharunderscore}{\kern0pt}def\ slope{\isacharunderscore}{\kern0pt}zero\ bounded{\isacharunderscore}{\kern0pt}slopeI{\isacharparenright}{\kern0pt}%
\endisatagproof
{\isafoldproof}%
%
\isadelimproof
%
\endisadelimproof
\isanewline
\isacommand{lemma}\isamarkupfalse%
\ zero{\isacharunderscore}{\kern0pt}iff{\isacharunderscore}{\kern0pt}bounded{\isacharprime}{\kern0pt}{\isacharcolon}{\kern0pt}\ {\isachardoublequoteopen}x\ {\isacharequal}{\kern0pt}\ {\isadigit{0}}\ {\isasymlongleftrightarrow}\ bounded\ {\isacharparenleft}{\kern0pt}rep{\isacharunderscore}{\kern0pt}real\ x{\isacharparenright}{\kern0pt}{\isachardoublequoteclose}%
\isadelimproof
\ %
\endisadelimproof
%
\isatagproof
\isacommand{by}\isamarkupfalse%
\ {\isacharparenleft}{\kern0pt}metis\ {\isacharparenleft}{\kern0pt}mono{\isacharunderscore}{\kern0pt}tags{\isacharparenright}{\kern0pt}\ abs{\isacharunderscore}{\kern0pt}real{\isacharunderscore}{\kern0pt}eq{\isacharunderscore}{\kern0pt}iff\ id{\isacharunderscore}{\kern0pt}apply\ rep{\isacharunderscore}{\kern0pt}real{\isacharunderscore}{\kern0pt}abs{\isacharunderscore}{\kern0pt}real{\isacharunderscore}{\kern0pt}refl\ rep{\isacharunderscore}{\kern0pt}real{\isacharunderscore}{\kern0pt}iff\ slope{\isacharunderscore}{\kern0pt}zero\ zero{\isacharunderscore}{\kern0pt}iff{\isacharunderscore}{\kern0pt}bounded\ zero{\isacharunderscore}{\kern0pt}real{\isacharunderscore}{\kern0pt}def{\isacharparenright}{\kern0pt}%
\endisatagproof
{\isafoldproof}%
%
\isadelimproof
%
\endisadelimproof
\isanewline
\isanewline
\isacommand{lemma}\isamarkupfalse%
\ zero{\isacharunderscore}{\kern0pt}def{\isacharcolon}{\kern0pt}\ {\isachardoublequoteopen}{\isadigit{0}}\ {\isacharequal}{\kern0pt}\ abs{\isacharunderscore}{\kern0pt}real\ {\isacharparenleft}{\kern0pt}{\isasymlambda}{\isacharunderscore}{\kern0pt}{\isachardot}{\kern0pt}\ {\isadigit{0}}{\isacharparenright}{\kern0pt}{\isachardoublequoteclose}%
\isadelimproof
\ %
\endisadelimproof
%
\isatagproof
\isacommand{unfolding}\isamarkupfalse%
\ zero{\isacharunderscore}{\kern0pt}real{\isacharunderscore}{\kern0pt}def\ \isacommand{by}\isamarkupfalse%
\ simp%
\endisatagproof
{\isafoldproof}%
%
\isadelimproof
%
\endisadelimproof
\isanewline
\isanewline
\isacommand{definition}\isamarkupfalse%
\ eudoxus{\isacharunderscore}{\kern0pt}plus\ {\isacharcolon}{\kern0pt}{\isacharcolon}{\kern0pt}\ {\isachardoublequoteopen}{\isacharparenleft}{\kern0pt}int\ {\isasymRightarrow}\ int{\isacharparenright}{\kern0pt}\ {\isasymRightarrow}\ {\isacharparenleft}{\kern0pt}int\ {\isasymRightarrow}\ int{\isacharparenright}{\kern0pt}\ {\isasymRightarrow}\ {\isacharparenleft}{\kern0pt}int\ {\isasymRightarrow}\ int{\isacharparenright}{\kern0pt}{\isachardoublequoteclose}\ {\isacharparenleft}{\kern0pt}\isakeyword{infixl}\ {\isachardoublequoteopen}{\isacharplus}{\kern0pt}\isactrlsub e{\isachardoublequoteclose}\ {\isadigit{6}}{\isadigit{0}}{\isacharparenright}{\kern0pt}\ \isakeyword{where}\isanewline
\ \ {\isachardoublequoteopen}{\isacharparenleft}{\kern0pt}f\ {\isacharcolon}{\kern0pt}{\isacharcolon}{\kern0pt}\ int\ {\isasymRightarrow}\ int{\isacharparenright}{\kern0pt}\ {\isacharplus}{\kern0pt}\isactrlsub e\ g\ {\isacharequal}{\kern0pt}\ {\isacharparenleft}{\kern0pt}{\isasymlambda}z{\isachardot}{\kern0pt}\ f\ z\ {\isacharplus}{\kern0pt}\ g\ z{\isacharparenright}{\kern0pt}{\isachardoublequoteclose}\isanewline
\isanewline
\isacommand{declare}\isamarkupfalse%
\ slope{\isacharunderscore}{\kern0pt}add{\isacharbrackleft}{\kern0pt}intro{\isacharcomma}{\kern0pt}\ simp{\isacharbrackright}{\kern0pt}\isanewline
\isanewline
\isacommand{quotient{\isacharunderscore}{\kern0pt}definition}\isamarkupfalse%
\isanewline
\ \ {\isachardoublequoteopen}{\isacharparenleft}{\kern0pt}{\isacharplus}{\kern0pt}{\isacharparenright}{\kern0pt}\ {\isacharcolon}{\kern0pt}{\isacharcolon}{\kern0pt}\ {\isacharparenleft}{\kern0pt}real\ {\isasymRightarrow}\ real\ {\isasymRightarrow}\ real{\isacharparenright}{\kern0pt}{\isachardoublequoteclose}\ \isakeyword{is}\ {\isachardoublequoteopen}{\isacharparenleft}{\kern0pt}{\isacharplus}{\kern0pt}\isactrlsub e{\isacharparenright}{\kern0pt}{\isachardoublequoteclose}\ \isanewline
%
\isadelimproof
%
\endisadelimproof
%
\isatagproof
\isacommand{proof}\isamarkupfalse%
\ {\isacharminus}{\kern0pt}\isanewline
\ \ \isacommand{fix}\isamarkupfalse%
\ x\ x{\isacharprime}{\kern0pt}\ y\ y{\isacharprime}{\kern0pt}\ \isacommand{assume}\isamarkupfalse%
\ {\isachardoublequoteopen}x\ {\isasymsim}\isactrlsub e\ x{\isacharprime}{\kern0pt}{\isachardoublequoteclose}\ {\isachardoublequoteopen}y\ {\isasymsim}\isactrlsub e\ y{\isacharprime}{\kern0pt}{\isachardoublequoteclose}\isanewline
\ \ \isacommand{hence}\isamarkupfalse%
\ rel{\isacharunderscore}{\kern0pt}x{\isacharcolon}{\kern0pt}\ {\isachardoublequoteopen}slope\ x{\isachardoublequoteclose}\ {\isachardoublequoteopen}slope\ x{\isacharprime}{\kern0pt}{\isachardoublequoteclose}\ {\isachardoublequoteopen}bounded\ {\isacharparenleft}{\kern0pt}{\isasymlambda}z{\isachardot}{\kern0pt}\ x\ z\ {\isacharminus}{\kern0pt}\ x{\isacharprime}{\kern0pt}\ z{\isacharparenright}{\kern0pt}{\isachardoublequoteclose}\ \isakeyword{and}\ rel{\isacharunderscore}{\kern0pt}y{\isacharcolon}{\kern0pt}\ {\isachardoublequoteopen}slope\ y{\isachardoublequoteclose}\ {\isachardoublequoteopen}slope\ y{\isacharprime}{\kern0pt}{\isachardoublequoteclose}\ {\isachardoublequoteopen}bounded\ {\isacharparenleft}{\kern0pt}{\isasymlambda}z{\isachardot}{\kern0pt}\ y\ z\ {\isacharminus}{\kern0pt}\ y{\isacharprime}{\kern0pt}\ z{\isacharparenright}{\kern0pt}{\isachardoublequoteclose}\ \isacommand{unfolding}\isamarkupfalse%
\ eudoxus{\isacharunderscore}{\kern0pt}rel{\isacharunderscore}{\kern0pt}def\ \isacommand{by}\isamarkupfalse%
\ blast{\isacharplus}{\kern0pt}\isanewline
\ \ \isacommand{thus}\isamarkupfalse%
\ {\isachardoublequoteopen}{\isacharparenleft}{\kern0pt}x\ {\isacharplus}{\kern0pt}\isactrlsub e\ y{\isacharparenright}{\kern0pt}\ {\isasymsim}\isactrlsub e\ {\isacharparenleft}{\kern0pt}x{\isacharprime}{\kern0pt}\ {\isacharplus}{\kern0pt}\isactrlsub e\ y{\isacharprime}{\kern0pt}{\isacharparenright}{\kern0pt}{\isachardoublequoteclose}\ \isacommand{unfolding}\isamarkupfalse%
\ eudoxus{\isacharunderscore}{\kern0pt}rel{\isacharunderscore}{\kern0pt}def\ eudoxus{\isacharunderscore}{\kern0pt}plus{\isacharunderscore}{\kern0pt}def\ \isacommand{by}\isamarkupfalse%
\ {\isacharparenleft}{\kern0pt}fastforce\ intro{\isacharcolon}{\kern0pt}\ back{\isacharunderscore}{\kern0pt}subst{\isacharbrackleft}{\kern0pt}of\ bounded{\isacharcomma}{\kern0pt}\ OF\ bounded{\isacharunderscore}{\kern0pt}add{\isacharbrackleft}{\kern0pt}OF\ rel{\isacharunderscore}{\kern0pt}x{\isacharparenleft}{\kern0pt}{\isadigit{3}}{\isacharparenright}{\kern0pt}\ rel{\isacharunderscore}{\kern0pt}y{\isacharparenleft}{\kern0pt}{\isadigit{3}}{\isacharparenright}{\kern0pt}{\isacharbrackright}{\kern0pt}{\isacharbrackright}{\kern0pt}{\isacharparenright}{\kern0pt}\isanewline
\isacommand{qed}\isamarkupfalse%
%
\endisatagproof
{\isafoldproof}%
%
\isadelimproof
\isanewline
%
\endisadelimproof
\isanewline
\isacommand{lemmas}\isamarkupfalse%
\ eudoxus{\isacharunderscore}{\kern0pt}plus{\isacharunderscore}{\kern0pt}cong\ {\isacharequal}{\kern0pt}\ apply{\isacharunderscore}{\kern0pt}rsp{\isacharprime}{\kern0pt}{\isacharbrackleft}{\kern0pt}OF\ plus{\isacharunderscore}{\kern0pt}real{\isachardot}{\kern0pt}rsp{\isacharcomma}{\kern0pt}\ THEN\ rel{\isacharunderscore}{\kern0pt}funD{\isacharcomma}{\kern0pt}\ intro{\isacharbrackright}{\kern0pt}\isanewline
\isanewline
\isacommand{lemma}\isamarkupfalse%
\ abs{\isacharunderscore}{\kern0pt}real{\isacharunderscore}{\kern0pt}plus{\isacharbrackleft}{\kern0pt}simp{\isacharbrackright}{\kern0pt}{\isacharcolon}{\kern0pt}\isanewline
\ \ \isakeyword{assumes}\ {\isachardoublequoteopen}slope\ f{\isachardoublequoteclose}\ {\isachardoublequoteopen}slope\ g{\isachardoublequoteclose}\isanewline
\ \ \isakeyword{shows}\ {\isachardoublequoteopen}abs{\isacharunderscore}{\kern0pt}real\ f\ {\isacharplus}{\kern0pt}\ abs{\isacharunderscore}{\kern0pt}real\ g\ {\isacharequal}{\kern0pt}\ abs{\isacharunderscore}{\kern0pt}real\ {\isacharparenleft}{\kern0pt}f\ {\isacharplus}{\kern0pt}\isactrlsub e\ g{\isacharparenright}{\kern0pt}{\isachardoublequoteclose}\isanewline
%
\isadelimproof
\ \ %
\endisadelimproof
%
\isatagproof
\isacommand{using}\isamarkupfalse%
\ assms\ \isacommand{unfolding}\isamarkupfalse%
\ plus{\isacharunderscore}{\kern0pt}real{\isacharunderscore}{\kern0pt}def\ \isacommand{by}\isamarkupfalse%
\ auto%
\endisatagproof
{\isafoldproof}%
%
\isadelimproof
\isanewline
%
\endisadelimproof
\isanewline
\isacommand{definition}\isamarkupfalse%
\ eudoxus{\isacharunderscore}{\kern0pt}uminus\ {\isacharcolon}{\kern0pt}{\isacharcolon}{\kern0pt}\ {\isachardoublequoteopen}{\isacharparenleft}{\kern0pt}int\ {\isasymRightarrow}\ int{\isacharparenright}{\kern0pt}\ {\isasymRightarrow}\ {\isacharparenleft}{\kern0pt}int\ {\isasymRightarrow}\ int{\isacharparenright}{\kern0pt}{\isachardoublequoteclose}\ {\isacharparenleft}{\kern0pt}{\isachardoublequoteopen}{\isacharminus}{\kern0pt}\isactrlsub e{\isachardoublequoteclose}{\isacharparenright}{\kern0pt}\ \isakeyword{where}\isanewline
\ \ {\isachardoublequoteopen}{\isacharminus}{\kern0pt}\isactrlsub e\ {\isacharparenleft}{\kern0pt}f\ {\isacharcolon}{\kern0pt}{\isacharcolon}{\kern0pt}\ int\ {\isasymRightarrow}\ int{\isacharparenright}{\kern0pt}\ {\isacharequal}{\kern0pt}\ {\isacharparenleft}{\kern0pt}{\isasymlambda}x{\isachardot}{\kern0pt}\ {\isacharminus}{\kern0pt}\ f\ x{\isacharparenright}{\kern0pt}{\isachardoublequoteclose}\isanewline
\isanewline
\isacommand{declare}\isamarkupfalse%
\ slope{\isacharunderscore}{\kern0pt}uminus{\isacharprime}{\kern0pt}{\isacharbrackleft}{\kern0pt}intro{\isacharcomma}{\kern0pt}\ simp{\isacharbrackright}{\kern0pt}\isanewline
\isanewline
\isacommand{quotient{\isacharunderscore}{\kern0pt}definition}\isamarkupfalse%
\isanewline
\ \ {\isachardoublequoteopen}{\isacharparenleft}{\kern0pt}uminus{\isacharparenright}{\kern0pt}\ {\isacharcolon}{\kern0pt}{\isacharcolon}{\kern0pt}\ {\isacharparenleft}{\kern0pt}real\ {\isasymRightarrow}\ real{\isacharparenright}{\kern0pt}{\isachardoublequoteclose}\ \isakeyword{is}\ {\isachardoublequoteopen}{\isacharminus}{\kern0pt}\isactrlsub e{\isachardoublequoteclose}\isanewline
%
\isadelimproof
%
\endisadelimproof
%
\isatagproof
\isacommand{proof}\isamarkupfalse%
\ {\isacharminus}{\kern0pt}\isanewline
\ \ \isacommand{fix}\isamarkupfalse%
\ x\ x{\isacharprime}{\kern0pt}\ \isacommand{assume}\isamarkupfalse%
\ {\isachardoublequoteopen}x\ {\isasymsim}\isactrlsub e\ x{\isacharprime}{\kern0pt}{\isachardoublequoteclose}\isanewline
\ \ \isacommand{hence}\isamarkupfalse%
\ rel{\isacharunderscore}{\kern0pt}x{\isacharcolon}{\kern0pt}\ {\isachardoublequoteopen}slope\ x{\isachardoublequoteclose}\ {\isachardoublequoteopen}slope\ x{\isacharprime}{\kern0pt}{\isachardoublequoteclose}\ {\isachardoublequoteopen}bounded\ {\isacharparenleft}{\kern0pt}{\isasymlambda}z{\isachardot}{\kern0pt}\ x\ z\ {\isacharminus}{\kern0pt}\ x{\isacharprime}{\kern0pt}\ z{\isacharparenright}{\kern0pt}{\isachardoublequoteclose}\ \isacommand{unfolding}\isamarkupfalse%
\ eudoxus{\isacharunderscore}{\kern0pt}rel{\isacharunderscore}{\kern0pt}def\ \isacommand{by}\isamarkupfalse%
\ blast{\isacharplus}{\kern0pt}\isanewline
\ \ \isacommand{thus}\isamarkupfalse%
\ {\isachardoublequoteopen}{\isacharminus}{\kern0pt}\isactrlsub e\ x\ {\isasymsim}\isactrlsub e\ {\isacharminus}{\kern0pt}\isactrlsub e\ x{\isacharprime}{\kern0pt}{\isachardoublequoteclose}\ \isacommand{unfolding}\isamarkupfalse%
\ eudoxus{\isacharunderscore}{\kern0pt}rel{\isacharunderscore}{\kern0pt}def\ eudoxus{\isacharunderscore}{\kern0pt}uminus{\isacharunderscore}{\kern0pt}def\ \isacommand{by}\isamarkupfalse%
\ {\isacharparenleft}{\kern0pt}fastforce\ intro{\isacharcolon}{\kern0pt}\ back{\isacharunderscore}{\kern0pt}subst{\isacharbrackleft}{\kern0pt}of\ bounded{\isacharcomma}{\kern0pt}\ OF\ bounded{\isacharunderscore}{\kern0pt}uminus{\isacharbrackleft}{\kern0pt}OF\ rel{\isacharunderscore}{\kern0pt}x{\isacharparenleft}{\kern0pt}{\isadigit{3}}{\isacharparenright}{\kern0pt}{\isacharbrackright}{\kern0pt}{\isacharbrackright}{\kern0pt}{\isacharparenright}{\kern0pt}\isanewline
\isacommand{qed}\isamarkupfalse%
%
\endisatagproof
{\isafoldproof}%
%
\isadelimproof
\isanewline
%
\endisadelimproof
\isanewline
\isacommand{lemmas}\isamarkupfalse%
\ eudoxus{\isacharunderscore}{\kern0pt}uminus{\isacharunderscore}{\kern0pt}cong\ {\isacharequal}{\kern0pt}\ apply{\isacharunderscore}{\kern0pt}rsp{\isacharprime}{\kern0pt}{\isacharbrackleft}{\kern0pt}OF\ uminus{\isacharunderscore}{\kern0pt}real{\isachardot}{\kern0pt}rsp{\isacharcomma}{\kern0pt}\ simplified{\isacharcomma}{\kern0pt}\ intro{\isacharbrackright}{\kern0pt}\isanewline
\isanewline
\isacommand{lemma}\isamarkupfalse%
\ abs{\isacharunderscore}{\kern0pt}real{\isacharunderscore}{\kern0pt}uminus{\isacharbrackleft}{\kern0pt}simp{\isacharbrackright}{\kern0pt}{\isacharcolon}{\kern0pt}\ \isanewline
\ \ \isakeyword{assumes}\ {\isachardoublequoteopen}slope\ f{\isachardoublequoteclose}\isanewline
\ \ \isakeyword{shows}\ {\isachardoublequoteopen}{\isacharminus}{\kern0pt}\ abs{\isacharunderscore}{\kern0pt}real\ f\ {\isacharequal}{\kern0pt}\ abs{\isacharunderscore}{\kern0pt}real\ {\isacharparenleft}{\kern0pt}{\isacharminus}{\kern0pt}\isactrlsub e\ f{\isacharparenright}{\kern0pt}{\isachardoublequoteclose}\isanewline
%
\isadelimproof
\ \ %
\endisadelimproof
%
\isatagproof
\isacommand{using}\isamarkupfalse%
\ assms\ \isacommand{unfolding}\isamarkupfalse%
\ uminus{\isacharunderscore}{\kern0pt}real{\isacharunderscore}{\kern0pt}def\ \isacommand{by}\isamarkupfalse%
\ auto%
\endisatagproof
{\isafoldproof}%
%
\isadelimproof
\isanewline
%
\endisadelimproof
\isanewline
\isacommand{definition}\isamarkupfalse%
\ {\isachardoublequoteopen}x\ {\isacharminus}{\kern0pt}\ {\isacharparenleft}{\kern0pt}y{\isacharcolon}{\kern0pt}{\isacharcolon}{\kern0pt}real{\isacharparenright}{\kern0pt}\ {\isacharequal}{\kern0pt}\ x\ {\isacharplus}{\kern0pt}\ {\isacharminus}{\kern0pt}\ y{\isachardoublequoteclose}\isanewline
\isanewline
\isacommand{declare}\isamarkupfalse%
\ slope{\isacharunderscore}{\kern0pt}minus{\isacharbrackleft}{\kern0pt}intro{\isacharcomma}{\kern0pt}\ simp{\isacharbrackright}{\kern0pt}\isanewline
\isanewline
\isacommand{lemma}\isamarkupfalse%
\ abs{\isacharunderscore}{\kern0pt}real{\isacharunderscore}{\kern0pt}minus{\isacharbrackleft}{\kern0pt}simp{\isacharbrackright}{\kern0pt}{\isacharcolon}{\kern0pt}\isanewline
\ \ \isakeyword{assumes}\ {\isachardoublequoteopen}slope\ g{\isachardoublequoteclose}\ {\isachardoublequoteopen}slope\ f{\isachardoublequoteclose}\isanewline
\ \ \isakeyword{shows}\ {\isachardoublequoteopen}abs{\isacharunderscore}{\kern0pt}real\ g\ {\isacharminus}{\kern0pt}\ abs{\isacharunderscore}{\kern0pt}real\ f\ {\isacharequal}{\kern0pt}\ abs{\isacharunderscore}{\kern0pt}real\ {\isacharparenleft}{\kern0pt}g\ {\isacharplus}{\kern0pt}\isactrlsub e\ {\isacharparenleft}{\kern0pt}{\isacharminus}{\kern0pt}\isactrlsub e\ f{\isacharparenright}{\kern0pt}{\isacharparenright}{\kern0pt}{\isachardoublequoteclose}\ \isanewline
%
\isadelimproof
\ \ %
\endisadelimproof
%
\isatagproof
\isacommand{using}\isamarkupfalse%
\ assms\ \isacommand{by}\isamarkupfalse%
\ {\isacharparenleft}{\kern0pt}simp\ add{\isacharcolon}{\kern0pt}\ minus{\isacharunderscore}{\kern0pt}real{\isacharunderscore}{\kern0pt}def\ slope{\isacharunderscore}{\kern0pt}refl\ eudoxus{\isacharunderscore}{\kern0pt}uminus{\isacharunderscore}{\kern0pt}cong{\isacharparenright}{\kern0pt}%
\endisatagproof
{\isafoldproof}%
%
\isadelimproof
\isanewline
%
\endisadelimproof
\isanewline
\isacommand{instance}\isamarkupfalse%
%
\isadelimproof
\ %
\endisadelimproof
%
\isatagproof
\isacommand{{\isachardot}{\kern0pt}{\isachardot}{\kern0pt}}\isamarkupfalse%
%
\endisatagproof
{\isafoldproof}%
%
\isadelimproof
%
\endisadelimproof
\isanewline
\isacommand{end}\isamarkupfalse%
%
\begin{isamarkuptext}%
The Eudoxus reals equipped with addition and negation specified as above constitute an Abelian group.%
\end{isamarkuptext}\isamarkuptrue%
\isacommand{instance}\isamarkupfalse%
\ real\ {\isacharcolon}{\kern0pt}{\isacharcolon}{\kern0pt}\ ab{\isacharunderscore}{\kern0pt}group{\isacharunderscore}{\kern0pt}add\isanewline
%
\isadelimproof
%
\endisadelimproof
%
\isatagproof
\isacommand{proof}\isamarkupfalse%
\isanewline
\ \ \isacommand{fix}\isamarkupfalse%
\ x\ y\ z\ {\isacharcolon}{\kern0pt}{\isacharcolon}{\kern0pt}\ real\isanewline
\ \ \isacommand{show}\isamarkupfalse%
\ {\isachardoublequoteopen}x\ {\isacharplus}{\kern0pt}\ y\ {\isacharplus}{\kern0pt}\ z\ {\isacharequal}{\kern0pt}\ x\ {\isacharplus}{\kern0pt}\ {\isacharparenleft}{\kern0pt}y\ {\isacharplus}{\kern0pt}\ z{\isacharparenright}{\kern0pt}{\isachardoublequoteclose}\ \isacommand{by}\isamarkupfalse%
\ {\isacharparenleft}{\kern0pt}induct\ x{\isacharcomma}{\kern0pt}\ induct\ y{\isacharcomma}{\kern0pt}\ induct\ z{\isacharparenright}{\kern0pt}\ {\isacharparenleft}{\kern0pt}simp\ add{\isacharcolon}{\kern0pt}\ eudoxus{\isacharunderscore}{\kern0pt}plus{\isacharunderscore}{\kern0pt}cong\ eudoxus{\isacharunderscore}{\kern0pt}plus{\isacharunderscore}{\kern0pt}def\ add{\isachardot}{\kern0pt}assoc{\isacharparenright}{\kern0pt}\isanewline
\ \ \isacommand{show}\isamarkupfalse%
\ {\isachardoublequoteopen}x\ {\isacharplus}{\kern0pt}\ y\ {\isacharequal}{\kern0pt}\ y\ {\isacharplus}{\kern0pt}\ x{\isachardoublequoteclose}\ \isacommand{by}\isamarkupfalse%
\ {\isacharparenleft}{\kern0pt}induct\ x{\isacharcomma}{\kern0pt}\ induct\ y{\isacharparenright}{\kern0pt}\ {\isacharparenleft}{\kern0pt}simp\ add{\isacharcolon}{\kern0pt}\ eudoxus{\isacharunderscore}{\kern0pt}plus{\isacharunderscore}{\kern0pt}def\ add{\isachardot}{\kern0pt}commute{\isacharparenright}{\kern0pt}\isanewline
\ \ \isacommand{show}\isamarkupfalse%
\ {\isachardoublequoteopen}{\isadigit{0}}\ {\isacharplus}{\kern0pt}\ x\ {\isacharequal}{\kern0pt}\ x{\isachardoublequoteclose}\ \isacommand{by}\isamarkupfalse%
\ {\isacharparenleft}{\kern0pt}induct\ x{\isacharparenright}{\kern0pt}\ {\isacharparenleft}{\kern0pt}simp\ add{\isacharcolon}{\kern0pt}\ zero{\isacharunderscore}{\kern0pt}real{\isacharunderscore}{\kern0pt}def\ eudoxus{\isacharunderscore}{\kern0pt}plus{\isacharunderscore}{\kern0pt}def{\isacharparenright}{\kern0pt}\isanewline
\ \ \isacommand{show}\isamarkupfalse%
\ {\isachardoublequoteopen}{\isacharminus}{\kern0pt}\ x\ {\isacharplus}{\kern0pt}\ x\ {\isacharequal}{\kern0pt}\ {\isadigit{0}}{\isachardoublequoteclose}\ \isacommand{by}\isamarkupfalse%
\ {\isacharparenleft}{\kern0pt}induct\ x{\isacharparenright}{\kern0pt}\ {\isacharparenleft}{\kern0pt}simp\ add{\isacharcolon}{\kern0pt}\ eudoxus{\isacharunderscore}{\kern0pt}uminus{\isacharunderscore}{\kern0pt}cong{\isacharcomma}{\kern0pt}\ simp\ add{\isacharcolon}{\kern0pt}\ zero{\isacharunderscore}{\kern0pt}real{\isacharunderscore}{\kern0pt}def\ eudoxus{\isacharunderscore}{\kern0pt}plus{\isacharunderscore}{\kern0pt}def\ eudoxus{\isacharunderscore}{\kern0pt}uminus{\isacharunderscore}{\kern0pt}def{\isacharparenright}{\kern0pt}\isanewline
\isacommand{qed}\isamarkupfalse%
\ {\isacharparenleft}{\kern0pt}simp\ add{\isacharcolon}{\kern0pt}\ minus{\isacharunderscore}{\kern0pt}real{\isacharunderscore}{\kern0pt}def{\isacharparenright}{\kern0pt}%
\endisatagproof
{\isafoldproof}%
%
\isadelimproof
%
\endisadelimproof
%
\isadelimdocument
%
\endisadelimdocument
%
\isatagdocument
%
\isamarkupsubsection{Multiplication%
}
\isamarkuptrue%
%
\endisatagdocument
{\isafolddocument}%
%
\isadelimdocument
%
\endisadelimdocument
%
\begin{isamarkuptext}%
We define multiplication as the composition of two slopes.%
\end{isamarkuptext}\isamarkuptrue%
\isacommand{instantiation}\isamarkupfalse%
\ real\ {\isacharcolon}{\kern0pt}{\isacharcolon}{\kern0pt}\ {\isachardoublequoteopen}{\isacharbraceleft}{\kern0pt}one{\isacharcomma}{\kern0pt}\ times{\isacharbraceright}{\kern0pt}{\isachardoublequoteclose}\isanewline
\isakeyword{begin}\isanewline
\isanewline
\isacommand{quotient{\isacharunderscore}{\kern0pt}definition}\isamarkupfalse%
\isanewline
\ \ {\isachardoublequoteopen}{\isadigit{1}}\ {\isacharcolon}{\kern0pt}{\isacharcolon}{\kern0pt}\ real{\isachardoublequoteclose}\ \isakeyword{is}\ {\isachardoublequoteopen}abs{\isacharunderscore}{\kern0pt}real\ id{\isachardoublequoteclose}%
\isadelimproof
\ %
\endisadelimproof
%
\isatagproof
\isacommand{{\isachardot}{\kern0pt}}\isamarkupfalse%
%
\endisatagproof
{\isafoldproof}%
%
\isadelimproof
%
\endisadelimproof
\isanewline
\isanewline
\isacommand{declare}\isamarkupfalse%
\ slope{\isacharunderscore}{\kern0pt}one{\isacharbrackleft}{\kern0pt}intro{\isacharbang}{\kern0pt}{\isacharcomma}{\kern0pt}\ simp{\isacharbrackright}{\kern0pt}\isanewline
\isanewline
\isacommand{lemma}\isamarkupfalse%
\ one{\isacharunderscore}{\kern0pt}def{\isacharcolon}{\kern0pt}\ {\isachardoublequoteopen}{\isadigit{1}}\ {\isacharequal}{\kern0pt}\ abs{\isacharunderscore}{\kern0pt}real\ id{\isachardoublequoteclose}%
\isadelimproof
\ %
\endisadelimproof
%
\isatagproof
\isacommand{unfolding}\isamarkupfalse%
\ one{\isacharunderscore}{\kern0pt}real{\isacharunderscore}{\kern0pt}def\ \isacommand{by}\isamarkupfalse%
\ simp%
\endisatagproof
{\isafoldproof}%
%
\isadelimproof
%
\endisadelimproof
\isanewline
\isanewline
\isacommand{definition}\isamarkupfalse%
\ eudoxus{\isacharunderscore}{\kern0pt}times\ {\isacharcolon}{\kern0pt}{\isacharcolon}{\kern0pt}\ {\isachardoublequoteopen}{\isacharparenleft}{\kern0pt}int\ {\isasymRightarrow}\ int{\isacharparenright}{\kern0pt}\ {\isasymRightarrow}\ {\isacharparenleft}{\kern0pt}int\ {\isasymRightarrow}\ int{\isacharparenright}{\kern0pt}\ {\isasymRightarrow}\ int\ {\isasymRightarrow}\ int{\isachardoublequoteclose}\ {\isacharparenleft}{\kern0pt}\isakeyword{infixl}\ {\isachardoublequoteopen}{\isacharasterisk}{\kern0pt}\isactrlsub e{\isachardoublequoteclose}\ {\isadigit{6}}{\isadigit{0}}{\isacharparenright}{\kern0pt}\ \isakeyword{where}\isanewline
\ \ {\isachardoublequoteopen}f\ {\isacharasterisk}{\kern0pt}\isactrlsub e\ g\ {\isacharequal}{\kern0pt}\ f\ o\ g{\isachardoublequoteclose}\isanewline
\isanewline
\isacommand{declare}\isamarkupfalse%
\ slope{\isacharunderscore}{\kern0pt}comp{\isacharbrackleft}{\kern0pt}intro{\isacharcomma}{\kern0pt}\ simp{\isacharbrackright}{\kern0pt}\isanewline
\isacommand{declare}\isamarkupfalse%
\ slope{\isacharunderscore}{\kern0pt}scale{\isacharbrackleft}{\kern0pt}intro{\isacharcomma}{\kern0pt}\ simp{\isacharbrackright}{\kern0pt}\isanewline
\isanewline
\isacommand{quotient{\isacharunderscore}{\kern0pt}definition}\isamarkupfalse%
\isanewline
\ \ {\isachardoublequoteopen}{\isacharparenleft}{\kern0pt}{\isacharasterisk}{\kern0pt}{\isacharparenright}{\kern0pt}\ {\isacharcolon}{\kern0pt}{\isacharcolon}{\kern0pt}\ real\ {\isasymRightarrow}\ real\ {\isasymRightarrow}\ real{\isachardoublequoteclose}\ \isakeyword{is}\ {\isachardoublequoteopen}{\isacharparenleft}{\kern0pt}{\isacharasterisk}{\kern0pt}\isactrlsub e{\isacharparenright}{\kern0pt}{\isachardoublequoteclose}\isanewline
%
\isadelimproof
%
\endisadelimproof
%
\isatagproof
\isacommand{proof}\isamarkupfalse%
\ {\isacharminus}{\kern0pt}\isanewline
\ \ \isacommand{fix}\isamarkupfalse%
\ x\ x{\isacharprime}{\kern0pt}\ y\ y{\isacharprime}{\kern0pt}\ \isacommand{assume}\isamarkupfalse%
\ {\isachardoublequoteopen}x\ {\isasymsim}\isactrlsub e\ x{\isacharprime}{\kern0pt}{\isachardoublequoteclose}\ {\isachardoublequoteopen}y\ {\isasymsim}\isactrlsub e\ y{\isacharprime}{\kern0pt}{\isachardoublequoteclose}\isanewline
\ \ \isacommand{hence}\isamarkupfalse%
\ rel{\isacharunderscore}{\kern0pt}x{\isacharcolon}{\kern0pt}\ {\isachardoublequoteopen}slope\ x{\isachardoublequoteclose}\ {\isachardoublequoteopen}slope\ x{\isacharprime}{\kern0pt}{\isachardoublequoteclose}\ {\isachardoublequoteopen}bounded\ {\isacharparenleft}{\kern0pt}{\isasymlambda}z{\isachardot}{\kern0pt}\ x\ z\ {\isacharminus}{\kern0pt}\ x{\isacharprime}{\kern0pt}\ z{\isacharparenright}{\kern0pt}{\isachardoublequoteclose}\ \isakeyword{and}\ rel{\isacharunderscore}{\kern0pt}y{\isacharcolon}{\kern0pt}\ {\isachardoublequoteopen}slope\ y{\isachardoublequoteclose}\ {\isachardoublequoteopen}slope\ y{\isacharprime}{\kern0pt}{\isachardoublequoteclose}\ {\isachardoublequoteopen}bounded\ {\isacharparenleft}{\kern0pt}{\isasymlambda}z{\isachardot}{\kern0pt}\ y\ z\ {\isacharminus}{\kern0pt}\ y{\isacharprime}{\kern0pt}\ z{\isacharparenright}{\kern0pt}{\isachardoublequoteclose}\ \isacommand{unfolding}\isamarkupfalse%
\ eudoxus{\isacharunderscore}{\kern0pt}rel{\isacharunderscore}{\kern0pt}def\ \isacommand{by}\isamarkupfalse%
\ blast{\isacharplus}{\kern0pt}\isanewline
\isanewline
\ \ \isacommand{obtain}\isamarkupfalse%
\ C\ \isakeyword{where}\ x{\isacharprime}{\kern0pt}{\isacharunderscore}{\kern0pt}bound{\isacharcolon}{\kern0pt}\ {\isachardoublequoteopen}{\isasymbar}x{\isacharprime}{\kern0pt}\ {\isacharparenleft}{\kern0pt}m\ {\isacharplus}{\kern0pt}\ n{\isacharparenright}{\kern0pt}\ {\isacharminus}{\kern0pt}\ {\isacharparenleft}{\kern0pt}x{\isacharprime}{\kern0pt}\ m\ {\isacharplus}{\kern0pt}\ x{\isacharprime}{\kern0pt}\ n{\isacharparenright}{\kern0pt}{\isasymbar}\ {\isasymle}\ C{\isachardoublequoteclose}\ \isakeyword{for}\ m\ n\ \isacommand{using}\isamarkupfalse%
\ rel{\isacharunderscore}{\kern0pt}x{\isacharparenleft}{\kern0pt}{\isadigit{2}}{\isacharparenright}{\kern0pt}\ \isacommand{unfolding}\isamarkupfalse%
\ slope{\isacharunderscore}{\kern0pt}def\ \isacommand{by}\isamarkupfalse%
\ fastforce\isanewline
\ \ \isanewline
\ \ \isacommand{obtain}\isamarkupfalse%
\ A\ B\ \isakeyword{where}\ x{\isacharprime}{\kern0pt}{\isacharunderscore}{\kern0pt}lin{\isacharunderscore}{\kern0pt}bound{\isacharcolon}{\kern0pt}\ {\isachardoublequoteopen}{\isasymbar}x{\isacharprime}{\kern0pt}\ n{\isasymbar}\ {\isasymle}\ A\ {\isacharasterisk}{\kern0pt}\ {\isasymbar}n{\isasymbar}\ {\isacharplus}{\kern0pt}\ B{\isachardoublequoteclose}\ {\isachardoublequoteopen}{\isadigit{0}}\ {\isasymle}\ A{\isachardoublequoteclose}\ {\isachardoublequoteopen}{\isadigit{0}}\ {\isasymle}\ B{\isachardoublequoteclose}\ \isakeyword{for}\ n\ \isacommand{using}\isamarkupfalse%
\ slope{\isacharunderscore}{\kern0pt}linear{\isacharunderscore}{\kern0pt}bound{\isacharbrackleft}{\kern0pt}OF\ rel{\isacharunderscore}{\kern0pt}x{\isacharparenleft}{\kern0pt}{\isadigit{2}}{\isacharparenright}{\kern0pt}{\isacharbrackright}{\kern0pt}\ \isacommand{by}\isamarkupfalse%
\ blast\isanewline
\isanewline
\ \ \isacommand{obtain}\isamarkupfalse%
\ C{\isacharprime}{\kern0pt}\ \isakeyword{where}\ y{\isacharunderscore}{\kern0pt}y{\isacharprime}{\kern0pt}{\isacharunderscore}{\kern0pt}bound{\isacharcolon}{\kern0pt}\ {\isachardoublequoteopen}{\isasymbar}y\ z\ {\isacharminus}{\kern0pt}\ y{\isacharprime}{\kern0pt}\ z{\isasymbar}\ {\isasymle}\ C{\isacharprime}{\kern0pt}{\isachardoublequoteclose}\ \isakeyword{for}\ z\ \isacommand{using}\isamarkupfalse%
\ rel{\isacharunderscore}{\kern0pt}y{\isacharparenleft}{\kern0pt}{\isadigit{3}}{\isacharparenright}{\kern0pt}\ \isacommand{unfolding}\isamarkupfalse%
\ slope{\isacharunderscore}{\kern0pt}def\ \isacommand{by}\isamarkupfalse%
\ fastforce\isanewline
\isanewline
\ \ \isacommand{have}\isamarkupfalse%
\ {\isachardoublequoteopen}bounded\ {\isacharparenleft}{\kern0pt}{\isasymlambda}z{\isachardot}{\kern0pt}\ x{\isacharprime}{\kern0pt}\ {\isacharparenleft}{\kern0pt}y\ z{\isacharparenright}{\kern0pt}\ {\isacharminus}{\kern0pt}\ x{\isacharprime}{\kern0pt}\ {\isacharparenleft}{\kern0pt}y{\isacharprime}{\kern0pt}\ z{\isacharparenright}{\kern0pt}{\isacharparenright}{\kern0pt}{\isachardoublequoteclose}\ \isanewline
\ \ \isacommand{proof}\isamarkupfalse%
\ {\isacharparenleft}{\kern0pt}rule\ boundedI{\isacharparenright}{\kern0pt}\isanewline
\ \ \ \ \isacommand{fix}\isamarkupfalse%
\ z\isanewline
\ \ \ \ \isacommand{have}\isamarkupfalse%
\ {\isachardoublequoteopen}{\isasymbar}x{\isacharprime}{\kern0pt}\ {\isacharparenleft}{\kern0pt}y\ z{\isacharparenright}{\kern0pt}\ {\isacharminus}{\kern0pt}\ x{\isacharprime}{\kern0pt}\ {\isacharparenleft}{\kern0pt}y{\isacharprime}{\kern0pt}\ z{\isacharparenright}{\kern0pt}{\isasymbar}\ {\isasymle}\ {\isasymbar}x{\isacharprime}{\kern0pt}\ {\isacharparenleft}{\kern0pt}y\ z\ {\isacharminus}{\kern0pt}\ y{\isacharprime}{\kern0pt}\ z{\isacharparenright}{\kern0pt}{\isasymbar}\ {\isacharplus}{\kern0pt}\ C{\isachardoublequoteclose}\ \isacommand{using}\isamarkupfalse%
\ x{\isacharprime}{\kern0pt}{\isacharunderscore}{\kern0pt}bound{\isacharbrackleft}{\kern0pt}of\ {\isachardoublequoteopen}y\ z\ {\isacharminus}{\kern0pt}\ y{\isacharprime}{\kern0pt}\ z{\isachardoublequoteclose}\ {\isachardoublequoteopen}y{\isacharprime}{\kern0pt}\ z{\isachardoublequoteclose}{\isacharbrackright}{\kern0pt}\ \isacommand{by}\isamarkupfalse%
\ fastforce\isanewline
\ \ \ \ \isacommand{also}\isamarkupfalse%
\ \isacommand{have}\isamarkupfalse%
\ {\isachardoublequoteopen}{\isachardot}{\kern0pt}{\isachardot}{\kern0pt}{\isachardot}{\kern0pt}\ {\isasymle}\ A\ {\isacharasterisk}{\kern0pt}\ {\isasymbar}y\ z\ {\isacharminus}{\kern0pt}\ y{\isacharprime}{\kern0pt}\ z{\isasymbar}\ {\isacharplus}{\kern0pt}\ B\ {\isacharplus}{\kern0pt}\ C{\isachardoublequoteclose}\ \isacommand{using}\isamarkupfalse%
\ x{\isacharprime}{\kern0pt}{\isacharunderscore}{\kern0pt}lin{\isacharunderscore}{\kern0pt}bound\ \isacommand{by}\isamarkupfalse%
\ force\isanewline
\ \ \ \ \isacommand{also}\isamarkupfalse%
\ \isacommand{have}\isamarkupfalse%
\ {\isachardoublequoteopen}{\isachardot}{\kern0pt}{\isachardot}{\kern0pt}{\isachardot}{\kern0pt}\ {\isasymle}\ A\ {\isacharasterisk}{\kern0pt}\ C{\isacharprime}{\kern0pt}\ {\isacharplus}{\kern0pt}\ B\ {\isacharplus}{\kern0pt}\ C{\isachardoublequoteclose}\ \isacommand{using}\isamarkupfalse%
\ mult{\isacharunderscore}{\kern0pt}left{\isacharunderscore}{\kern0pt}mono{\isacharbrackleft}{\kern0pt}OF\ y{\isacharunderscore}{\kern0pt}y{\isacharprime}{\kern0pt}{\isacharunderscore}{\kern0pt}bound\ x{\isacharprime}{\kern0pt}{\isacharunderscore}{\kern0pt}lin{\isacharunderscore}{\kern0pt}bound{\isacharparenleft}{\kern0pt}{\isadigit{2}}{\isacharparenright}{\kern0pt}{\isacharbrackright}{\kern0pt}\ \isacommand{by}\isamarkupfalse%
\ fastforce\isanewline
\ \ \ \ \isacommand{finally}\isamarkupfalse%
\ \isacommand{show}\isamarkupfalse%
\ {\isachardoublequoteopen}{\isasymbar}x{\isacharprime}{\kern0pt}\ {\isacharparenleft}{\kern0pt}y\ z{\isacharparenright}{\kern0pt}\ {\isacharminus}{\kern0pt}\ x{\isacharprime}{\kern0pt}\ {\isacharparenleft}{\kern0pt}y{\isacharprime}{\kern0pt}\ z{\isacharparenright}{\kern0pt}{\isasymbar}\ {\isasymle}\ A\ {\isacharasterisk}{\kern0pt}\ C{\isacharprime}{\kern0pt}\ {\isacharplus}{\kern0pt}\ B\ {\isacharplus}{\kern0pt}\ C{\isachardoublequoteclose}\ \isacommand{by}\isamarkupfalse%
\ blast\isanewline
\ \ \isacommand{qed}\isamarkupfalse%
\isanewline
\ \ \isacommand{hence}\isamarkupfalse%
\ {\isachardoublequoteopen}bounded\ {\isacharparenleft}{\kern0pt}{\isasymlambda}z{\isachardot}{\kern0pt}\ x\ {\isacharparenleft}{\kern0pt}y\ z{\isacharparenright}{\kern0pt}\ {\isacharminus}{\kern0pt}\ x{\isacharprime}{\kern0pt}\ {\isacharparenleft}{\kern0pt}y{\isacharprime}{\kern0pt}\ z{\isacharparenright}{\kern0pt}{\isacharparenright}{\kern0pt}{\isachardoublequoteclose}\ \isacommand{using}\isamarkupfalse%
\ bounded{\isacharunderscore}{\kern0pt}add{\isacharbrackleft}{\kern0pt}OF\ bounded{\isacharunderscore}{\kern0pt}comp{\isacharparenleft}{\kern0pt}{\isadigit{1}}{\isacharparenright}{\kern0pt}{\isacharbrackleft}{\kern0pt}OF\ rel{\isacharunderscore}{\kern0pt}x{\isacharparenleft}{\kern0pt}{\isadigit{3}}{\isacharparenright}{\kern0pt}{\isacharcomma}{\kern0pt}\ of\ y{\isacharbrackright}{\kern0pt}{\isacharbrackright}{\kern0pt}\ \isacommand{by}\isamarkupfalse%
\ force\isanewline
\ \ \isacommand{thus}\isamarkupfalse%
\ {\isachardoublequoteopen}{\isacharparenleft}{\kern0pt}x\ {\isacharasterisk}{\kern0pt}\isactrlsub e\ y{\isacharparenright}{\kern0pt}\ {\isasymsim}\isactrlsub e\ {\isacharparenleft}{\kern0pt}x{\isacharprime}{\kern0pt}\ {\isacharasterisk}{\kern0pt}\isactrlsub e\ y{\isacharprime}{\kern0pt}{\isacharparenright}{\kern0pt}{\isachardoublequoteclose}\ \isacommand{unfolding}\isamarkupfalse%
\ eudoxus{\isacharunderscore}{\kern0pt}rel{\isacharunderscore}{\kern0pt}def\ eudoxus{\isacharunderscore}{\kern0pt}times{\isacharunderscore}{\kern0pt}def\ \isacommand{using}\isamarkupfalse%
\ rel{\isacharunderscore}{\kern0pt}x\ rel{\isacharunderscore}{\kern0pt}y\ \isacommand{by}\isamarkupfalse%
\ simp\isanewline
\isacommand{qed}\isamarkupfalse%
%
\endisatagproof
{\isafoldproof}%
%
\isadelimproof
\isanewline
%
\endisadelimproof
\isanewline
\isacommand{lemmas}\isamarkupfalse%
\ eudoxus{\isacharunderscore}{\kern0pt}times{\isacharunderscore}{\kern0pt}cong\ {\isacharequal}{\kern0pt}\ apply{\isacharunderscore}{\kern0pt}rsp{\isacharprime}{\kern0pt}{\isacharbrackleft}{\kern0pt}OF\ times{\isacharunderscore}{\kern0pt}real{\isachardot}{\kern0pt}rsp{\isacharcomma}{\kern0pt}\ THEN\ rel{\isacharunderscore}{\kern0pt}funD{\isacharcomma}{\kern0pt}\ intro{\isacharbrackright}{\kern0pt}\isanewline
\isacommand{lemmas}\isamarkupfalse%
\ eudoxus{\isacharunderscore}{\kern0pt}rel{\isacharunderscore}{\kern0pt}comp\ {\isacharequal}{\kern0pt}\ eudoxus{\isacharunderscore}{\kern0pt}times{\isacharunderscore}{\kern0pt}cong{\isacharbrackleft}{\kern0pt}unfolded\ eudoxus{\isacharunderscore}{\kern0pt}times{\isacharunderscore}{\kern0pt}def{\isacharbrackright}{\kern0pt}\isanewline
\isanewline
\isacommand{lemma}\isamarkupfalse%
\ eudoxus{\isacharunderscore}{\kern0pt}times{\isacharunderscore}{\kern0pt}commute{\isacharcolon}{\kern0pt}\isanewline
\ \ \isakeyword{assumes}\ {\isachardoublequoteopen}slope\ f{\isachardoublequoteclose}\ {\isachardoublequoteopen}slope\ g{\isachardoublequoteclose}\isanewline
\ \ \isakeyword{shows}\ {\isachardoublequoteopen}{\isacharparenleft}{\kern0pt}f\ {\isacharasterisk}{\kern0pt}\isactrlsub e\ g{\isacharparenright}{\kern0pt}\ {\isasymsim}\isactrlsub e\ {\isacharparenleft}{\kern0pt}g\ {\isacharasterisk}{\kern0pt}\isactrlsub e\ f{\isacharparenright}{\kern0pt}{\isachardoublequoteclose}\isanewline
%
\isadelimproof
\ \ %
\endisadelimproof
%
\isatagproof
\isacommand{unfolding}\isamarkupfalse%
\ eudoxus{\isacharunderscore}{\kern0pt}rel{\isacharunderscore}{\kern0pt}def\ eudoxus{\isacharunderscore}{\kern0pt}times{\isacharunderscore}{\kern0pt}def\isanewline
\ \ \isacommand{using}\isamarkupfalse%
\ slope{\isacharunderscore}{\kern0pt}comp\ slope{\isacharunderscore}{\kern0pt}comp{\isacharunderscore}{\kern0pt}commute\ assms\ \isacommand{by}\isamarkupfalse%
\ blast%
\endisatagproof
{\isafoldproof}%
%
\isadelimproof
\isanewline
%
\endisadelimproof
\isanewline
\isacommand{lemma}\isamarkupfalse%
\ abs{\isacharunderscore}{\kern0pt}real{\isacharunderscore}{\kern0pt}times{\isacharbrackleft}{\kern0pt}simp{\isacharbrackright}{\kern0pt}{\isacharcolon}{\kern0pt}\ \isanewline
\ \ \isakeyword{assumes}\ {\isachardoublequoteopen}slope\ f{\isachardoublequoteclose}\ {\isachardoublequoteopen}slope\ g{\isachardoublequoteclose}\isanewline
\ \ \isakeyword{shows}\ {\isachardoublequoteopen}abs{\isacharunderscore}{\kern0pt}real\ f\ {\isacharasterisk}{\kern0pt}\ abs{\isacharunderscore}{\kern0pt}real\ g\ {\isacharequal}{\kern0pt}\ abs{\isacharunderscore}{\kern0pt}real\ {\isacharparenleft}{\kern0pt}f\ {\isacharasterisk}{\kern0pt}\isactrlsub e\ g{\isacharparenright}{\kern0pt}{\isachardoublequoteclose}\isanewline
%
\isadelimproof
\ \ %
\endisadelimproof
%
\isatagproof
\isacommand{using}\isamarkupfalse%
\ assms\ \isacommand{unfolding}\isamarkupfalse%
\ times{\isacharunderscore}{\kern0pt}real{\isacharunderscore}{\kern0pt}def\ \isacommand{by}\isamarkupfalse%
\ auto%
\endisatagproof
{\isafoldproof}%
%
\isadelimproof
\isanewline
%
\endisadelimproof
\isanewline
\isacommand{instance}\isamarkupfalse%
%
\isadelimproof
\ %
\endisadelimproof
%
\isatagproof
\isacommand{{\isachardot}{\kern0pt}{\isachardot}{\kern0pt}}\isamarkupfalse%
%
\endisatagproof
{\isafoldproof}%
%
\isadelimproof
%
\endisadelimproof
\isanewline
\isacommand{end}\isamarkupfalse%
\isanewline
\isanewline
\isacommand{lemma}\isamarkupfalse%
\ neg{\isacharunderscore}{\kern0pt}one{\isacharunderscore}{\kern0pt}def{\isacharcolon}{\kern0pt}\ {\isachardoublequoteopen}{\isacharminus}{\kern0pt}\ {\isadigit{1}}\ {\isacharequal}{\kern0pt}\ abs{\isacharunderscore}{\kern0pt}real\ {\isacharparenleft}{\kern0pt}{\isacharminus}{\kern0pt}\isactrlsub e\ id{\isacharparenright}{\kern0pt}{\isachardoublequoteclose}%
\isadelimproof
\ %
\endisadelimproof
%
\isatagproof
\isacommand{unfolding}\isamarkupfalse%
\ one{\isacharunderscore}{\kern0pt}real{\isacharunderscore}{\kern0pt}def\ \isacommand{by}\isamarkupfalse%
\ {\isacharparenleft}{\kern0pt}simp\ add{\isacharcolon}{\kern0pt}\ eudoxus{\isacharunderscore}{\kern0pt}uminus{\isacharunderscore}{\kern0pt}def{\isacharparenright}{\kern0pt}%
\endisatagproof
{\isafoldproof}%
%
\isadelimproof
%
\endisadelimproof
\isanewline
\isacommand{lemma}\isamarkupfalse%
\ slope{\isacharunderscore}{\kern0pt}neg{\isacharunderscore}{\kern0pt}one{\isacharbrackleft}{\kern0pt}intro{\isacharcomma}{\kern0pt}\ simp{\isacharbrackright}{\kern0pt}{\isacharcolon}{\kern0pt}\ {\isachardoublequoteopen}slope\ {\isacharparenleft}{\kern0pt}{\isacharminus}{\kern0pt}\isactrlsub e\ id{\isacharparenright}{\kern0pt}{\isachardoublequoteclose}%
\isadelimproof
\ %
\endisadelimproof
%
\isatagproof
\isacommand{using}\isamarkupfalse%
\ slope{\isacharunderscore}{\kern0pt}refl\ \isacommand{by}\isamarkupfalse%
\ blast%
\endisatagproof
{\isafoldproof}%
%
\isadelimproof
%
\endisadelimproof
%
\begin{isamarkuptext}%
With the definitions provided above, the Eudoxus reals are a commutative ring with unity.%
\end{isamarkuptext}\isamarkuptrue%
\isacommand{instance}\isamarkupfalse%
\ real\ {\isacharcolon}{\kern0pt}{\isacharcolon}{\kern0pt}\ comm{\isacharunderscore}{\kern0pt}ring{\isacharunderscore}{\kern0pt}{\isadigit{1}}\isanewline
%
\isadelimproof
%
\endisadelimproof
%
\isatagproof
\isacommand{proof}\isamarkupfalse%
\isanewline
\ \ \isacommand{fix}\isamarkupfalse%
\ x\ y\ z\ {\isacharcolon}{\kern0pt}{\isacharcolon}{\kern0pt}\ real\isanewline
\ \ \isacommand{show}\isamarkupfalse%
\ {\isachardoublequoteopen}x\ {\isacharasterisk}{\kern0pt}\ y\ {\isacharasterisk}{\kern0pt}\ z\ {\isacharequal}{\kern0pt}\ x\ {\isacharasterisk}{\kern0pt}\ {\isacharparenleft}{\kern0pt}y\ {\isacharasterisk}{\kern0pt}\ z{\isacharparenright}{\kern0pt}{\isachardoublequoteclose}\ \isacommand{by}\isamarkupfalse%
\ {\isacharparenleft}{\kern0pt}induct\ x{\isacharcomma}{\kern0pt}\ induct\ y{\isacharcomma}{\kern0pt}\ induct\ z{\isacharparenright}{\kern0pt}\ {\isacharparenleft}{\kern0pt}simp\ add{\isacharcolon}{\kern0pt}\ eudoxus{\isacharunderscore}{\kern0pt}times{\isacharunderscore}{\kern0pt}cong\ eudoxus{\isacharunderscore}{\kern0pt}times{\isacharunderscore}{\kern0pt}def\ comp{\isacharunderscore}{\kern0pt}assoc{\isacharparenright}{\kern0pt}\isanewline
\ \ \isacommand{show}\isamarkupfalse%
\ {\isachardoublequoteopen}x\ {\isacharasterisk}{\kern0pt}\ y\ {\isacharequal}{\kern0pt}\ y\ {\isacharasterisk}{\kern0pt}\ x{\isachardoublequoteclose}\ \isacommand{by}\isamarkupfalse%
\ {\isacharparenleft}{\kern0pt}induct\ x{\isacharcomma}{\kern0pt}\ induct\ y{\isacharparenright}{\kern0pt}\ {\isacharparenleft}{\kern0pt}force\ simp\ add{\isacharcolon}{\kern0pt}\ slope{\isacharunderscore}{\kern0pt}refl\ eudoxus{\isacharunderscore}{\kern0pt}times{\isacharunderscore}{\kern0pt}commute{\isacharparenright}{\kern0pt}\isanewline
\ \ \isacommand{show}\isamarkupfalse%
\ {\isachardoublequoteopen}{\isadigit{1}}\ {\isacharasterisk}{\kern0pt}\ x\ {\isacharequal}{\kern0pt}\ x{\isachardoublequoteclose}\ \isacommand{by}\isamarkupfalse%
\ {\isacharparenleft}{\kern0pt}induct\ x{\isacharparenright}{\kern0pt}\ {\isacharparenleft}{\kern0pt}simp\ add{\isacharcolon}{\kern0pt}\ one{\isacharunderscore}{\kern0pt}real{\isacharunderscore}{\kern0pt}def\ eudoxus{\isacharunderscore}{\kern0pt}times{\isacharunderscore}{\kern0pt}def{\isacharparenright}{\kern0pt}\isanewline
\ \ \isacommand{show}\isamarkupfalse%
\ {\isachardoublequoteopen}{\isacharparenleft}{\kern0pt}x\ {\isacharplus}{\kern0pt}\ y{\isacharparenright}{\kern0pt}\ {\isacharasterisk}{\kern0pt}\ z\ {\isacharequal}{\kern0pt}\ x\ {\isacharasterisk}{\kern0pt}\ z\ {\isacharplus}{\kern0pt}\ y\ {\isacharasterisk}{\kern0pt}\ z{\isachardoublequoteclose}\ \isacommand{by}\isamarkupfalse%
\ {\isacharparenleft}{\kern0pt}induct\ x{\isacharcomma}{\kern0pt}\ induct\ y{\isacharcomma}{\kern0pt}\ induct\ z{\isacharparenright}{\kern0pt}\ {\isacharparenleft}{\kern0pt}simp\ add{\isacharcolon}{\kern0pt}\ eudoxus{\isacharunderscore}{\kern0pt}times{\isacharunderscore}{\kern0pt}cong\ eudoxus{\isacharunderscore}{\kern0pt}plus{\isacharunderscore}{\kern0pt}cong{\isacharcomma}{\kern0pt}\ simp\ add{\isacharcolon}{\kern0pt}\ eudoxus{\isacharunderscore}{\kern0pt}times{\isacharunderscore}{\kern0pt}def\ eudoxus{\isacharunderscore}{\kern0pt}plus{\isacharunderscore}{\kern0pt}def\ comp{\isacharunderscore}{\kern0pt}def{\isacharparenright}{\kern0pt}\isanewline
\ \ \isacommand{have}\isamarkupfalse%
\ {\isachardoublequoteopen}{\isasymnot}bounded\ {\isacharparenleft}{\kern0pt}{\isasymlambda}x{\isachardot}{\kern0pt}\ x{\isacharparenright}{\kern0pt}{\isachardoublequoteclose}\ \isacommand{by}\isamarkupfalse%
\ {\isacharparenleft}{\kern0pt}metis\ add{\isachardot}{\kern0pt}inverse{\isacharunderscore}{\kern0pt}inverse\ boundedE{\isacharunderscore}{\kern0pt}strict\ less{\isacharunderscore}{\kern0pt}irrefl\ neg{\isacharunderscore}{\kern0pt}less{\isacharunderscore}{\kern0pt}{\isadigit{0}}{\isacharunderscore}{\kern0pt}iff{\isacharunderscore}{\kern0pt}less\ zabs{\isacharunderscore}{\kern0pt}def{\isacharparenright}{\kern0pt}\isanewline
\ \ \isacommand{thus}\isamarkupfalse%
\ {\isachardoublequoteopen}{\isacharparenleft}{\kern0pt}{\isadigit{0}}\ {\isacharcolon}{\kern0pt}{\isacharcolon}{\kern0pt}\ real{\isacharparenright}{\kern0pt}\ {\isasymnoteq}\ {\isacharparenleft}{\kern0pt}{\isadigit{1}}\ {\isacharcolon}{\kern0pt}{\isacharcolon}{\kern0pt}\ real{\isacharparenright}{\kern0pt}{\isachardoublequoteclose}\ \isacommand{using}\isamarkupfalse%
\ abs{\isacharunderscore}{\kern0pt}real{\isacharunderscore}{\kern0pt}eq{\isacharunderscore}{\kern0pt}iff{\isacharbrackleft}{\kern0pt}of\ id\ {\isachardoublequoteopen}{\isasymlambda}{\isacharunderscore}{\kern0pt}{\isachardot}{\kern0pt}\ {\isadigit{0}}{\isachardoublequoteclose}{\isacharbrackright}{\kern0pt}\ \isacommand{unfolding}\isamarkupfalse%
\ one{\isacharunderscore}{\kern0pt}real{\isacharunderscore}{\kern0pt}def\ zero{\isacharunderscore}{\kern0pt}real{\isacharunderscore}{\kern0pt}def\ eudoxus{\isacharunderscore}{\kern0pt}rel{\isacharunderscore}{\kern0pt}def\ \isacommand{by}\isamarkupfalse%
\ simp\isanewline
\isacommand{qed}\isamarkupfalse%
%
\endisatagproof
{\isafoldproof}%
%
\isadelimproof
\isanewline
%
\endisadelimproof
\isanewline
\isacommand{lemma}\isamarkupfalse%
\ real{\isacharunderscore}{\kern0pt}of{\isacharunderscore}{\kern0pt}nat{\isacharcolon}{\kern0pt}\isanewline
\ \ {\isachardoublequoteopen}of{\isacharunderscore}{\kern0pt}nat\ n\ {\isacharequal}{\kern0pt}\ abs{\isacharunderscore}{\kern0pt}real\ {\isacharparenleft}{\kern0pt}{\isacharparenleft}{\kern0pt}{\isacharasterisk}{\kern0pt}{\isacharparenright}{\kern0pt}\ {\isacharparenleft}{\kern0pt}of{\isacharunderscore}{\kern0pt}nat\ n{\isacharparenright}{\kern0pt}{\isacharparenright}{\kern0pt}{\isachardoublequoteclose}\isanewline
%
\isadelimproof
%
\endisadelimproof
%
\isatagproof
\isacommand{proof}\isamarkupfalse%
\ {\isacharparenleft}{\kern0pt}induction\ n{\isacharparenright}{\kern0pt}\isanewline
\ \ \isacommand{case}\isamarkupfalse%
\ {\isadigit{0}}\isanewline
\ \ \isacommand{then}\isamarkupfalse%
\ \isacommand{show}\isamarkupfalse%
\ {\isacharquery}{\kern0pt}case\ \isacommand{by}\isamarkupfalse%
\ {\isacharparenleft}{\kern0pt}simp\ add{\isacharcolon}{\kern0pt}\ zero{\isacharunderscore}{\kern0pt}real{\isacharunderscore}{\kern0pt}def{\isacharparenright}{\kern0pt}\isanewline
\isacommand{next}\isamarkupfalse%
\isanewline
\ \ \isacommand{case}\isamarkupfalse%
\ {\isacharparenleft}{\kern0pt}Suc\ n{\isacharparenright}{\kern0pt}\isanewline
\ \ \isacommand{then}\isamarkupfalse%
\ \isacommand{show}\isamarkupfalse%
\ {\isacharquery}{\kern0pt}case\ \isacommand{by}\isamarkupfalse%
\ {\isacharparenleft}{\kern0pt}simp\ add{\isacharcolon}{\kern0pt}\ one{\isacharunderscore}{\kern0pt}real{\isacharunderscore}{\kern0pt}def\ distrib{\isacharunderscore}{\kern0pt}right\ eudoxus{\isacharunderscore}{\kern0pt}plus{\isacharunderscore}{\kern0pt}def{\isacharparenright}{\kern0pt}\isanewline
\isacommand{qed}\isamarkupfalse%
%
\endisatagproof
{\isafoldproof}%
%
\isadelimproof
\isanewline
%
\endisadelimproof
\isanewline
\isacommand{lemma}\isamarkupfalse%
\ real{\isacharunderscore}{\kern0pt}of{\isacharunderscore}{\kern0pt}int{\isacharcolon}{\kern0pt}\isanewline
\ \ {\isachardoublequoteopen}of{\isacharunderscore}{\kern0pt}int\ z\ {\isacharequal}{\kern0pt}\ abs{\isacharunderscore}{\kern0pt}real\ {\isacharparenleft}{\kern0pt}{\isacharparenleft}{\kern0pt}{\isacharasterisk}{\kern0pt}{\isacharparenright}{\kern0pt}\ z{\isacharparenright}{\kern0pt}{\isachardoublequoteclose}\isanewline
%
\isadelimproof
%
\endisadelimproof
%
\isatagproof
\isacommand{proof}\isamarkupfalse%
\ {\isacharparenleft}{\kern0pt}induction\ z\ rule{\isacharcolon}{\kern0pt}\ int{\isacharunderscore}{\kern0pt}induct{\isacharbrackleft}{\kern0pt}\isakeyword{where}\ {\isacharquery}{\kern0pt}k{\isacharequal}{\kern0pt}{\isadigit{0}}{\isacharbrackright}{\kern0pt}{\isacharparenright}{\kern0pt}\isanewline
\ \ \isacommand{case}\isamarkupfalse%
\ base\isanewline
\ \ \isacommand{then}\isamarkupfalse%
\ \isacommand{show}\isamarkupfalse%
\ {\isacharquery}{\kern0pt}case\ \isacommand{by}\isamarkupfalse%
\ {\isacharparenleft}{\kern0pt}simp\ add{\isacharcolon}{\kern0pt}\ zero{\isacharunderscore}{\kern0pt}real{\isacharunderscore}{\kern0pt}def{\isacharparenright}{\kern0pt}\isanewline
\isacommand{next}\isamarkupfalse%
\isanewline
\ \ \isacommand{case}\isamarkupfalse%
\ {\isacharparenleft}{\kern0pt}step{\isadigit{1}}\ i{\isacharparenright}{\kern0pt}\isanewline
\ \ \isacommand{then}\isamarkupfalse%
\ \isacommand{show}\isamarkupfalse%
\ {\isacharquery}{\kern0pt}case\ \isacommand{by}\isamarkupfalse%
\ {\isacharparenleft}{\kern0pt}simp\ add{\isacharcolon}{\kern0pt}\ one{\isacharunderscore}{\kern0pt}real{\isacharunderscore}{\kern0pt}def\ distrib{\isacharunderscore}{\kern0pt}right\ eudoxus{\isacharunderscore}{\kern0pt}plus{\isacharunderscore}{\kern0pt}def{\isacharparenright}{\kern0pt}\isanewline
\isacommand{next}\isamarkupfalse%
\isanewline
\ \ \isacommand{case}\isamarkupfalse%
\ {\isacharparenleft}{\kern0pt}step{\isadigit{2}}\ i{\isacharparenright}{\kern0pt}\isanewline
\ \ \isacommand{then}\isamarkupfalse%
\ \isacommand{show}\isamarkupfalse%
\ {\isacharquery}{\kern0pt}case\ \isacommand{by}\isamarkupfalse%
\ {\isacharparenleft}{\kern0pt}simp\ add{\isacharcolon}{\kern0pt}\ one{\isacharunderscore}{\kern0pt}real{\isacharunderscore}{\kern0pt}def\ eudoxus{\isacharunderscore}{\kern0pt}plus{\isacharunderscore}{\kern0pt}def\ left{\isacharunderscore}{\kern0pt}diff{\isacharunderscore}{\kern0pt}distrib\ eudoxus{\isacharunderscore}{\kern0pt}uminus{\isacharunderscore}{\kern0pt}def{\isacharparenright}{\kern0pt}\isanewline
\isacommand{qed}\isamarkupfalse%
%
\endisatagproof
{\isafoldproof}%
%
\isadelimproof
%
\endisadelimproof
%
\begin{isamarkuptext}%
The Eudoxus reals are a ring of characteristic \isa{{\isadigit{0}}{\isacharcolon}{\kern0pt}{\isacharcolon}{\kern0pt}{\isacharprime}{\kern0pt}a}.%
\end{isamarkuptext}\isamarkuptrue%
\isacommand{instance}\isamarkupfalse%
\ real\ {\isacharcolon}{\kern0pt}{\isacharcolon}{\kern0pt}\ ring{\isacharunderscore}{\kern0pt}char{\isacharunderscore}{\kern0pt}{\isadigit{0}}\isanewline
%
\isadelimproof
%
\endisadelimproof
%
\isatagproof
\isacommand{proof}\isamarkupfalse%
\isanewline
\ \ \isacommand{show}\isamarkupfalse%
\ {\isachardoublequoteopen}inj\ {\isacharparenleft}{\kern0pt}{\isasymlambda}n{\isachardot}{\kern0pt}\ of{\isacharunderscore}{\kern0pt}nat\ n\ {\isacharcolon}{\kern0pt}{\isacharcolon}{\kern0pt}\ real{\isacharparenright}{\kern0pt}{\isachardoublequoteclose}\isanewline
\ \ \isacommand{proof}\isamarkupfalse%
\ {\isacharparenleft}{\kern0pt}intro\ inj{\isacharunderscore}{\kern0pt}onI{\isacharparenright}{\kern0pt}\isanewline
\ \ \ \ \isacommand{fix}\isamarkupfalse%
\ x\ y\ \isacommand{assume}\isamarkupfalse%
\ {\isachardoublequoteopen}{\isacharparenleft}{\kern0pt}of{\isacharunderscore}{\kern0pt}nat\ x\ {\isacharcolon}{\kern0pt}{\isacharcolon}{\kern0pt}\ real{\isacharparenright}{\kern0pt}\ {\isacharequal}{\kern0pt}\ of{\isacharunderscore}{\kern0pt}nat\ y{\isachardoublequoteclose}\isanewline
\ \ \ \ \isacommand{hence}\isamarkupfalse%
\ {\isachardoublequoteopen}{\isacharparenleft}{\kern0pt}{\isacharparenleft}{\kern0pt}{\isacharasterisk}{\kern0pt}{\isacharparenright}{\kern0pt}\ {\isacharparenleft}{\kern0pt}int\ x{\isacharparenright}{\kern0pt}{\isacharparenright}{\kern0pt}\ {\isasymsim}\isactrlsub e\ {\isacharparenleft}{\kern0pt}{\isacharparenleft}{\kern0pt}{\isacharasterisk}{\kern0pt}{\isacharparenright}{\kern0pt}\ {\isacharparenleft}{\kern0pt}int\ y{\isacharparenright}{\kern0pt}{\isacharparenright}{\kern0pt}{\isachardoublequoteclose}\ \isacommand{unfolding}\isamarkupfalse%
\ abs{\isacharunderscore}{\kern0pt}real{\isacharunderscore}{\kern0pt}eq{\isacharunderscore}{\kern0pt}iff\ real{\isacharunderscore}{\kern0pt}of{\isacharunderscore}{\kern0pt}nat\ \isacommand{using}\isamarkupfalse%
\ slope{\isacharunderscore}{\kern0pt}scale\ \isacommand{by}\isamarkupfalse%
\ blast\isanewline
\ \ \ \ \isacommand{hence}\isamarkupfalse%
\ {\isachardoublequoteopen}bounded\ {\isacharparenleft}{\kern0pt}{\isasymlambda}z{\isachardot}{\kern0pt}\ {\isacharparenleft}{\kern0pt}int\ x\ {\isacharminus}{\kern0pt}\ int\ y{\isacharparenright}{\kern0pt}\ {\isacharasterisk}{\kern0pt}\ z{\isacharparenright}{\kern0pt}{\isachardoublequoteclose}\ \isacommand{unfolding}\isamarkupfalse%
\ eudoxus{\isacharunderscore}{\kern0pt}rel{\isacharunderscore}{\kern0pt}def\ \isacommand{by}\isamarkupfalse%
\ {\isacharparenleft}{\kern0pt}simp\ add{\isacharcolon}{\kern0pt}\ left{\isacharunderscore}{\kern0pt}diff{\isacharunderscore}{\kern0pt}distrib{\isacharparenright}{\kern0pt}\isanewline
\ \ \ \ \isacommand{then}\isamarkupfalse%
\ \isacommand{obtain}\isamarkupfalse%
\ C\ \isakeyword{where}\ bound{\isacharcolon}{\kern0pt}\ {\isachardoublequoteopen}{\isasymbar}{\isacharparenleft}{\kern0pt}int\ x\ {\isacharminus}{\kern0pt}\ int\ y{\isacharparenright}{\kern0pt}\ {\isacharasterisk}{\kern0pt}\ z{\isasymbar}\ {\isasymle}\ C{\isachardoublequoteclose}\ \isakeyword{and}\ C{\isacharunderscore}{\kern0pt}nonneg{\isacharcolon}{\kern0pt}\ {\isachardoublequoteopen}{\isadigit{0}}\ {\isasymle}\ C{\isachardoublequoteclose}\ \isakeyword{for}\ z\ \isacommand{by}\isamarkupfalse%
\ blast\isanewline
\ \ \ \ \isacommand{hence}\isamarkupfalse%
\ {\isachardoublequoteopen}{\isasymbar}int\ x\ {\isacharminus}{\kern0pt}\ int\ y{\isasymbar}\ {\isacharasterisk}{\kern0pt}\ {\isasymbar}C\ {\isacharplus}{\kern0pt}\ {\isadigit{1}}{\isasymbar}\ {\isasymle}\ C{\isachardoublequoteclose}\ \ \isacommand{using}\isamarkupfalse%
\ abs{\isacharunderscore}{\kern0pt}mult\ \isacommand{by}\isamarkupfalse%
\ metis\isanewline
\ \ \ \ \isacommand{hence}\isamarkupfalse%
\ {\isacharasterisk}{\kern0pt}{\isacharcolon}{\kern0pt}\ {\isachardoublequoteopen}{\isasymbar}int\ x\ {\isacharminus}{\kern0pt}\ int\ y{\isasymbar}\ {\isacharasterisk}{\kern0pt}\ {\isacharparenleft}{\kern0pt}C\ {\isacharplus}{\kern0pt}\ {\isadigit{1}}{\isacharparenright}{\kern0pt}\ {\isasymle}\ C{\isachardoublequoteclose}\ \isacommand{using}\isamarkupfalse%
\ C{\isacharunderscore}{\kern0pt}nonneg\ \isacommand{by}\isamarkupfalse%
\ force\isanewline
\ \ \ \ \isacommand{thus}\isamarkupfalse%
\ {\isachardoublequoteopen}x\ {\isacharequal}{\kern0pt}\ y{\isachardoublequoteclose}\ \isacommand{using}\isamarkupfalse%
\ order{\isacharunderscore}{\kern0pt}trans{\isacharbrackleft}{\kern0pt}OF\ mult{\isacharunderscore}{\kern0pt}right{\isacharunderscore}{\kern0pt}mono\ {\isacharasterisk}{\kern0pt}{\isacharcomma}{\kern0pt}\ of\ {\isadigit{1}}{\isacharbrackright}{\kern0pt}\ C{\isacharunderscore}{\kern0pt}nonneg\ \isacommand{by}\isamarkupfalse%
\ fastforce\isanewline
\ \ \isacommand{qed}\isamarkupfalse%
\isanewline
\isacommand{qed}\isamarkupfalse%
%
\endisatagproof
{\isafoldproof}%
%
\isadelimproof
%
\endisadelimproof
%
\isadelimdocument
%
\endisadelimdocument
%
\isatagdocument
%
\isamarkupsubsection{Ordering%
}
\isamarkuptrue%
%
\endisatagdocument
{\isafolddocument}%
%
\isadelimdocument
%
\endisadelimdocument
%
\begin{isamarkuptext}%
We call a slope positive, if it tends to infinity. Similarly, we call a slope negative if it tends to negative infinity.%
\end{isamarkuptext}\isamarkuptrue%
\isacommand{instantiation}\isamarkupfalse%
\ real\ {\isacharcolon}{\kern0pt}{\isacharcolon}{\kern0pt}\ {\isachardoublequoteopen}{\isacharbraceleft}{\kern0pt}ord{\isacharcomma}{\kern0pt}\ abs{\isacharcomma}{\kern0pt}\ sgn{\isacharbraceright}{\kern0pt}{\isachardoublequoteclose}\isanewline
\isakeyword{begin}\isanewline
\isanewline
\isacommand{definition}\isamarkupfalse%
\ pos\ {\isacharcolon}{\kern0pt}{\isacharcolon}{\kern0pt}\ {\isachardoublequoteopen}{\isacharparenleft}{\kern0pt}int\ {\isasymRightarrow}\ int{\isacharparenright}{\kern0pt}\ {\isasymRightarrow}\ bool{\isachardoublequoteclose}\ \isakeyword{where}\isanewline
\ \ {\isachardoublequoteopen}pos\ f\ {\isacharequal}{\kern0pt}\ {\isacharparenleft}{\kern0pt}{\isasymforall}C\ {\isasymge}\ {\isadigit{0}}{\isachardot}{\kern0pt}\ {\isasymexists}N{\isachardot}{\kern0pt}\ {\isasymforall}n\ {\isasymge}\ N{\isachardot}{\kern0pt}\ f\ n\ {\isasymge}\ C{\isacharparenright}{\kern0pt}{\isachardoublequoteclose}\isanewline
\isanewline
\isacommand{definition}\isamarkupfalse%
\ neg\ {\isacharcolon}{\kern0pt}{\isacharcolon}{\kern0pt}\ {\isachardoublequoteopen}{\isacharparenleft}{\kern0pt}int\ {\isasymRightarrow}\ int{\isacharparenright}{\kern0pt}\ {\isasymRightarrow}\ bool{\isachardoublequoteclose}\ \isakeyword{where}\isanewline
\ \ {\isachardoublequoteopen}neg\ f\ {\isacharequal}{\kern0pt}\ {\isacharparenleft}{\kern0pt}{\isasymforall}C\ {\isasymge}\ {\isadigit{0}}{\isachardot}{\kern0pt}\ {\isasymexists}N{\isachardot}{\kern0pt}\ {\isasymforall}n\ {\isasymge}\ N{\isachardot}{\kern0pt}\ f\ n\ {\isasymle}\ {\isacharminus}{\kern0pt}C{\isacharparenright}{\kern0pt}{\isachardoublequoteclose}\isanewline
\ \ \ \ \ \ \ \ \ \ \ \ \ \ \ \ \ \ \ \ \ \ \ \ \ \ \ \ \ \ \ \ \ \ \isanewline
\isacommand{lemma}\isamarkupfalse%
\ pos{\isacharunderscore}{\kern0pt}neg{\isacharunderscore}{\kern0pt}exclusive{\isacharcolon}{\kern0pt}\ {\isachardoublequoteopen}{\isasymnot}\ {\isacharparenleft}{\kern0pt}pos\ f\ {\isasymand}\ neg\ f{\isacharparenright}{\kern0pt}{\isachardoublequoteclose}%
\isadelimproof
\ %
\endisadelimproof
%
\isatagproof
\isacommand{unfolding}\isamarkupfalse%
\ neg{\isacharunderscore}{\kern0pt}def\ pos{\isacharunderscore}{\kern0pt}def\ \isacommand{by}\isamarkupfalse%
\ {\isacharparenleft}{\kern0pt}metis\ int{\isacharunderscore}{\kern0pt}one{\isacharunderscore}{\kern0pt}le{\isacharunderscore}{\kern0pt}iff{\isacharunderscore}{\kern0pt}zero{\isacharunderscore}{\kern0pt}less\ linorder{\isacharunderscore}{\kern0pt}not{\isacharunderscore}{\kern0pt}less\ nle{\isacharunderscore}{\kern0pt}le\ uminus{\isacharunderscore}{\kern0pt}int{\isacharunderscore}{\kern0pt}code{\isacharparenleft}{\kern0pt}{\isadigit{1}}{\isacharparenright}{\kern0pt}\ zero{\isacharunderscore}{\kern0pt}less{\isacharunderscore}{\kern0pt}one{\isacharunderscore}{\kern0pt}class{\isachardot}{\kern0pt}zero{\isacharunderscore}{\kern0pt}le{\isacharunderscore}{\kern0pt}one{\isacharparenright}{\kern0pt}%
\endisatagproof
{\isafoldproof}%
%
\isadelimproof
%
\endisadelimproof
\isanewline
\isanewline
\isacommand{lemma}\isamarkupfalse%
\ pos{\isacharunderscore}{\kern0pt}iff{\isacharunderscore}{\kern0pt}neg{\isacharunderscore}{\kern0pt}uminus{\isacharcolon}{\kern0pt}\ {\isachardoublequoteopen}pos\ f\ {\isacharequal}{\kern0pt}\ neg\ {\isacharparenleft}{\kern0pt}{\isacharminus}{\kern0pt}\isactrlsub e\ f{\isacharparenright}{\kern0pt}{\isachardoublequoteclose}%
\isadelimproof
\ %
\endisadelimproof
%
\isatagproof
\isacommand{unfolding}\isamarkupfalse%
\ neg{\isacharunderscore}{\kern0pt}def\ pos{\isacharunderscore}{\kern0pt}def\ eudoxus{\isacharunderscore}{\kern0pt}uminus{\isacharunderscore}{\kern0pt}def\ \isacommand{by}\isamarkupfalse%
\ simp%
\endisatagproof
{\isafoldproof}%
%
\isadelimproof
%
\endisadelimproof
\isanewline
\isanewline
\isacommand{lemma}\isamarkupfalse%
\ neg{\isacharunderscore}{\kern0pt}iff{\isacharunderscore}{\kern0pt}pos{\isacharunderscore}{\kern0pt}uminus{\isacharcolon}{\kern0pt}\ {\isachardoublequoteopen}neg\ f\ {\isacharequal}{\kern0pt}\ pos\ {\isacharparenleft}{\kern0pt}{\isacharminus}{\kern0pt}\isactrlsub e\ f{\isacharparenright}{\kern0pt}{\isachardoublequoteclose}%
\isadelimproof
\ %
\endisadelimproof
%
\isatagproof
\isacommand{unfolding}\isamarkupfalse%
\ neg{\isacharunderscore}{\kern0pt}def\ pos{\isacharunderscore}{\kern0pt}def\ eudoxus{\isacharunderscore}{\kern0pt}uminus{\isacharunderscore}{\kern0pt}def\ \isacommand{by}\isamarkupfalse%
\ fastforce%
\endisatagproof
{\isafoldproof}%
%
\isadelimproof
%
\endisadelimproof
\isanewline
\isanewline
\isacommand{lemma}\isamarkupfalse%
\ pos{\isacharunderscore}{\kern0pt}iff{\isacharcolon}{\kern0pt}\isanewline
\ \ \isakeyword{assumes}\ {\isachardoublequoteopen}slope\ f{\isachardoublequoteclose}\isanewline
\ \ \isakeyword{shows}\ {\isachardoublequoteopen}pos\ f\ {\isacharequal}{\kern0pt}\ infinite\ {\isacharparenleft}{\kern0pt}f\ {\isacharbackquote}{\kern0pt}\ {\isacharbraceleft}{\kern0pt}{\isadigit{0}}{\isachardot}{\kern0pt}{\isachardot}{\kern0pt}{\isacharbraceright}{\kern0pt}\ {\isasyminter}\ {\isacharbraceleft}{\kern0pt}{\isadigit{0}}{\isacharless}{\kern0pt}{\isachardot}{\kern0pt}{\isachardot}{\kern0pt}{\isacharbraceright}{\kern0pt}{\isacharparenright}{\kern0pt}{\isachardoublequoteclose}\ {\isacharparenleft}{\kern0pt}\isakeyword{is}\ {\isachardoublequoteopen}{\isacharquery}{\kern0pt}lhs\ {\isacharequal}{\kern0pt}\ {\isacharquery}{\kern0pt}rhs{\isachardoublequoteclose}{\isacharparenright}{\kern0pt}\isanewline
%
\isadelimproof
%
\endisadelimproof
%
\isatagproof
\isacommand{proof}\isamarkupfalse%
\ {\isacharparenleft}{\kern0pt}rule\ iffI{\isacharparenright}{\kern0pt}\isanewline
\ \ \isacommand{assume}\isamarkupfalse%
\ pos{\isacharcolon}{\kern0pt}\ {\isacharquery}{\kern0pt}lhs\isanewline
\ \ \isacommand{{\isacharbraceleft}{\kern0pt}}\isamarkupfalse%
\isanewline
\ \ \ \ \isacommand{fix}\isamarkupfalse%
\ C\ \isacommand{assume}\isamarkupfalse%
\ C{\isacharunderscore}{\kern0pt}nonneg{\isacharcolon}{\kern0pt}\ {\isachardoublequoteopen}{\isadigit{0}}\ {\isasymle}\ {\isacharparenleft}{\kern0pt}C\ {\isacharcolon}{\kern0pt}{\isacharcolon}{\kern0pt}\ int{\isacharparenright}{\kern0pt}{\isachardoublequoteclose}\isanewline
\ \ \ \ \isacommand{hence}\isamarkupfalse%
\ {\isachardoublequoteopen}{\isasymexists}z\ {\isasymge}\ {\isadigit{0}}{\isachardot}{\kern0pt}\ {\isacharparenleft}{\kern0pt}C\ {\isacharplus}{\kern0pt}\ {\isadigit{1}}{\isacharparenright}{\kern0pt}\ {\isasymle}\ f\ z{\isachardoublequoteclose}\ \isacommand{by}\isamarkupfalse%
\ {\isacharparenleft}{\kern0pt}metis\ add{\isacharunderscore}{\kern0pt}increasing{\isadigit{2}}\ nle{\isacharunderscore}{\kern0pt}le\ zero{\isacharunderscore}{\kern0pt}less{\isacharunderscore}{\kern0pt}one{\isacharunderscore}{\kern0pt}class{\isachardot}{\kern0pt}zero{\isacharunderscore}{\kern0pt}le{\isacharunderscore}{\kern0pt}one\ pos\ pos{\isacharunderscore}{\kern0pt}def{\isacharparenright}{\kern0pt}\isanewline
\ \ \ \ \isacommand{hence}\isamarkupfalse%
\ {\isachardoublequoteopen}{\isasymexists}z\ {\isasymge}\ {\isadigit{0}}{\isachardot}{\kern0pt}\ C\ {\isasymle}\ f\ z\ {\isasymand}\ {\isadigit{0}}\ {\isacharless}{\kern0pt}\ f\ z{\isachardoublequoteclose}\ \isacommand{using}\isamarkupfalse%
\ C{\isacharunderscore}{\kern0pt}nonneg\ \isacommand{by}\isamarkupfalse%
\ fastforce\isanewline
\ \ \ \ \isacommand{hence}\isamarkupfalse%
\ {\isachardoublequoteopen}{\isasymexists}N{\isasymge}C{\isachardot}{\kern0pt}\ {\isasymexists}z{\isachardot}{\kern0pt}\ N\ {\isacharequal}{\kern0pt}\ f\ z\ {\isasymand}\ {\isadigit{0}}\ {\isacharless}{\kern0pt}\ f\ z\ {\isasymand}\ {\isadigit{0}}\ {\isasymle}\ z{\isachardoublequoteclose}\ \isacommand{by}\isamarkupfalse%
\ blast\isanewline
\ \ \isacommand{{\isacharbraceright}{\kern0pt}}\isamarkupfalse%
\isanewline
\ \ \isacommand{thus}\isamarkupfalse%
\ {\isacharquery}{\kern0pt}rhs\ \isacommand{by}\isamarkupfalse%
\ {\isacharparenleft}{\kern0pt}blast\ intro{\isacharbang}{\kern0pt}{\isacharcolon}{\kern0pt}\ int{\isacharunderscore}{\kern0pt}set{\isacharunderscore}{\kern0pt}infiniteI{\isacharparenright}{\kern0pt}\isanewline
\isacommand{next}\isamarkupfalse%
\isanewline
\ \ \isacommand{assume}\isamarkupfalse%
\ infinite{\isacharcolon}{\kern0pt}\ {\isacharquery}{\kern0pt}rhs\isanewline
\ \ \isacommand{then}\isamarkupfalse%
\ \isacommand{obtain}\isamarkupfalse%
\ D\ \isakeyword{where}\ D{\isacharunderscore}{\kern0pt}bound{\isacharcolon}{\kern0pt}\ {\isachardoublequoteopen}{\isasymbar}f\ {\isacharparenleft}{\kern0pt}m\ {\isacharplus}{\kern0pt}\ n{\isacharparenright}{\kern0pt}\ {\isacharminus}{\kern0pt}\ {\isacharparenleft}{\kern0pt}f\ m\ {\isacharplus}{\kern0pt}\ f\ n{\isacharparenright}{\kern0pt}{\isasymbar}\ {\isacharless}{\kern0pt}\ D{\isachardoublequoteclose}\ {\isachardoublequoteopen}{\isadigit{0}}\ {\isacharless}{\kern0pt}\ D{\isachardoublequoteclose}\ \isakeyword{for}\ m\ n\ \isacommand{using}\isamarkupfalse%
\ assms\ \isacommand{by}\isamarkupfalse%
\ {\isacharparenleft}{\kern0pt}fastforce\ simp{\isacharcolon}{\kern0pt}\ slope{\isacharunderscore}{\kern0pt}def\ elim{\isacharcolon}{\kern0pt}\ boundedE{\isacharunderscore}{\kern0pt}strict{\isacharparenright}{\kern0pt}\isanewline
\ \isanewline
\ \ \isacommand{obtain}\isamarkupfalse%
\ M\ \isakeyword{where}\ M{\isacharunderscore}{\kern0pt}bound{\isacharcolon}{\kern0pt}\ {\isachardoublequoteopen}{\isasymforall}m{\isachargreater}{\kern0pt}{\isadigit{0}}{\isachardot}{\kern0pt}\ {\isacharparenleft}{\kern0pt}m\ {\isacharplus}{\kern0pt}\ {\isadigit{1}}{\isacharparenright}{\kern0pt}\ {\isacharasterisk}{\kern0pt}\ D\ {\isasymle}\ f\ {\isacharparenleft}{\kern0pt}m\ {\isacharasterisk}{\kern0pt}\ M{\isacharparenright}{\kern0pt}{\isachardoublequoteclose}\ {\isachardoublequoteopen}{\isadigit{0}}\ {\isacharless}{\kern0pt}\ M{\isachardoublequoteclose}\ \isacommand{using}\isamarkupfalse%
\ slope{\isacharunderscore}{\kern0pt}positive{\isacharunderscore}{\kern0pt}lower{\isacharunderscore}{\kern0pt}bound{\isacharbrackleft}{\kern0pt}OF\ assms\ infinite{\isacharbrackright}{\kern0pt}\ D{\isacharunderscore}{\kern0pt}bound{\isacharparenleft}{\kern0pt}{\isadigit{2}}{\isacharparenright}{\kern0pt}\ \isacommand{by}\isamarkupfalse%
\ blast\isanewline
\isanewline
\ \ \isacommand{define}\isamarkupfalse%
\ g\ \isakeyword{where}\ {\isachardoublequoteopen}g\ {\isacharequal}{\kern0pt}\ {\isacharparenleft}{\kern0pt}{\isasymlambda}z{\isachardot}{\kern0pt}\ f\ {\isacharparenleft}{\kern0pt}{\isacharparenleft}{\kern0pt}z\ div\ M{\isacharparenright}{\kern0pt}\ {\isacharasterisk}{\kern0pt}\ M{\isacharparenright}{\kern0pt}{\isacharparenright}{\kern0pt}{\isachardoublequoteclose}\isanewline
\ \ \isacommand{define}\isamarkupfalse%
\ E\ \isakeyword{where}\ {\isachardoublequoteopen}E\ {\isacharequal}{\kern0pt}\ Sup\ {\isacharparenleft}{\kern0pt}{\isacharparenleft}{\kern0pt}abs\ o\ f{\isacharparenright}{\kern0pt}\ {\isacharbackquote}{\kern0pt}\ {\isacharbraceleft}{\kern0pt}z{\isachardot}{\kern0pt}\ {\isadigit{0}}\ {\isasymle}\ z\ {\isasymand}\ z\ {\isacharless}{\kern0pt}\ M{\isacharbraceright}{\kern0pt}{\isacharparenright}{\kern0pt}{\isachardoublequoteclose}\isanewline
\ \ \isanewline
\ \ \isacommand{have}\isamarkupfalse%
\ E{\isacharunderscore}{\kern0pt}bound{\isacharcolon}{\kern0pt}\ {\isachardoublequoteopen}{\isasymbar}f\ {\isacharparenleft}{\kern0pt}z\ mod\ M{\isacharparenright}{\kern0pt}{\isasymbar}\ {\isasymle}\ E{\isachardoublequoteclose}\ \isakeyword{for}\ z\isanewline
\ \ \isacommand{proof}\isamarkupfalse%
\ {\isacharminus}{\kern0pt}\isanewline
\ \ \ \ \isacommand{have}\isamarkupfalse%
\ {\isachardoublequoteopen}{\isacharparenleft}{\kern0pt}z\ mod\ M{\isacharparenright}{\kern0pt}\ {\isasymin}\ {\isacharbraceleft}{\kern0pt}z{\isachardot}{\kern0pt}\ {\isadigit{0}}\ {\isasymle}\ z\ {\isasymand}\ z\ {\isacharless}{\kern0pt}\ M{\isacharbraceright}{\kern0pt}{\isachardoublequoteclose}\ \isacommand{by}\isamarkupfalse%
\ {\isacharparenleft}{\kern0pt}simp\ add{\isacharcolon}{\kern0pt}\ M{\isacharunderscore}{\kern0pt}bound{\isacharparenleft}{\kern0pt}{\isadigit{2}}{\isacharparenright}{\kern0pt}{\isacharparenright}{\kern0pt}\isanewline
\ \ \ \ \isacommand{hence}\isamarkupfalse%
\ {\isachardoublequoteopen}{\isasymbar}f\ {\isacharparenleft}{\kern0pt}z\ mod\ M{\isacharparenright}{\kern0pt}{\isasymbar}\ {\isasymin}\ {\isacharparenleft}{\kern0pt}abs\ o\ f{\isacharparenright}{\kern0pt}\ {\isacharbackquote}{\kern0pt}\ {\isacharbraceleft}{\kern0pt}z{\isachardot}{\kern0pt}\ {\isadigit{0}}\ {\isasymle}\ z\ {\isasymand}\ z\ {\isacharless}{\kern0pt}\ M{\isacharbraceright}{\kern0pt}{\isachardoublequoteclose}\ \isacommand{by}\isamarkupfalse%
\ fastforce\isanewline
\ \ \ \ \isacommand{thus}\isamarkupfalse%
\ {\isachardoublequoteopen}{\isasymbar}f\ {\isacharparenleft}{\kern0pt}z\ mod\ M{\isacharparenright}{\kern0pt}{\isasymbar}\ {\isasymle}\ E{\isachardoublequoteclose}\ \isacommand{unfolding}\isamarkupfalse%
\ E{\isacharunderscore}{\kern0pt}def\ \isacommand{by}\isamarkupfalse%
\ {\isacharparenleft}{\kern0pt}simp\ add{\isacharcolon}{\kern0pt}\ le{\isacharunderscore}{\kern0pt}cSup{\isacharunderscore}{\kern0pt}finite{\isacharparenright}{\kern0pt}\isanewline
\ \ \isacommand{qed}\isamarkupfalse%
\isanewline
\ \ \isacommand{hence}\isamarkupfalse%
\ E{\isacharunderscore}{\kern0pt}nonneg{\isacharcolon}{\kern0pt}\ {\isachardoublequoteopen}{\isadigit{0}}\ {\isasymle}\ E{\isachardoublequoteclose}\ \isacommand{by}\isamarkupfalse%
\ fastforce\isanewline
\isanewline
\ \ \isacommand{have}\isamarkupfalse%
\ diff{\isacharunderscore}{\kern0pt}bound{\isacharcolon}{\kern0pt}\ {\isachardoublequoteopen}{\isasymbar}f\ z\ {\isacharminus}{\kern0pt}\ g\ z{\isasymbar}\ {\isasymle}\ E\ {\isacharplus}{\kern0pt}\ D{\isachardoublequoteclose}\ \isakeyword{for}\ z\isanewline
\ \ \isacommand{proof}\isamarkupfalse%
{\isacharminus}{\kern0pt}\isanewline
\ \ \ \ \isacommand{let}\isamarkupfalse%
\ {\isacharquery}{\kern0pt}d\ {\isacharequal}{\kern0pt}\ {\isachardoublequoteopen}z\ div\ M{\isachardoublequoteclose}\ \isakeyword{and}\ {\isacharquery}{\kern0pt}r\ {\isacharequal}{\kern0pt}\ {\isachardoublequoteopen}z\ mod\ M{\isachardoublequoteclose}\isanewline
\ \ \ \ \isacommand{have}\isamarkupfalse%
\ z{\isacharunderscore}{\kern0pt}is{\isacharcolon}{\kern0pt}\ {\isachardoublequoteopen}z\ {\isacharequal}{\kern0pt}\ {\isacharquery}{\kern0pt}d\ {\isacharasterisk}{\kern0pt}\ M\ {\isacharplus}{\kern0pt}\ {\isacharquery}{\kern0pt}r{\isachardoublequoteclose}\ \isacommand{by}\isamarkupfalse%
\ presburger\isanewline
\ \ \ \ \isacommand{hence}\isamarkupfalse%
\ {\isachardoublequoteopen}{\isasymbar}f\ z\ {\isacharminus}{\kern0pt}\ g\ z{\isasymbar}\ {\isacharequal}{\kern0pt}\ {\isasymbar}f\ {\isacharparenleft}{\kern0pt}{\isacharquery}{\kern0pt}d\ {\isacharasterisk}{\kern0pt}\ M\ {\isacharplus}{\kern0pt}\ {\isacharquery}{\kern0pt}r{\isacharparenright}{\kern0pt}\ {\isacharminus}{\kern0pt}\ g\ {\isacharparenleft}{\kern0pt}{\isacharquery}{\kern0pt}d\ {\isacharasterisk}{\kern0pt}\ M\ {\isacharplus}{\kern0pt}\ {\isacharquery}{\kern0pt}r{\isacharparenright}{\kern0pt}{\isasymbar}{\isachardoublequoteclose}\ \isacommand{by}\isamarkupfalse%
\ argo\isanewline
\ \ \ \ \isacommand{also}\isamarkupfalse%
\ \isacommand{have}\isamarkupfalse%
\ {\isachardoublequoteopen}{\isachardot}{\kern0pt}{\isachardot}{\kern0pt}{\isachardot}{\kern0pt}\ {\isacharequal}{\kern0pt}\ {\isasymbar}{\isacharparenleft}{\kern0pt}f\ {\isacharparenleft}{\kern0pt}{\isacharquery}{\kern0pt}d{\isacharasterisk}{\kern0pt}M\ {\isacharplus}{\kern0pt}\ {\isacharquery}{\kern0pt}r{\isacharparenright}{\kern0pt}\ {\isacharminus}{\kern0pt}\ {\isacharparenleft}{\kern0pt}f\ {\isacharparenleft}{\kern0pt}{\isacharquery}{\kern0pt}d\ {\isacharasterisk}{\kern0pt}\ M{\isacharparenright}{\kern0pt}\ {\isacharplus}{\kern0pt}\ f\ {\isacharquery}{\kern0pt}r{\isacharparenright}{\kern0pt}{\isacharparenright}{\kern0pt}\ {\isacharplus}{\kern0pt}\ {\isacharparenleft}{\kern0pt}f\ {\isacharparenleft}{\kern0pt}{\isacharquery}{\kern0pt}d\ {\isacharasterisk}{\kern0pt}\ M{\isacharparenright}{\kern0pt}\ {\isacharplus}{\kern0pt}\ f\ {\isacharquery}{\kern0pt}r{\isacharparenright}{\kern0pt}\ {\isacharminus}{\kern0pt}\ g\ {\isacharparenleft}{\kern0pt}{\isacharquery}{\kern0pt}d\ {\isacharasterisk}{\kern0pt}\ M\ {\isacharplus}{\kern0pt}\ {\isacharquery}{\kern0pt}r{\isacharparenright}{\kern0pt}{\isasymbar}{\isachardoublequoteclose}\ \isacommand{by}\isamarkupfalse%
\ auto\isanewline
\ \ \ \ \isacommand{also}\isamarkupfalse%
\ \isacommand{have}\isamarkupfalse%
\ {\isachardoublequoteopen}{\isachardot}{\kern0pt}{\isachardot}{\kern0pt}{\isachardot}{\kern0pt}\ {\isacharequal}{\kern0pt}\ {\isasymbar}f\ {\isacharquery}{\kern0pt}r\ {\isacharplus}{\kern0pt}\ {\isacharparenleft}{\kern0pt}f\ {\isacharparenleft}{\kern0pt}{\isacharquery}{\kern0pt}d{\isacharasterisk}{\kern0pt}M\ {\isacharplus}{\kern0pt}\ {\isacharquery}{\kern0pt}r{\isacharparenright}{\kern0pt}\ {\isacharminus}{\kern0pt}\ {\isacharparenleft}{\kern0pt}f\ {\isacharparenleft}{\kern0pt}{\isacharquery}{\kern0pt}d\ {\isacharasterisk}{\kern0pt}\ M{\isacharparenright}{\kern0pt}\ {\isacharplus}{\kern0pt}\ f\ {\isacharquery}{\kern0pt}r{\isacharparenright}{\kern0pt}{\isacharparenright}{\kern0pt}{\isasymbar}{\isachardoublequoteclose}\ \isacommand{unfolding}\isamarkupfalse%
\ g{\isacharunderscore}{\kern0pt}def\ \isacommand{by}\isamarkupfalse%
\ force\isanewline
\ \ \ \ \isacommand{also}\isamarkupfalse%
\ \isacommand{have}\isamarkupfalse%
\ {\isachardoublequoteopen}{\isachardot}{\kern0pt}{\isachardot}{\kern0pt}{\isachardot}{\kern0pt}\ {\isasymle}\ {\isasymbar}f\ {\isacharquery}{\kern0pt}r{\isasymbar}\ {\isacharplus}{\kern0pt}\ D{\isachardoublequoteclose}\ \isacommand{using}\isamarkupfalse%
\ D{\isacharunderscore}{\kern0pt}bound{\isacharparenleft}{\kern0pt}{\isadigit{1}}{\isacharparenright}{\kern0pt}{\isacharbrackleft}{\kern0pt}of\ {\isachardoublequoteopen}{\isacharquery}{\kern0pt}d\ {\isacharasterisk}{\kern0pt}\ M{\isachardoublequoteclose}\ {\isachardoublequoteopen}{\isacharquery}{\kern0pt}r{\isachardoublequoteclose}{\isacharbrackright}{\kern0pt}\ \isacommand{by}\isamarkupfalse%
\ linarith\isanewline
\ \ \ \ \isacommand{also}\isamarkupfalse%
\ \isacommand{have}\isamarkupfalse%
\ {\isachardoublequoteopen}{\isachardot}{\kern0pt}{\isachardot}{\kern0pt}{\isachardot}{\kern0pt}\ {\isasymle}\ E\ {\isacharplus}{\kern0pt}\ D{\isachardoublequoteclose}\ \isacommand{using}\isamarkupfalse%
\ E{\isacharunderscore}{\kern0pt}bound\ \isacommand{by}\isamarkupfalse%
\ simp\isanewline
\ \ \ \ \isacommand{finally}\isamarkupfalse%
\ \isacommand{show}\isamarkupfalse%
\ {\isachardoublequoteopen}{\isasymbar}f\ z\ {\isacharminus}{\kern0pt}\ g\ z{\isasymbar}\ {\isasymle}\ E\ {\isacharplus}{\kern0pt}\ D{\isachardoublequoteclose}\ \isacommand{{\isachardot}{\kern0pt}}\isamarkupfalse%
\isanewline
\ \ \isacommand{qed}\isamarkupfalse%
\isanewline
\ \ \isacommand{{\isacharbraceleft}{\kern0pt}}\isamarkupfalse%
\isanewline
\ \ \ \ \isacommand{fix}\isamarkupfalse%
\ C\ \isacommand{assume}\isamarkupfalse%
\ C{\isacharunderscore}{\kern0pt}nonneg{\isacharcolon}{\kern0pt}\ {\isachardoublequoteopen}{\isadigit{0}}\ {\isasymle}\ {\isacharparenleft}{\kern0pt}C\ {\isacharcolon}{\kern0pt}{\isacharcolon}{\kern0pt}\ int{\isacharparenright}{\kern0pt}{\isachardoublequoteclose}\isanewline
\ \ \isanewline
\ \ \ \ \isacommand{define}\isamarkupfalse%
\ n\ \isakeyword{where}\ {\isachardoublequoteopen}n\ {\isacharequal}{\kern0pt}\ {\isacharparenleft}{\kern0pt}E\ {\isacharplus}{\kern0pt}\ D\ {\isacharplus}{\kern0pt}\ C{\isacharparenright}{\kern0pt}\ div\ D{\isachardoublequoteclose}\isanewline
\ \ \ \ \isacommand{hence}\isamarkupfalse%
\ zero{\isacharunderscore}{\kern0pt}less{\isacharunderscore}{\kern0pt}n{\isacharcolon}{\kern0pt}\ {\isachardoublequoteopen}n\ {\isachargreater}{\kern0pt}\ {\isadigit{0}}{\isachardoublequoteclose}\ \isacommand{using}\isamarkupfalse%
\ D{\isacharunderscore}{\kern0pt}bound{\isacharparenleft}{\kern0pt}{\isadigit{2}}{\isacharparenright}{\kern0pt}\ E{\isacharunderscore}{\kern0pt}nonneg\ C{\isacharunderscore}{\kern0pt}nonneg\ \isacommand{using}\isamarkupfalse%
\ pos{\isacharunderscore}{\kern0pt}imp{\isacharunderscore}{\kern0pt}zdiv{\isacharunderscore}{\kern0pt}pos{\isacharunderscore}{\kern0pt}iff\ \isacommand{by}\isamarkupfalse%
\ fastforce\isanewline
\ \ \ \isanewline
\ \ \ \ \isacommand{have}\isamarkupfalse%
\ {\isachardoublequoteopen}E\ {\isacharplus}{\kern0pt}\ C\ {\isacharless}{\kern0pt}\ E\ {\isacharplus}{\kern0pt}\ D\ {\isacharplus}{\kern0pt}\ C\ {\isacharminus}{\kern0pt}\ {\isacharparenleft}{\kern0pt}E\ {\isacharplus}{\kern0pt}\ D\ {\isacharplus}{\kern0pt}\ C{\isacharparenright}{\kern0pt}\ mod\ D{\isachardoublequoteclose}\ \isacommand{using}\isamarkupfalse%
\ diff{\isacharunderscore}{\kern0pt}strict{\isacharunderscore}{\kern0pt}left{\isacharunderscore}{\kern0pt}mono{\isacharbrackleft}{\kern0pt}OF\ pos{\isacharunderscore}{\kern0pt}mod{\isacharunderscore}{\kern0pt}bound{\isacharbrackleft}{\kern0pt}OF\ D{\isacharunderscore}{\kern0pt}bound{\isacharparenleft}{\kern0pt}{\isadigit{2}}{\isacharparenright}{\kern0pt}{\isacharbrackright}{\kern0pt}{\isacharbrackright}{\kern0pt}\ \isacommand{by}\isamarkupfalse%
\ simp\isanewline
\ \ \ \ \isacommand{also}\isamarkupfalse%
\ \isacommand{have}\isamarkupfalse%
\ {\isachardoublequoteopen}{\isachardot}{\kern0pt}{\isachardot}{\kern0pt}{\isachardot}{\kern0pt}\ {\isacharequal}{\kern0pt}\ n\ {\isacharasterisk}{\kern0pt}\ D{\isachardoublequoteclose}\ \isacommand{unfolding}\isamarkupfalse%
\ n{\isacharunderscore}{\kern0pt}def\ \isacommand{using}\isamarkupfalse%
\ div{\isacharunderscore}{\kern0pt}mod{\isacharunderscore}{\kern0pt}decomp{\isacharunderscore}{\kern0pt}int{\isacharbrackleft}{\kern0pt}of\ {\isachardoublequoteopen}E\ {\isacharplus}{\kern0pt}\ D\ {\isacharplus}{\kern0pt}\ C{\isachardoublequoteclose}\ D{\isacharbrackright}{\kern0pt}\ \isacommand{by}\isamarkupfalse%
\ algebra\isanewline
\ \ \ \ \isacommand{finally}\isamarkupfalse%
\ \isacommand{have}\isamarkupfalse%
\ {\isacharasterisk}{\kern0pt}{\isacharcolon}{\kern0pt}\ {\isachardoublequoteopen}{\isacharparenleft}{\kern0pt}n\ {\isacharplus}{\kern0pt}\ {\isadigit{1}}{\isacharparenright}{\kern0pt}\ {\isacharasterisk}{\kern0pt}\ D\ {\isachargreater}{\kern0pt}\ E\ {\isacharplus}{\kern0pt}\ D\ {\isacharplus}{\kern0pt}\ C{\isachardoublequoteclose}\ \isacommand{by}\isamarkupfalse%
\ {\isacharparenleft}{\kern0pt}simp\ add{\isacharcolon}{\kern0pt}\ add{\isachardot}{\kern0pt}commute\ distrib{\isacharunderscore}{\kern0pt}right{\isacharparenright}{\kern0pt}\isanewline
\ \ \isanewline
\ \ \ \ \isacommand{have}\isamarkupfalse%
\ {\isachardoublequoteopen}C\ {\isasymle}\ f\ m{\isachardoublequoteclose}\ \isakeyword{if}\ {\isachardoublequoteopen}m\ {\isasymge}\ n\ {\isacharasterisk}{\kern0pt}\ M{\isachardoublequoteclose}\ \isakeyword{for}\ m\isanewline
\ \ \ \ \isacommand{proof}\isamarkupfalse%
\ {\isacharminus}{\kern0pt}\isanewline
\ \ \ \ \ \ \isacommand{let}\isamarkupfalse%
\ {\isacharquery}{\kern0pt}d\ {\isacharequal}{\kern0pt}\ {\isachardoublequoteopen}m\ div\ M{\isachardoublequoteclose}\ \isakeyword{and}\ {\isacharquery}{\kern0pt}r\ {\isacharequal}{\kern0pt}\ {\isachardoublequoteopen}m\ mod\ M{\isachardoublequoteclose}\isanewline
\ \ \ \ \ \ \isacommand{have}\isamarkupfalse%
\ d{\isacharunderscore}{\kern0pt}pos{\isacharcolon}{\kern0pt}\ {\isachardoublequoteopen}{\isacharquery}{\kern0pt}d\ {\isachargreater}{\kern0pt}\ {\isadigit{0}}{\isachardoublequoteclose}\ \isacommand{using}\isamarkupfalse%
\ zero{\isacharunderscore}{\kern0pt}less{\isacharunderscore}{\kern0pt}n\ M{\isacharunderscore}{\kern0pt}bound\ that\ dual{\isacharunderscore}{\kern0pt}order{\isachardot}{\kern0pt}trans\ pos{\isacharunderscore}{\kern0pt}imp{\isacharunderscore}{\kern0pt}zdiv{\isacharunderscore}{\kern0pt}pos{\isacharunderscore}{\kern0pt}iff\ \isacommand{by}\isamarkupfalse%
\ fastforce\isanewline
\ \ \ \ \ \ \isacommand{have}\isamarkupfalse%
\ n{\isacharunderscore}{\kern0pt}le{\isacharunderscore}{\kern0pt}d{\isacharcolon}{\kern0pt}\ {\isachardoublequoteopen}{\isacharquery}{\kern0pt}d\ {\isasymge}\ n{\isachardoublequoteclose}\ \isacommand{using}\isamarkupfalse%
\ zdiv{\isacharunderscore}{\kern0pt}mono{\isadigit{1}}\ M{\isacharunderscore}{\kern0pt}bound\ that\ \isacommand{by}\isamarkupfalse%
\ fastforce\isanewline
\ \ \ \ \ \ \isacommand{have}\isamarkupfalse%
\ {\isachardoublequoteopen}E\ {\isacharplus}{\kern0pt}\ D\ {\isacharplus}{\kern0pt}\ C\ {\isacharless}{\kern0pt}\ {\isacharparenleft}{\kern0pt}{\isacharquery}{\kern0pt}d\ {\isacharplus}{\kern0pt}\ {\isadigit{1}}{\isacharparenright}{\kern0pt}\ {\isacharasterisk}{\kern0pt}\ D{\isachardoublequoteclose}\ \isacommand{using}\isamarkupfalse%
\ D{\isacharunderscore}{\kern0pt}bound\ n{\isacharunderscore}{\kern0pt}le{\isacharunderscore}{\kern0pt}d\ \isacommand{by}\isamarkupfalse%
\ {\isacharparenleft}{\kern0pt}intro\ {\isacharasterisk}{\kern0pt}{\isacharbrackleft}{\kern0pt}THEN\ order{\isachardot}{\kern0pt}strict{\isacharunderscore}{\kern0pt}trans{\isadigit{2}}{\isacharbrackright}{\kern0pt}{\isacharparenright}{\kern0pt}\ simp\isanewline
\ \ \ \ \ \ \isacommand{also}\isamarkupfalse%
\ \isacommand{have}\isamarkupfalse%
\ {\isachardoublequoteopen}{\isachardot}{\kern0pt}{\isachardot}{\kern0pt}{\isachardot}{\kern0pt}\ {\isasymle}\ g\ m{\isachardoublequoteclose}\ \isacommand{unfolding}\isamarkupfalse%
\ g{\isacharunderscore}{\kern0pt}def\ \isacommand{using}\isamarkupfalse%
\ M{\isacharunderscore}{\kern0pt}bound\ d{\isacharunderscore}{\kern0pt}pos\ \isacommand{by}\isamarkupfalse%
\ blast\isanewline
\ \ \ \ \ \ \isacommand{finally}\isamarkupfalse%
\ \isacommand{have}\isamarkupfalse%
\ {\isachardoublequoteopen}E\ {\isacharplus}{\kern0pt}\ D\ {\isacharplus}{\kern0pt}\ C\ {\isacharless}{\kern0pt}\ g\ m{\isachardoublequoteclose}\ \isacommand{{\isachardot}{\kern0pt}}\isamarkupfalse%
\isanewline
\ \ \ \ \ \ \isacommand{hence}\isamarkupfalse%
\ {\isachardoublequoteopen}{\isasymbar}f\ m\ {\isacharminus}{\kern0pt}\ g\ m{\isasymbar}\ {\isacharplus}{\kern0pt}\ C\ {\isacharless}{\kern0pt}\ g\ m{\isachardoublequoteclose}\ \isacommand{using}\isamarkupfalse%
\ diff{\isacharunderscore}{\kern0pt}bound{\isacharbrackleft}{\kern0pt}of\ m{\isacharbrackright}{\kern0pt}\ \isacommand{by}\isamarkupfalse%
\ fastforce\isanewline
\ \ \ \ \ \ \isacommand{thus}\isamarkupfalse%
\ {\isacharquery}{\kern0pt}thesis\ \isacommand{by}\isamarkupfalse%
\ fastforce\isanewline
\ \ \ \ \isacommand{qed}\isamarkupfalse%
\isanewline
\ \ \ \ \isacommand{hence}\isamarkupfalse%
\ {\isachardoublequoteopen}{\isasymexists}N{\isachardot}{\kern0pt}\ {\isasymforall}p{\isasymge}N{\isachardot}{\kern0pt}\ C\ {\isasymle}\ f\ p{\isachardoublequoteclose}\ \isacommand{using}\isamarkupfalse%
\ add{\isadigit{1}}{\isacharunderscore}{\kern0pt}zle{\isacharunderscore}{\kern0pt}eq\ \isacommand{by}\isamarkupfalse%
\ blast\isanewline
\ \ \isacommand{{\isacharbraceright}{\kern0pt}}\isamarkupfalse%
\isanewline
\ \ \isacommand{thus}\isamarkupfalse%
\ {\isacharquery}{\kern0pt}lhs\ \isacommand{unfolding}\isamarkupfalse%
\ pos{\isacharunderscore}{\kern0pt}def\ \isacommand{by}\isamarkupfalse%
\ blast\isanewline
\isacommand{qed}\isamarkupfalse%
%
\endisatagproof
{\isafoldproof}%
%
\isadelimproof
\isanewline
%
\endisadelimproof
\isanewline
\isacommand{lemma}\isamarkupfalse%
\ neg{\isacharunderscore}{\kern0pt}iff{\isacharcolon}{\kern0pt}\isanewline
\ \ \isakeyword{assumes}\ {\isachardoublequoteopen}slope\ f{\isachardoublequoteclose}\isanewline
\ \ \isakeyword{shows}\ {\isachardoublequoteopen}neg\ f\ {\isacharequal}{\kern0pt}\ infinite\ {\isacharparenleft}{\kern0pt}f\ {\isacharbackquote}{\kern0pt}\ {\isacharbraceleft}{\kern0pt}{\isadigit{0}}{\isachardot}{\kern0pt}{\isachardot}{\kern0pt}{\isacharbraceright}{\kern0pt}\ {\isasyminter}\ {\isacharbraceleft}{\kern0pt}{\isachardot}{\kern0pt}{\isachardot}{\kern0pt}{\isacharless}{\kern0pt}{\isadigit{0}}{\isacharbraceright}{\kern0pt}{\isacharparenright}{\kern0pt}{\isachardoublequoteclose}\ {\isacharparenleft}{\kern0pt}\isakeyword{is}\ {\isachardoublequoteopen}{\isacharquery}{\kern0pt}lhs\ {\isacharequal}{\kern0pt}\ {\isacharquery}{\kern0pt}rhs{\isachardoublequoteclose}{\isacharparenright}{\kern0pt}\isanewline
%
\isadelimproof
%
\endisadelimproof
%
\isatagproof
\isacommand{proof}\isamarkupfalse%
\ {\isacharparenleft}{\kern0pt}rule\ iffI{\isacharparenright}{\kern0pt}\isanewline
\ \ \isacommand{assume}\isamarkupfalse%
\ {\isacharquery}{\kern0pt}lhs\isanewline
\ \ \isacommand{hence}\isamarkupfalse%
\ {\isachardoublequoteopen}infinite\ {\isacharparenleft}{\kern0pt}{\isacharparenleft}{\kern0pt}{\isacharminus}{\kern0pt}\ f{\isacharparenright}{\kern0pt}\ {\isacharbackquote}{\kern0pt}\ {\isacharbraceleft}{\kern0pt}{\isadigit{0}}{\isachardot}{\kern0pt}{\isachardot}{\kern0pt}{\isacharbraceright}{\kern0pt}\ {\isasyminter}\ {\isacharbraceleft}{\kern0pt}{\isadigit{0}}{\isacharless}{\kern0pt}{\isachardot}{\kern0pt}{\isachardot}{\kern0pt}{\isacharbraceright}{\kern0pt}{\isacharparenright}{\kern0pt}{\isachardoublequoteclose}\ \isacommand{using}\isamarkupfalse%
\ pos{\isacharunderscore}{\kern0pt}iff{\isacharbrackleft}{\kern0pt}OF\ slope{\isacharunderscore}{\kern0pt}uminus{\isacharprime}{\kern0pt}{\isacharbrackleft}{\kern0pt}OF\ assms{\isacharbrackright}{\kern0pt}{\isacharbrackright}{\kern0pt}\ \isacommand{unfolding}\isamarkupfalse%
\ neg{\isacharunderscore}{\kern0pt}def\ pos{\isacharunderscore}{\kern0pt}def\ \isacommand{by}\isamarkupfalse%
\ fastforce\isanewline
\ \ \isacommand{moreover}\isamarkupfalse%
\ \isacommand{have}\isamarkupfalse%
\ {\isachardoublequoteopen}inj\ {\isacharparenleft}{\kern0pt}uminus\ {\isacharcolon}{\kern0pt}{\isacharcolon}{\kern0pt}\ int\ {\isasymRightarrow}\ int{\isacharparenright}{\kern0pt}{\isachardoublequoteclose}\ \isacommand{by}\isamarkupfalse%
\ simp\isanewline
\ \ \isacommand{moreover}\isamarkupfalse%
\ \isacommand{have}\isamarkupfalse%
\ {\isachardoublequoteopen}{\isacharparenleft}{\kern0pt}{\isacharminus}{\kern0pt}\ f{\isacharparenright}{\kern0pt}\ {\isacharbackquote}{\kern0pt}\ {\isacharbraceleft}{\kern0pt}{\isadigit{0}}{\isachardot}{\kern0pt}{\isachardot}{\kern0pt}{\isacharbraceright}{\kern0pt}\ {\isasyminter}\ {\isacharbraceleft}{\kern0pt}{\isadigit{0}}{\isacharless}{\kern0pt}{\isachardot}{\kern0pt}{\isachardot}{\kern0pt}{\isacharbraceright}{\kern0pt}\ {\isacharequal}{\kern0pt}\ uminus\ {\isacharbackquote}{\kern0pt}\ {\isacharparenleft}{\kern0pt}f\ {\isacharbackquote}{\kern0pt}\ {\isacharbraceleft}{\kern0pt}{\isadigit{0}}{\isachardot}{\kern0pt}{\isachardot}{\kern0pt}{\isacharbraceright}{\kern0pt}\ {\isasyminter}\ {\isacharbraceleft}{\kern0pt}{\isachardot}{\kern0pt}{\isachardot}{\kern0pt}{\isacharless}{\kern0pt}{\isadigit{0}}{\isacharbraceright}{\kern0pt}{\isacharparenright}{\kern0pt}{\isachardoublequoteclose}\ \isacommand{by}\isamarkupfalse%
\ fastforce\isanewline
\ \ \isacommand{ultimately}\isamarkupfalse%
\ \isacommand{show}\isamarkupfalse%
\ {\isacharquery}{\kern0pt}rhs\ \isacommand{using}\isamarkupfalse%
\ finite{\isacharunderscore}{\kern0pt}imageD\ \isacommand{by}\isamarkupfalse%
\ fastforce\isanewline
\isacommand{next}\isamarkupfalse%
\isanewline
\ \ \isacommand{assume}\isamarkupfalse%
\ {\isacharquery}{\kern0pt}rhs\isanewline
\ \ \isacommand{moreover}\isamarkupfalse%
\ \isacommand{have}\isamarkupfalse%
\ {\isachardoublequoteopen}inj\ {\isacharparenleft}{\kern0pt}uminus\ {\isacharcolon}{\kern0pt}{\isacharcolon}{\kern0pt}\ int\ {\isasymRightarrow}\ int{\isacharparenright}{\kern0pt}{\isachardoublequoteclose}\ \isacommand{by}\isamarkupfalse%
\ simp\isanewline
\ \ \isacommand{moreover}\isamarkupfalse%
\ \isacommand{have}\isamarkupfalse%
\ {\isachardoublequoteopen}f\ {\isacharbackquote}{\kern0pt}\ {\isacharbraceleft}{\kern0pt}{\isadigit{0}}{\isachardot}{\kern0pt}{\isachardot}{\kern0pt}{\isacharbraceright}{\kern0pt}\ {\isasyminter}\ {\isacharbraceleft}{\kern0pt}{\isachardot}{\kern0pt}{\isachardot}{\kern0pt}{\isacharless}{\kern0pt}{\isadigit{0}}{\isacharbraceright}{\kern0pt}\ {\isacharequal}{\kern0pt}\ uminus\ {\isacharbackquote}{\kern0pt}\ {\isacharparenleft}{\kern0pt}{\isacharparenleft}{\kern0pt}{\isacharminus}{\kern0pt}\ f{\isacharparenright}{\kern0pt}\ {\isacharbackquote}{\kern0pt}\ {\isacharbraceleft}{\kern0pt}{\isadigit{0}}{\isachardot}{\kern0pt}{\isachardot}{\kern0pt}{\isacharbraceright}{\kern0pt}\ {\isasyminter}\ {\isacharbraceleft}{\kern0pt}{\isadigit{0}}{\isacharless}{\kern0pt}{\isachardot}{\kern0pt}{\isachardot}{\kern0pt}{\isacharbraceright}{\kern0pt}{\isacharparenright}{\kern0pt}{\isachardoublequoteclose}\ \isacommand{by}\isamarkupfalse%
\ force\isanewline
\ \ \isacommand{ultimately}\isamarkupfalse%
\ \isacommand{have}\isamarkupfalse%
\ {\isachardoublequoteopen}infinite\ {\isacharparenleft}{\kern0pt}{\isacharparenleft}{\kern0pt}{\isacharminus}{\kern0pt}\ f{\isacharparenright}{\kern0pt}\ {\isacharbackquote}{\kern0pt}\ {\isacharbraceleft}{\kern0pt}{\isadigit{0}}{\isachardot}{\kern0pt}{\isachardot}{\kern0pt}{\isacharbraceright}{\kern0pt}\ {\isasyminter}\ {\isacharbraceleft}{\kern0pt}{\isadigit{0}}{\isacharless}{\kern0pt}{\isachardot}{\kern0pt}{\isachardot}{\kern0pt}{\isacharbraceright}{\kern0pt}{\isacharparenright}{\kern0pt}{\isachardoublequoteclose}\ \isacommand{using}\isamarkupfalse%
\ finite{\isacharunderscore}{\kern0pt}imageD\ \isacommand{by}\isamarkupfalse%
\ force\isanewline
\ \ \isacommand{thus}\isamarkupfalse%
\ {\isacharquery}{\kern0pt}lhs\ \isacommand{using}\isamarkupfalse%
\ pos{\isacharunderscore}{\kern0pt}iff{\isacharbrackleft}{\kern0pt}OF\ slope{\isacharunderscore}{\kern0pt}uminus{\isacharprime}{\kern0pt}{\isacharbrackleft}{\kern0pt}OF\ assms{\isacharbrackright}{\kern0pt}{\isacharbrackright}{\kern0pt}\ \isacommand{unfolding}\isamarkupfalse%
\ pos{\isacharunderscore}{\kern0pt}def\ neg{\isacharunderscore}{\kern0pt}def\ \isacommand{by}\isamarkupfalse%
\ fastforce\isanewline
\isacommand{qed}\isamarkupfalse%
%
\endisatagproof
{\isafoldproof}%
%
\isadelimproof
\isanewline
%
\endisadelimproof
\isanewline
\isacommand{lemma}\isamarkupfalse%
\ pos{\isacharunderscore}{\kern0pt}cong{\isacharcolon}{\kern0pt}\isanewline
\ \ \isakeyword{assumes}\ {\isachardoublequoteopen}f\ {\isasymsim}\isactrlsub e\ g{\isachardoublequoteclose}\isanewline
\ \ \isakeyword{shows}\ {\isachardoublequoteopen}pos\ f\ {\isacharequal}{\kern0pt}\ pos\ g{\isachardoublequoteclose}\isanewline
%
\isadelimproof
%
\endisadelimproof
%
\isatagproof
\isacommand{proof}\isamarkupfalse%
\ {\isacharminus}{\kern0pt}\isanewline
\ \ \isacommand{{\isacharbraceleft}{\kern0pt}}\isamarkupfalse%
\ \isanewline
\ \ \ \ \isacommand{fix}\isamarkupfalse%
\ x\ y\ \isacommand{assume}\isamarkupfalse%
\ asm{\isacharcolon}{\kern0pt}\ {\isachardoublequoteopen}pos\ x{\isachardoublequoteclose}\ {\isachardoublequoteopen}x\ {\isasymsim}\isactrlsub e\ y{\isachardoublequoteclose}\isanewline
\ \ \ \ \isacommand{fix}\isamarkupfalse%
\ D\ \isacommand{assume}\isamarkupfalse%
\ D{\isacharcolon}{\kern0pt}\ {\isachardoublequoteopen}{\isadigit{0}}\ {\isasymle}\ D{\isachardoublequoteclose}\ {\isachardoublequoteopen}{\isasymforall}N{\isachardot}{\kern0pt}\ {\isasymexists}p{\isasymge}N{\isachardot}{\kern0pt}\ {\isasymnot}\ D\ {\isasymle}\ y\ p{\isachardoublequoteclose}\isanewline
\ \ \ \ \isacommand{obtain}\isamarkupfalse%
\ C\ \isakeyword{where}\ bounds{\isacharcolon}{\kern0pt}\ {\isachardoublequoteopen}{\isasymforall}n{\isachardot}{\kern0pt}\ {\isasymbar}x\ n\ {\isacharminus}{\kern0pt}\ y\ n{\isasymbar}\ {\isasymle}\ C{\isachardoublequoteclose}\ {\isachardoublequoteopen}{\isadigit{0}}\ {\isasymle}\ C{\isachardoublequoteclose}\ \isacommand{using}\isamarkupfalse%
\ asm\ \isacommand{unfolding}\isamarkupfalse%
\ eudoxus{\isacharunderscore}{\kern0pt}rel{\isacharunderscore}{\kern0pt}def\ \isacommand{by}\isamarkupfalse%
\ blast\isanewline
\ \ \ \ \isacommand{obtain}\isamarkupfalse%
\ N\ \isakeyword{where}\ {\isachardoublequoteopen}{\isasymforall}p{\isasymge}N{\isachardot}{\kern0pt}\ C\ {\isacharplus}{\kern0pt}\ D\ {\isasymle}\ x\ p{\isachardoublequoteclose}\ \isacommand{using}\isamarkupfalse%
\ D\ bounds\ asm\ \isacommand{by}\isamarkupfalse%
\ {\isacharparenleft}{\kern0pt}fastforce\ simp\ add{\isacharcolon}{\kern0pt}\ pos{\isacharunderscore}{\kern0pt}def{\isacharparenright}{\kern0pt}\isanewline
\ \ \ \ \isacommand{hence}\isamarkupfalse%
\ {\isachardoublequoteopen}{\isasymforall}p{\isasymge}N{\isachardot}{\kern0pt}\ {\isasymbar}x\ p\ {\isacharminus}{\kern0pt}\ y\ p{\isasymbar}\ {\isacharplus}{\kern0pt}\ D\ {\isasymle}\ x\ p{\isachardoublequoteclose}\ \isacommand{by}\isamarkupfalse%
\ {\isacharparenleft}{\kern0pt}metis\ add{\isachardot}{\kern0pt}commute\ add{\isacharunderscore}{\kern0pt}left{\isacharunderscore}{\kern0pt}mono\ bounds{\isacharparenleft}{\kern0pt}{\isadigit{1}}{\isacharparenright}{\kern0pt}\ dual{\isacharunderscore}{\kern0pt}order{\isachardot}{\kern0pt}trans{\isacharparenright}{\kern0pt}\isanewline
\ \ \ \ \isacommand{hence}\isamarkupfalse%
\ {\isachardoublequoteopen}{\isasymforall}p{\isasymge}N{\isachardot}{\kern0pt}\ D\ {\isasymle}\ y\ p{\isachardoublequoteclose}\ \isacommand{by}\isamarkupfalse%
\ force\isanewline
\ \ \ \ \isacommand{hence}\isamarkupfalse%
\ False\ \isacommand{using}\isamarkupfalse%
\ D\ \isacommand{by}\isamarkupfalse%
\ blast\isanewline
\ \ \isacommand{{\isacharbraceright}{\kern0pt}}\isamarkupfalse%
\isanewline
\ \ \isacommand{hence}\isamarkupfalse%
\ {\isachardoublequoteopen}pos\ x\ {\isasymLongrightarrow}\ pos\ y{\isachardoublequoteclose}\ \isakeyword{if}\ {\isachardoublequoteopen}x\ {\isasymsim}\isactrlsub e\ y{\isachardoublequoteclose}\ \isakeyword{for}\ x\ y\ \isacommand{using}\isamarkupfalse%
\ that\ \isacommand{unfolding}\isamarkupfalse%
\ pos{\isacharunderscore}{\kern0pt}def\ \isacommand{by}\isamarkupfalse%
\ metis\isanewline
\ \ \isacommand{thus}\isamarkupfalse%
\ {\isacharquery}{\kern0pt}thesis\ \isacommand{by}\isamarkupfalse%
\ {\isacharparenleft}{\kern0pt}metis\ assms\ eudoxus{\isacharunderscore}{\kern0pt}rel{\isacharunderscore}{\kern0pt}equivp\ part{\isacharunderscore}{\kern0pt}equivp{\isacharunderscore}{\kern0pt}symp{\isacharparenright}{\kern0pt}\isanewline
\isacommand{qed}\isamarkupfalse%
%
\endisatagproof
{\isafoldproof}%
%
\isadelimproof
\isanewline
%
\endisadelimproof
\isanewline
\isacommand{lemma}\isamarkupfalse%
\ neg{\isacharunderscore}{\kern0pt}cong{\isacharcolon}{\kern0pt}\isanewline
\ \ \isakeyword{assumes}\ {\isachardoublequoteopen}f\ {\isasymsim}\isactrlsub e\ g{\isachardoublequoteclose}\isanewline
\ \ \isakeyword{shows}\ {\isachardoublequoteopen}neg\ f\ {\isacharequal}{\kern0pt}\ neg\ g{\isachardoublequoteclose}\isanewline
%
\isadelimproof
%
\endisadelimproof
%
\isatagproof
\isacommand{proof}\isamarkupfalse%
\ {\isacharminus}{\kern0pt}\isanewline
\ \ \isacommand{{\isacharbraceleft}{\kern0pt}}\isamarkupfalse%
\ \isanewline
\ \ \ \ \isacommand{fix}\isamarkupfalse%
\ x\ y\ \isacommand{assume}\isamarkupfalse%
\ asm{\isacharcolon}{\kern0pt}\ {\isachardoublequoteopen}neg\ x{\isachardoublequoteclose}\ {\isachardoublequoteopen}x\ {\isasymsim}\isactrlsub e\ y{\isachardoublequoteclose}\isanewline
\ \ \ \ \isacommand{fix}\isamarkupfalse%
\ D\ \isacommand{assume}\isamarkupfalse%
\ D{\isacharcolon}{\kern0pt}\ {\isachardoublequoteopen}{\isadigit{0}}\ {\isasymle}\ D{\isachardoublequoteclose}\ {\isachardoublequoteopen}{\isasymforall}N{\isachardot}{\kern0pt}\ {\isasymexists}p{\isasymge}N{\isachardot}{\kern0pt}\ {\isasymnot}\ {\isacharminus}{\kern0pt}\ D\ {\isasymge}\ y\ p{\isachardoublequoteclose}\isanewline
\ \ \ \ \isacommand{obtain}\isamarkupfalse%
\ C\ \isakeyword{where}\ bounds{\isacharcolon}{\kern0pt}\ {\isachardoublequoteopen}{\isasymbar}x\ n\ {\isacharminus}{\kern0pt}\ y\ n{\isasymbar}\ {\isasymle}\ C{\isachardoublequoteclose}\ {\isachardoublequoteopen}{\isadigit{0}}\ {\isasymle}\ C{\isachardoublequoteclose}\ \isakeyword{for}\ n\ \isacommand{using}\isamarkupfalse%
\ asm\ \isacommand{unfolding}\isamarkupfalse%
\ eudoxus{\isacharunderscore}{\kern0pt}rel{\isacharunderscore}{\kern0pt}def\ \isacommand{by}\isamarkupfalse%
\ blast\isanewline
\ \ \ \ \isacommand{obtain}\isamarkupfalse%
\ N\ \isakeyword{where}\ {\isachardoublequoteopen}{\isasymforall}p{\isasymge}N{\isachardot}{\kern0pt}\ {\isacharminus}{\kern0pt}\ {\isacharparenleft}{\kern0pt}C\ {\isacharplus}{\kern0pt}\ D{\isacharparenright}{\kern0pt}\ {\isasymge}\ x\ p{\isachardoublequoteclose}\ \isacommand{using}\isamarkupfalse%
\ D\ bounds\ asm\ add{\isacharunderscore}{\kern0pt}increasing{\isadigit{2}}\ \isacommand{unfolding}\isamarkupfalse%
\ neg{\isacharunderscore}{\kern0pt}def\ \isacommand{by}\isamarkupfalse%
\ meson\isanewline
\ \ \ \ \isacommand{hence}\isamarkupfalse%
\ {\isachardoublequoteopen}{\isasymforall}p{\isasymge}N{\isachardot}{\kern0pt}\ {\isacharminus}{\kern0pt}\ {\isasymbar}x\ p\ {\isacharminus}{\kern0pt}\ y\ p{\isasymbar}\ {\isacharminus}{\kern0pt}\ D\ {\isasymge}\ x\ p{\isachardoublequoteclose}\ \isacommand{using}\isamarkupfalse%
\ bounds{\isacharparenleft}{\kern0pt}{\isadigit{1}}{\isacharparenright}{\kern0pt}{\isacharbrackleft}{\kern0pt}THEN\ le{\isacharunderscore}{\kern0pt}imp{\isacharunderscore}{\kern0pt}neg{\isacharunderscore}{\kern0pt}le{\isacharcomma}{\kern0pt}\ THEN\ diff{\isacharunderscore}{\kern0pt}right{\isacharunderscore}{\kern0pt}mono{\isacharcomma}{\kern0pt}\ THEN\ dual{\isacharunderscore}{\kern0pt}order{\isachardot}{\kern0pt}trans{\isacharbrackright}{\kern0pt}\ \isacommand{by}\isamarkupfalse%
\ simp\isanewline
\ \ \ \ \isacommand{hence}\isamarkupfalse%
\ {\isachardoublequoteopen}{\isasymforall}p{\isasymge}N{\isachardot}{\kern0pt}\ {\isacharminus}{\kern0pt}\ D\ {\isasymge}\ y\ p{\isachardoublequoteclose}\ \isacommand{by}\isamarkupfalse%
\ force\isanewline
\ \ \ \ \isacommand{hence}\isamarkupfalse%
\ False\ \isacommand{using}\isamarkupfalse%
\ D\ \isacommand{by}\isamarkupfalse%
\ blast\isanewline
\ \ \isacommand{{\isacharbraceright}{\kern0pt}}\isamarkupfalse%
\isanewline
\ \ \isacommand{hence}\isamarkupfalse%
\ {\isachardoublequoteopen}neg\ x\ {\isasymLongrightarrow}\ neg\ y{\isachardoublequoteclose}\ \isakeyword{if}\ {\isachardoublequoteopen}x\ {\isasymsim}\isactrlsub e\ y{\isachardoublequoteclose}\ \isakeyword{for}\ x\ y\ \isacommand{using}\isamarkupfalse%
\ that\ \isacommand{unfolding}\isamarkupfalse%
\ neg{\isacharunderscore}{\kern0pt}def\ \isacommand{by}\isamarkupfalse%
\ metis\isanewline
\ \ \isacommand{thus}\isamarkupfalse%
\ {\isacharquery}{\kern0pt}thesis\ \isacommand{by}\isamarkupfalse%
\ {\isacharparenleft}{\kern0pt}metis\ assms\ eudoxus{\isacharunderscore}{\kern0pt}rel{\isacharunderscore}{\kern0pt}equivp\ part{\isacharunderscore}{\kern0pt}equivp{\isacharunderscore}{\kern0pt}symp{\isacharparenright}{\kern0pt}\isanewline
\isacommand{qed}\isamarkupfalse%
%
\endisatagproof
{\isafoldproof}%
%
\isadelimproof
\isanewline
%
\endisadelimproof
\isanewline
\isacommand{lemma}\isamarkupfalse%
\ pos{\isacharunderscore}{\kern0pt}iff{\isacharunderscore}{\kern0pt}nonneg{\isacharunderscore}{\kern0pt}nonzero{\isacharcolon}{\kern0pt}\ \isanewline
\ \ \isakeyword{assumes}\ {\isachardoublequoteopen}slope\ f{\isachardoublequoteclose}\isanewline
\ \ \isakeyword{shows}\ {\isachardoublequoteopen}pos\ f\ {\isasymlongleftrightarrow}\ {\isacharparenleft}{\kern0pt}{\isasymnot}\ neg\ f{\isacharparenright}{\kern0pt}\ {\isasymand}\ {\isacharparenleft}{\kern0pt}{\isasymnot}\ bounded\ f{\isacharparenright}{\kern0pt}{\isachardoublequoteclose}\ {\isacharparenleft}{\kern0pt}\isakeyword{is}\ {\isachardoublequoteopen}{\isacharquery}{\kern0pt}lhs\ {\isasymlongleftrightarrow}\ {\isacharquery}{\kern0pt}rhs{\isachardoublequoteclose}{\isacharparenright}{\kern0pt}\isanewline
%
\isadelimproof
%
\endisadelimproof
%
\isatagproof
\isacommand{proof}\isamarkupfalse%
\ {\isacharparenleft}{\kern0pt}rule\ iffI{\isacharparenright}{\kern0pt}\isanewline
\ \ \isacommand{assume}\isamarkupfalse%
\ pos{\isacharcolon}{\kern0pt}\ {\isacharquery}{\kern0pt}lhs\isanewline
\ \ \isacommand{then}\isamarkupfalse%
\ \isacommand{obtain}\isamarkupfalse%
\ N\ \isakeyword{where}\ {\isachardoublequoteopen}{\isasymforall}n{\isasymge}N{\isachardot}{\kern0pt}\ f\ n\ {\isachargreater}{\kern0pt}\ {\isadigit{0}}{\isachardoublequoteclose}\ \isacommand{unfolding}\isamarkupfalse%
\ pos{\isacharunderscore}{\kern0pt}def\ \isacommand{by}\isamarkupfalse%
\ {\isacharparenleft}{\kern0pt}metis\ int{\isacharunderscore}{\kern0pt}one{\isacharunderscore}{\kern0pt}le{\isacharunderscore}{\kern0pt}iff{\isacharunderscore}{\kern0pt}zero{\isacharunderscore}{\kern0pt}less\ zero{\isacharunderscore}{\kern0pt}less{\isacharunderscore}{\kern0pt}one{\isacharunderscore}{\kern0pt}class{\isachardot}{\kern0pt}zero{\isacharunderscore}{\kern0pt}le{\isacharunderscore}{\kern0pt}one{\isacharparenright}{\kern0pt}\isanewline
\ \ \isacommand{hence}\isamarkupfalse%
\ {\isachardoublequoteopen}f\ {\isacharparenleft}{\kern0pt}max\ N\ m{\isacharparenright}{\kern0pt}\ {\isachargreater}{\kern0pt}\ {\isadigit{0}}{\isachardoublequoteclose}\ \isakeyword{for}\ m\ \isacommand{by}\isamarkupfalse%
\ simp\isanewline
\ \ \isacommand{hence}\isamarkupfalse%
\ {\isachardoublequoteopen}{\isasymnot}\ neg\ f{\isachardoublequoteclose}\ \isacommand{unfolding}\isamarkupfalse%
\ neg{\isacharunderscore}{\kern0pt}def\ \isacommand{by}\isamarkupfalse%
\ {\isacharparenleft}{\kern0pt}metis\ add{\isachardot}{\kern0pt}inverse{\isacharunderscore}{\kern0pt}neutral\ dual{\isacharunderscore}{\kern0pt}order{\isachardot}{\kern0pt}refl\ linorder{\isacharunderscore}{\kern0pt}not{\isacharunderscore}{\kern0pt}le\ max{\isachardot}{\kern0pt}cobounded{\isadigit{2}}{\isacharparenright}{\kern0pt}\isanewline
\ \ \isacommand{thus}\isamarkupfalse%
\ {\isacharquery}{\kern0pt}rhs\ \isacommand{using}\isamarkupfalse%
\ pos\ \isacommand{unfolding}\isamarkupfalse%
\ pos{\isacharunderscore}{\kern0pt}def\ bounded{\isacharunderscore}{\kern0pt}def\ bdd{\isacharunderscore}{\kern0pt}above{\isacharunderscore}{\kern0pt}def\ \isacommand{by}\isamarkupfalse%
\ {\isacharparenleft}{\kern0pt}metis\ abs{\isacharunderscore}{\kern0pt}ge{\isacharunderscore}{\kern0pt}self\ dual{\isacharunderscore}{\kern0pt}order{\isachardot}{\kern0pt}trans\ gt{\isacharunderscore}{\kern0pt}ex\ imageI\ iso{\isacharunderscore}{\kern0pt}tuple{\isacharunderscore}{\kern0pt}UNIV{\isacharunderscore}{\kern0pt}I\ order{\isachardot}{\kern0pt}strict{\isacharunderscore}{\kern0pt}iff{\isacharunderscore}{\kern0pt}not{\isacharparenright}{\kern0pt}\isanewline
\isacommand{next}\isamarkupfalse%
\isanewline
\ \ \isacommand{assume}\isamarkupfalse%
\ nonneg{\isacharunderscore}{\kern0pt}nonzero{\isacharcolon}{\kern0pt}\ {\isacharquery}{\kern0pt}rhs\isanewline
\ \ \isacommand{hence}\isamarkupfalse%
\ finite{\isacharcolon}{\kern0pt}\ {\isachardoublequoteopen}finite\ {\isacharparenleft}{\kern0pt}f\ {\isacharbackquote}{\kern0pt}\ {\isacharbraceleft}{\kern0pt}{\isadigit{0}}{\isachardot}{\kern0pt}{\isachardot}{\kern0pt}{\isacharbraceright}{\kern0pt}\ {\isasyminter}\ {\isacharbraceleft}{\kern0pt}{\isachardot}{\kern0pt}{\isachardot}{\kern0pt}{\isacharless}{\kern0pt}{\isadigit{0}}{\isacharbraceright}{\kern0pt}{\isacharparenright}{\kern0pt}{\isachardoublequoteclose}\ \isacommand{using}\isamarkupfalse%
\ neg{\isacharunderscore}{\kern0pt}iff\ assms\ \isacommand{by}\isamarkupfalse%
\ blast\isanewline
\ \ \isacommand{moreover}\isamarkupfalse%
\ \isacommand{have}\isamarkupfalse%
\ unbounded{\isacharcolon}{\kern0pt}\ {\isachardoublequoteopen}infinite\ {\isacharparenleft}{\kern0pt}f\ {\isacharbackquote}{\kern0pt}\ {\isacharbraceleft}{\kern0pt}{\isadigit{0}}{\isachardot}{\kern0pt}{\isachardot}{\kern0pt}{\isacharbraceright}{\kern0pt}{\isacharparenright}{\kern0pt}{\isachardoublequoteclose}\ \isacommand{using}\isamarkupfalse%
\ nonneg{\isacharunderscore}{\kern0pt}nonzero\ bounded{\isacharunderscore}{\kern0pt}iff{\isacharunderscore}{\kern0pt}finite{\isacharunderscore}{\kern0pt}range\ slope{\isacharunderscore}{\kern0pt}finite{\isacharunderscore}{\kern0pt}range{\isacharunderscore}{\kern0pt}iff\ assms\ \isacommand{by}\isamarkupfalse%
\ blast\isanewline
\ \ \isacommand{ultimately}\isamarkupfalse%
\ \isacommand{have}\isamarkupfalse%
\ {\isachardoublequoteopen}infinite\ {\isacharparenleft}{\kern0pt}f\ {\isacharbackquote}{\kern0pt}\ {\isacharbraceleft}{\kern0pt}{\isadigit{0}}{\isachardot}{\kern0pt}{\isachardot}{\kern0pt}{\isacharbraceright}{\kern0pt}\ {\isasyminter}\ {\isacharbraceleft}{\kern0pt}{\isadigit{0}}{\isachardot}{\kern0pt}{\isachardot}{\kern0pt}{\isacharbraceright}{\kern0pt}{\isacharparenright}{\kern0pt}{\isachardoublequoteclose}\ \isacommand{by}\isamarkupfalse%
\ {\isacharparenleft}{\kern0pt}metis\ Compl{\isacharunderscore}{\kern0pt}atLeast\ Diff{\isacharunderscore}{\kern0pt}Diff{\isacharunderscore}{\kern0pt}Int\ Diff{\isacharunderscore}{\kern0pt}eq\ Diff{\isacharunderscore}{\kern0pt}infinite{\isacharunderscore}{\kern0pt}finite{\isacharparenright}{\kern0pt}\isanewline
\ \ \isacommand{moreover}\isamarkupfalse%
\ \isacommand{have}\isamarkupfalse%
\ {\isachardoublequoteopen}f\ {\isacharbackquote}{\kern0pt}\ {\isacharbraceleft}{\kern0pt}{\isadigit{0}}{\isachardot}{\kern0pt}{\isachardot}{\kern0pt}{\isacharbraceright}{\kern0pt}\ {\isasyminter}\ {\isacharbraceleft}{\kern0pt}{\isadigit{0}}{\isacharless}{\kern0pt}{\isachardot}{\kern0pt}{\isachardot}{\kern0pt}{\isacharbraceright}{\kern0pt}\ {\isacharequal}{\kern0pt}\ f\ {\isacharbackquote}{\kern0pt}\ {\isacharbraceleft}{\kern0pt}{\isadigit{0}}{\isachardot}{\kern0pt}{\isachardot}{\kern0pt}{\isacharbraceright}{\kern0pt}\ {\isasyminter}\ {\isacharbraceleft}{\kern0pt}{\isadigit{0}}{\isachardot}{\kern0pt}{\isachardot}{\kern0pt}{\isacharbraceright}{\kern0pt}\ {\isacharminus}{\kern0pt}\ {\isacharbraceleft}{\kern0pt}{\isadigit{0}}{\isacharbraceright}{\kern0pt}{\isachardoublequoteclose}\ \isacommand{by}\isamarkupfalse%
\ force\isanewline
\ \ \isacommand{ultimately}\isamarkupfalse%
\ \isacommand{show}\isamarkupfalse%
\ {\isacharquery}{\kern0pt}lhs\ \isacommand{unfolding}\isamarkupfalse%
\ pos{\isacharunderscore}{\kern0pt}iff{\isacharbrackleft}{\kern0pt}OF\ assms{\isacharbrackright}{\kern0pt}\ \isacommand{by}\isamarkupfalse%
\ simp\isanewline
\isacommand{qed}\isamarkupfalse%
%
\endisatagproof
{\isafoldproof}%
%
\isadelimproof
\isanewline
%
\endisadelimproof
\isanewline
\isacommand{lemma}\isamarkupfalse%
\ neg{\isacharunderscore}{\kern0pt}iff{\isacharunderscore}{\kern0pt}nonpos{\isacharunderscore}{\kern0pt}nonzero{\isacharcolon}{\kern0pt}\ \isanewline
\ \ \isakeyword{assumes}\ {\isachardoublequoteopen}slope\ f{\isachardoublequoteclose}\isanewline
\ \ \isakeyword{shows}\ {\isachardoublequoteopen}neg\ f\ {\isasymlongleftrightarrow}\ {\isacharparenleft}{\kern0pt}{\isasymnot}\ pos\ f{\isacharparenright}{\kern0pt}\ {\isasymand}\ {\isacharparenleft}{\kern0pt}{\isasymnot}\ bounded\ f{\isacharparenright}{\kern0pt}{\isachardoublequoteclose}\isanewline
%
\isadelimproof
\ \ %
\endisadelimproof
%
\isatagproof
\isacommand{unfolding}\isamarkupfalse%
\ pos{\isacharunderscore}{\kern0pt}iff{\isacharunderscore}{\kern0pt}nonneg{\isacharunderscore}{\kern0pt}nonzero{\isacharbrackleft}{\kern0pt}OF\ assms{\isacharbrackright}{\kern0pt}\ neg{\isacharunderscore}{\kern0pt}iff{\isacharunderscore}{\kern0pt}pos{\isacharunderscore}{\kern0pt}uminus\ uminus{\isacharunderscore}{\kern0pt}apply\ \isanewline
\ \ \ \ \ \ \ \ \ \ \ \ eudoxus{\isacharunderscore}{\kern0pt}uminus{\isacharunderscore}{\kern0pt}def\ pos{\isacharunderscore}{\kern0pt}iff{\isacharunderscore}{\kern0pt}nonneg{\isacharunderscore}{\kern0pt}nonzero{\isacharbrackleft}{\kern0pt}OF\ slope{\isacharunderscore}{\kern0pt}uminus{\isacharprime}{\kern0pt}{\isacharcomma}{\kern0pt}\ OF\ assms{\isacharbrackright}{\kern0pt}\isanewline
\ \ \isacommand{by}\isamarkupfalse%
\ {\isacharparenleft}{\kern0pt}force\ simp\ add{\isacharcolon}{\kern0pt}\ bounded{\isacharunderscore}{\kern0pt}def\ bdd{\isacharunderscore}{\kern0pt}above{\isacharunderscore}{\kern0pt}def{\isacharparenright}{\kern0pt}%
\endisatagproof
{\isafoldproof}%
%
\isadelimproof
%
\endisadelimproof
%
\begin{isamarkuptext}%
We define the sign of a slope to be \isa{id} if it is positive, \isa{{\isacharminus}{\kern0pt}\isactrlsub e\ id} if it is negative and \isa{{\isasymlambda}{\isacharunderscore}{\kern0pt}{\isachardot}{\kern0pt}\ {\isadigit{0}}{\isacharcolon}{\kern0pt}{\isacharcolon}{\kern0pt}{\isacharprime}{\kern0pt}b} otherwise.%
\end{isamarkuptext}\isamarkuptrue%
\isacommand{definition}\isamarkupfalse%
\ eudoxus{\isacharunderscore}{\kern0pt}sgn\ {\isacharcolon}{\kern0pt}{\isacharcolon}{\kern0pt}\ {\isachardoublequoteopen}{\isacharparenleft}{\kern0pt}int\ {\isasymRightarrow}\ int{\isacharparenright}{\kern0pt}\ {\isasymRightarrow}\ {\isacharparenleft}{\kern0pt}int\ {\isasymRightarrow}\ int{\isacharparenright}{\kern0pt}{\isachardoublequoteclose}\ \isakeyword{where}\ \isanewline
\ \ {\isachardoublequoteopen}eudoxus{\isacharunderscore}{\kern0pt}sgn\ f\ {\isacharequal}{\kern0pt}\ {\isacharparenleft}{\kern0pt}if\ pos\ f\ then\ id\ else\ if\ neg\ f\ then\ {\isacharminus}{\kern0pt}\isactrlsub e\ id\ else\ {\isacharparenleft}{\kern0pt}{\isasymlambda}{\isacharunderscore}{\kern0pt}{\isachardot}{\kern0pt}\ {\isadigit{0}}{\isacharparenright}{\kern0pt}{\isacharparenright}{\kern0pt}{\isachardoublequoteclose}\isanewline
\isanewline
\isacommand{lemma}\isamarkupfalse%
\ eudoxus{\isacharunderscore}{\kern0pt}sgn{\isacharunderscore}{\kern0pt}iff{\isacharcolon}{\kern0pt}\isanewline
\ \ \isakeyword{assumes}\ {\isachardoublequoteopen}slope\ f{\isachardoublequoteclose}\isanewline
\ \ \isakeyword{shows}\ {\isachardoublequoteopen}eudoxus{\isacharunderscore}{\kern0pt}sgn\ f\ {\isacharequal}{\kern0pt}\ {\isacharparenleft}{\kern0pt}{\isasymlambda}{\isacharunderscore}{\kern0pt}{\isachardot}{\kern0pt}\ {\isadigit{0}}{\isacharparenright}{\kern0pt}\ {\isasymlongleftrightarrow}\ bounded\ f{\isachardoublequoteclose}\isanewline
\ \ \ \ \ \ \ \ {\isachardoublequoteopen}eudoxus{\isacharunderscore}{\kern0pt}sgn\ f\ {\isacharequal}{\kern0pt}\ id\ {\isasymlongleftrightarrow}\ pos\ f{\isachardoublequoteclose}\isanewline
\ \ \ \ \ \ \ \ {\isachardoublequoteopen}eudoxus{\isacharunderscore}{\kern0pt}sgn\ f\ {\isacharequal}{\kern0pt}\ {\isacharparenleft}{\kern0pt}{\isacharminus}{\kern0pt}\isactrlsub e\ id{\isacharparenright}{\kern0pt}\ {\isasymlongleftrightarrow}\ neg\ f{\isachardoublequoteclose}\ \isanewline
%
\isadelimproof
\ \ %
\endisadelimproof
%
\isatagproof
\isacommand{using}\isamarkupfalse%
\ eudoxus{\isacharunderscore}{\kern0pt}sgn{\isacharunderscore}{\kern0pt}def\ neg{\isacharunderscore}{\kern0pt}one{\isacharunderscore}{\kern0pt}def\ one{\isacharunderscore}{\kern0pt}def\ zero{\isacharunderscore}{\kern0pt}def\ assms\ neg{\isacharunderscore}{\kern0pt}iff{\isacharunderscore}{\kern0pt}nonpos{\isacharunderscore}{\kern0pt}nonzero\ pos{\isacharunderscore}{\kern0pt}iff{\isacharunderscore}{\kern0pt}nonneg{\isacharunderscore}{\kern0pt}nonzero\ \isacommand{by}\isamarkupfalse%
\ auto%
\endisatagproof
{\isafoldproof}%
%
\isadelimproof
\isanewline
%
\endisadelimproof
\isanewline
\isacommand{quotient{\isacharunderscore}{\kern0pt}definition}\isamarkupfalse%
\isanewline
\ \ {\isachardoublequoteopen}{\isacharparenleft}{\kern0pt}sgn\ {\isacharcolon}{\kern0pt}{\isacharcolon}{\kern0pt}\ real\ {\isasymRightarrow}\ real{\isacharparenright}{\kern0pt}{\isachardoublequoteclose}\ \isakeyword{is}\ eudoxus{\isacharunderscore}{\kern0pt}sgn\isanewline
%
\isadelimproof
\ \ %
\endisadelimproof
%
\isatagproof
\isacommand{unfolding}\isamarkupfalse%
\ eudoxus{\isacharunderscore}{\kern0pt}sgn{\isacharunderscore}{\kern0pt}def\isanewline
\ \ \isacommand{using}\isamarkupfalse%
\ eudoxus{\isacharunderscore}{\kern0pt}uminus{\isacharunderscore}{\kern0pt}cong\ neg{\isacharunderscore}{\kern0pt}cong\ pos{\isacharunderscore}{\kern0pt}cong\ slope{\isacharunderscore}{\kern0pt}one\ slope{\isacharunderscore}{\kern0pt}refl\ \isacommand{by}\isamarkupfalse%
\ fastforce%
\endisatagproof
{\isafoldproof}%
%
\isadelimproof
\isanewline
%
\endisadelimproof
\isanewline
\isacommand{lemmas}\isamarkupfalse%
\ eudoxus{\isacharunderscore}{\kern0pt}sgn{\isacharunderscore}{\kern0pt}cong\ {\isacharequal}{\kern0pt}\ apply{\isacharunderscore}{\kern0pt}rsp{\isacharprime}{\kern0pt}{\isacharbrackleft}{\kern0pt}OF\ sgn{\isacharunderscore}{\kern0pt}real{\isachardot}{\kern0pt}rsp{\isacharcomma}{\kern0pt}\ intro{\isacharbrackright}{\kern0pt}\isanewline
\isanewline
\isacommand{lemma}\isamarkupfalse%
\ eudoxus{\isacharunderscore}{\kern0pt}sgn{\isacharunderscore}{\kern0pt}cong{\isacharprime}{\kern0pt}{\isacharbrackleft}{\kern0pt}cong{\isacharbrackright}{\kern0pt}{\isacharcolon}{\kern0pt}\isanewline
\ \ \isakeyword{assumes}\ {\isachardoublequoteopen}f\ {\isasymsim}\isactrlsub e\ g{\isachardoublequoteclose}\isanewline
\ \ \isakeyword{shows}\ {\isachardoublequoteopen}eudoxus{\isacharunderscore}{\kern0pt}sgn\ f\ {\isacharequal}{\kern0pt}\ eudoxus{\isacharunderscore}{\kern0pt}sgn\ g{\isachardoublequoteclose}\ \isanewline
%
\isadelimproof
\ \ %
\endisadelimproof
%
\isatagproof
\isacommand{using}\isamarkupfalse%
\ assms\ eudoxus{\isacharunderscore}{\kern0pt}sgn{\isacharunderscore}{\kern0pt}def\ neg{\isacharunderscore}{\kern0pt}cong\ pos{\isacharunderscore}{\kern0pt}cong\ \isacommand{by}\isamarkupfalse%
\ presburger%
\endisatagproof
{\isafoldproof}%
%
\isadelimproof
\ \isanewline
%
\endisadelimproof
\isanewline
\isacommand{lemma}\isamarkupfalse%
\ sgn{\isacharunderscore}{\kern0pt}range{\isacharcolon}{\kern0pt}\ {\isachardoublequoteopen}sgn\ {\isacharparenleft}{\kern0pt}x\ {\isacharcolon}{\kern0pt}{\isacharcolon}{\kern0pt}\ real{\isacharparenright}{\kern0pt}\ {\isasymin}\ {\isacharbraceleft}{\kern0pt}{\isacharminus}{\kern0pt}{\isadigit{1}}{\isacharcomma}{\kern0pt}\ {\isadigit{0}}{\isacharcomma}{\kern0pt}\ {\isadigit{1}}{\isacharbraceright}{\kern0pt}{\isachardoublequoteclose}%
\isadelimproof
\ %
\endisadelimproof
%
\isatagproof
\isacommand{unfolding}\isamarkupfalse%
\ sgn{\isacharunderscore}{\kern0pt}real{\isacharunderscore}{\kern0pt}def\ zero{\isacharunderscore}{\kern0pt}def\ one{\isacharunderscore}{\kern0pt}def\ neg{\isacharunderscore}{\kern0pt}one{\isacharunderscore}{\kern0pt}def\ eudoxus{\isacharunderscore}{\kern0pt}sgn{\isacharunderscore}{\kern0pt}def\ \isacommand{by}\isamarkupfalse%
\ simp%
\endisatagproof
{\isafoldproof}%
%
\isadelimproof
%
\endisadelimproof
\isanewline
\isanewline
\isacommand{lemma}\isamarkupfalse%
\ sgn{\isacharunderscore}{\kern0pt}abs{\isacharunderscore}{\kern0pt}real{\isacharunderscore}{\kern0pt}zero{\isacharunderscore}{\kern0pt}iff{\isacharcolon}{\kern0pt}\isanewline
\ \ \isakeyword{assumes}\ {\isachardoublequoteopen}slope\ f{\isachardoublequoteclose}\isanewline
\ \ \isakeyword{shows}\ {\isachardoublequoteopen}sgn\ {\isacharparenleft}{\kern0pt}abs{\isacharunderscore}{\kern0pt}real\ f{\isacharparenright}{\kern0pt}\ {\isacharequal}{\kern0pt}\ {\isadigit{0}}\ {\isasymlongleftrightarrow}\ {\isacharparenleft}{\kern0pt}eudoxus{\isacharunderscore}{\kern0pt}sgn\ f\ {\isacharequal}{\kern0pt}\ {\isacharparenleft}{\kern0pt}{\isasymlambda}{\isacharunderscore}{\kern0pt}{\isachardot}{\kern0pt}\ {\isadigit{0}}{\isacharparenright}{\kern0pt}{\isacharparenright}{\kern0pt}{\isachardoublequoteclose}\ {\isacharparenleft}{\kern0pt}\isakeyword{is}\ {\isachardoublequoteopen}{\isacharquery}{\kern0pt}lhs\ {\isasymlongleftrightarrow}\ {\isacharquery}{\kern0pt}rhs{\isachardoublequoteclose}{\isacharparenright}{\kern0pt}\isanewline
%
\isadelimproof
\ \ %
\endisadelimproof
%
\isatagproof
\isacommand{using}\isamarkupfalse%
\ eudoxus{\isacharunderscore}{\kern0pt}sgn{\isacharunderscore}{\kern0pt}cong{\isacharbrackleft}{\kern0pt}OF\ rep{\isacharunderscore}{\kern0pt}real{\isacharunderscore}{\kern0pt}abs{\isacharunderscore}{\kern0pt}real{\isacharunderscore}{\kern0pt}refl{\isacharcomma}{\kern0pt}\ OF\ assms{\isacharbrackright}{\kern0pt}\ abs{\isacharunderscore}{\kern0pt}real{\isacharunderscore}{\kern0pt}eqI\ eudoxus{\isacharunderscore}{\kern0pt}sgn{\isacharunderscore}{\kern0pt}def\ neg{\isacharunderscore}{\kern0pt}one{\isacharunderscore}{\kern0pt}def\ one{\isacharunderscore}{\kern0pt}def\ zero{\isacharunderscore}{\kern0pt}def\ \isanewline
\ \ \isacommand{by}\isamarkupfalse%
\ {\isacharparenleft}{\kern0pt}auto\ simp\ add{\isacharcolon}{\kern0pt}\ sgn{\isacharunderscore}{\kern0pt}real{\isacharunderscore}{\kern0pt}def{\isacharparenright}{\kern0pt}%
\endisatagproof
{\isafoldproof}%
%
\isadelimproof
\isanewline
%
\endisadelimproof
\isanewline
\isacommand{lemma}\isamarkupfalse%
\ sgn{\isacharunderscore}{\kern0pt}zero{\isacharunderscore}{\kern0pt}iff{\isacharbrackleft}{\kern0pt}simp{\isacharbrackright}{\kern0pt}{\isacharcolon}{\kern0pt}\ {\isachardoublequoteopen}sgn\ {\isacharparenleft}{\kern0pt}x\ {\isacharcolon}{\kern0pt}{\isacharcolon}{\kern0pt}\ real{\isacharparenright}{\kern0pt}\ {\isacharequal}{\kern0pt}\ {\isadigit{0}}\ {\isasymlongleftrightarrow}\ x\ {\isacharequal}{\kern0pt}\ {\isadigit{0}}{\isachardoublequoteclose}\isanewline
%
\isadelimproof
\ \ %
\endisadelimproof
%
\isatagproof
\isacommand{using}\isamarkupfalse%
\ eudoxus{\isacharunderscore}{\kern0pt}sgn{\isacharunderscore}{\kern0pt}iff{\isacharparenleft}{\kern0pt}{\isadigit{1}}{\isacharparenright}{\kern0pt}\ sgn{\isacharunderscore}{\kern0pt}abs{\isacharunderscore}{\kern0pt}real{\isacharunderscore}{\kern0pt}zero{\isacharunderscore}{\kern0pt}iff\ zero{\isacharunderscore}{\kern0pt}iff{\isacharunderscore}{\kern0pt}bounded{\isacharprime}{\kern0pt}\ slope{\isacharunderscore}{\kern0pt}refl\isanewline
\ \ \isacommand{by}\isamarkupfalse%
\ {\isacharparenleft}{\kern0pt}induct\ x{\isacharparenright}{\kern0pt}\ {\isacharparenleft}{\kern0pt}metis\ {\isacharparenleft}{\kern0pt}mono{\isacharunderscore}{\kern0pt}tags{\isacharparenright}{\kern0pt}\ rep{\isacharunderscore}{\kern0pt}real{\isacharunderscore}{\kern0pt}abs{\isacharunderscore}{\kern0pt}real{\isacharunderscore}{\kern0pt}refl\ rep{\isacharunderscore}{\kern0pt}real{\isacharunderscore}{\kern0pt}iff{\isacharparenright}{\kern0pt}%
\endisatagproof
{\isafoldproof}%
%
\isadelimproof
\isanewline
%
\endisadelimproof
\isanewline
\isacommand{lemma}\isamarkupfalse%
\ sgn{\isacharunderscore}{\kern0pt}zero{\isacharbrackleft}{\kern0pt}simp{\isacharbrackright}{\kern0pt}{\isacharcolon}{\kern0pt}\ {\isachardoublequoteopen}sgn\ {\isacharparenleft}{\kern0pt}{\isadigit{0}}\ {\isacharcolon}{\kern0pt}{\isacharcolon}{\kern0pt}\ real{\isacharparenright}{\kern0pt}\ {\isacharequal}{\kern0pt}\ {\isadigit{0}}{\isachardoublequoteclose}%
\isadelimproof
\ %
\endisadelimproof
%
\isatagproof
\isacommand{by}\isamarkupfalse%
\ simp%
\endisatagproof
{\isafoldproof}%
%
\isadelimproof
%
\endisadelimproof
\isanewline
\isanewline
\isacommand{lemma}\isamarkupfalse%
\ sgn{\isacharunderscore}{\kern0pt}abs{\isacharunderscore}{\kern0pt}real{\isacharunderscore}{\kern0pt}one{\isacharunderscore}{\kern0pt}iff{\isacharcolon}{\kern0pt}\ \isanewline
\ \ \isakeyword{assumes}\ {\isachardoublequoteopen}slope\ f{\isachardoublequoteclose}\isanewline
\ \ \isakeyword{shows}\ {\isachardoublequoteopen}sgn\ {\isacharparenleft}{\kern0pt}abs{\isacharunderscore}{\kern0pt}real\ f{\isacharparenright}{\kern0pt}\ {\isacharequal}{\kern0pt}\ {\isadigit{1}}\ {\isasymlongleftrightarrow}\ pos\ f{\isachardoublequoteclose}\isanewline
%
\isadelimproof
\ \ %
\endisadelimproof
%
\isatagproof
\isacommand{using}\isamarkupfalse%
\ eudoxus{\isacharunderscore}{\kern0pt}sgn{\isacharunderscore}{\kern0pt}cong{\isacharbrackleft}{\kern0pt}OF\ rep{\isacharunderscore}{\kern0pt}real{\isacharunderscore}{\kern0pt}abs{\isacharunderscore}{\kern0pt}real{\isacharunderscore}{\kern0pt}refl{\isacharcomma}{\kern0pt}\ OF\ assms{\isacharbrackright}{\kern0pt}\ abs{\isacharunderscore}{\kern0pt}real{\isacharunderscore}{\kern0pt}eqI\ eudoxus{\isacharunderscore}{\kern0pt}sgn{\isacharunderscore}{\kern0pt}def\ neg{\isacharunderscore}{\kern0pt}one{\isacharunderscore}{\kern0pt}def\ one{\isacharunderscore}{\kern0pt}def\ zero{\isacharunderscore}{\kern0pt}def\ \isanewline
\ \ \isacommand{by}\isamarkupfalse%
\ {\isacharparenleft}{\kern0pt}auto\ simp\ add{\isacharcolon}{\kern0pt}\ sgn{\isacharunderscore}{\kern0pt}real{\isacharunderscore}{\kern0pt}def{\isacharparenright}{\kern0pt}%
\endisatagproof
{\isafoldproof}%
%
\isadelimproof
\isanewline
%
\endisadelimproof
\isanewline
\isacommand{lemmas}\isamarkupfalse%
\ sgn{\isacharunderscore}{\kern0pt}pos\ {\isacharequal}{\kern0pt}\ sgn{\isacharunderscore}{\kern0pt}abs{\isacharunderscore}{\kern0pt}real{\isacharunderscore}{\kern0pt}one{\isacharunderscore}{\kern0pt}iff{\isacharbrackleft}{\kern0pt}THEN\ iffD{\isadigit{2}}{\isacharcomma}{\kern0pt}\ simp{\isacharbrackright}{\kern0pt}\isanewline
\isanewline
\isacommand{lemma}\isamarkupfalse%
\ sgn{\isacharunderscore}{\kern0pt}one{\isacharbrackleft}{\kern0pt}simp{\isacharbrackright}{\kern0pt}{\isacharcolon}{\kern0pt}\ {\isachardoublequoteopen}sgn\ {\isacharparenleft}{\kern0pt}{\isadigit{1}}\ {\isacharcolon}{\kern0pt}{\isacharcolon}{\kern0pt}\ real{\isacharparenright}{\kern0pt}\ {\isacharequal}{\kern0pt}\ {\isadigit{1}}{\isachardoublequoteclose}%
\isadelimproof
\ %
\endisadelimproof
%
\isatagproof
\isacommand{by}\isamarkupfalse%
\ {\isacharparenleft}{\kern0pt}subst\ one{\isacharunderscore}{\kern0pt}def{\isacharparenright}{\kern0pt}\ {\isacharparenleft}{\kern0pt}fastforce\ simp\ add{\isacharcolon}{\kern0pt}\ pos{\isacharunderscore}{\kern0pt}def\ iff{\isacharcolon}{\kern0pt}\ sgn{\isacharunderscore}{\kern0pt}abs{\isacharunderscore}{\kern0pt}real{\isacharunderscore}{\kern0pt}one{\isacharunderscore}{\kern0pt}iff{\isacharparenright}{\kern0pt}%
\endisatagproof
{\isafoldproof}%
%
\isadelimproof
%
\endisadelimproof
\isanewline
\isanewline
\isacommand{lemma}\isamarkupfalse%
\ sgn{\isacharunderscore}{\kern0pt}abs{\isacharunderscore}{\kern0pt}real{\isacharunderscore}{\kern0pt}neg{\isacharunderscore}{\kern0pt}one{\isacharunderscore}{\kern0pt}iff{\isacharcolon}{\kern0pt}\isanewline
\ \ \isakeyword{assumes}\ {\isachardoublequoteopen}slope\ f{\isachardoublequoteclose}\isanewline
\ \ \isakeyword{shows}\ {\isachardoublequoteopen}sgn\ {\isacharparenleft}{\kern0pt}abs{\isacharunderscore}{\kern0pt}real\ f{\isacharparenright}{\kern0pt}\ {\isacharequal}{\kern0pt}\ {\isacharminus}{\kern0pt}\ {\isadigit{1}}\ {\isasymlongleftrightarrow}\ neg\ f{\isachardoublequoteclose}\isanewline
%
\isadelimproof
\ \ %
\endisadelimproof
%
\isatagproof
\isacommand{using}\isamarkupfalse%
\ eudoxus{\isacharunderscore}{\kern0pt}sgn{\isacharunderscore}{\kern0pt}cong{\isacharbrackleft}{\kern0pt}OF\ rep{\isacharunderscore}{\kern0pt}real{\isacharunderscore}{\kern0pt}abs{\isacharunderscore}{\kern0pt}real{\isacharunderscore}{\kern0pt}refl{\isacharcomma}{\kern0pt}\ OF\ assms{\isacharbrackright}{\kern0pt}\ abs{\isacharunderscore}{\kern0pt}real{\isacharunderscore}{\kern0pt}eqI\ eudoxus{\isacharunderscore}{\kern0pt}sgn{\isacharunderscore}{\kern0pt}def\ neg{\isacharunderscore}{\kern0pt}one{\isacharunderscore}{\kern0pt}def\ one{\isacharunderscore}{\kern0pt}def\ zero{\isacharunderscore}{\kern0pt}def\ pos{\isacharunderscore}{\kern0pt}neg{\isacharunderscore}{\kern0pt}exclusive\isanewline
\ \ \isacommand{by}\isamarkupfalse%
\ {\isacharparenleft}{\kern0pt}auto\ simp\ add{\isacharcolon}{\kern0pt}\ sgn{\isacharunderscore}{\kern0pt}real{\isacharunderscore}{\kern0pt}def{\isacharparenright}{\kern0pt}%
\endisatagproof
{\isafoldproof}%
%
\isadelimproof
\isanewline
%
\endisadelimproof
\isanewline
\isacommand{lemmas}\isamarkupfalse%
\ sgn{\isacharunderscore}{\kern0pt}neg\ {\isacharequal}{\kern0pt}\ sgn{\isacharunderscore}{\kern0pt}abs{\isacharunderscore}{\kern0pt}real{\isacharunderscore}{\kern0pt}neg{\isacharunderscore}{\kern0pt}one{\isacharunderscore}{\kern0pt}iff{\isacharbrackleft}{\kern0pt}THEN\ iffD{\isadigit{2}}{\isacharcomma}{\kern0pt}\ simp{\isacharbrackright}{\kern0pt}\isanewline
\isanewline
\isacommand{lemma}\isamarkupfalse%
\ sgn{\isacharunderscore}{\kern0pt}neg{\isacharunderscore}{\kern0pt}one{\isacharbrackleft}{\kern0pt}simp{\isacharbrackright}{\kern0pt}{\isacharcolon}{\kern0pt}\ {\isachardoublequoteopen}sgn\ {\isacharparenleft}{\kern0pt}{\isacharminus}{\kern0pt}\ {\isadigit{1}}\ {\isacharcolon}{\kern0pt}{\isacharcolon}{\kern0pt}\ real{\isacharparenright}{\kern0pt}\ {\isacharequal}{\kern0pt}\ {\isacharminus}{\kern0pt}\ {\isadigit{1}}{\isachardoublequoteclose}%
\isadelimproof
\ %
\endisadelimproof
%
\isatagproof
\isacommand{by}\isamarkupfalse%
\ {\isacharparenleft}{\kern0pt}subst\ neg{\isacharunderscore}{\kern0pt}one{\isacharunderscore}{\kern0pt}def{\isacharparenright}{\kern0pt}\ {\isacharparenleft}{\kern0pt}fastforce\ simp\ add{\isacharcolon}{\kern0pt}\ neg{\isacharunderscore}{\kern0pt}def\ eudoxus{\isacharunderscore}{\kern0pt}uminus{\isacharunderscore}{\kern0pt}def\ iff{\isacharcolon}{\kern0pt}\ sgn{\isacharunderscore}{\kern0pt}abs{\isacharunderscore}{\kern0pt}real{\isacharunderscore}{\kern0pt}neg{\isacharunderscore}{\kern0pt}one{\isacharunderscore}{\kern0pt}iff{\isacharparenright}{\kern0pt}%
\endisatagproof
{\isafoldproof}%
%
\isadelimproof
%
\endisadelimproof
\isanewline
\isanewline
\isacommand{lemma}\isamarkupfalse%
\ sgn{\isacharunderscore}{\kern0pt}plus{\isacharcolon}{\kern0pt}\isanewline
\ \ \isakeyword{assumes}\ {\isachardoublequoteopen}sgn\ x\ {\isacharequal}{\kern0pt}\ {\isacharparenleft}{\kern0pt}{\isadigit{1}}\ {\isacharcolon}{\kern0pt}{\isacharcolon}{\kern0pt}\ real{\isacharparenright}{\kern0pt}{\isachardoublequoteclose}\ {\isachardoublequoteopen}sgn\ y\ {\isacharequal}{\kern0pt}\ {\isadigit{1}}{\isachardoublequoteclose}\isanewline
\ \ \isakeyword{shows}\ {\isachardoublequoteopen}sgn\ {\isacharparenleft}{\kern0pt}x\ {\isacharplus}{\kern0pt}\ y{\isacharparenright}{\kern0pt}\ {\isacharequal}{\kern0pt}\ {\isadigit{1}}{\isachardoublequoteclose}\isanewline
%
\isadelimproof
%
\endisadelimproof
%
\isatagproof
\isacommand{proof}\isamarkupfalse%
\ {\isacharminus}{\kern0pt}\isanewline
\ \ \isacommand{have}\isamarkupfalse%
\ pos{\isacharcolon}{\kern0pt}\ {\isachardoublequoteopen}pos\ {\isacharparenleft}{\kern0pt}rep{\isacharunderscore}{\kern0pt}real\ x{\isacharparenright}{\kern0pt}{\isachardoublequoteclose}\ {\isachardoublequoteopen}pos\ {\isacharparenleft}{\kern0pt}rep{\isacharunderscore}{\kern0pt}real\ y{\isacharparenright}{\kern0pt}{\isachardoublequoteclose}\ \isacommand{using}\isamarkupfalse%
\ assms\ sgn{\isacharunderscore}{\kern0pt}abs{\isacharunderscore}{\kern0pt}real{\isacharunderscore}{\kern0pt}one{\isacharunderscore}{\kern0pt}iff{\isacharbrackleft}{\kern0pt}OF\ slope{\isacharunderscore}{\kern0pt}rep{\isacharunderscore}{\kern0pt}real{\isacharbrackright}{\kern0pt}\ \isacommand{by}\isamarkupfalse%
\ simp{\isacharplus}{\kern0pt}\isanewline
\ \ \isacommand{{\isacharbraceleft}{\kern0pt}}\isamarkupfalse%
\isanewline
\ \ \ \ \isacommand{fix}\isamarkupfalse%
\ C\ {\isacharcolon}{\kern0pt}{\isacharcolon}{\kern0pt}\ int\ \isacommand{assume}\isamarkupfalse%
\ C{\isacharunderscore}{\kern0pt}nonneg{\isacharcolon}{\kern0pt}\ {\isachardoublequoteopen}C\ {\isasymge}\ {\isadigit{0}}{\isachardoublequoteclose}\isanewline
\ \ \ \ \isacommand{then}\isamarkupfalse%
\ \isacommand{obtain}\isamarkupfalse%
\ N\ M\ \isakeyword{where}\ {\isachardoublequoteopen}{\isasymforall}n{\isasymge}N{\isachardot}{\kern0pt}\ rep{\isacharunderscore}{\kern0pt}real\ x\ n\ {\isasymge}\ C{\isachardoublequoteclose}\ {\isachardoublequoteopen}{\isasymforall}n{\isasymge}M{\isachardot}{\kern0pt}\ rep{\isacharunderscore}{\kern0pt}real\ y\ n\ {\isasymge}\ C{\isachardoublequoteclose}\ \isacommand{using}\isamarkupfalse%
\ pos\ \isacommand{unfolding}\isamarkupfalse%
\ pos{\isacharunderscore}{\kern0pt}def\ \isacommand{by}\isamarkupfalse%
\ presburger\isanewline
\ \ \ \ \isacommand{hence}\isamarkupfalse%
\ {\isachardoublequoteopen}{\isasymforall}n{\isasymge}\ max\ N\ M{\isachardot}{\kern0pt}\ {\isacharparenleft}{\kern0pt}rep{\isacharunderscore}{\kern0pt}real\ x\ {\isacharplus}{\kern0pt}\isactrlsub e\ rep{\isacharunderscore}{\kern0pt}real\ y{\isacharparenright}{\kern0pt}\ n\ {\isasymge}\ C{\isachardoublequoteclose}\ \isacommand{using}\isamarkupfalse%
\ C{\isacharunderscore}{\kern0pt}nonneg\ \isacommand{unfolding}\isamarkupfalse%
\ eudoxus{\isacharunderscore}{\kern0pt}plus{\isacharunderscore}{\kern0pt}def\ \isacommand{by}\isamarkupfalse%
\ fastforce\isanewline
\ \ \ \ \isacommand{hence}\isamarkupfalse%
\ {\isachardoublequoteopen}{\isasymexists}N{\isachardot}{\kern0pt}\ {\isasymforall}n\ {\isasymge}\ N{\isachardot}{\kern0pt}\ {\isacharparenleft}{\kern0pt}rep{\isacharunderscore}{\kern0pt}real\ x\ {\isacharplus}{\kern0pt}\isactrlsub e\ rep{\isacharunderscore}{\kern0pt}real\ y{\isacharparenright}{\kern0pt}\ n\ {\isasymge}\ C{\isachardoublequoteclose}\ \isacommand{by}\isamarkupfalse%
\ blast\isanewline
\ \ \isacommand{{\isacharbraceright}{\kern0pt}}\isamarkupfalse%
\isanewline
\ \ \isacommand{thus}\isamarkupfalse%
\ {\isacharquery}{\kern0pt}thesis\ \isacommand{using}\isamarkupfalse%
\ pos{\isacharunderscore}{\kern0pt}def\ \isacommand{by}\isamarkupfalse%
\ {\isacharparenleft}{\kern0pt}simp\ add{\isacharcolon}{\kern0pt}\ eudoxus{\isacharunderscore}{\kern0pt}plus{\isacharunderscore}{\kern0pt}cong\ plus{\isacharunderscore}{\kern0pt}real{\isacharunderscore}{\kern0pt}def{\isacharparenright}{\kern0pt}\isanewline
\isacommand{qed}\isamarkupfalse%
%
\endisatagproof
{\isafoldproof}%
%
\isadelimproof
\isanewline
%
\endisadelimproof
\isanewline
\isacommand{lemma}\isamarkupfalse%
\ sgn{\isacharunderscore}{\kern0pt}times{\isacharcolon}{\kern0pt}\ {\isachardoublequoteopen}sgn\ {\isacharparenleft}{\kern0pt}{\isacharparenleft}{\kern0pt}x\ {\isacharcolon}{\kern0pt}{\isacharcolon}{\kern0pt}\ real{\isacharparenright}{\kern0pt}\ {\isacharasterisk}{\kern0pt}\ y{\isacharparenright}{\kern0pt}\ {\isacharequal}{\kern0pt}\ sgn\ x\ {\isacharasterisk}{\kern0pt}\ sgn\ y{\isachardoublequoteclose}\isanewline
%
\isadelimproof
%
\endisadelimproof
%
\isatagproof
\isacommand{proof}\isamarkupfalse%
\ {\isacharparenleft}{\kern0pt}cases\ {\isachardoublequoteopen}x\ {\isacharequal}{\kern0pt}\ {\isadigit{0}}\ {\isasymor}\ y\ {\isacharequal}{\kern0pt}\ {\isadigit{0}}{\isachardoublequoteclose}{\isacharparenright}{\kern0pt}\isanewline
\ \ \isacommand{case}\isamarkupfalse%
\ False\isanewline
\ \ \isacommand{have}\isamarkupfalse%
\ {\isacharasterisk}{\kern0pt}{\isacharcolon}{\kern0pt}\ {\isachardoublequoteopen}{\isasymlbrakk}x\ {\isasymnoteq}\ {\isadigit{0}}{\isacharsemicolon}{\kern0pt}\ pos\ {\isacharparenleft}{\kern0pt}rep{\isacharunderscore}{\kern0pt}real\ y{\isacharparenright}{\kern0pt}{\isasymrbrakk}\ {\isasymLongrightarrow}\ sgn\ {\isacharparenleft}{\kern0pt}{\isacharparenleft}{\kern0pt}x\ {\isacharcolon}{\kern0pt}{\isacharcolon}{\kern0pt}\ real{\isacharparenright}{\kern0pt}\ {\isacharasterisk}{\kern0pt}\ y{\isacharparenright}{\kern0pt}\ {\isacharequal}{\kern0pt}\ sgn\ x\ {\isacharasterisk}{\kern0pt}\ sgn\ y{\isachardoublequoteclose}\ \isakeyword{for}\ x\ y\isanewline
\ \ \isacommand{proof}\isamarkupfalse%
\ {\isacharparenleft}{\kern0pt}induct\ x\ rule{\isacharcolon}{\kern0pt}\ slope{\isacharunderscore}{\kern0pt}induct{\isacharcomma}{\kern0pt}\ induct\ y\ rule{\isacharcolon}{\kern0pt}\ slope{\isacharunderscore}{\kern0pt}induct{\isacharparenright}{\kern0pt}\isanewline
\ \ \ \ \isacommand{case}\isamarkupfalse%
\ {\isacharparenleft}{\kern0pt}slope\ y\ x{\isacharparenright}{\kern0pt}\isanewline
\ \ \ \ \isacommand{hence}\isamarkupfalse%
\ pos{\isacharunderscore}{\kern0pt}y{\isacharcolon}{\kern0pt}\ {\isachardoublequoteopen}pos\ y{\isachardoublequoteclose}\ \isacommand{using}\isamarkupfalse%
\ pos{\isacharunderscore}{\kern0pt}cong\ \isacommand{by}\isamarkupfalse%
\ blast\isanewline
\ \ \ \ \isacommand{show}\isamarkupfalse%
\ {\isacharquery}{\kern0pt}case\isanewline
\ \ \ \ \isacommand{proof}\isamarkupfalse%
\ {\isacharparenleft}{\kern0pt}cases\ {\isachardoublequoteopen}pos\ x{\isachardoublequoteclose}{\isacharparenright}{\kern0pt}\isanewline
\ \ \ \ \ \ \isacommand{case}\isamarkupfalse%
\ pos{\isacharunderscore}{\kern0pt}x{\isacharcolon}{\kern0pt}\ True\isanewline
\ \ \ \ \ \ \isacommand{{\isacharbraceleft}{\kern0pt}}\isamarkupfalse%
\isanewline
\ \ \ \ \ \ \ \ \isacommand{fix}\isamarkupfalse%
\ C\ {\isacharcolon}{\kern0pt}{\isacharcolon}{\kern0pt}\ int\ \isacommand{assume}\isamarkupfalse%
\ asm{\isacharcolon}{\kern0pt}\ {\isachardoublequoteopen}C\ {\isasymge}\ {\isadigit{0}}{\isachardoublequoteclose}\isanewline
\ \ \ \ \ \ \ \ \isacommand{then}\isamarkupfalse%
\ \isacommand{obtain}\isamarkupfalse%
\ N\ \isakeyword{where}\ N{\isacharcolon}{\kern0pt}\ {\isachardoublequoteopen}{\isasymforall}n\ {\isasymge}\ N{\isachardot}{\kern0pt}\ x\ n\ {\isasymge}\ C{\isachardoublequoteclose}\ \isacommand{using}\isamarkupfalse%
\ pos{\isacharunderscore}{\kern0pt}x\ \isacommand{unfolding}\isamarkupfalse%
\ pos{\isacharunderscore}{\kern0pt}def\ \isacommand{by}\isamarkupfalse%
\ blast\isanewline
\ \ \ \ \ \ \ \ \isacommand{then}\isamarkupfalse%
\ \isacommand{obtain}\isamarkupfalse%
\ N{\isacharprime}{\kern0pt}\ \isakeyword{where}\ {\isachardoublequoteopen}{\isasymforall}n{\isasymge}\ N{\isacharprime}{\kern0pt}{\isachardot}{\kern0pt}\ y\ n\ {\isasymge}\ max\ {\isadigit{0}}\ N{\isachardoublequoteclose}\ \isacommand{using}\isamarkupfalse%
\ pos{\isacharunderscore}{\kern0pt}y\ \isacommand{unfolding}\isamarkupfalse%
\ pos{\isacharunderscore}{\kern0pt}def\ \isacommand{by}\isamarkupfalse%
\ {\isacharparenleft}{\kern0pt}meson\ max{\isachardot}{\kern0pt}cobounded{\isadigit{1}}{\isacharparenright}{\kern0pt}\isanewline
\ \ \ \ \ \ \ \ \isacommand{hence}\isamarkupfalse%
\ {\isachardoublequoteopen}{\isasymexists}N{\isacharprime}{\kern0pt}{\isachardot}{\kern0pt}\ {\isasymforall}n{\isasymge}\ N{\isacharprime}{\kern0pt}{\isachardot}{\kern0pt}\ x\ {\isacharparenleft}{\kern0pt}y\ n{\isacharparenright}{\kern0pt}\ {\isasymge}\ C{\isachardoublequoteclose}\ \isacommand{using}\isamarkupfalse%
\ N\ \isacommand{by}\isamarkupfalse%
\ force\isanewline
\ \ \ \ \ \ \isacommand{{\isacharbraceright}{\kern0pt}}\isamarkupfalse%
\isanewline
\ \ \ \ \ \ \isacommand{hence}\isamarkupfalse%
\ {\isachardoublequoteopen}pos\ {\isacharparenleft}{\kern0pt}x\ {\isacharasterisk}{\kern0pt}\isactrlsub e\ y{\isacharparenright}{\kern0pt}{\isachardoublequoteclose}\ \isacommand{unfolding}\isamarkupfalse%
\ pos{\isacharunderscore}{\kern0pt}def\ eudoxus{\isacharunderscore}{\kern0pt}times{\isacharunderscore}{\kern0pt}def\ \isacommand{by}\isamarkupfalse%
\ simp\isanewline
\ \ \ \ \ \ \isacommand{thus}\isamarkupfalse%
\ {\isacharquery}{\kern0pt}thesis\ \isacommand{using}\isamarkupfalse%
\ pos{\isacharunderscore}{\kern0pt}x\ pos{\isacharunderscore}{\kern0pt}y\ slope\ \isacommand{by}\isamarkupfalse%
\ {\isacharparenleft}{\kern0pt}simp\ add{\isacharcolon}{\kern0pt}\ eudoxus{\isacharunderscore}{\kern0pt}times{\isacharunderscore}{\kern0pt}def{\isacharparenright}{\kern0pt}\isanewline
\ \ \ \ \isacommand{next}\isamarkupfalse%
\isanewline
\ \ \ \ \ \ \isacommand{case}\isamarkupfalse%
\ {\isacharunderscore}{\kern0pt}{\isacharcolon}{\kern0pt}\ False\isanewline
\ \ \ \ \ \ \isacommand{hence}\isamarkupfalse%
\ neg{\isacharunderscore}{\kern0pt}x{\isacharcolon}{\kern0pt}\ {\isachardoublequoteopen}neg\ x{\isachardoublequoteclose}\ \isacommand{using}\isamarkupfalse%
\ slope\ \isacommand{by}\isamarkupfalse%
\ {\isacharparenleft}{\kern0pt}metis\ abs{\isacharunderscore}{\kern0pt}real{\isacharunderscore}{\kern0pt}eqI\ neg{\isacharunderscore}{\kern0pt}iff{\isacharunderscore}{\kern0pt}nonpos{\isacharunderscore}{\kern0pt}nonzero\ zero{\isacharunderscore}{\kern0pt}def\ zero{\isacharunderscore}{\kern0pt}iff{\isacharunderscore}{\kern0pt}bounded{\isacharparenright}{\kern0pt}\isanewline
\ \ \ \ \ \ \isacommand{{\isacharbraceleft}{\kern0pt}}\isamarkupfalse%
\isanewline
\ \ \ \ \ \ \ \ \isacommand{fix}\isamarkupfalse%
\ C\ {\isacharcolon}{\kern0pt}{\isacharcolon}{\kern0pt}\ int\ \isacommand{assume}\isamarkupfalse%
\ {\isachardoublequoteopen}C\ {\isasymge}\ {\isadigit{0}}{\isachardoublequoteclose}\isanewline
\ \ \ \ \ \ \ \ \isacommand{then}\isamarkupfalse%
\ \isacommand{obtain}\isamarkupfalse%
\ N\ \isakeyword{where}\ N{\isacharcolon}{\kern0pt}\ {\isachardoublequoteopen}{\isasymforall}n\ {\isasymge}\ N{\isachardot}{\kern0pt}\ x\ n\ {\isasymle}\ {\isacharminus}{\kern0pt}\ C{\isachardoublequoteclose}\ \isacommand{using}\isamarkupfalse%
\ neg{\isacharunderscore}{\kern0pt}x\ \isacommand{unfolding}\isamarkupfalse%
\ neg{\isacharunderscore}{\kern0pt}def\ \isacommand{by}\isamarkupfalse%
\ blast\isanewline
\ \ \ \ \ \ \ \ \isacommand{then}\isamarkupfalse%
\ \isacommand{obtain}\isamarkupfalse%
\ N{\isacharprime}{\kern0pt}\ \isakeyword{where}\ {\isachardoublequoteopen}{\isasymforall}n{\isasymge}\ N{\isacharprime}{\kern0pt}{\isachardot}{\kern0pt}\ y\ n\ {\isasymge}\ max\ {\isadigit{0}}\ N{\isachardoublequoteclose}\ \isacommand{using}\isamarkupfalse%
\ pos{\isacharunderscore}{\kern0pt}y\ \isacommand{unfolding}\isamarkupfalse%
\ pos{\isacharunderscore}{\kern0pt}def\ \isacommand{by}\isamarkupfalse%
\ {\isacharparenleft}{\kern0pt}meson\ max{\isachardot}{\kern0pt}cobounded{\isadigit{1}}{\isacharparenright}{\kern0pt}\isanewline
\ \ \ \ \ \ \ \ \isacommand{hence}\isamarkupfalse%
\ {\isachardoublequoteopen}{\isasymexists}N{\isacharprime}{\kern0pt}{\isachardot}{\kern0pt}\ {\isasymforall}n{\isasymge}\ N{\isacharprime}{\kern0pt}{\isachardot}{\kern0pt}\ x\ {\isacharparenleft}{\kern0pt}y\ n{\isacharparenright}{\kern0pt}\ {\isasymle}\ {\isacharminus}{\kern0pt}C{\isachardoublequoteclose}\ \isacommand{using}\isamarkupfalse%
\ N\ \isacommand{by}\isamarkupfalse%
\ force\isanewline
\ \ \ \ \ \ \isacommand{{\isacharbraceright}{\kern0pt}}\isamarkupfalse%
\isanewline
\ \ \ \ \ \ \isacommand{hence}\isamarkupfalse%
\ {\isachardoublequoteopen}neg\ {\isacharparenleft}{\kern0pt}x\ {\isacharasterisk}{\kern0pt}\isactrlsub e\ y{\isacharparenright}{\kern0pt}{\isachardoublequoteclose}\ \isacommand{unfolding}\isamarkupfalse%
\ neg{\isacharunderscore}{\kern0pt}def\ eudoxus{\isacharunderscore}{\kern0pt}times{\isacharunderscore}{\kern0pt}def\ \isacommand{by}\isamarkupfalse%
\ simp\isanewline
\ \ \ \ \ \ \isacommand{thus}\isamarkupfalse%
\ {\isacharquery}{\kern0pt}thesis\ \isacommand{using}\isamarkupfalse%
\ neg{\isacharunderscore}{\kern0pt}x\ pos{\isacharunderscore}{\kern0pt}y\ slope\ \isacommand{by}\isamarkupfalse%
\ {\isacharparenleft}{\kern0pt}simp\ add{\isacharcolon}{\kern0pt}\ eudoxus{\isacharunderscore}{\kern0pt}times{\isacharunderscore}{\kern0pt}def{\isacharparenright}{\kern0pt}\isanewline
\ \ \ \ \isacommand{qed}\isamarkupfalse%
\isanewline
\ \ \isacommand{qed}\isamarkupfalse%
\isanewline
\ \ \isacommand{moreover}\isamarkupfalse%
\ \isacommand{have}\isamarkupfalse%
\ {\isachardoublequoteopen}sgn\ {\isacharparenleft}{\kern0pt}{\isacharparenleft}{\kern0pt}x\ {\isacharcolon}{\kern0pt}{\isacharcolon}{\kern0pt}\ real{\isacharparenright}{\kern0pt}\ {\isacharasterisk}{\kern0pt}\ y{\isacharparenright}{\kern0pt}\ {\isacharequal}{\kern0pt}\ sgn\ x\ {\isacharasterisk}{\kern0pt}\ sgn\ y{\isachardoublequoteclose}\ \isakeyword{if}\ neg{\isacharunderscore}{\kern0pt}x{\isacharcolon}{\kern0pt}\ {\isachardoublequoteopen}neg\ {\isacharparenleft}{\kern0pt}rep{\isacharunderscore}{\kern0pt}real\ x{\isacharparenright}{\kern0pt}{\isachardoublequoteclose}\ \isakeyword{and}\ neg{\isacharunderscore}{\kern0pt}y{\isacharcolon}{\kern0pt}\ {\isachardoublequoteopen}neg\ {\isacharparenleft}{\kern0pt}rep{\isacharunderscore}{\kern0pt}real\ y{\isacharparenright}{\kern0pt}{\isachardoublequoteclose}\ \isakeyword{for}\ x\ y\isanewline
\ \ \isacommand{proof}\isamarkupfalse%
\ {\isacharminus}{\kern0pt}\isanewline
\ \ \ \ \isacommand{have}\isamarkupfalse%
\ pos{\isacharunderscore}{\kern0pt}uminus{\isacharunderscore}{\kern0pt}y{\isacharcolon}{\kern0pt}\ {\isachardoublequoteopen}pos\ {\isacharparenleft}{\kern0pt}rep{\isacharunderscore}{\kern0pt}real\ {\isacharparenleft}{\kern0pt}{\isacharminus}{\kern0pt}\ y{\isacharparenright}{\kern0pt}{\isacharparenright}{\kern0pt}{\isachardoublequoteclose}\ \isacommand{by}\isamarkupfalse%
\ {\isacharparenleft}{\kern0pt}metis\ abs{\isacharunderscore}{\kern0pt}real{\isacharunderscore}{\kern0pt}eq{\isacharunderscore}{\kern0pt}iff\ eudoxus{\isacharunderscore}{\kern0pt}uminus{\isacharunderscore}{\kern0pt}cong\ map{\isacharunderscore}{\kern0pt}fun{\isacharunderscore}{\kern0pt}apply\ neg{\isacharunderscore}{\kern0pt}iff{\isacharunderscore}{\kern0pt}pos{\isacharunderscore}{\kern0pt}uminus\ neg{\isacharunderscore}{\kern0pt}y\ pos{\isacharunderscore}{\kern0pt}cong\ rep{\isacharunderscore}{\kern0pt}real{\isacharunderscore}{\kern0pt}abs{\isacharunderscore}{\kern0pt}real{\isacharunderscore}{\kern0pt}refl\ rep{\isacharunderscore}{\kern0pt}real{\isacharunderscore}{\kern0pt}iff\ uminus{\isacharunderscore}{\kern0pt}real{\isacharunderscore}{\kern0pt}def{\isacharparenright}{\kern0pt}\isanewline
\ \ \ \ \isacommand{moreover}\isamarkupfalse%
\ \isacommand{have}\isamarkupfalse%
\ {\isachardoublequoteopen}x\ {\isasymnoteq}\ {\isadigit{0}}{\isachardoublequoteclose}\ \isacommand{using}\isamarkupfalse%
\ neg{\isacharunderscore}{\kern0pt}iff{\isacharunderscore}{\kern0pt}nonpos{\isacharunderscore}{\kern0pt}nonzero\ neg{\isacharunderscore}{\kern0pt}x\ zero{\isacharunderscore}{\kern0pt}iff{\isacharunderscore}{\kern0pt}bounded{\isacharprime}{\kern0pt}\ \isacommand{by}\isamarkupfalse%
\ fastforce\isanewline
\ \ \ \ \isacommand{ultimately}\isamarkupfalse%
\ \isacommand{have}\isamarkupfalse%
\ {\isachardoublequoteopen}sgn\ {\isacharparenleft}{\kern0pt}{\isacharminus}{\kern0pt}\ {\isacharparenleft}{\kern0pt}x\ {\isacharasterisk}{\kern0pt}\ y{\isacharparenright}{\kern0pt}{\isacharparenright}{\kern0pt}\ {\isacharequal}{\kern0pt}\ {\isacharminus}{\kern0pt}\ {\isadigit{1}}{\isachardoublequoteclose}\ \isacommand{using}\isamarkupfalse%
\ sgn{\isacharunderscore}{\kern0pt}neg{\isacharbrackleft}{\kern0pt}OF\ slope{\isacharunderscore}{\kern0pt}rep{\isacharunderscore}{\kern0pt}real\ neg{\isacharunderscore}{\kern0pt}x{\isacharbrackright}{\kern0pt}\ sgn{\isacharunderscore}{\kern0pt}pos{\isacharbrackleft}{\kern0pt}OF\ slope{\isacharunderscore}{\kern0pt}rep{\isacharunderscore}{\kern0pt}real\ pos{\isacharunderscore}{\kern0pt}uminus{\isacharunderscore}{\kern0pt}y{\isacharbrackright}{\kern0pt}\ {\isacharasterisk}{\kern0pt}\ \isacommand{by}\isamarkupfalse%
\ fastforce\isanewline
\ \ \ \ \isacommand{hence}\isamarkupfalse%
\ {\isachardoublequoteopen}pos\ {\isacharparenleft}{\kern0pt}rep{\isacharunderscore}{\kern0pt}real\ {\isacharparenleft}{\kern0pt}x\ {\isacharasterisk}{\kern0pt}\ y{\isacharparenright}{\kern0pt}{\isacharparenright}{\kern0pt}{\isachardoublequoteclose}\ \isacommand{by}\isamarkupfalse%
\ {\isacharparenleft}{\kern0pt}metis\ eudoxus{\isacharunderscore}{\kern0pt}uminus{\isacharunderscore}{\kern0pt}cong\ map{\isacharunderscore}{\kern0pt}fun{\isacharunderscore}{\kern0pt}apply\ pos{\isacharunderscore}{\kern0pt}iff{\isacharunderscore}{\kern0pt}neg{\isacharunderscore}{\kern0pt}uminus\ sgn{\isacharunderscore}{\kern0pt}abs{\isacharunderscore}{\kern0pt}real{\isacharunderscore}{\kern0pt}neg{\isacharunderscore}{\kern0pt}one{\isacharunderscore}{\kern0pt}iff\ slope{\isacharunderscore}{\kern0pt}refl\ slope{\isacharunderscore}{\kern0pt}rep{\isacharunderscore}{\kern0pt}real\ uminus{\isacharunderscore}{\kern0pt}real{\isacharunderscore}{\kern0pt}def{\isacharparenright}{\kern0pt}\isanewline
\ \ \ \ \isacommand{thus}\isamarkupfalse%
\ {\isacharquery}{\kern0pt}thesis\ \isacommand{using}\isamarkupfalse%
\ sgn{\isacharunderscore}{\kern0pt}neg{\isacharbrackleft}{\kern0pt}OF\ slope{\isacharunderscore}{\kern0pt}rep{\isacharunderscore}{\kern0pt}real{\isacharbrackright}{\kern0pt}\ sgn{\isacharunderscore}{\kern0pt}pos{\isacharbrackleft}{\kern0pt}OF\ slope{\isacharunderscore}{\kern0pt}rep{\isacharunderscore}{\kern0pt}real{\isacharbrackright}{\kern0pt}\ neg{\isacharunderscore}{\kern0pt}x\ neg{\isacharunderscore}{\kern0pt}y\ \isacommand{by}\isamarkupfalse%
\ simp\isanewline
\ \ \isacommand{qed}\isamarkupfalse%
\isanewline
\ \ \isacommand{ultimately}\isamarkupfalse%
\ \isacommand{show}\isamarkupfalse%
\ {\isacharquery}{\kern0pt}thesis\ \isacommand{using}\isamarkupfalse%
\ False\ neg{\isacharunderscore}{\kern0pt}iff{\isacharunderscore}{\kern0pt}nonpos{\isacharunderscore}{\kern0pt}nonzero{\isacharbrackleft}{\kern0pt}OF\ slope{\isacharunderscore}{\kern0pt}rep{\isacharunderscore}{\kern0pt}real{\isacharbrackright}{\kern0pt}\ zero{\isacharunderscore}{\kern0pt}iff{\isacharunderscore}{\kern0pt}bounded{\isacharprime}{\kern0pt}\isanewline
\ \ \ \ \isacommand{by}\isamarkupfalse%
\ {\isacharparenleft}{\kern0pt}cases\ {\isachardoublequoteopen}pos\ {\isacharparenleft}{\kern0pt}rep{\isacharunderscore}{\kern0pt}real\ x{\isacharparenright}{\kern0pt}{\isachardoublequoteclose}\ {\isacharsemicolon}{\kern0pt}\ cases\ {\isachardoublequoteopen}pos\ {\isacharparenleft}{\kern0pt}rep{\isacharunderscore}{\kern0pt}real\ y{\isacharparenright}{\kern0pt}{\isachardoublequoteclose}{\isacharparenright}{\kern0pt}\ {\isacharparenleft}{\kern0pt}fastforce\ simp\ add{\isacharcolon}{\kern0pt}\ mult{\isachardot}{\kern0pt}commute{\isacharparenright}{\kern0pt}{\isacharplus}{\kern0pt}\isanewline
\isacommand{qed}\isamarkupfalse%
\ {\isacharparenleft}{\kern0pt}force{\isacharparenright}{\kern0pt}%
\endisatagproof
{\isafoldproof}%
%
\isadelimproof
\isanewline
%
\endisadelimproof
\isanewline
\isacommand{lemma}\isamarkupfalse%
\ sgn{\isacharunderscore}{\kern0pt}uminus{\isacharcolon}{\kern0pt}\ {\isachardoublequoteopen}sgn\ {\isacharparenleft}{\kern0pt}{\isacharminus}{\kern0pt}\ {\isacharparenleft}{\kern0pt}x\ {\isacharcolon}{\kern0pt}{\isacharcolon}{\kern0pt}\ real{\isacharparenright}{\kern0pt}{\isacharparenright}{\kern0pt}\ {\isacharequal}{\kern0pt}\ {\isacharminus}{\kern0pt}\ sgn\ x{\isachardoublequoteclose}%
\isadelimproof
\ %
\endisadelimproof
%
\isatagproof
\isacommand{by}\isamarkupfalse%
\ {\isacharparenleft}{\kern0pt}metis\ {\isacharparenleft}{\kern0pt}mono{\isacharunderscore}{\kern0pt}tags{\isacharcomma}{\kern0pt}\ lifting{\isacharparenright}{\kern0pt}\ mult{\isacharunderscore}{\kern0pt}minus{\isadigit{1}}\ sgn{\isacharunderscore}{\kern0pt}neg{\isacharunderscore}{\kern0pt}one\ sgn{\isacharunderscore}{\kern0pt}times{\isacharparenright}{\kern0pt}%
\endisatagproof
{\isafoldproof}%
%
\isadelimproof
%
\endisadelimproof
\isanewline
\isanewline
\isacommand{lemma}\isamarkupfalse%
\ sgn{\isacharunderscore}{\kern0pt}plus{\isacharprime}{\kern0pt}{\isacharcolon}{\kern0pt}\isanewline
\ \ \isakeyword{assumes}\ {\isachardoublequoteopen}sgn\ x\ {\isacharequal}{\kern0pt}\ {\isacharparenleft}{\kern0pt}{\isacharminus}{\kern0pt}{\isadigit{1}}\ {\isacharcolon}{\kern0pt}{\isacharcolon}{\kern0pt}\ real{\isacharparenright}{\kern0pt}{\isachardoublequoteclose}\ {\isachardoublequoteopen}sgn\ y\ {\isacharequal}{\kern0pt}\ {\isacharminus}{\kern0pt}{\isadigit{1}}{\isachardoublequoteclose}\isanewline
\ \ \isakeyword{shows}\ {\isachardoublequoteopen}sgn\ {\isacharparenleft}{\kern0pt}x\ {\isacharplus}{\kern0pt}\ y{\isacharparenright}{\kern0pt}\ {\isacharequal}{\kern0pt}\ {\isacharminus}{\kern0pt}{\isadigit{1}}{\isachardoublequoteclose}\isanewline
%
\isadelimproof
\ \ %
\endisadelimproof
%
\isatagproof
\isacommand{using}\isamarkupfalse%
\ assms\ sgn{\isacharunderscore}{\kern0pt}uminus{\isacharbrackleft}{\kern0pt}of\ x{\isacharbrackright}{\kern0pt}\ sgn{\isacharunderscore}{\kern0pt}uminus{\isacharbrackleft}{\kern0pt}of\ y{\isacharbrackright}{\kern0pt}\ sgn{\isacharunderscore}{\kern0pt}uminus{\isacharbrackleft}{\kern0pt}of\ {\isachardoublequoteopen}x\ {\isacharplus}{\kern0pt}\ y{\isachardoublequoteclose}{\isacharbrackright}{\kern0pt}\ sgn{\isacharunderscore}{\kern0pt}plus{\isacharbrackleft}{\kern0pt}of\ {\isachardoublequoteopen}{\isacharminus}{\kern0pt}\ x{\isachardoublequoteclose}\ {\isachardoublequoteopen}{\isacharminus}{\kern0pt}\ y{\isachardoublequoteclose}{\isacharbrackright}{\kern0pt}\isanewline
\ \ \isacommand{by}\isamarkupfalse%
\ {\isacharparenleft}{\kern0pt}simp\ add{\isacharcolon}{\kern0pt}\ equation{\isacharunderscore}{\kern0pt}minus{\isacharunderscore}{\kern0pt}iff{\isacharparenright}{\kern0pt}%
\endisatagproof
{\isafoldproof}%
%
\isadelimproof
\isanewline
%
\endisadelimproof
\isanewline
\isacommand{lemma}\isamarkupfalse%
\ pos{\isacharunderscore}{\kern0pt}dual{\isacharunderscore}{\kern0pt}def{\isacharcolon}{\kern0pt}\ \isanewline
\ \ \isakeyword{assumes}\ {\isachardoublequoteopen}slope\ f{\isachardoublequoteclose}\isanewline
\ \ \isakeyword{shows}\ {\isachardoublequoteopen}pos\ f\ {\isacharequal}{\kern0pt}\ {\isacharparenleft}{\kern0pt}{\isasymforall}C\ {\isasymge}\ {\isadigit{0}}{\isachardot}{\kern0pt}\ {\isasymexists}N{\isachardot}{\kern0pt}\ {\isasymforall}n\ {\isasymle}\ N{\isachardot}{\kern0pt}\ f\ n\ {\isasymle}\ {\isacharminus}{\kern0pt}C{\isacharparenright}{\kern0pt}{\isachardoublequoteclose}\isanewline
%
\isadelimproof
%
\endisadelimproof
%
\isatagproof
\isacommand{proof}\isamarkupfalse%
{\isacharminus}{\kern0pt}\isanewline
\ \ \isacommand{have}\isamarkupfalse%
\ {\isachardoublequoteopen}pos\ f\ {\isacharequal}{\kern0pt}\ neg\ {\isacharparenleft}{\kern0pt}f\ {\isacharasterisk}{\kern0pt}\isactrlsub e\ {\isacharparenleft}{\kern0pt}{\isacharminus}{\kern0pt}\isactrlsub e\ id{\isacharparenright}{\kern0pt}{\isacharparenright}{\kern0pt}{\isachardoublequoteclose}\ \isacommand{by}\isamarkupfalse%
\ {\isacharparenleft}{\kern0pt}metis\ abs{\isacharunderscore}{\kern0pt}real{\isacharunderscore}{\kern0pt}eq{\isacharunderscore}{\kern0pt}iff\ abs{\isacharunderscore}{\kern0pt}real{\isacharunderscore}{\kern0pt}times\ add{\isachardot}{\kern0pt}inverse{\isacharunderscore}{\kern0pt}inverse\ assms\ eudoxus{\isacharunderscore}{\kern0pt}times{\isacharunderscore}{\kern0pt}commute\ mult{\isacharunderscore}{\kern0pt}minus{\isadigit{1}}{\isacharunderscore}{\kern0pt}right\ neg{\isacharunderscore}{\kern0pt}one{\isacharunderscore}{\kern0pt}def\ sgn{\isacharunderscore}{\kern0pt}abs{\isacharunderscore}{\kern0pt}real{\isacharunderscore}{\kern0pt}neg{\isacharunderscore}{\kern0pt}one{\isacharunderscore}{\kern0pt}iff\ sgn{\isacharunderscore}{\kern0pt}abs{\isacharunderscore}{\kern0pt}real{\isacharunderscore}{\kern0pt}one{\isacharunderscore}{\kern0pt}iff\ sgn{\isacharunderscore}{\kern0pt}uminus\ slope{\isacharunderscore}{\kern0pt}neg{\isacharunderscore}{\kern0pt}one{\isacharparenright}{\kern0pt}\isanewline
\ \ \isacommand{also}\isamarkupfalse%
\ \isacommand{have}\isamarkupfalse%
\ {\isachardoublequoteopen}{\isachardot}{\kern0pt}{\isachardot}{\kern0pt}{\isachardot}{\kern0pt}\ {\isacharequal}{\kern0pt}\ {\isacharparenleft}{\kern0pt}{\isasymforall}C\ {\isasymge}\ {\isadigit{0}}{\isachardot}{\kern0pt}\ {\isasymexists}N{\isachardot}{\kern0pt}\ {\isasymforall}n\ {\isasymge}\ N{\isachardot}{\kern0pt}\ {\isacharparenleft}{\kern0pt}f\ {\isacharparenleft}{\kern0pt}{\isacharminus}{\kern0pt}\ n{\isacharparenright}{\kern0pt}{\isacharparenright}{\kern0pt}\ {\isasymle}\ {\isacharminus}{\kern0pt}C{\isacharparenright}{\kern0pt}{\isachardoublequoteclose}\ \isacommand{unfolding}\isamarkupfalse%
\ neg{\isacharunderscore}{\kern0pt}def\ eudoxus{\isacharunderscore}{\kern0pt}times{\isacharunderscore}{\kern0pt}def\ eudoxus{\isacharunderscore}{\kern0pt}uminus{\isacharunderscore}{\kern0pt}def\ \isacommand{by}\isamarkupfalse%
\ simp\isanewline
\ \ \isacommand{also}\isamarkupfalse%
\ \isacommand{have}\isamarkupfalse%
\ {\isachardoublequoteopen}{\isachardot}{\kern0pt}{\isachardot}{\kern0pt}{\isachardot}{\kern0pt}\ {\isacharequal}{\kern0pt}\ {\isacharparenleft}{\kern0pt}{\isasymforall}C\ {\isasymge}\ {\isadigit{0}}{\isachardot}{\kern0pt}\ {\isasymexists}N{\isachardot}{\kern0pt}\ {\isasymforall}n\ {\isasymle}\ N{\isachardot}{\kern0pt}\ f\ n\ {\isasymle}\ {\isacharminus}{\kern0pt}C{\isacharparenright}{\kern0pt}{\isachardoublequoteclose}\ \isacommand{by}\isamarkupfalse%
\ {\isacharparenleft}{\kern0pt}metis\ add{\isachardot}{\kern0pt}inverse{\isacharunderscore}{\kern0pt}inverse\ minus{\isacharunderscore}{\kern0pt}le{\isacharunderscore}{\kern0pt}iff{\isacharparenright}{\kern0pt}\isanewline
\ \ \isacommand{finally}\isamarkupfalse%
\ \isacommand{show}\isamarkupfalse%
\ {\isacharquery}{\kern0pt}thesis\ \isacommand{{\isachardot}{\kern0pt}}\isamarkupfalse%
\isanewline
\isacommand{qed}\isamarkupfalse%
%
\endisatagproof
{\isafoldproof}%
%
\isadelimproof
\isanewline
%
\endisadelimproof
\isanewline
\isacommand{lemma}\isamarkupfalse%
\ neg{\isacharunderscore}{\kern0pt}dual{\isacharunderscore}{\kern0pt}def{\isacharcolon}{\kern0pt}\isanewline
\ \ \isakeyword{assumes}\ {\isachardoublequoteopen}slope\ f{\isachardoublequoteclose}\isanewline
\ \ \isakeyword{shows}\ {\isachardoublequoteopen}neg\ f\ {\isacharequal}{\kern0pt}\ {\isacharparenleft}{\kern0pt}{\isasymforall}C\ {\isasymge}\ {\isadigit{0}}{\isachardot}{\kern0pt}\ {\isasymexists}N{\isachardot}{\kern0pt}\ {\isasymforall}n\ {\isasymle}\ N{\isachardot}{\kern0pt}\ f\ n\ {\isasymge}\ C{\isacharparenright}{\kern0pt}{\isachardoublequoteclose}\isanewline
%
\isadelimproof
\ \ %
\endisadelimproof
%
\isatagproof
\isacommand{unfolding}\isamarkupfalse%
\ neg{\isacharunderscore}{\kern0pt}iff{\isacharunderscore}{\kern0pt}pos{\isacharunderscore}{\kern0pt}uminus\ \isacommand{using}\isamarkupfalse%
\ assms\ \isacommand{by}\isamarkupfalse%
\ {\isacharparenleft}{\kern0pt}subst\ pos{\isacharunderscore}{\kern0pt}dual{\isacharunderscore}{\kern0pt}def{\isacharparenright}{\kern0pt}\ {\isacharparenleft}{\kern0pt}auto\ simp\ add{\isacharcolon}{\kern0pt}\ eudoxus{\isacharunderscore}{\kern0pt}uminus{\isacharunderscore}{\kern0pt}def{\isacharparenright}{\kern0pt}%
\endisatagproof
{\isafoldproof}%
%
\isadelimproof
\isanewline
%
\endisadelimproof
\isanewline
\isacommand{lemma}\isamarkupfalse%
\ pos{\isacharunderscore}{\kern0pt}representative{\isacharcolon}{\kern0pt}\isanewline
\ \ \isakeyword{assumes}\ {\isachardoublequoteopen}slope\ f{\isachardoublequoteclose}\ {\isachardoublequoteopen}pos\ f{\isachardoublequoteclose}\isanewline
\ \ \isakeyword{obtains}\ g\ \isakeyword{where}\ {\isachardoublequoteopen}f\ {\isasymsim}\isactrlsub e\ g{\isachardoublequoteclose}\ {\isachardoublequoteopen}{\isasymAnd}n{\isachardot}{\kern0pt}\ n\ {\isasymge}\ N\ {\isasymLongrightarrow}\ g\ n\ {\isasymge}\ C{\isachardoublequoteclose}\isanewline
%
\isadelimproof
%
\endisadelimproof
%
\isatagproof
\isacommand{proof}\isamarkupfalse%
\ {\isacharminus}{\kern0pt}\isanewline
\ \ \isacommand{obtain}\isamarkupfalse%
\ N{\isacharprime}{\kern0pt}\ \isakeyword{where}\ N{\isacharprime}{\kern0pt}{\isacharcolon}{\kern0pt}\ {\isachardoublequoteopen}{\isasymforall}z{\isasymge}N{\isacharprime}{\kern0pt}{\isachardot}{\kern0pt}\ f\ z\ {\isasymge}\ max\ {\isadigit{0}}\ C{\isachardoublequoteclose}\ \isacommand{using}\isamarkupfalse%
\ assms\ \isacommand{unfolding}\isamarkupfalse%
\ pos{\isacharunderscore}{\kern0pt}def\ \isacommand{by}\isamarkupfalse%
\ {\isacharparenleft}{\kern0pt}meson\ max{\isachardot}{\kern0pt}cobounded{\isadigit{1}}{\isacharparenright}{\kern0pt}\isanewline
\ \ \isacommand{have}\isamarkupfalse%
\ {\isacharasterisk}{\kern0pt}{\isacharcolon}{\kern0pt}\ {\isachardoublequoteopen}{\isadigit{1}}\ {\isacharequal}{\kern0pt}\ abs{\isacharunderscore}{\kern0pt}real\ {\isacharparenleft}{\kern0pt}{\isasymlambda}x{\isachardot}{\kern0pt}\ x\ {\isacharplus}{\kern0pt}\ N{\isacharprime}{\kern0pt}\ {\isacharminus}{\kern0pt}\ N{\isacharparenright}{\kern0pt}{\isachardoublequoteclose}\ {\isachardoublequoteopen}slope\ {\isacharparenleft}{\kern0pt}{\isasymlambda}x{\isachardot}{\kern0pt}\ x\ {\isacharplus}{\kern0pt}\ N{\isacharprime}{\kern0pt}\ {\isacharminus}{\kern0pt}\ N{\isacharparenright}{\kern0pt}{\isachardoublequoteclose}\ \isacommand{unfolding}\isamarkupfalse%
\ one{\isacharunderscore}{\kern0pt}def\ \isacommand{by}\isamarkupfalse%
\ {\isacharparenleft}{\kern0pt}intro\ abs{\isacharunderscore}{\kern0pt}real{\isacharunderscore}{\kern0pt}eqI{\isacharparenright}{\kern0pt}\ {\isacharparenleft}{\kern0pt}auto\ simp\ add{\isacharcolon}{\kern0pt}\ eudoxus{\isacharunderscore}{\kern0pt}rel{\isacharunderscore}{\kern0pt}def\ slope{\isacharunderscore}{\kern0pt}def\ intro{\isacharbang}{\kern0pt}{\isacharcolon}{\kern0pt}\ boundedI{\isacharparenright}{\kern0pt}\isanewline
\ \ \isacommand{hence}\isamarkupfalse%
\ {\isachardoublequoteopen}abs{\isacharunderscore}{\kern0pt}real\ f\ {\isacharasterisk}{\kern0pt}\ {\isadigit{1}}\ {\isacharequal}{\kern0pt}\ abs{\isacharunderscore}{\kern0pt}real\ {\isacharparenleft}{\kern0pt}f\ {\isacharasterisk}{\kern0pt}\isactrlsub e\ {\isacharparenleft}{\kern0pt}{\isasymlambda}x{\isachardot}{\kern0pt}\ x\ {\isacharplus}{\kern0pt}\ N{\isacharprime}{\kern0pt}\ {\isacharminus}{\kern0pt}\ N{\isacharparenright}{\kern0pt}{\isacharparenright}{\kern0pt}{\isachardoublequoteclose}\ \isacommand{using}\isamarkupfalse%
\ abs{\isacharunderscore}{\kern0pt}real{\isacharunderscore}{\kern0pt}times{\isacharbrackleft}{\kern0pt}OF\ assms{\isacharparenleft}{\kern0pt}{\isadigit{1}}{\isacharparenright}{\kern0pt}\ {\isacharasterisk}{\kern0pt}{\isacharparenleft}{\kern0pt}{\isadigit{2}}{\isacharparenright}{\kern0pt}{\isacharbrackright}{\kern0pt}\ \isacommand{by}\isamarkupfalse%
\ simp\isanewline
\ \ \isacommand{hence}\isamarkupfalse%
\ {\isachardoublequoteopen}f\ {\isasymsim}\isactrlsub e\ {\isacharparenleft}{\kern0pt}f\ {\isacharasterisk}{\kern0pt}\isactrlsub e\ {\isacharparenleft}{\kern0pt}{\isasymlambda}x{\isachardot}{\kern0pt}\ x\ {\isacharplus}{\kern0pt}\ N{\isacharprime}{\kern0pt}\ {\isacharminus}{\kern0pt}\ N{\isacharparenright}{\kern0pt}{\isacharparenright}{\kern0pt}{\isachardoublequoteclose}\ \isacommand{using}\isamarkupfalse%
\ assms\ {\isacharasterisk}{\kern0pt}\ \isacommand{by}\isamarkupfalse%
\ {\isacharparenleft}{\kern0pt}metis\ abs{\isacharunderscore}{\kern0pt}real{\isacharunderscore}{\kern0pt}eq{\isacharunderscore}{\kern0pt}iff\ eudoxus{\isacharunderscore}{\kern0pt}times{\isacharunderscore}{\kern0pt}commute\ mult{\isachardot}{\kern0pt}right{\isacharunderscore}{\kern0pt}neutral{\isacharparenright}{\kern0pt}\isanewline
\ \ \isacommand{moreover}\isamarkupfalse%
\ \isacommand{have}\isamarkupfalse%
\ {\isachardoublequoteopen}{\isasymforall}z{\isasymge}N{\isachardot}{\kern0pt}\ {\isacharparenleft}{\kern0pt}f\ {\isacharasterisk}{\kern0pt}\isactrlsub e\ {\isacharparenleft}{\kern0pt}{\isasymlambda}x{\isachardot}{\kern0pt}\ x\ {\isacharplus}{\kern0pt}\ N{\isacharprime}{\kern0pt}\ {\isacharminus}{\kern0pt}\ N{\isacharparenright}{\kern0pt}{\isacharparenright}{\kern0pt}\ z\ {\isasymge}\ C{\isachardoublequoteclose}\ \isacommand{unfolding}\isamarkupfalse%
\ eudoxus{\isacharunderscore}{\kern0pt}times{\isacharunderscore}{\kern0pt}def\ \isacommand{using}\isamarkupfalse%
\ N{\isacharprime}{\kern0pt}\ \isacommand{by}\isamarkupfalse%
\ simp\isanewline
\ \ \isacommand{ultimately}\isamarkupfalse%
\ \isacommand{show}\isamarkupfalse%
\ {\isacharquery}{\kern0pt}thesis\ \isacommand{using}\isamarkupfalse%
\ that\ \isacommand{by}\isamarkupfalse%
\ blast\isanewline
\isacommand{qed}\isamarkupfalse%
%
\endisatagproof
{\isafoldproof}%
%
\isadelimproof
\isanewline
%
\endisadelimproof
\isanewline
\isacommand{lemma}\isamarkupfalse%
\ pos{\isacharunderscore}{\kern0pt}representative{\isacharprime}{\kern0pt}{\isacharcolon}{\kern0pt}\isanewline
\ \ \isakeyword{assumes}\ {\isachardoublequoteopen}slope\ f{\isachardoublequoteclose}\ {\isachardoublequoteopen}pos\ f{\isachardoublequoteclose}\isanewline
\ \ \isakeyword{obtains}\ g\ \isakeyword{where}\ {\isachardoublequoteopen}f\ {\isasymsim}\isactrlsub e\ g{\isachardoublequoteclose}\ {\isachardoublequoteopen}{\isasymAnd}n{\isachardot}{\kern0pt}\ g\ n\ {\isasymge}\ C\ {\isasymLongrightarrow}\ n\ {\isasymge}\ N{\isachardoublequoteclose}\isanewline
%
\isadelimproof
%
\endisadelimproof
%
\isatagproof
\isacommand{proof}\isamarkupfalse%
\ {\isacharminus}{\kern0pt}\isanewline
\ \ \isacommand{obtain}\isamarkupfalse%
\ N{\isacharprime}{\kern0pt}\ \isakeyword{where}\ \ {\isachardoublequoteopen}{\isasymforall}z\ {\isasymle}\ N{\isacharprime}{\kern0pt}{\isachardot}{\kern0pt}\ f\ z\ {\isasymle}\ {\isacharminus}{\kern0pt}\ {\isacharparenleft}{\kern0pt}max\ {\isadigit{0}}\ {\isacharparenleft}{\kern0pt}{\isacharminus}{\kern0pt}\ C{\isacharparenright}{\kern0pt}\ {\isacharplus}{\kern0pt}\ {\isadigit{1}}{\isacharparenright}{\kern0pt}{\isachardoublequoteclose}\ \isacommand{using}\isamarkupfalse%
\ assms\ \isacommand{unfolding}\isamarkupfalse%
\ pos{\isacharunderscore}{\kern0pt}dual{\isacharunderscore}{\kern0pt}def{\isacharbrackleft}{\kern0pt}OF\ assms{\isacharparenleft}{\kern0pt}{\isadigit{1}}{\isacharparenright}{\kern0pt}{\isacharbrackright}{\kern0pt}\ \isacommand{by}\isamarkupfalse%
\ {\isacharparenleft}{\kern0pt}metis\ max{\isachardot}{\kern0pt}cobounded{\isadigit{1}}\ add{\isacharunderscore}{\kern0pt}increasing{\isadigit{2}}\ zero{\isacharunderscore}{\kern0pt}less{\isacharunderscore}{\kern0pt}one{\isacharunderscore}{\kern0pt}class{\isachardot}{\kern0pt}zero{\isacharunderscore}{\kern0pt}le{\isacharunderscore}{\kern0pt}one{\isacharparenright}{\kern0pt}\isanewline
\ \ \isacommand{hence}\isamarkupfalse%
\ N{\isacharprime}{\kern0pt}{\isacharcolon}{\kern0pt}\ {\isachardoublequoteopen}{\isasymforall}z\ {\isasymle}\ N{\isacharprime}{\kern0pt}{\isachardot}{\kern0pt}\ f\ z\ {\isacharless}{\kern0pt}\ min\ {\isadigit{0}}\ C{\isachardoublequoteclose}\ \isacommand{by}\isamarkupfalse%
\ fastforce\isanewline
\ \ \isacommand{have}\isamarkupfalse%
\ {\isacharasterisk}{\kern0pt}{\isacharcolon}{\kern0pt}\ {\isachardoublequoteopen}{\isadigit{1}}\ {\isacharequal}{\kern0pt}\ abs{\isacharunderscore}{\kern0pt}real\ {\isacharparenleft}{\kern0pt}{\isasymlambda}x{\isachardot}{\kern0pt}\ x\ {\isacharplus}{\kern0pt}\ N{\isacharprime}{\kern0pt}\ {\isacharminus}{\kern0pt}\ N{\isacharparenright}{\kern0pt}{\isachardoublequoteclose}\ {\isachardoublequoteopen}slope\ {\isacharparenleft}{\kern0pt}{\isasymlambda}x{\isachardot}{\kern0pt}\ x\ {\isacharplus}{\kern0pt}\ N{\isacharprime}{\kern0pt}\ {\isacharminus}{\kern0pt}\ N{\isacharparenright}{\kern0pt}{\isachardoublequoteclose}\ \isacommand{unfolding}\isamarkupfalse%
\ one{\isacharunderscore}{\kern0pt}def\ \isacommand{by}\isamarkupfalse%
\ {\isacharparenleft}{\kern0pt}intro\ abs{\isacharunderscore}{\kern0pt}real{\isacharunderscore}{\kern0pt}eqI{\isacharparenright}{\kern0pt}\ {\isacharparenleft}{\kern0pt}auto\ simp\ add{\isacharcolon}{\kern0pt}\ eudoxus{\isacharunderscore}{\kern0pt}rel{\isacharunderscore}{\kern0pt}def\ slope{\isacharunderscore}{\kern0pt}def\ intro{\isacharbang}{\kern0pt}{\isacharcolon}{\kern0pt}\ boundedI{\isacharparenright}{\kern0pt}\isanewline
\ \ \isacommand{hence}\isamarkupfalse%
\ {\isachardoublequoteopen}abs{\isacharunderscore}{\kern0pt}real\ f\ {\isacharasterisk}{\kern0pt}\ {\isadigit{1}}\ {\isacharequal}{\kern0pt}\ abs{\isacharunderscore}{\kern0pt}real\ {\isacharparenleft}{\kern0pt}f\ {\isacharasterisk}{\kern0pt}\isactrlsub e\ {\isacharparenleft}{\kern0pt}{\isasymlambda}x{\isachardot}{\kern0pt}\ x\ {\isacharplus}{\kern0pt}\ N{\isacharprime}{\kern0pt}\ {\isacharminus}{\kern0pt}\ N{\isacharparenright}{\kern0pt}{\isacharparenright}{\kern0pt}{\isachardoublequoteclose}\ \isacommand{using}\isamarkupfalse%
\ abs{\isacharunderscore}{\kern0pt}real{\isacharunderscore}{\kern0pt}times{\isacharbrackleft}{\kern0pt}OF\ assms{\isacharparenleft}{\kern0pt}{\isadigit{1}}{\isacharparenright}{\kern0pt}\ {\isacharasterisk}{\kern0pt}{\isacharparenleft}{\kern0pt}{\isadigit{2}}{\isacharparenright}{\kern0pt}{\isacharbrackright}{\kern0pt}\ \isacommand{by}\isamarkupfalse%
\ simp\isanewline
\ \ \isacommand{hence}\isamarkupfalse%
\ {\isachardoublequoteopen}f\ {\isasymsim}\isactrlsub e\ {\isacharparenleft}{\kern0pt}f\ {\isacharasterisk}{\kern0pt}\isactrlsub e\ {\isacharparenleft}{\kern0pt}{\isasymlambda}x{\isachardot}{\kern0pt}\ x\ {\isacharplus}{\kern0pt}\ N{\isacharprime}{\kern0pt}\ {\isacharminus}{\kern0pt}\ N{\isacharparenright}{\kern0pt}{\isacharparenright}{\kern0pt}{\isachardoublequoteclose}\ \isacommand{using}\isamarkupfalse%
\ assms\ {\isacharasterisk}{\kern0pt}\ \isacommand{by}\isamarkupfalse%
\ {\isacharparenleft}{\kern0pt}metis\ abs{\isacharunderscore}{\kern0pt}real{\isacharunderscore}{\kern0pt}eq{\isacharunderscore}{\kern0pt}iff\ eudoxus{\isacharunderscore}{\kern0pt}times{\isacharunderscore}{\kern0pt}commute\ mult{\isachardot}{\kern0pt}right{\isacharunderscore}{\kern0pt}neutral{\isacharparenright}{\kern0pt}\isanewline
\ \ \isacommand{moreover}\isamarkupfalse%
\ \isacommand{have}\isamarkupfalse%
\ {\isachardoublequoteopen}{\isasymforall}z{\isacharless}{\kern0pt}N{\isachardot}{\kern0pt}\ {\isacharparenleft}{\kern0pt}f\ {\isacharasterisk}{\kern0pt}\isactrlsub e\ {\isacharparenleft}{\kern0pt}{\isasymlambda}x{\isachardot}{\kern0pt}\ x\ {\isacharplus}{\kern0pt}\ N{\isacharprime}{\kern0pt}\ {\isacharminus}{\kern0pt}\ N{\isacharparenright}{\kern0pt}{\isacharparenright}{\kern0pt}\ z\ {\isacharless}{\kern0pt}\ C{\isachardoublequoteclose}\ \isacommand{unfolding}\isamarkupfalse%
\ eudoxus{\isacharunderscore}{\kern0pt}times{\isacharunderscore}{\kern0pt}def\ \isacommand{using}\isamarkupfalse%
\ N{\isacharprime}{\kern0pt}\ \isacommand{by}\isamarkupfalse%
\ simp\isanewline
\ \ \isacommand{ultimately}\isamarkupfalse%
\ \isacommand{show}\isamarkupfalse%
\ {\isacharquery}{\kern0pt}thesis\ \isacommand{using}\isamarkupfalse%
\ that\ \isacommand{by}\isamarkupfalse%
\ {\isacharparenleft}{\kern0pt}meson\ linorder{\isacharunderscore}{\kern0pt}not{\isacharunderscore}{\kern0pt}less{\isacharparenright}{\kern0pt}\isanewline
\isacommand{qed}\isamarkupfalse%
%
\endisatagproof
{\isafoldproof}%
%
\isadelimproof
\isanewline
%
\endisadelimproof
\isanewline
\isacommand{lemma}\isamarkupfalse%
\ neg{\isacharunderscore}{\kern0pt}representative{\isacharcolon}{\kern0pt}\isanewline
\ \ \isakeyword{assumes}\ {\isachardoublequoteopen}slope\ f{\isachardoublequoteclose}\ {\isachardoublequoteopen}neg\ f{\isachardoublequoteclose}\isanewline
\ \ \isakeyword{obtains}\ g\ \isakeyword{where}\ {\isachardoublequoteopen}f\ {\isasymsim}\isactrlsub e\ g{\isachardoublequoteclose}\ {\isachardoublequoteopen}{\isasymAnd}n{\isachardot}{\kern0pt}\ n\ {\isasymge}\ N\ {\isasymLongrightarrow}\ g\ n\ {\isasymle}\ {\isacharminus}{\kern0pt}\ C{\isachardoublequoteclose}\isanewline
%
\isadelimproof
%
\endisadelimproof
%
\isatagproof
\isacommand{proof}\isamarkupfalse%
\ {\isacharminus}{\kern0pt}\isanewline
\ \ \isacommand{obtain}\isamarkupfalse%
\ N{\isacharprime}{\kern0pt}\ \isakeyword{where}\ {\isachardoublequoteopen}{\isasymforall}z{\isasymge}N{\isacharprime}{\kern0pt}{\isachardot}{\kern0pt}\ f\ z\ {\isasymle}\ {\isacharminus}{\kern0pt}\ max\ {\isadigit{0}}\ C{\isachardoublequoteclose}\ \isacommand{using}\isamarkupfalse%
\ assms\ \isacommand{unfolding}\isamarkupfalse%
\ neg{\isacharunderscore}{\kern0pt}def\ \isacommand{by}\isamarkupfalse%
\ {\isacharparenleft}{\kern0pt}meson\ max{\isachardot}{\kern0pt}cobounded{\isadigit{1}}{\isacharparenright}{\kern0pt}\isanewline
\ \ \isacommand{hence}\isamarkupfalse%
\ N{\isacharprime}{\kern0pt}{\isacharcolon}{\kern0pt}\ {\isachardoublequoteopen}{\isasymforall}z{\isasymge}N{\isacharprime}{\kern0pt}{\isachardot}{\kern0pt}\ f\ z\ {\isasymle}\ min\ {\isadigit{0}}\ {\isacharparenleft}{\kern0pt}{\isacharminus}{\kern0pt}\ C{\isacharparenright}{\kern0pt}{\isachardoublequoteclose}\ \isacommand{by}\isamarkupfalse%
\ force\isanewline
\ \ \isacommand{have}\isamarkupfalse%
\ {\isacharasterisk}{\kern0pt}{\isacharcolon}{\kern0pt}\ {\isachardoublequoteopen}{\isadigit{1}}\ {\isacharequal}{\kern0pt}\ abs{\isacharunderscore}{\kern0pt}real\ {\isacharparenleft}{\kern0pt}{\isasymlambda}x{\isachardot}{\kern0pt}\ x\ {\isacharplus}{\kern0pt}\ N{\isacharprime}{\kern0pt}\ {\isacharminus}{\kern0pt}\ N{\isacharparenright}{\kern0pt}{\isachardoublequoteclose}\ {\isachardoublequoteopen}slope\ {\isacharparenleft}{\kern0pt}{\isasymlambda}x{\isachardot}{\kern0pt}\ x\ {\isacharplus}{\kern0pt}\ N{\isacharprime}{\kern0pt}\ {\isacharminus}{\kern0pt}\ N{\isacharparenright}{\kern0pt}{\isachardoublequoteclose}\ \isacommand{unfolding}\isamarkupfalse%
\ one{\isacharunderscore}{\kern0pt}def\ \isacommand{by}\isamarkupfalse%
\ {\isacharparenleft}{\kern0pt}intro\ abs{\isacharunderscore}{\kern0pt}real{\isacharunderscore}{\kern0pt}eqI{\isacharparenright}{\kern0pt}\ {\isacharparenleft}{\kern0pt}auto\ simp\ add{\isacharcolon}{\kern0pt}\ eudoxus{\isacharunderscore}{\kern0pt}rel{\isacharunderscore}{\kern0pt}def\ slope{\isacharunderscore}{\kern0pt}def\ intro{\isacharbang}{\kern0pt}{\isacharcolon}{\kern0pt}\ boundedI{\isacharparenright}{\kern0pt}\isanewline
\ \ \isacommand{hence}\isamarkupfalse%
\ {\isachardoublequoteopen}abs{\isacharunderscore}{\kern0pt}real\ f\ {\isacharasterisk}{\kern0pt}\ {\isadigit{1}}\ {\isacharequal}{\kern0pt}\ abs{\isacharunderscore}{\kern0pt}real\ {\isacharparenleft}{\kern0pt}f\ {\isacharasterisk}{\kern0pt}\isactrlsub e\ {\isacharparenleft}{\kern0pt}{\isasymlambda}x{\isachardot}{\kern0pt}\ x\ {\isacharplus}{\kern0pt}\ N{\isacharprime}{\kern0pt}\ {\isacharminus}{\kern0pt}\ N{\isacharparenright}{\kern0pt}{\isacharparenright}{\kern0pt}{\isachardoublequoteclose}\ \isacommand{using}\isamarkupfalse%
\ abs{\isacharunderscore}{\kern0pt}real{\isacharunderscore}{\kern0pt}times{\isacharbrackleft}{\kern0pt}OF\ assms{\isacharparenleft}{\kern0pt}{\isadigit{1}}{\isacharparenright}{\kern0pt}\ {\isacharasterisk}{\kern0pt}{\isacharparenleft}{\kern0pt}{\isadigit{2}}{\isacharparenright}{\kern0pt}{\isacharbrackright}{\kern0pt}\ \isacommand{by}\isamarkupfalse%
\ simp\isanewline
\ \ \isacommand{hence}\isamarkupfalse%
\ {\isachardoublequoteopen}f\ {\isasymsim}\isactrlsub e\ {\isacharparenleft}{\kern0pt}f\ {\isacharasterisk}{\kern0pt}\isactrlsub e\ {\isacharparenleft}{\kern0pt}{\isasymlambda}x{\isachardot}{\kern0pt}\ x\ {\isacharplus}{\kern0pt}\ N{\isacharprime}{\kern0pt}\ {\isacharminus}{\kern0pt}\ N{\isacharparenright}{\kern0pt}{\isacharparenright}{\kern0pt}{\isachardoublequoteclose}\ \isacommand{using}\isamarkupfalse%
\ assms\ {\isacharasterisk}{\kern0pt}\ \isacommand{by}\isamarkupfalse%
\ {\isacharparenleft}{\kern0pt}metis\ abs{\isacharunderscore}{\kern0pt}real{\isacharunderscore}{\kern0pt}eq{\isacharunderscore}{\kern0pt}iff\ eudoxus{\isacharunderscore}{\kern0pt}times{\isacharunderscore}{\kern0pt}commute\ mult{\isachardot}{\kern0pt}right{\isacharunderscore}{\kern0pt}neutral{\isacharparenright}{\kern0pt}\isanewline
\ \ \isacommand{moreover}\isamarkupfalse%
\ \isacommand{have}\isamarkupfalse%
\ {\isachardoublequoteopen}{\isasymforall}z{\isasymge}N{\isachardot}{\kern0pt}\ {\isacharparenleft}{\kern0pt}f\ {\isacharasterisk}{\kern0pt}\isactrlsub e\ {\isacharparenleft}{\kern0pt}{\isasymlambda}x{\isachardot}{\kern0pt}\ x\ {\isacharplus}{\kern0pt}\ N{\isacharprime}{\kern0pt}\ {\isacharminus}{\kern0pt}\ N{\isacharparenright}{\kern0pt}{\isacharparenright}{\kern0pt}\ z\ {\isasymle}\ {\isacharminus}{\kern0pt}\ C{\isachardoublequoteclose}\ \isacommand{unfolding}\isamarkupfalse%
\ eudoxus{\isacharunderscore}{\kern0pt}times{\isacharunderscore}{\kern0pt}def\ \isacommand{using}\isamarkupfalse%
\ N{\isacharprime}{\kern0pt}\ \isacommand{by}\isamarkupfalse%
\ simp\isanewline
\ \ \isacommand{ultimately}\isamarkupfalse%
\ \isacommand{show}\isamarkupfalse%
\ {\isacharquery}{\kern0pt}thesis\ \isacommand{using}\isamarkupfalse%
\ that\ \isacommand{by}\isamarkupfalse%
\ blast\isanewline
\isacommand{qed}\isamarkupfalse%
%
\endisatagproof
{\isafoldproof}%
%
\isadelimproof
\isanewline
%
\endisadelimproof
\isanewline
\isacommand{lemma}\isamarkupfalse%
\ neg{\isacharunderscore}{\kern0pt}representative{\isacharprime}{\kern0pt}{\isacharcolon}{\kern0pt}\isanewline
\ \ \isakeyword{assumes}\ {\isachardoublequoteopen}slope\ f{\isachardoublequoteclose}\ {\isachardoublequoteopen}neg\ f{\isachardoublequoteclose}\isanewline
\ \ \isakeyword{obtains}\ g\ \isakeyword{where}\ {\isachardoublequoteopen}f\ {\isasymsim}\isactrlsub e\ g{\isachardoublequoteclose}\ {\isachardoublequoteopen}{\isasymAnd}n{\isachardot}{\kern0pt}\ g\ n\ {\isasymle}\ {\isacharminus}{\kern0pt}\ C\ {\isasymLongrightarrow}\ n\ {\isasymge}\ N{\isachardoublequoteclose}\isanewline
%
\isadelimproof
%
\endisadelimproof
%
\isatagproof
\isacommand{proof}\isamarkupfalse%
\ {\isacharminus}{\kern0pt}\isanewline
\ \ \isacommand{obtain}\isamarkupfalse%
\ N{\isacharprime}{\kern0pt}\ \isakeyword{where}\ {\isachardoublequoteopen}{\isasymforall}z\ {\isasymle}\ N{\isacharprime}{\kern0pt}{\isachardot}{\kern0pt}\ f\ z\ {\isasymge}\ max\ {\isadigit{0}}\ {\isacharparenleft}{\kern0pt}{\isacharminus}{\kern0pt}\ C{\isacharparenright}{\kern0pt}\ {\isacharplus}{\kern0pt}\ {\isadigit{1}}{\isachardoublequoteclose}\ \isacommand{using}\isamarkupfalse%
\ assms\ \isacommand{unfolding}\isamarkupfalse%
\ neg{\isacharunderscore}{\kern0pt}dual{\isacharunderscore}{\kern0pt}def{\isacharbrackleft}{\kern0pt}OF\ assms{\isacharparenleft}{\kern0pt}{\isadigit{1}}{\isacharparenright}{\kern0pt}{\isacharbrackright}{\kern0pt}\ \isacommand{by}\isamarkupfalse%
\ {\isacharparenleft}{\kern0pt}metis\ max{\isachardot}{\kern0pt}cobounded{\isadigit{1}}\ add{\isacharunderscore}{\kern0pt}increasing{\isadigit{2}}\ zero{\isacharunderscore}{\kern0pt}less{\isacharunderscore}{\kern0pt}one{\isacharunderscore}{\kern0pt}class{\isachardot}{\kern0pt}zero{\isacharunderscore}{\kern0pt}le{\isacharunderscore}{\kern0pt}one{\isacharparenright}{\kern0pt}\isanewline
\ \ \isacommand{hence}\isamarkupfalse%
\ N{\isacharprime}{\kern0pt}{\isacharcolon}{\kern0pt}\ {\isachardoublequoteopen}{\isasymforall}z\ {\isasymle}\ N{\isacharprime}{\kern0pt}{\isachardot}{\kern0pt}\ f\ z\ {\isachargreater}{\kern0pt}\ max\ {\isadigit{0}}\ {\isacharparenleft}{\kern0pt}{\isacharminus}{\kern0pt}\ C{\isacharparenright}{\kern0pt}{\isachardoublequoteclose}\ \isacommand{by}\isamarkupfalse%
\ fastforce\isanewline
\ \ \isacommand{have}\isamarkupfalse%
\ {\isacharasterisk}{\kern0pt}{\isacharcolon}{\kern0pt}\ {\isachardoublequoteopen}{\isadigit{1}}\ {\isacharequal}{\kern0pt}\ abs{\isacharunderscore}{\kern0pt}real\ {\isacharparenleft}{\kern0pt}{\isasymlambda}x{\isachardot}{\kern0pt}\ x\ {\isacharplus}{\kern0pt}\ N{\isacharprime}{\kern0pt}\ {\isacharminus}{\kern0pt}\ N{\isacharparenright}{\kern0pt}{\isachardoublequoteclose}\ {\isachardoublequoteopen}slope\ {\isacharparenleft}{\kern0pt}{\isasymlambda}x{\isachardot}{\kern0pt}\ x\ {\isacharplus}{\kern0pt}\ N{\isacharprime}{\kern0pt}\ {\isacharminus}{\kern0pt}\ N{\isacharparenright}{\kern0pt}{\isachardoublequoteclose}\ \isacommand{unfolding}\isamarkupfalse%
\ one{\isacharunderscore}{\kern0pt}def\ \isacommand{by}\isamarkupfalse%
\ {\isacharparenleft}{\kern0pt}intro\ abs{\isacharunderscore}{\kern0pt}real{\isacharunderscore}{\kern0pt}eqI{\isacharparenright}{\kern0pt}\ {\isacharparenleft}{\kern0pt}auto\ simp\ add{\isacharcolon}{\kern0pt}\ eudoxus{\isacharunderscore}{\kern0pt}rel{\isacharunderscore}{\kern0pt}def\ slope{\isacharunderscore}{\kern0pt}def\ intro{\isacharbang}{\kern0pt}{\isacharcolon}{\kern0pt}\ boundedI{\isacharparenright}{\kern0pt}\isanewline
\ \ \isacommand{hence}\isamarkupfalse%
\ {\isachardoublequoteopen}abs{\isacharunderscore}{\kern0pt}real\ f\ {\isacharasterisk}{\kern0pt}\ {\isadigit{1}}\ {\isacharequal}{\kern0pt}\ abs{\isacharunderscore}{\kern0pt}real\ {\isacharparenleft}{\kern0pt}f\ {\isacharasterisk}{\kern0pt}\isactrlsub e\ {\isacharparenleft}{\kern0pt}{\isasymlambda}x{\isachardot}{\kern0pt}\ x\ {\isacharplus}{\kern0pt}\ N{\isacharprime}{\kern0pt}\ {\isacharminus}{\kern0pt}\ N{\isacharparenright}{\kern0pt}{\isacharparenright}{\kern0pt}{\isachardoublequoteclose}\ \isacommand{using}\isamarkupfalse%
\ abs{\isacharunderscore}{\kern0pt}real{\isacharunderscore}{\kern0pt}times{\isacharbrackleft}{\kern0pt}OF\ assms{\isacharparenleft}{\kern0pt}{\isadigit{1}}{\isacharparenright}{\kern0pt}\ {\isacharasterisk}{\kern0pt}{\isacharparenleft}{\kern0pt}{\isadigit{2}}{\isacharparenright}{\kern0pt}{\isacharbrackright}{\kern0pt}\ \isacommand{by}\isamarkupfalse%
\ simp\isanewline
\ \ \isacommand{hence}\isamarkupfalse%
\ {\isachardoublequoteopen}f\ {\isasymsim}\isactrlsub e\ {\isacharparenleft}{\kern0pt}f\ {\isacharasterisk}{\kern0pt}\isactrlsub e\ {\isacharparenleft}{\kern0pt}{\isasymlambda}x{\isachardot}{\kern0pt}\ x\ {\isacharplus}{\kern0pt}\ N{\isacharprime}{\kern0pt}\ {\isacharminus}{\kern0pt}\ N{\isacharparenright}{\kern0pt}{\isacharparenright}{\kern0pt}{\isachardoublequoteclose}\ \isacommand{using}\isamarkupfalse%
\ assms\ {\isacharasterisk}{\kern0pt}\ \isacommand{by}\isamarkupfalse%
\ {\isacharparenleft}{\kern0pt}metis\ abs{\isacharunderscore}{\kern0pt}real{\isacharunderscore}{\kern0pt}eq{\isacharunderscore}{\kern0pt}iff\ eudoxus{\isacharunderscore}{\kern0pt}times{\isacharunderscore}{\kern0pt}commute\ mult{\isachardot}{\kern0pt}right{\isacharunderscore}{\kern0pt}neutral{\isacharparenright}{\kern0pt}\isanewline
\ \ \isacommand{moreover}\isamarkupfalse%
\ \isacommand{have}\isamarkupfalse%
\ {\isachardoublequoteopen}{\isasymforall}z\ {\isacharless}{\kern0pt}\ N{\isachardot}{\kern0pt}\ {\isacharparenleft}{\kern0pt}f\ {\isacharasterisk}{\kern0pt}\isactrlsub e\ {\isacharparenleft}{\kern0pt}{\isasymlambda}x{\isachardot}{\kern0pt}\ x\ {\isacharplus}{\kern0pt}\ N{\isacharprime}{\kern0pt}\ {\isacharminus}{\kern0pt}\ N{\isacharparenright}{\kern0pt}{\isacharparenright}{\kern0pt}\ z\ {\isachargreater}{\kern0pt}\ {\isacharminus}{\kern0pt}\ C{\isachardoublequoteclose}\ \isacommand{unfolding}\isamarkupfalse%
\ eudoxus{\isacharunderscore}{\kern0pt}times{\isacharunderscore}{\kern0pt}def\ \isacommand{using}\isamarkupfalse%
\ N{\isacharprime}{\kern0pt}\ \isacommand{by}\isamarkupfalse%
\ simp\isanewline
\ \ \isacommand{ultimately}\isamarkupfalse%
\ \isacommand{show}\isamarkupfalse%
\ {\isacharquery}{\kern0pt}thesis\ \isacommand{using}\isamarkupfalse%
\ that\ \isacommand{by}\isamarkupfalse%
\ {\isacharparenleft}{\kern0pt}meson\ linorder{\isacharunderscore}{\kern0pt}not{\isacharunderscore}{\kern0pt}less{\isacharparenright}{\kern0pt}\isanewline
\isacommand{qed}\isamarkupfalse%
%
\endisatagproof
{\isafoldproof}%
%
\isadelimproof
%
\endisadelimproof
%
\begin{isamarkuptext}%
We call a real \isa{x} less than another real \isa{y}, if their difference is positive.%
\end{isamarkuptext}\isamarkuptrue%
\isacommand{definition}\isamarkupfalse%
\isanewline
\ \ {\isachardoublequoteopen}x\ {\isacharless}{\kern0pt}\ {\isacharparenleft}{\kern0pt}y{\isacharcolon}{\kern0pt}{\isacharcolon}{\kern0pt}real{\isacharparenright}{\kern0pt}\ {\isasymequiv}\ sgn\ {\isacharparenleft}{\kern0pt}y\ {\isacharminus}{\kern0pt}\ x{\isacharparenright}{\kern0pt}\ {\isacharequal}{\kern0pt}\ {\isadigit{1}}{\isachardoublequoteclose}\isanewline
\isanewline
\isacommand{definition}\isamarkupfalse%
\isanewline
\ \ {\isachardoublequoteopen}x\ {\isasymle}\ {\isacharparenleft}{\kern0pt}y{\isacharcolon}{\kern0pt}{\isacharcolon}{\kern0pt}real{\isacharparenright}{\kern0pt}\ {\isasymequiv}\ x\ {\isacharless}{\kern0pt}\ y\ {\isasymor}\ x\ {\isacharequal}{\kern0pt}\ y{\isachardoublequoteclose}\isanewline
\isanewline
\isacommand{definition}\isamarkupfalse%
\isanewline
\ \ abs{\isacharunderscore}{\kern0pt}real{\isacharcolon}{\kern0pt}\ {\isachardoublequoteopen}{\isasymbar}x\ {\isacharcolon}{\kern0pt}{\isacharcolon}{\kern0pt}\ real{\isasymbar}\ {\isacharequal}{\kern0pt}\ {\isacharparenleft}{\kern0pt}if\ {\isadigit{0}}\ {\isasymle}\ x\ then\ x\ else\ {\isacharminus}{\kern0pt}\ x{\isacharparenright}{\kern0pt}{\isachardoublequoteclose}\isanewline
\isanewline
\isacommand{instance}\isamarkupfalse%
%
\isadelimproof
\ %
\endisadelimproof
%
\isatagproof
\isacommand{{\isachardot}{\kern0pt}{\isachardot}{\kern0pt}}\isamarkupfalse%
%
\endisatagproof
{\isafoldproof}%
%
\isadelimproof
%
\endisadelimproof
\isanewline
\isacommand{end}\isamarkupfalse%
\isanewline
\isanewline
\isacommand{instance}\isamarkupfalse%
\ real\ {\isacharcolon}{\kern0pt}{\isacharcolon}{\kern0pt}\ linorder\isanewline
%
\isadelimproof
%
\endisadelimproof
%
\isatagproof
\isacommand{proof}\isamarkupfalse%
\isanewline
\ \ \isacommand{fix}\isamarkupfalse%
\ x\ y\ z\ {\isacharcolon}{\kern0pt}{\isacharcolon}{\kern0pt}\ real\isanewline
\ \ \isacommand{show}\isamarkupfalse%
\ {\isachardoublequoteopen}{\isacharparenleft}{\kern0pt}x\ {\isacharless}{\kern0pt}\ y{\isacharparenright}{\kern0pt}\ {\isacharequal}{\kern0pt}\ {\isacharparenleft}{\kern0pt}x\ {\isasymle}\ y\ {\isasymand}\ {\isasymnot}\ y\ {\isasymle}\ x{\isacharparenright}{\kern0pt}{\isachardoublequoteclose}\ \isacommand{unfolding}\isamarkupfalse%
\ less{\isacharunderscore}{\kern0pt}eq{\isacharunderscore}{\kern0pt}real{\isacharunderscore}{\kern0pt}def\ less{\isacharunderscore}{\kern0pt}real{\isacharunderscore}{\kern0pt}def\ \isacommand{using}\isamarkupfalse%
\ sgn{\isacharunderscore}{\kern0pt}times{\isacharbrackleft}{\kern0pt}of\ {\isachardoublequoteopen}{\isacharminus}{\kern0pt}{\isadigit{1}}{\isachardoublequoteclose}\ {\isachardoublequoteopen}x\ {\isacharminus}{\kern0pt}\ y{\isachardoublequoteclose}{\isacharbrackright}{\kern0pt}\ \isacommand{by}\isamarkupfalse%
\ fastforce\isanewline
\ \ \isacommand{show}\isamarkupfalse%
\ {\isachardoublequoteopen}x\ {\isasymle}\ x{\isachardoublequoteclose}\ \isacommand{unfolding}\isamarkupfalse%
\ less{\isacharunderscore}{\kern0pt}eq{\isacharunderscore}{\kern0pt}real{\isacharunderscore}{\kern0pt}def\ \isacommand{by}\isamarkupfalse%
\ blast\isanewline
\ \ \isacommand{show}\isamarkupfalse%
\ {\isachardoublequoteopen}{\isasymlbrakk}x\ {\isasymle}\ y{\isacharsemicolon}{\kern0pt}\ y\ {\isasymle}\ z{\isasymrbrakk}\ {\isasymLongrightarrow}\ x\ {\isasymle}\ z{\isachardoublequoteclose}\ \isacommand{unfolding}\isamarkupfalse%
\ less{\isacharunderscore}{\kern0pt}eq{\isacharunderscore}{\kern0pt}real{\isacharunderscore}{\kern0pt}def\ less{\isacharunderscore}{\kern0pt}real{\isacharunderscore}{\kern0pt}def\ \isacommand{using}\isamarkupfalse%
\ sgn{\isacharunderscore}{\kern0pt}plus\ \isacommand{by}\isamarkupfalse%
\ fastforce\isanewline
\ \ \isacommand{show}\isamarkupfalse%
\ {\isachardoublequoteopen}{\isasymlbrakk}x\ {\isasymle}\ y{\isacharsemicolon}{\kern0pt}\ y\ {\isasymle}\ x{\isasymrbrakk}\ {\isasymLongrightarrow}\ x\ {\isacharequal}{\kern0pt}\ y{\isachardoublequoteclose}\ \isacommand{unfolding}\isamarkupfalse%
\ less{\isacharunderscore}{\kern0pt}eq{\isacharunderscore}{\kern0pt}real{\isacharunderscore}{\kern0pt}def\ less{\isacharunderscore}{\kern0pt}real{\isacharunderscore}{\kern0pt}def\ \isacommand{using}\isamarkupfalse%
\ sgn{\isacharunderscore}{\kern0pt}times{\isacharbrackleft}{\kern0pt}of\ {\isachardoublequoteopen}{\isacharminus}{\kern0pt}{\isadigit{1}}{\isachardoublequoteclose}\ {\isachardoublequoteopen}x\ {\isacharminus}{\kern0pt}\ y{\isachardoublequoteclose}{\isacharbrackright}{\kern0pt}\ \isacommand{by}\isamarkupfalse%
\ fastforce\isanewline
\ \ \isacommand{show}\isamarkupfalse%
\ {\isachardoublequoteopen}x\ {\isasymle}\ y\ {\isasymor}\ y\ {\isasymle}\ x{\isachardoublequoteclose}\ \isacommand{unfolding}\isamarkupfalse%
\ less{\isacharunderscore}{\kern0pt}eq{\isacharunderscore}{\kern0pt}real{\isacharunderscore}{\kern0pt}def\ less{\isacharunderscore}{\kern0pt}real{\isacharunderscore}{\kern0pt}def\ \isacommand{using}\isamarkupfalse%
\ sgn{\isacharunderscore}{\kern0pt}times{\isacharbrackleft}{\kern0pt}of\ {\isachardoublequoteopen}{\isacharminus}{\kern0pt}{\isadigit{1}}{\isachardoublequoteclose}\ {\isachardoublequoteopen}x\ {\isacharminus}{\kern0pt}\ y{\isachardoublequoteclose}{\isacharbrackright}{\kern0pt}\ sgn{\isacharunderscore}{\kern0pt}range\ \isacommand{by}\isamarkupfalse%
\ force\isanewline
\isacommand{qed}\isamarkupfalse%
%
\endisatagproof
{\isafoldproof}%
%
\isadelimproof
\isanewline
%
\endisadelimproof
\isanewline
\isanewline
\isacommand{lemma}\isamarkupfalse%
\ real{\isacharunderscore}{\kern0pt}leI{\isacharcolon}{\kern0pt}\isanewline
\ \ \isakeyword{assumes}\ {\isachardoublequoteopen}sgn\ {\isacharparenleft}{\kern0pt}y\ {\isacharminus}{\kern0pt}\ x{\isacharparenright}{\kern0pt}\ {\isasymin}\ {\isacharbraceleft}{\kern0pt}{\isadigit{0}}\ {\isacharcolon}{\kern0pt}{\isacharcolon}{\kern0pt}\ real{\isacharcomma}{\kern0pt}\ {\isadigit{1}}{\isacharbraceright}{\kern0pt}{\isachardoublequoteclose}\isanewline
\ \ \isakeyword{shows}\ {\isachardoublequoteopen}x\ {\isasymle}\ y{\isachardoublequoteclose}\isanewline
%
\isadelimproof
\ \ %
\endisadelimproof
%
\isatagproof
\isacommand{using}\isamarkupfalse%
\ assms\ \isacommand{unfolding}\isamarkupfalse%
\ less{\isacharunderscore}{\kern0pt}eq{\isacharunderscore}{\kern0pt}real{\isacharunderscore}{\kern0pt}def\ less{\isacharunderscore}{\kern0pt}real{\isacharunderscore}{\kern0pt}def\ \isacommand{by}\isamarkupfalse%
\ force%
\endisatagproof
{\isafoldproof}%
%
\isadelimproof
\isanewline
%
\endisadelimproof
\isanewline
\isacommand{lemma}\isamarkupfalse%
\ real{\isacharunderscore}{\kern0pt}lessI{\isacharcolon}{\kern0pt}\isanewline
\ \ \isakeyword{assumes}\ {\isachardoublequoteopen}sgn\ {\isacharparenleft}{\kern0pt}y\ {\isacharminus}{\kern0pt}\ x{\isacharparenright}{\kern0pt}\ {\isacharequal}{\kern0pt}\ {\isacharparenleft}{\kern0pt}{\isadigit{1}}\ {\isacharcolon}{\kern0pt}{\isacharcolon}{\kern0pt}\ real{\isacharparenright}{\kern0pt}{\isachardoublequoteclose}\isanewline
\ \ \isakeyword{shows}\ {\isachardoublequoteopen}x\ {\isacharless}{\kern0pt}\ y{\isachardoublequoteclose}\isanewline
%
\isadelimproof
\ \ %
\endisadelimproof
%
\isatagproof
\isacommand{using}\isamarkupfalse%
\ assms\ \isacommand{unfolding}\isamarkupfalse%
\ less{\isacharunderscore}{\kern0pt}real{\isacharunderscore}{\kern0pt}def\ \isacommand{by}\isamarkupfalse%
\ blast%
\endisatagproof
{\isafoldproof}%
%
\isadelimproof
\isanewline
%
\endisadelimproof
\isanewline
\isacommand{lemma}\isamarkupfalse%
\ abs{\isacharunderscore}{\kern0pt}real{\isacharunderscore}{\kern0pt}leI{\isacharcolon}{\kern0pt}\isanewline
\ \ \isakeyword{assumes}\ {\isachardoublequoteopen}slope\ f{\isachardoublequoteclose}\ {\isachardoublequoteopen}slope\ g{\isachardoublequoteclose}\ {\isachardoublequoteopen}{\isasymAnd}z{\isachardot}{\kern0pt}\ z\ {\isasymge}\ N\ {\isasymLongrightarrow}\ f\ z\ {\isasymge}\ g\ z{\isachardoublequoteclose}\isanewline
\ \ \isakeyword{shows}\ {\isachardoublequoteopen}abs{\isacharunderscore}{\kern0pt}real\ f\ {\isasymge}\ abs{\isacharunderscore}{\kern0pt}real\ g{\isachardoublequoteclose}\isanewline
%
\isadelimproof
%
\endisadelimproof
%
\isatagproof
\isacommand{proof}\isamarkupfalse%
\ {\isacharminus}{\kern0pt}\isanewline
\ \ \isacommand{{\isacharbraceleft}{\kern0pt}}\isamarkupfalse%
\isanewline
\ \ \ \ \isacommand{assume}\isamarkupfalse%
\ {\isachardoublequoteopen}abs{\isacharunderscore}{\kern0pt}real\ f\ {\isasymnoteq}\ abs{\isacharunderscore}{\kern0pt}real\ g{\isachardoublequoteclose}\isanewline
\ \ \ \ \isacommand{hence}\isamarkupfalse%
\ {\isachardoublequoteopen}abs{\isacharunderscore}{\kern0pt}real\ {\isacharparenleft}{\kern0pt}f\ {\isacharplus}{\kern0pt}\isactrlsub e\ {\isacharminus}{\kern0pt}\isactrlsub e\ g{\isacharparenright}{\kern0pt}\ {\isasymnoteq}\ {\isadigit{0}}{\isachardoublequoteclose}\ \isacommand{by}\isamarkupfalse%
\ {\isacharparenleft}{\kern0pt}metis\ abs{\isacharunderscore}{\kern0pt}real{\isacharunderscore}{\kern0pt}minus\ assms{\isacharparenleft}{\kern0pt}{\isadigit{1}}{\isacharcomma}{\kern0pt}{\isadigit{2}}{\isacharparenright}{\kern0pt}\ eq{\isacharunderscore}{\kern0pt}iff{\isacharunderscore}{\kern0pt}diff{\isacharunderscore}{\kern0pt}eq{\isacharunderscore}{\kern0pt}{\isadigit{0}}{\isacharparenright}{\kern0pt}\isanewline
\ \ \ \ \isacommand{hence}\isamarkupfalse%
\ {\isachardoublequoteopen}{\isasymnot}\ bounded\ {\isacharparenleft}{\kern0pt}f\ {\isacharplus}{\kern0pt}\isactrlsub e\ {\isacharminus}{\kern0pt}\isactrlsub e\ g{\isacharparenright}{\kern0pt}{\isachardoublequoteclose}\ \isacommand{by}\isamarkupfalse%
\ {\isacharparenleft}{\kern0pt}metis\ abs{\isacharunderscore}{\kern0pt}real{\isacharunderscore}{\kern0pt}eqI\ zero{\isacharunderscore}{\kern0pt}def\ zero{\isacharunderscore}{\kern0pt}iff{\isacharunderscore}{\kern0pt}bounded{\isacharparenright}{\kern0pt}\isanewline
\ \ \ \ \isacommand{hence}\isamarkupfalse%
\ {\isachardoublequoteopen}pos\ {\isacharparenleft}{\kern0pt}f\ {\isacharplus}{\kern0pt}\isactrlsub e\ {\isacharminus}{\kern0pt}\isactrlsub e\ g{\isacharparenright}{\kern0pt}\ {\isasymor}\ neg\ {\isacharparenleft}{\kern0pt}f\ {\isacharplus}{\kern0pt}\isactrlsub e\ {\isacharminus}{\kern0pt}\isactrlsub e\ g{\isacharparenright}{\kern0pt}{\isachardoublequoteclose}\ \isacommand{using}\isamarkupfalse%
\ assms\ eudoxus{\isacharunderscore}{\kern0pt}plus{\isacharunderscore}{\kern0pt}cong\ eudoxus{\isacharunderscore}{\kern0pt}uminus{\isacharunderscore}{\kern0pt}cong\ neg{\isacharunderscore}{\kern0pt}iff{\isacharunderscore}{\kern0pt}nonpos{\isacharunderscore}{\kern0pt}nonzero\ slope{\isacharunderscore}{\kern0pt}refl\ \isacommand{by}\isamarkupfalse%
\ auto\isanewline
\ \ \ \ \isacommand{moreover}\isamarkupfalse%
\isanewline
\ \ \ \ \isacommand{{\isacharbraceleft}{\kern0pt}}\isamarkupfalse%
\isanewline
\ \ \ \ \ \ \isacommand{assume}\isamarkupfalse%
\ {\isachardoublequoteopen}neg\ {\isacharparenleft}{\kern0pt}f\ {\isacharplus}{\kern0pt}\isactrlsub e\ {\isacharminus}{\kern0pt}\isactrlsub e\ g{\isacharparenright}{\kern0pt}{\isachardoublequoteclose}\isanewline
\ \ \ \ \ \ \isacommand{then}\isamarkupfalse%
\ \isacommand{obtain}\isamarkupfalse%
\ N{\isacharprime}{\kern0pt}\ \isakeyword{where}\ {\isachardoublequoteopen}{\isacharparenleft}{\kern0pt}f\ {\isacharplus}{\kern0pt}\isactrlsub e\ {\isacharminus}{\kern0pt}\isactrlsub e\ g{\isacharparenright}{\kern0pt}\ z\ {\isasymle}\ {\isacharminus}{\kern0pt}\ {\isadigit{1}}{\isachardoublequoteclose}\ \isakeyword{if}\ {\isachardoublequoteopen}z\ {\isasymge}\ N{\isacharprime}{\kern0pt}{\isachardoublequoteclose}\ \isakeyword{for}\ z\ \isacommand{unfolding}\isamarkupfalse%
\ neg{\isacharunderscore}{\kern0pt}def\ \isacommand{by}\isamarkupfalse%
\ fastforce\isanewline
\ \ \ \ \ \ \isacommand{hence}\isamarkupfalse%
\ {\isachardoublequoteopen}f\ z\ {\isacharless}{\kern0pt}\ g\ z{\isachardoublequoteclose}\ \isakeyword{if}\ {\isachardoublequoteopen}z\ {\isasymge}\ N{\isacharprime}{\kern0pt}{\isachardoublequoteclose}\ \isakeyword{for}\ z\ \isacommand{using}\isamarkupfalse%
\ that\ \isacommand{unfolding}\isamarkupfalse%
\ eudoxus{\isacharunderscore}{\kern0pt}plus{\isacharunderscore}{\kern0pt}def\ eudoxus{\isacharunderscore}{\kern0pt}uminus{\isacharunderscore}{\kern0pt}def\ \isacommand{by}\isamarkupfalse%
\ fastforce\isanewline
\ \ \ \ \ \ \isacommand{hence}\isamarkupfalse%
\ False\ \isacommand{using}\isamarkupfalse%
\ assms\ \isacommand{by}\isamarkupfalse%
\ {\isacharparenleft}{\kern0pt}metis\ linorder{\isacharunderscore}{\kern0pt}not{\isacharunderscore}{\kern0pt}less\ nle{\isacharunderscore}{\kern0pt}le{\isacharparenright}{\kern0pt}\ \ \ \ \ \ \isanewline
\ \ \ \ \isacommand{{\isacharbraceright}{\kern0pt}}\isamarkupfalse%
\isanewline
\ \ \ \ \isacommand{ultimately}\isamarkupfalse%
\ \isacommand{have}\isamarkupfalse%
\ {\isachardoublequoteopen}abs{\isacharunderscore}{\kern0pt}real\ f\ {\isachargreater}{\kern0pt}\ abs{\isacharunderscore}{\kern0pt}real\ g{\isachardoublequoteclose}\ \isacommand{using}\isamarkupfalse%
\ assms\ \isacommand{by}\isamarkupfalse%
\ {\isacharparenleft}{\kern0pt}fastforce\ intro{\isacharcolon}{\kern0pt}\ real{\isacharunderscore}{\kern0pt}lessI\ sgn{\isacharunderscore}{\kern0pt}pos\ simp\ add{\isacharcolon}{\kern0pt}\ eudoxus{\isacharunderscore}{\kern0pt}plus{\isacharunderscore}{\kern0pt}def\ eudoxus{\isacharunderscore}{\kern0pt}uminus{\isacharunderscore}{\kern0pt}def{\isacharparenright}{\kern0pt}\isanewline
\ \ \isacommand{{\isacharbraceright}{\kern0pt}}\isamarkupfalse%
\isanewline
\ \ \isacommand{thus}\isamarkupfalse%
\ {\isacharquery}{\kern0pt}thesis\ \isacommand{unfolding}\isamarkupfalse%
\ less{\isacharunderscore}{\kern0pt}eq{\isacharunderscore}{\kern0pt}real{\isacharunderscore}{\kern0pt}def\ \isacommand{by}\isamarkupfalse%
\ argo\isanewline
\isacommand{qed}\isamarkupfalse%
%
\endisatagproof
{\isafoldproof}%
%
\isadelimproof
\isanewline
%
\endisadelimproof
\isanewline
\isacommand{lemma}\isamarkupfalse%
\ abs{\isacharunderscore}{\kern0pt}real{\isacharunderscore}{\kern0pt}lessI{\isacharcolon}{\kern0pt}\isanewline
\ \ \isakeyword{assumes}\ {\isachardoublequoteopen}slope\ f{\isachardoublequoteclose}\ {\isachardoublequoteopen}slope\ g{\isachardoublequoteclose}\ {\isachardoublequoteopen}{\isasymAnd}z{\isachardot}{\kern0pt}\ z\ {\isasymge}\ N\ {\isasymLongrightarrow}\ f\ z\ {\isasymge}\ g\ z{\isachardoublequoteclose}\ {\isachardoublequoteopen}{\isasymAnd}C{\isachardot}{\kern0pt}\ C\ {\isasymge}\ {\isadigit{0}}\ {\isasymLongrightarrow}\ {\isasymexists}z{\isachardot}{\kern0pt}\ f\ z\ {\isasymge}\ g\ z\ {\isacharplus}{\kern0pt}\ C{\isachardoublequoteclose}\isanewline
\ \ \isakeyword{shows}\ {\isachardoublequoteopen}abs{\isacharunderscore}{\kern0pt}real\ f\ {\isachargreater}{\kern0pt}\ abs{\isacharunderscore}{\kern0pt}real\ g{\isachardoublequoteclose}\isanewline
%
\isadelimproof
%
\endisadelimproof
%
\isatagproof
\isacommand{proof}\isamarkupfalse%
\ {\isacharminus}{\kern0pt}\isanewline
\ \ \isacommand{{\isacharbraceleft}{\kern0pt}}\isamarkupfalse%
\isanewline
\ \ \ \ \isacommand{assume}\isamarkupfalse%
\ {\isachardoublequoteopen}bounded\ {\isacharparenleft}{\kern0pt}f\ {\isacharplus}{\kern0pt}\isactrlsub e\ {\isacharminus}{\kern0pt}\isactrlsub e\ g{\isacharparenright}{\kern0pt}{\isachardoublequoteclose}\isanewline
\ \ \ \ \isacommand{then}\isamarkupfalse%
\ \isacommand{obtain}\isamarkupfalse%
\ C\ \isakeyword{where}\ {\isachardoublequoteopen}{\isasymbar}f\ z\ {\isacharminus}{\kern0pt}\ g\ z{\isasymbar}\ {\isasymle}\ C{\isachardoublequoteclose}\ {\isachardoublequoteopen}C\ {\isasymge}\ {\isadigit{0}}{\isachardoublequoteclose}\ \isakeyword{for}\ z\ \isacommand{unfolding}\isamarkupfalse%
\ eudoxus{\isacharunderscore}{\kern0pt}plus{\isacharunderscore}{\kern0pt}def\ eudoxus{\isacharunderscore}{\kern0pt}uminus{\isacharunderscore}{\kern0pt}def\ \isacommand{by}\isamarkupfalse%
\ auto\isanewline
\ \ \ \ \isacommand{moreover}\isamarkupfalse%
\ \isacommand{obtain}\isamarkupfalse%
\ z\ \isakeyword{where}\ {\isachardoublequoteopen}f\ z\ {\isasymge}\ g\ z\ {\isacharplus}{\kern0pt}\ {\isacharparenleft}{\kern0pt}C\ {\isacharplus}{\kern0pt}\ {\isadigit{1}}{\isacharparenright}{\kern0pt}{\isachardoublequoteclose}\ \isacommand{using}\isamarkupfalse%
\ assms{\isacharparenleft}{\kern0pt}{\isadigit{4}}{\isacharparenright}{\kern0pt}{\isacharbrackleft}{\kern0pt}of\ {\isachardoublequoteopen}C\ {\isacharplus}{\kern0pt}\ {\isadigit{1}}{\isachardoublequoteclose}{\isacharbrackright}{\kern0pt}\ calculation\ \isacommand{by}\isamarkupfalse%
\ auto\isanewline
\ \ \ \ \isacommand{ultimately}\isamarkupfalse%
\ \isacommand{have}\isamarkupfalse%
\ False\ \isacommand{by}\isamarkupfalse%
\ {\isacharparenleft}{\kern0pt}metis\ abs{\isacharunderscore}{\kern0pt}le{\isacharunderscore}{\kern0pt}D{\isadigit{1}}\ add{\isachardot}{\kern0pt}commute\ dual{\isacharunderscore}{\kern0pt}order{\isachardot}{\kern0pt}trans\ le{\isacharunderscore}{\kern0pt}diff{\isacharunderscore}{\kern0pt}eq\ linorder{\isacharunderscore}{\kern0pt}not{\isacharunderscore}{\kern0pt}less\ zless{\isacharunderscore}{\kern0pt}add{\isadigit{1}}{\isacharunderscore}{\kern0pt}eq{\isacharparenright}{\kern0pt}\isanewline
\ \ \isacommand{{\isacharbraceright}{\kern0pt}}\isamarkupfalse%
\isanewline
\ \ \isacommand{moreover}\isamarkupfalse%
\ \isacommand{have}\isamarkupfalse%
\ {\isachardoublequoteopen}abs{\isacharunderscore}{\kern0pt}real\ f\ {\isasymge}\ abs{\isacharunderscore}{\kern0pt}real\ g{\isachardoublequoteclose}\ \isacommand{using}\isamarkupfalse%
\ assms\ abs{\isacharunderscore}{\kern0pt}real{\isacharunderscore}{\kern0pt}leI\ \isacommand{by}\isamarkupfalse%
\ blast\isanewline
\ \ \isacommand{ultimately}\isamarkupfalse%
\ \isacommand{show}\isamarkupfalse%
\ {\isacharquery}{\kern0pt}thesis\ \isacommand{by}\isamarkupfalse%
\ {\isacharparenleft}{\kern0pt}metis\ abs{\isacharunderscore}{\kern0pt}real{\isacharunderscore}{\kern0pt}minus\ assms{\isacharparenleft}{\kern0pt}{\isadigit{1}}{\isacharcomma}{\kern0pt}{\isadigit{2}}{\isacharparenright}{\kern0pt}\ eq{\isacharunderscore}{\kern0pt}iff{\isacharunderscore}{\kern0pt}diff{\isacharunderscore}{\kern0pt}eq{\isacharunderscore}{\kern0pt}{\isadigit{0}}\ eudoxus{\isacharunderscore}{\kern0pt}plus{\isacharunderscore}{\kern0pt}cong\ eudoxus{\isacharunderscore}{\kern0pt}sgn{\isacharunderscore}{\kern0pt}iff{\isacharparenleft}{\kern0pt}{\isadigit{1}}{\isacharparenright}{\kern0pt}\ eudoxus{\isacharunderscore}{\kern0pt}uminus{\isacharunderscore}{\kern0pt}cong\ order{\isacharunderscore}{\kern0pt}le{\isacharunderscore}{\kern0pt}imp{\isacharunderscore}{\kern0pt}less{\isacharunderscore}{\kern0pt}or{\isacharunderscore}{\kern0pt}eq\ sgn{\isacharunderscore}{\kern0pt}abs{\isacharunderscore}{\kern0pt}real{\isacharunderscore}{\kern0pt}zero{\isacharunderscore}{\kern0pt}iff\ sgn{\isacharunderscore}{\kern0pt}zero\ slope{\isacharunderscore}{\kern0pt}refl{\isacharparenright}{\kern0pt}\isanewline
\isacommand{qed}\isamarkupfalse%
%
\endisatagproof
{\isafoldproof}%
%
\isadelimproof
\isanewline
%
\endisadelimproof
\isanewline
\isacommand{lemma}\isamarkupfalse%
\ abs{\isacharunderscore}{\kern0pt}real{\isacharunderscore}{\kern0pt}lessD{\isacharcolon}{\kern0pt}\isanewline
\ \ \isakeyword{assumes}\ {\isachardoublequoteopen}slope\ f{\isachardoublequoteclose}\ {\isachardoublequoteopen}slope\ g{\isachardoublequoteclose}\ {\isachardoublequoteopen}abs{\isacharunderscore}{\kern0pt}real\ f\ {\isachargreater}{\kern0pt}\ abs{\isacharunderscore}{\kern0pt}real\ g{\isachardoublequoteclose}\isanewline
\ \ \isakeyword{obtains}\ z\ \isakeyword{where}\ {\isachardoublequoteopen}z\ {\isasymge}\ N{\isachardoublequoteclose}\ {\isachardoublequoteopen}f\ z\ {\isachargreater}{\kern0pt}\ g\ z{\isachardoublequoteclose}\isanewline
%
\isadelimproof
%
\endisadelimproof
%
\isatagproof
\isacommand{proof}\isamarkupfalse%
\ {\isacharminus}{\kern0pt}\isanewline
\ \ \isacommand{{\isacharbraceleft}{\kern0pt}}\isamarkupfalse%
\isanewline
\ \ \ \ \isacommand{assume}\isamarkupfalse%
\ {\isachardoublequoteopen}{\isasymexists}N{\isachardot}{\kern0pt}\ {\isasymforall}z{\isasymge}N{\isachardot}{\kern0pt}\ f\ z\ {\isasymle}\ g\ z{\isachardoublequoteclose}\isanewline
\ \ \ \ \isacommand{then}\isamarkupfalse%
\ \isacommand{obtain}\isamarkupfalse%
\ N\ \isakeyword{where}\ {\isachardoublequoteopen}f\ z\ {\isasymle}\ g\ z{\isachardoublequoteclose}\ \isakeyword{if}\ {\isachardoublequoteopen}z\ {\isasymge}\ N{\isachardoublequoteclose}\ \isakeyword{for}\ z\ \isacommand{by}\isamarkupfalse%
\ fastforce\isanewline
\ \ \ \ \isacommand{hence}\isamarkupfalse%
\ False\ \isacommand{using}\isamarkupfalse%
\ assms\ abs{\isacharunderscore}{\kern0pt}real{\isacharunderscore}{\kern0pt}leI\ \isacommand{by}\isamarkupfalse%
\ {\isacharparenleft}{\kern0pt}metis\ linorder{\isacharunderscore}{\kern0pt}not{\isacharunderscore}{\kern0pt}le{\isacharparenright}{\kern0pt}\isanewline
\ \ \isacommand{{\isacharbraceright}{\kern0pt}}\isamarkupfalse%
\isanewline
\ \ \isacommand{thus}\isamarkupfalse%
\ {\isacharquery}{\kern0pt}thesis\ \isacommand{using}\isamarkupfalse%
\ that\ \isacommand{by}\isamarkupfalse%
\ fastforce\isanewline
\isacommand{qed}\isamarkupfalse%
%
\endisatagproof
{\isafoldproof}%
%
\isadelimproof
%
\endisadelimproof
%
\isadelimdocument
%
\endisadelimdocument
%
\isatagdocument
%
\isamarkupsubsection{Multiplicative Inverse%
}
\isamarkuptrue%
%
\endisatagdocument
{\isafolddocument}%
%
\isadelimdocument
%
\endisadelimdocument
%
\begin{isamarkuptext}%
We now define the multiplicative inverse. We start by constructing a candidate for positive slopes first and then extend it to the entire domain using the choice function \isa{Eps}.%
\end{isamarkuptext}\isamarkuptrue%
\isacommand{instantiation}\isamarkupfalse%
\ real\ {\isacharcolon}{\kern0pt}{\isacharcolon}{\kern0pt}\ {\isachardoublequoteopen}{\isacharbraceleft}{\kern0pt}inverse{\isacharbraceright}{\kern0pt}{\isachardoublequoteclose}\isanewline
\isakeyword{begin}\isanewline
\isanewline
\isacommand{definition}\isamarkupfalse%
\ eudoxus{\isacharunderscore}{\kern0pt}pos{\isacharunderscore}{\kern0pt}inverse\ {\isacharcolon}{\kern0pt}{\isacharcolon}{\kern0pt}\ {\isachardoublequoteopen}{\isacharparenleft}{\kern0pt}int\ {\isasymRightarrow}\ int{\isacharparenright}{\kern0pt}\ {\isasymRightarrow}\ {\isacharparenleft}{\kern0pt}int\ {\isasymRightarrow}\ int{\isacharparenright}{\kern0pt}{\isachardoublequoteclose}\ \isakeyword{where}\isanewline
\ \ {\isachardoublequoteopen}eudoxus{\isacharunderscore}{\kern0pt}pos{\isacharunderscore}{\kern0pt}inverse\ f\ z\ {\isacharequal}{\kern0pt}\ sgn\ z\ {\isacharasterisk}{\kern0pt}\ Inf\ {\isacharparenleft}{\kern0pt}{\isacharbraceleft}{\kern0pt}{\isadigit{0}}{\isachardot}{\kern0pt}{\isachardot}{\kern0pt}{\isacharbraceright}{\kern0pt}\ {\isasyminter}\ {\isacharbraceleft}{\kern0pt}n{\isachardot}{\kern0pt}\ f\ n\ {\isasymge}\ {\isasymbar}z{\isasymbar}{\isacharbraceright}{\kern0pt}{\isacharparenright}{\kern0pt}{\isachardoublequoteclose}\isanewline
\isanewline
\isacommand{lemma}\isamarkupfalse%
\ eudoxus{\isacharunderscore}{\kern0pt}pos{\isacharunderscore}{\kern0pt}inverse{\isacharcolon}{\kern0pt}\isanewline
\ \ \isakeyword{assumes}\ {\isachardoublequoteopen}slope\ f{\isachardoublequoteclose}\ {\isachardoublequoteopen}pos\ f{\isachardoublequoteclose}\isanewline
\ \ \isakeyword{obtains}\ g\ \isakeyword{where}\ {\isachardoublequoteopen}f\ {\isasymsim}\isactrlsub e\ g{\isachardoublequoteclose}\ {\isachardoublequoteopen}slope\ {\isacharparenleft}{\kern0pt}eudoxus{\isacharunderscore}{\kern0pt}pos{\isacharunderscore}{\kern0pt}inverse\ g{\isacharparenright}{\kern0pt}{\isachardoublequoteclose}\ {\isachardoublequoteopen}eudoxus{\isacharunderscore}{\kern0pt}pos{\isacharunderscore}{\kern0pt}inverse\ g\ {\isacharasterisk}{\kern0pt}\isactrlsub e\ f\ {\isasymsim}\isactrlsub e\ id{\isachardoublequoteclose}\ \isanewline
%
\isadelimproof
%
\endisadelimproof
%
\isatagproof
\isacommand{proof}\isamarkupfalse%
\ {\isacharminus}{\kern0pt}\isanewline
\ \ \isacommand{let}\isamarkupfalse%
\ {\isacharquery}{\kern0pt}{\isasymphi}\ {\isacharequal}{\kern0pt}\ eudoxus{\isacharunderscore}{\kern0pt}pos{\isacharunderscore}{\kern0pt}inverse\isanewline
\ \ \isacommand{obtain}\isamarkupfalse%
\ g\ \isakeyword{where}\ g{\isacharcolon}{\kern0pt}\ {\isachardoublequoteopen}f\ {\isasymsim}\isactrlsub e\ g{\isachardoublequoteclose}\ {\isachardoublequoteopen}g\ z\ {\isasymge}\ {\isadigit{0}}\ {\isasymLongrightarrow}\ z\ {\isachargreater}{\kern0pt}\ {\isadigit{1}}{\isachardoublequoteclose}\ \isakeyword{for}\ z\ \isacommand{using}\isamarkupfalse%
\ pos{\isacharunderscore}{\kern0pt}representative{\isacharprime}{\kern0pt}{\isacharbrackleft}{\kern0pt}OF\ assms{\isacharbrackright}{\kern0pt}\ \isacommand{by}\isamarkupfalse%
\ {\isacharparenleft}{\kern0pt}metis\ gt{\isacharunderscore}{\kern0pt}ex\ order{\isacharunderscore}{\kern0pt}less{\isacharunderscore}{\kern0pt}le{\isacharunderscore}{\kern0pt}trans{\isacharparenright}{\kern0pt}\isanewline
\ \ \isacommand{hence}\isamarkupfalse%
\ pos{\isacharunderscore}{\kern0pt}g{\isacharcolon}{\kern0pt}\ {\isachardoublequoteopen}pos\ g{\isachardoublequoteclose}\ \isacommand{using}\isamarkupfalse%
\ assms\ pos{\isacharunderscore}{\kern0pt}cong\ \isacommand{by}\isamarkupfalse%
\ blast\isanewline
\ \ \isacommand{have}\isamarkupfalse%
\ slope{\isacharunderscore}{\kern0pt}g{\isacharcolon}{\kern0pt}\ {\isachardoublequoteopen}slope\ g{\isachardoublequoteclose}\ \isacommand{using}\isamarkupfalse%
\ g\ \isacommand{unfolding}\isamarkupfalse%
\ eudoxus{\isacharunderscore}{\kern0pt}rel{\isacharunderscore}{\kern0pt}def\ \isacommand{by}\isamarkupfalse%
\ simp\isanewline
\isanewline
\ \ \isacommand{have}\isamarkupfalse%
\ {\isachardoublequoteopen}{\isasymexists}n\ {\isasymge}\ {\isadigit{0}}{\isachardot}{\kern0pt}\ g\ n\ {\isasymge}\ {\isasymbar}z{\isasymbar}{\isachardoublequoteclose}\ \isakeyword{for}\ z\ \isacommand{using}\isamarkupfalse%
\ pos{\isacharunderscore}{\kern0pt}g\ \isacommand{unfolding}\isamarkupfalse%
\ pos{\isacharunderscore}{\kern0pt}def\ \isacommand{by}\isamarkupfalse%
\ {\isacharparenleft}{\kern0pt}metis\ abs{\isacharunderscore}{\kern0pt}ge{\isacharunderscore}{\kern0pt}self\ order{\isacharunderscore}{\kern0pt}less{\isacharunderscore}{\kern0pt}imp{\isacharunderscore}{\kern0pt}le\ zero{\isacharunderscore}{\kern0pt}less{\isacharunderscore}{\kern0pt}abs{\isacharunderscore}{\kern0pt}iff{\isacharparenright}{\kern0pt}\isanewline
\ \ \isacommand{hence}\isamarkupfalse%
\ nonempty{\isacharunderscore}{\kern0pt}{\isasymphi}{\isacharcolon}{\kern0pt}\ {\isachardoublequoteopen}{\isacharbraceleft}{\kern0pt}{\isadigit{0}}{\isachardot}{\kern0pt}{\isachardot}{\kern0pt}{\isacharbraceright}{\kern0pt}\ {\isasyminter}\ {\isacharbraceleft}{\kern0pt}n{\isachardot}{\kern0pt}\ {\isasymbar}z{\isasymbar}\ {\isasymle}\ g\ n{\isacharbraceright}{\kern0pt}\ {\isasymnoteq}\ {\isacharbraceleft}{\kern0pt}{\isacharbraceright}{\kern0pt}{\isachardoublequoteclose}\ \isakeyword{for}\ z\ \isacommand{by}\isamarkupfalse%
\ blast\isanewline
\ \ \isacommand{have}\isamarkupfalse%
\ bdd{\isacharunderscore}{\kern0pt}below{\isacharunderscore}{\kern0pt}{\isasymphi}{\isacharcolon}{\kern0pt}\ {\isachardoublequoteopen}bdd{\isacharunderscore}{\kern0pt}below\ {\isacharparenleft}{\kern0pt}{\isacharbraceleft}{\kern0pt}{\isadigit{0}}{\isachardot}{\kern0pt}{\isachardot}{\kern0pt}{\isacharbraceright}{\kern0pt}\ {\isasyminter}\ {\isacharbraceleft}{\kern0pt}n{\isachardot}{\kern0pt}\ g\ n\ {\isasymge}\ {\isasymbar}z{\isasymbar}{\isacharbraceright}{\kern0pt}{\isacharparenright}{\kern0pt}{\isachardoublequoteclose}\ \isakeyword{for}\ z\ \isacommand{by}\isamarkupfalse%
\ simp\isanewline
\ \ \isacommand{have}\isamarkupfalse%
\ {\isasymphi}{\isacharunderscore}{\kern0pt}bound{\isacharcolon}{\kern0pt}\ {\isachardoublequoteopen}g\ n\ {\isasymge}\ z\ {\isasymLongrightarrow}\ {\isacharquery}{\kern0pt}{\isasymphi}\ g\ z\ {\isasymle}\ n{\isachardoublequoteclose}\ \isakeyword{if}\ {\isachardoublequoteopen}z\ {\isasymge}\ {\isadigit{0}}{\isachardoublequoteclose}\ {\isachardoublequoteopen}n\ {\isasymge}\ {\isadigit{0}}{\isachardoublequoteclose}\ \isakeyword{for}\ n\ z\ \isacommand{unfolding}\isamarkupfalse%
\ eudoxus{\isacharunderscore}{\kern0pt}pos{\isacharunderscore}{\kern0pt}inverse{\isacharunderscore}{\kern0pt}def\ \isacommand{using}\isamarkupfalse%
\ cInf{\isacharunderscore}{\kern0pt}lower{\isacharbrackleft}{\kern0pt}OF\ {\isacharunderscore}{\kern0pt}\ bdd{\isacharunderscore}{\kern0pt}below{\isacharunderscore}{\kern0pt}{\isasymphi}{\isacharcomma}{\kern0pt}\ of\ n\ z{\isacharbrackright}{\kern0pt}\ that\ abs{\isacharunderscore}{\kern0pt}of{\isacharunderscore}{\kern0pt}nonneg\ zsgn{\isacharunderscore}{\kern0pt}def\ \isacommand{by}\isamarkupfalse%
\ simp\isanewline
\ \ \isacommand{hence}\isamarkupfalse%
\ {\isasymphi}{\isacharunderscore}{\kern0pt}bound{\isacharprime}{\kern0pt}{\isacharcolon}{\kern0pt}\ {\isachardoublequoteopen}{\isacharquery}{\kern0pt}{\isasymphi}\ g\ z\ {\isachargreater}{\kern0pt}\ n\ {\isasymLongrightarrow}\ g\ n\ {\isacharless}{\kern0pt}\ z{\isachardoublequoteclose}\ \isakeyword{if}\ {\isachardoublequoteopen}z\ {\isasymge}\ {\isadigit{0}}{\isachardoublequoteclose}\ {\isachardoublequoteopen}n\ {\isasymge}\ {\isadigit{0}}{\isachardoublequoteclose}\ \isakeyword{for}\ z\ n\ \isacommand{using}\isamarkupfalse%
\ that\ linorder{\isacharunderscore}{\kern0pt}not{\isacharunderscore}{\kern0pt}less\ \isacommand{by}\isamarkupfalse%
\ blast\isanewline
\isanewline
\ \ \isacommand{have}\isamarkupfalse%
\ {\isasymphi}{\isacharunderscore}{\kern0pt}mem{\isacharcolon}{\kern0pt}\ {\isachardoublequoteopen}z\ {\isachargreater}{\kern0pt}\ {\isadigit{0}}\ {\isasymLongrightarrow}\ {\isacharquery}{\kern0pt}{\isasymphi}\ g\ z\ {\isasymin}\ {\isacharbraceleft}{\kern0pt}{\isadigit{0}}{\isachardot}{\kern0pt}{\isachardot}{\kern0pt}{\isacharbraceright}{\kern0pt}\ {\isasyminter}\ {\isacharbraceleft}{\kern0pt}n{\isachardot}{\kern0pt}\ g\ n\ {\isasymge}\ {\isasymbar}z{\isasymbar}{\isacharbraceright}{\kern0pt}{\isachardoublequoteclose}\ \isakeyword{for}\ z\ \isacommand{unfolding}\isamarkupfalse%
\ eudoxus{\isacharunderscore}{\kern0pt}pos{\isacharunderscore}{\kern0pt}inverse{\isacharunderscore}{\kern0pt}def\ \isacommand{using}\isamarkupfalse%
\ int{\isacharunderscore}{\kern0pt}Inf{\isacharunderscore}{\kern0pt}mem{\isacharbrackleft}{\kern0pt}OF\ nonempty{\isacharunderscore}{\kern0pt}{\isasymphi}\ bdd{\isacharunderscore}{\kern0pt}below{\isacharunderscore}{\kern0pt}{\isasymphi}{\isacharcomma}{\kern0pt}\ of\ z{\isacharbrackright}{\kern0pt}\ \isacommand{by}\isamarkupfalse%
\ simp\isanewline
\isanewline
\ \ \isacommand{obtain}\isamarkupfalse%
\ L\ \isakeyword{where}\ {\isachardoublequoteopen}{\isasymbar}g\ {\isacharparenleft}{\kern0pt}{\isadigit{1}}\ {\isacharplus}{\kern0pt}\ {\isacharparenleft}{\kern0pt}z\ {\isacharminus}{\kern0pt}\ {\isadigit{1}}{\isacharparenright}{\kern0pt}{\isacharparenright}{\kern0pt}\ {\isacharminus}{\kern0pt}\ {\isacharparenleft}{\kern0pt}g\ {\isadigit{1}}\ {\isacharplus}{\kern0pt}\ g\ {\isacharparenleft}{\kern0pt}z\ {\isacharminus}{\kern0pt}\ {\isadigit{1}}{\isacharparenright}{\kern0pt}{\isacharparenright}{\kern0pt}{\isasymbar}\ {\isasymle}\ L{\isachardoublequoteclose}\ \isakeyword{for}\ z\ \isacommand{using}\isamarkupfalse%
\ slope{\isacharunderscore}{\kern0pt}g\ \isacommand{by}\isamarkupfalse%
\ fast\isanewline
\ \ \isacommand{hence}\isamarkupfalse%
\ {\isacharasterisk}{\kern0pt}{\isacharcolon}{\kern0pt}\ {\isachardoublequoteopen}{\isasymbar}g\ z\ {\isacharminus}{\kern0pt}\ {\isacharparenleft}{\kern0pt}g\ {\isadigit{1}}\ {\isacharplus}{\kern0pt}\ g\ {\isacharparenleft}{\kern0pt}z\ {\isacharminus}{\kern0pt}\ {\isadigit{1}}{\isacharparenright}{\kern0pt}{\isacharparenright}{\kern0pt}{\isasymbar}\ {\isasymle}\ L{\isachardoublequoteclose}\ \isakeyword{for}\ z\ \isacommand{by}\isamarkupfalse%
\ simp\isanewline
\ \ \isacommand{hence}\isamarkupfalse%
\ L{\isacharcolon}{\kern0pt}\ {\isachardoublequoteopen}g\ z\ {\isasymle}\ g\ {\isacharparenleft}{\kern0pt}z\ {\isacharminus}{\kern0pt}\ {\isadigit{1}}{\isacharparenright}{\kern0pt}\ {\isacharplus}{\kern0pt}\ {\isacharparenleft}{\kern0pt}L\ {\isacharplus}{\kern0pt}\ g\ {\isadigit{1}}{\isacharparenright}{\kern0pt}{\isachardoublequoteclose}\ \isakeyword{for}\ z\ \isacommand{using}\isamarkupfalse%
\ abs{\isacharunderscore}{\kern0pt}le{\isacharunderscore}{\kern0pt}D{\isadigit{1}}\ {\isacharasterisk}{\kern0pt}{\isacharbrackleft}{\kern0pt}of\ z{\isacharbrackright}{\kern0pt}\ \isacommand{by}\isamarkupfalse%
\ linarith\isanewline
\isanewline
\ \ \isacommand{let}\isamarkupfalse%
\ {\isacharquery}{\kern0pt}{\isasymgamma}\ {\isacharequal}{\kern0pt}\ {\isachardoublequoteopen}{\isasymlambda}m\ n{\isachardot}{\kern0pt}\ {\isacharparenleft}{\kern0pt}g\ {\isacharparenleft}{\kern0pt}m\ {\isacharplus}{\kern0pt}\ {\isacharparenleft}{\kern0pt}{\isacharminus}{\kern0pt}\ n{\isacharparenright}{\kern0pt}{\isacharparenright}{\kern0pt}\ {\isacharminus}{\kern0pt}\ {\isacharparenleft}{\kern0pt}g\ m\ {\isacharplus}{\kern0pt}\ g\ {\isacharparenleft}{\kern0pt}{\isacharminus}{\kern0pt}\ n{\isacharparenright}{\kern0pt}{\isacharparenright}{\kern0pt}{\isacharparenright}{\kern0pt}\ {\isacharminus}{\kern0pt}\ {\isacharparenleft}{\kern0pt}g\ {\isacharparenleft}{\kern0pt}n\ {\isacharplus}{\kern0pt}\ {\isacharparenleft}{\kern0pt}{\isacharminus}{\kern0pt}\ n{\isacharparenright}{\kern0pt}{\isacharparenright}{\kern0pt}\ {\isacharminus}{\kern0pt}\ {\isacharparenleft}{\kern0pt}g\ n\ {\isacharplus}{\kern0pt}\ g\ {\isacharparenleft}{\kern0pt}{\isacharminus}{\kern0pt}\ n{\isacharparenright}{\kern0pt}{\isacharparenright}{\kern0pt}{\isacharparenright}{\kern0pt}\ {\isacharplus}{\kern0pt}\ g\ {\isadigit{0}}{\isachardoublequoteclose}\isanewline
\ \ \isanewline
\ \ \isacommand{obtain}\isamarkupfalse%
\ c\ \isakeyword{where}\ c{\isacharcolon}{\kern0pt}\ {\isachardoublequoteopen}{\isasymbar}g\ {\isacharparenleft}{\kern0pt}m\ {\isacharplus}{\kern0pt}\ {\isacharparenleft}{\kern0pt}{\isacharminus}{\kern0pt}\ n{\isacharparenright}{\kern0pt}{\isacharparenright}{\kern0pt}\ {\isacharminus}{\kern0pt}\ {\isacharparenleft}{\kern0pt}g\ m\ {\isacharplus}{\kern0pt}\ g\ {\isacharparenleft}{\kern0pt}{\isacharminus}{\kern0pt}\ n{\isacharparenright}{\kern0pt}{\isacharparenright}{\kern0pt}{\isasymbar}\ {\isasymle}\ c{\isachardoublequoteclose}\ \isakeyword{for}\ m\ n\ \isacommand{using}\isamarkupfalse%
\ slope{\isacharunderscore}{\kern0pt}g\ \isacommand{by}\isamarkupfalse%
\ fast\isanewline
\ \ \isacommand{obtain}\isamarkupfalse%
\ c{\isacharprime}{\kern0pt}\ \isakeyword{where}\ c{\isacharprime}{\kern0pt}{\isacharcolon}{\kern0pt}\ {\isachardoublequoteopen}{\isasymbar}g\ {\isacharparenleft}{\kern0pt}n\ {\isacharplus}{\kern0pt}\ {\isacharparenleft}{\kern0pt}{\isacharminus}{\kern0pt}\ n{\isacharparenright}{\kern0pt}{\isacharparenright}{\kern0pt}\ {\isacharminus}{\kern0pt}\ {\isacharparenleft}{\kern0pt}g\ n\ {\isacharplus}{\kern0pt}\ g\ {\isacharparenleft}{\kern0pt}{\isacharminus}{\kern0pt}\ n{\isacharparenright}{\kern0pt}{\isacharparenright}{\kern0pt}{\isasymbar}\ {\isasymle}\ c{\isacharprime}{\kern0pt}{\isachardoublequoteclose}\ \isakeyword{for}\ n\ \isacommand{using}\isamarkupfalse%
\ slope{\isacharunderscore}{\kern0pt}g\ \isacommand{by}\isamarkupfalse%
\ fast\isanewline
\ \ \isacommand{have}\isamarkupfalse%
\ {\isachardoublequoteopen}{\isasymbar}{\isacharquery}{\kern0pt}{\isasymgamma}\ m\ n{\isasymbar}\ {\isasymle}\ {\isasymbar}g\ {\isacharparenleft}{\kern0pt}m\ {\isacharplus}{\kern0pt}\ {\isacharparenleft}{\kern0pt}{\isacharminus}{\kern0pt}\ n{\isacharparenright}{\kern0pt}{\isacharparenright}{\kern0pt}\ {\isacharminus}{\kern0pt}\ {\isacharparenleft}{\kern0pt}g\ m\ {\isacharplus}{\kern0pt}\ g\ {\isacharparenleft}{\kern0pt}{\isacharminus}{\kern0pt}\ n{\isacharparenright}{\kern0pt}{\isacharparenright}{\kern0pt}{\isasymbar}\ {\isacharplus}{\kern0pt}\ {\isasymbar}g\ {\isacharparenleft}{\kern0pt}n\ {\isacharplus}{\kern0pt}\ {\isacharparenleft}{\kern0pt}{\isacharminus}{\kern0pt}\ n{\isacharparenright}{\kern0pt}{\isacharparenright}{\kern0pt}\ {\isacharminus}{\kern0pt}\ {\isacharparenleft}{\kern0pt}g\ n\ {\isacharplus}{\kern0pt}\ g\ {\isacharparenleft}{\kern0pt}{\isacharminus}{\kern0pt}\ n{\isacharparenright}{\kern0pt}{\isacharparenright}{\kern0pt}{\isasymbar}\ {\isacharplus}{\kern0pt}\ {\isasymbar}g\ {\isadigit{0}}{\isasymbar}{\isachardoublequoteclose}\ \isakeyword{for}\ m\ n\ \isacommand{by}\isamarkupfalse%
\ linarith\isanewline
\ \ \isacommand{hence}\isamarkupfalse%
\ {\isacharasterisk}{\kern0pt}{\isacharcolon}{\kern0pt}\ {\isachardoublequoteopen}{\isasymbar}{\isacharquery}{\kern0pt}{\isasymgamma}\ m\ n{\isasymbar}\ {\isasymle}\ c\ {\isacharplus}{\kern0pt}\ c{\isacharprime}{\kern0pt}\ {\isacharplus}{\kern0pt}\ {\isasymbar}g\ {\isadigit{0}}{\isasymbar}{\isachardoublequoteclose}\ \isakeyword{for}\ m\ n\ \isacommand{using}\isamarkupfalse%
\ c{\isacharbrackleft}{\kern0pt}of\ m\ n{\isacharbrackright}{\kern0pt}\ c{\isacharprime}{\kern0pt}{\isacharbrackleft}{\kern0pt}of\ n{\isacharbrackright}{\kern0pt}\ \isacommand{by}\isamarkupfalse%
\ linarith\isanewline
\isanewline
\ \ \isacommand{define}\isamarkupfalse%
\ C\ \isakeyword{where}\ {\isachardoublequoteopen}C\ {\isacharequal}{\kern0pt}\ {\isadigit{2}}\ {\isacharasterisk}{\kern0pt}\ {\isacharparenleft}{\kern0pt}c\ {\isacharplus}{\kern0pt}\ c{\isacharprime}{\kern0pt}\ {\isacharplus}{\kern0pt}\ {\isasymbar}g\ {\isadigit{0}}{\isasymbar}{\isacharparenright}{\kern0pt}{\isachardoublequoteclose}\isanewline
\ \ \isacommand{have}\isamarkupfalse%
\ {\isachardoublequoteopen}g\ {\isacharparenleft}{\kern0pt}m\ {\isacharminus}{\kern0pt}\ {\isacharparenleft}{\kern0pt}n\ {\isacharplus}{\kern0pt}\ p{\isacharparenright}{\kern0pt}{\isacharparenright}{\kern0pt}\ {\isacharminus}{\kern0pt}\ {\isacharparenleft}{\kern0pt}g\ m\ {\isacharminus}{\kern0pt}\ {\isacharparenleft}{\kern0pt}g\ n\ {\isacharplus}{\kern0pt}\ g\ p{\isacharparenright}{\kern0pt}{\isacharparenright}{\kern0pt}\ {\isacharequal}{\kern0pt}\ {\isacharquery}{\kern0pt}{\isasymgamma}\ {\isacharparenleft}{\kern0pt}m\ {\isacharminus}{\kern0pt}\ n{\isacharparenright}{\kern0pt}\ p\ {\isacharplus}{\kern0pt}\ {\isacharquery}{\kern0pt}{\isasymgamma}\ m\ n{\isachardoublequoteclose}\ \isakeyword{for}\ m\ n\ p\ \isacommand{by}\isamarkupfalse%
\ {\isacharparenleft}{\kern0pt}simp\ add{\isacharcolon}{\kern0pt}\ algebra{\isacharunderscore}{\kern0pt}simps{\isacharparenright}{\kern0pt}\isanewline
\ \ \isacommand{hence}\isamarkupfalse%
\ {\isachardoublequoteopen}{\isasymbar}g\ {\isacharparenleft}{\kern0pt}m\ {\isacharminus}{\kern0pt}\ {\isacharparenleft}{\kern0pt}n\ {\isacharplus}{\kern0pt}\ p{\isacharparenright}{\kern0pt}{\isacharparenright}{\kern0pt}\ {\isacharminus}{\kern0pt}\ {\isacharparenleft}{\kern0pt}g\ m\ {\isacharminus}{\kern0pt}\ {\isacharparenleft}{\kern0pt}g\ n\ {\isacharplus}{\kern0pt}\ g\ p{\isacharparenright}{\kern0pt}{\isacharparenright}{\kern0pt}{\isasymbar}\ {\isasymle}\ {\isacharparenleft}{\kern0pt}c\ {\isacharplus}{\kern0pt}\ c{\isacharprime}{\kern0pt}\ {\isacharplus}{\kern0pt}\ {\isasymbar}g\ {\isadigit{0}}{\isasymbar}{\isacharparenright}{\kern0pt}\ {\isacharplus}{\kern0pt}\ {\isacharparenleft}{\kern0pt}c\ {\isacharplus}{\kern0pt}\ c{\isacharprime}{\kern0pt}\ {\isacharplus}{\kern0pt}\ {\isasymbar}g\ {\isadigit{0}}{\isasymbar}{\isacharparenright}{\kern0pt}{\isachardoublequoteclose}\ \isakeyword{for}\ m\ n\ p\ \isacommand{using}\isamarkupfalse%
\ {\isacharasterisk}{\kern0pt}{\isacharbrackleft}{\kern0pt}of\ {\isachardoublequoteopen}m\ {\isacharminus}{\kern0pt}\ n{\isachardoublequoteclose}\ p{\isacharbrackright}{\kern0pt}\ {\isacharasterisk}{\kern0pt}{\isacharbrackleft}{\kern0pt}of\ m\ n{\isacharbrackright}{\kern0pt}\ \isacommand{by}\isamarkupfalse%
\ simp\isanewline
\ \ \isacommand{hence}\isamarkupfalse%
\ {\isacharasterisk}{\kern0pt}{\isacharcolon}{\kern0pt}\ {\isachardoublequoteopen}{\isasymbar}g\ {\isacharparenleft}{\kern0pt}m\ {\isacharminus}{\kern0pt}\ {\isacharparenleft}{\kern0pt}n\ {\isacharplus}{\kern0pt}\ p{\isacharparenright}{\kern0pt}{\isacharparenright}{\kern0pt}\ {\isacharminus}{\kern0pt}\ {\isacharparenleft}{\kern0pt}g\ m\ {\isacharminus}{\kern0pt}\ {\isacharparenleft}{\kern0pt}g\ n\ {\isacharplus}{\kern0pt}\ g\ p{\isacharparenright}{\kern0pt}{\isacharparenright}{\kern0pt}{\isasymbar}\ {\isasymle}\ C{\isachardoublequoteclose}\ \isakeyword{for}\ m\ n\ p\ \isacommand{unfolding}\isamarkupfalse%
\ C{\isacharunderscore}{\kern0pt}def\ \isacommand{by}\isamarkupfalse%
\ {\isacharparenleft}{\kern0pt}metis\ mult{\isacharunderscore}{\kern0pt}{\isadigit{2}}{\isacharparenright}{\kern0pt}\isanewline
\ \ \isacommand{have}\isamarkupfalse%
\ C{\isacharcolon}{\kern0pt}\ {\isachardoublequoteopen}g\ {\isacharparenleft}{\kern0pt}m\ {\isacharminus}{\kern0pt}\ {\isacharparenleft}{\kern0pt}n\ {\isacharplus}{\kern0pt}\ p{\isacharparenright}{\kern0pt}{\isacharparenright}{\kern0pt}\ {\isasymle}\ g\ m\ {\isacharminus}{\kern0pt}\ {\isacharparenleft}{\kern0pt}g\ n\ {\isacharplus}{\kern0pt}\ g\ p{\isacharparenright}{\kern0pt}\ {\isacharplus}{\kern0pt}\ C{\isachardoublequoteclose}\ {\isachardoublequoteopen}g\ m\ {\isacharminus}{\kern0pt}\ {\isacharparenleft}{\kern0pt}g\ n\ {\isacharplus}{\kern0pt}\ g\ p{\isacharparenright}{\kern0pt}\ {\isacharplus}{\kern0pt}\ {\isacharparenleft}{\kern0pt}{\isacharminus}{\kern0pt}\ C{\isacharparenright}{\kern0pt}\ {\isasymle}\ g\ {\isacharparenleft}{\kern0pt}m\ {\isacharminus}{\kern0pt}\ {\isacharparenleft}{\kern0pt}n\ {\isacharplus}{\kern0pt}\ p{\isacharparenright}{\kern0pt}{\isacharparenright}{\kern0pt}{\isachardoublequoteclose}\ \isakeyword{for}\ m\ n\ p\ \isacommand{using}\isamarkupfalse%
\ {\isacharasterisk}{\kern0pt}{\isacharbrackleft}{\kern0pt}of\ m\ n\ p{\isacharbrackright}{\kern0pt}\ abs{\isacharunderscore}{\kern0pt}le{\isacharunderscore}{\kern0pt}D{\isadigit{1}}\ abs{\isacharunderscore}{\kern0pt}le{\isacharunderscore}{\kern0pt}D{\isadigit{2}}\ \isacommand{by}\isamarkupfalse%
\ linarith{\isacharplus}{\kern0pt}\isanewline
\isanewline
\ \ \isacommand{have}\isamarkupfalse%
\ bounded{\isacharcolon}{\kern0pt}\ {\isachardoublequoteopen}bounded\ h{\isachardoublequoteclose}\ \isakeyword{if}\ bounded{\isacharcolon}{\kern0pt}\ {\isachardoublequoteopen}bounded\ {\isacharparenleft}{\kern0pt}g\ o\ h{\isacharparenright}{\kern0pt}{\isachardoublequoteclose}\ \isakeyword{for}\ h\ {\isacharcolon}{\kern0pt}{\isacharcolon}{\kern0pt}\ {\isachardoublequoteopen}{\isacharprime}{\kern0pt}a\ {\isasymRightarrow}\ int{\isachardoublequoteclose}\isanewline
\ \ \isacommand{proof}\isamarkupfalse%
\ {\isacharparenleft}{\kern0pt}rule\ ccontr{\isacharparenright}{\kern0pt}\isanewline
\ \ \ \ \isacommand{assume}\isamarkupfalse%
\ asm{\isacharcolon}{\kern0pt}\ {\isachardoublequoteopen}{\isasymnot}\ bounded\ h{\isachardoublequoteclose}\isanewline
\ \ \ \ \isacommand{obtain}\isamarkupfalse%
\ C\ \isakeyword{where}\ C{\isacharcolon}{\kern0pt}\ {\isachardoublequoteopen}{\isasymbar}g\ {\isacharparenleft}{\kern0pt}h\ z{\isacharparenright}{\kern0pt}{\isasymbar}\ {\isasymle}\ C{\isachardoublequoteclose}\ {\isachardoublequoteopen}C\ {\isasymge}\ {\isadigit{0}}{\isachardoublequoteclose}\ \isakeyword{for}\ z\ \isacommand{using}\isamarkupfalse%
\ bounded\ \isacommand{by}\isamarkupfalse%
\ fastforce\isanewline
\ \ \ \ \isacommand{obtain}\isamarkupfalse%
\ N\ \isakeyword{where}\ N{\isacharcolon}{\kern0pt}\ {\isachardoublequoteopen}g\ z\ {\isasymge}\ C\ {\isacharplus}{\kern0pt}\ {\isadigit{1}}{\isachardoublequoteclose}\ \isakeyword{if}\ {\isachardoublequoteopen}z\ {\isasymge}\ N{\isachardoublequoteclose}\ \isakeyword{for}\ z\ \isacommand{using}\isamarkupfalse%
\ C\ pos{\isacharunderscore}{\kern0pt}g\ \isacommand{unfolding}\isamarkupfalse%
\ pos{\isacharunderscore}{\kern0pt}def\ \isacommand{by}\isamarkupfalse%
\ fastforce\isanewline
\ \ \ \ \isacommand{obtain}\isamarkupfalse%
\ N{\isacharprime}{\kern0pt}\ \isakeyword{where}\ N{\isacharprime}{\kern0pt}{\isacharcolon}{\kern0pt}\ {\isachardoublequoteopen}g\ z\ {\isasymle}\ {\isacharminus}{\kern0pt}\ {\isacharparenleft}{\kern0pt}C\ {\isacharplus}{\kern0pt}\ {\isadigit{1}}{\isacharparenright}{\kern0pt}{\isachardoublequoteclose}\ \isakeyword{if}\ {\isachardoublequoteopen}z\ {\isasymle}\ N{\isacharprime}{\kern0pt}{\isachardoublequoteclose}\ \isakeyword{for}\ z\ \isacommand{using}\isamarkupfalse%
\ C\ pos{\isacharunderscore}{\kern0pt}g\ \isacommand{unfolding}\isamarkupfalse%
\ pos{\isacharunderscore}{\kern0pt}dual{\isacharunderscore}{\kern0pt}def{\isacharbrackleft}{\kern0pt}OF\ slope{\isacharunderscore}{\kern0pt}g{\isacharbrackright}{\kern0pt}\ \isacommand{by}\isamarkupfalse%
\ {\isacharparenleft}{\kern0pt}meson\ add{\isacharunderscore}{\kern0pt}increasing{\isadigit{2}}\ linordered{\isacharunderscore}{\kern0pt}nonzero{\isacharunderscore}{\kern0pt}semiring{\isacharunderscore}{\kern0pt}class{\isachardot}{\kern0pt}zero{\isacharunderscore}{\kern0pt}le{\isacharunderscore}{\kern0pt}one{\isacharparenright}{\kern0pt}\isanewline
\ \ \ \ \isacommand{obtain}\isamarkupfalse%
\ z\ \isakeyword{where}\ {\isachardoublequoteopen}{\isasymbar}h\ z{\isasymbar}\ {\isachargreater}{\kern0pt}\ max\ {\isasymbar}N{\isasymbar}\ {\isasymbar}N{\isacharprime}{\kern0pt}{\isasymbar}{\isachardoublequoteclose}\ \isacommand{using}\isamarkupfalse%
\ asm\ \isacommand{unfolding}\isamarkupfalse%
\ bounded{\isacharunderscore}{\kern0pt}alt{\isacharunderscore}{\kern0pt}def\ \isacommand{by}\isamarkupfalse%
\ {\isacharparenleft}{\kern0pt}meson\ leI{\isacharparenright}{\kern0pt}\isanewline
\ \ \ \ \isacommand{hence}\isamarkupfalse%
\ {\isachardoublequoteopen}h\ z\ {\isasymin}\ {\isacharbraceleft}{\kern0pt}{\isachardot}{\kern0pt}{\isachardot}{\kern0pt}N{\isacharprime}{\kern0pt}{\isacharbraceright}{\kern0pt}\ {\isasymunion}\ {\isacharbraceleft}{\kern0pt}N{\isachardot}{\kern0pt}{\isachardot}{\kern0pt}{\isacharbraceright}{\kern0pt}{\isachardoublequoteclose}\ \isacommand{by}\isamarkupfalse%
\ fastforce\isanewline
\ \ \ \ \isacommand{hence}\isamarkupfalse%
\ {\isachardoublequoteopen}g\ {\isacharparenleft}{\kern0pt}h\ z{\isacharparenright}{\kern0pt}\ {\isasymin}\ {\isacharbraceleft}{\kern0pt}{\isachardot}{\kern0pt}{\isachardot}{\kern0pt}{\isacharminus}{\kern0pt}\ {\isacharparenleft}{\kern0pt}C\ {\isacharplus}{\kern0pt}\ {\isadigit{1}}{\isacharparenright}{\kern0pt}{\isacharbraceright}{\kern0pt}\ {\isasymunion}\ {\isacharbraceleft}{\kern0pt}C\ {\isacharplus}{\kern0pt}\ {\isadigit{1}}{\isachardot}{\kern0pt}{\isachardot}{\kern0pt}{\isacharbraceright}{\kern0pt}{\isachardoublequoteclose}\ \isacommand{using}\isamarkupfalse%
\ N\ N{\isacharprime}{\kern0pt}\ \isacommand{by}\isamarkupfalse%
\ blast\isanewline
\ \ \ \ \isacommand{hence}\isamarkupfalse%
\ {\isachardoublequoteopen}{\isasymbar}g\ {\isacharparenleft}{\kern0pt}h\ z{\isacharparenright}{\kern0pt}{\isasymbar}\ {\isasymge}\ C\ {\isacharplus}{\kern0pt}\ {\isadigit{1}}{\isachardoublequoteclose}\ \isacommand{by}\isamarkupfalse%
\ fastforce\isanewline
\ \ \ \ \isacommand{thus}\isamarkupfalse%
\ False\ \isacommand{using}\isamarkupfalse%
\ C{\isacharparenleft}{\kern0pt}{\isadigit{1}}{\isacharparenright}{\kern0pt}{\isacharbrackleft}{\kern0pt}of\ z{\isacharbrackright}{\kern0pt}\ \isacommand{by}\isamarkupfalse%
\ simp\isanewline
\ \ \isacommand{qed}\isamarkupfalse%
\isanewline
\isanewline
\ \ \isacommand{define}\isamarkupfalse%
\ D\ \isakeyword{where}\ {\isachardoublequoteopen}D\ {\isacharequal}{\kern0pt}\ max\ {\isasymbar}{\isacharminus}{\kern0pt}\ {\isacharparenleft}{\kern0pt}C\ {\isacharplus}{\kern0pt}\ {\isacharparenleft}{\kern0pt}L\ {\isacharplus}{\kern0pt}\ g\ {\isadigit{1}}{\isacharparenright}{\kern0pt}\ {\isacharplus}{\kern0pt}\ {\isacharparenleft}{\kern0pt}L\ {\isacharplus}{\kern0pt}\ g\ {\isadigit{1}}{\isacharparenright}{\kern0pt}{\isacharparenright}{\kern0pt}{\isasymbar}\ {\isasymbar}C\ {\isacharplus}{\kern0pt}\ L\ {\isacharplus}{\kern0pt}\ g\ {\isadigit{1}}{\isasymbar}{\isachardoublequoteclose}\isanewline
\ \ \isacommand{{\isacharbraceleft}{\kern0pt}}\isamarkupfalse%
\isanewline
\ \ \ \ \isacommand{fix}\isamarkupfalse%
\ m\ n\ {\isacharcolon}{\kern0pt}{\isacharcolon}{\kern0pt}\ int\isanewline
\ \ \ \ \isacommand{assume}\isamarkupfalse%
\ asm{\isacharcolon}{\kern0pt}\ {\isachardoublequoteopen}m\ {\isachargreater}{\kern0pt}\ {\isadigit{0}}{\isachardoublequoteclose}\ {\isachardoublequoteopen}n\ {\isachargreater}{\kern0pt}\ {\isadigit{0}}{\isachardoublequoteclose}\isanewline
\isanewline
\ \ \ \ \isacommand{have}\isamarkupfalse%
\ {\isachardoublequoteopen}g\ {\isacharparenleft}{\kern0pt}{\isacharquery}{\kern0pt}{\isasymphi}\ g\ m{\isacharparenright}{\kern0pt}\ {\isasymge}\ m{\isachardoublequoteclose}\ \isacommand{using}\isamarkupfalse%
\ {\isasymphi}{\isacharunderscore}{\kern0pt}mem\ asm\ \isacommand{by}\isamarkupfalse%
\ simp\isanewline
\ \ \ \ \isacommand{moreover}\isamarkupfalse%
\ \isacommand{have}\isamarkupfalse%
\ {\isachardoublequoteopen}{\isacharquery}{\kern0pt}{\isasymphi}\ g\ m\ {\isachargreater}{\kern0pt}\ {\isadigit{1}}{\isachardoublequoteclose}\ \isacommand{using}\isamarkupfalse%
\ calculation\ g\ asm\ \isacommand{by}\isamarkupfalse%
\ simp\isanewline
\ \ \ \ \isacommand{moreover}\isamarkupfalse%
\ \isacommand{have}\isamarkupfalse%
\ {\isachardoublequoteopen}m\ {\isachargreater}{\kern0pt}\ g\ {\isacharparenleft}{\kern0pt}{\isacharquery}{\kern0pt}{\isasymphi}\ g\ m\ {\isacharminus}{\kern0pt}\ {\isadigit{1}}{\isacharparenright}{\kern0pt}{\isachardoublequoteclose}\ \isacommand{using}\isamarkupfalse%
\ asm\ calculation\ \isacommand{by}\isamarkupfalse%
\ {\isacharparenleft}{\kern0pt}intro\ {\isasymphi}{\isacharunderscore}{\kern0pt}bound{\isacharprime}{\kern0pt}{\isacharparenright}{\kern0pt}\ auto\isanewline
\ \ \ \ \isacommand{ultimately}\isamarkupfalse%
\ \isacommand{have}\isamarkupfalse%
\ m{\isacharcolon}{\kern0pt}\ {\isachardoublequoteopen}m\ {\isasymin}\ {\isacharbraceleft}{\kern0pt}g\ {\isacharparenleft}{\kern0pt}{\isacharquery}{\kern0pt}{\isasymphi}\ g\ m\ {\isacharminus}{\kern0pt}\ {\isadigit{1}}{\isacharparenright}{\kern0pt}{\isacharless}{\kern0pt}{\isachardot}{\kern0pt}{\isachardot}{\kern0pt}g\ {\isacharparenleft}{\kern0pt}{\isacharquery}{\kern0pt}{\isasymphi}\ g\ m{\isacharparenright}{\kern0pt}{\isacharbraceright}{\kern0pt}{\isachardoublequoteclose}\ \isacommand{by}\isamarkupfalse%
\ simp\isanewline
\isanewline
\ \ \ \ \isacommand{have}\isamarkupfalse%
\ {\isachardoublequoteopen}g\ {\isacharparenleft}{\kern0pt}{\isacharquery}{\kern0pt}{\isasymphi}\ g\ n{\isacharparenright}{\kern0pt}\ {\isasymge}\ n{\isachardoublequoteclose}\ \isacommand{using}\isamarkupfalse%
\ {\isasymphi}{\isacharunderscore}{\kern0pt}mem\ asm\ \isacommand{by}\isamarkupfalse%
\ simp\isanewline
\ \ \ \ \isacommand{moreover}\isamarkupfalse%
\ \isacommand{have}\isamarkupfalse%
\ {\isachardoublequoteopen}{\isacharquery}{\kern0pt}{\isasymphi}\ g\ n\ {\isachargreater}{\kern0pt}\ {\isadigit{1}}{\isachardoublequoteclose}\ \isacommand{using}\isamarkupfalse%
\ calculation\ g\ asm\ \isacommand{by}\isamarkupfalse%
\ simp\isanewline
\ \ \ \ \isacommand{moreover}\isamarkupfalse%
\ \isacommand{have}\isamarkupfalse%
\ {\isachardoublequoteopen}n\ {\isachargreater}{\kern0pt}\ g\ {\isacharparenleft}{\kern0pt}{\isacharquery}{\kern0pt}{\isasymphi}\ g\ n\ {\isacharminus}{\kern0pt}\ {\isadigit{1}}{\isacharparenright}{\kern0pt}{\isachardoublequoteclose}\ \isacommand{using}\isamarkupfalse%
\ asm\ calculation\ \isacommand{by}\isamarkupfalse%
\ {\isacharparenleft}{\kern0pt}intro\ {\isasymphi}{\isacharunderscore}{\kern0pt}bound{\isacharprime}{\kern0pt}{\isacharparenright}{\kern0pt}\ auto\isanewline
\ \ \ \ \isacommand{ultimately}\isamarkupfalse%
\ \isacommand{have}\isamarkupfalse%
\ n{\isacharcolon}{\kern0pt}\ {\isachardoublequoteopen}n\ {\isasymin}\ {\isacharbraceleft}{\kern0pt}g\ {\isacharparenleft}{\kern0pt}{\isacharquery}{\kern0pt}{\isasymphi}\ g\ n\ {\isacharminus}{\kern0pt}\ {\isadigit{1}}{\isacharparenright}{\kern0pt}{\isacharless}{\kern0pt}{\isachardot}{\kern0pt}{\isachardot}{\kern0pt}g\ {\isacharparenleft}{\kern0pt}{\isacharquery}{\kern0pt}{\isasymphi}\ g\ n{\isacharparenright}{\kern0pt}{\isacharbraceright}{\kern0pt}{\isachardoublequoteclose}\ \isacommand{by}\isamarkupfalse%
\ simp\isanewline
\isanewline
\ \ \ \ \isacommand{have}\isamarkupfalse%
\ {\isachardoublequoteopen}g\ {\isacharparenleft}{\kern0pt}{\isacharquery}{\kern0pt}{\isasymphi}\ g\ {\isacharparenleft}{\kern0pt}m\ {\isacharplus}{\kern0pt}\ n{\isacharparenright}{\kern0pt}{\isacharparenright}{\kern0pt}\ {\isasymge}\ m\ {\isacharplus}{\kern0pt}\ n{\isachardoublequoteclose}\ \isacommand{using}\isamarkupfalse%
\ {\isasymphi}{\isacharunderscore}{\kern0pt}mem\ asm\ \isacommand{by}\isamarkupfalse%
\ simp\isanewline
\ \ \ \ \isacommand{moreover}\isamarkupfalse%
\ \isacommand{have}\isamarkupfalse%
\ {\isachardoublequoteopen}{\isacharquery}{\kern0pt}{\isasymphi}\ g\ {\isacharparenleft}{\kern0pt}m\ {\isacharplus}{\kern0pt}\ n{\isacharparenright}{\kern0pt}\ {\isachargreater}{\kern0pt}\ {\isadigit{1}}{\isachardoublequoteclose}\ \isacommand{using}\isamarkupfalse%
\ calculation\ g\ asm\ \isacommand{by}\isamarkupfalse%
\ simp\isanewline
\ \ \ \ \isacommand{moreover}\isamarkupfalse%
\ \isacommand{have}\isamarkupfalse%
\ {\isachardoublequoteopen}{\isacharparenleft}{\kern0pt}m\ {\isacharplus}{\kern0pt}\ n{\isacharparenright}{\kern0pt}\ {\isachargreater}{\kern0pt}\ g\ {\isacharparenleft}{\kern0pt}{\isacharquery}{\kern0pt}{\isasymphi}\ g\ {\isacharparenleft}{\kern0pt}m\ {\isacharplus}{\kern0pt}\ n{\isacharparenright}{\kern0pt}\ {\isacharminus}{\kern0pt}\ {\isadigit{1}}{\isacharparenright}{\kern0pt}{\isachardoublequoteclose}\ \isacommand{using}\isamarkupfalse%
\ asm\ calculation\ \isacommand{by}\isamarkupfalse%
\ {\isacharparenleft}{\kern0pt}intro\ {\isasymphi}{\isacharunderscore}{\kern0pt}bound{\isacharprime}{\kern0pt}{\isacharparenright}{\kern0pt}\ auto\isanewline
\ \ \ \ \isacommand{ultimately}\isamarkupfalse%
\ \isacommand{have}\isamarkupfalse%
\ m{\isacharunderscore}{\kern0pt}n{\isacharcolon}{\kern0pt}\ {\isachardoublequoteopen}m\ {\isacharplus}{\kern0pt}\ n\ {\isasymin}\ {\isacharbraceleft}{\kern0pt}g\ {\isacharparenleft}{\kern0pt}{\isacharquery}{\kern0pt}{\isasymphi}\ g\ {\isacharparenleft}{\kern0pt}m\ {\isacharplus}{\kern0pt}\ n{\isacharparenright}{\kern0pt}\ {\isacharminus}{\kern0pt}\ {\isadigit{1}}{\isacharparenright}{\kern0pt}{\isacharless}{\kern0pt}{\isachardot}{\kern0pt}{\isachardot}{\kern0pt}g\ {\isacharparenleft}{\kern0pt}{\isacharquery}{\kern0pt}{\isasymphi}\ g\ {\isacharparenleft}{\kern0pt}m\ {\isacharplus}{\kern0pt}\ n{\isacharparenright}{\kern0pt}{\isacharparenright}{\kern0pt}{\isacharbraceright}{\kern0pt}{\isachardoublequoteclose}\ \isacommand{by}\isamarkupfalse%
\ simp\isanewline
\isanewline
\ \ \ \ \isacommand{have}\isamarkupfalse%
\ {\isacharasterisk}{\kern0pt}{\isacharcolon}{\kern0pt}\ {\isachardoublequoteopen}g\ {\isacharparenleft}{\kern0pt}{\isacharquery}{\kern0pt}{\isasymphi}\ g\ {\isacharparenleft}{\kern0pt}m\ {\isacharplus}{\kern0pt}\ n{\isacharparenright}{\kern0pt}{\isacharparenright}{\kern0pt}\ {\isacharminus}{\kern0pt}\ {\isacharparenleft}{\kern0pt}g\ {\isacharparenleft}{\kern0pt}{\isacharquery}{\kern0pt}{\isasymphi}\ g\ m\ {\isacharminus}{\kern0pt}\ {\isadigit{1}}{\isacharparenright}{\kern0pt}\ {\isacharplus}{\kern0pt}\ g\ {\isacharparenleft}{\kern0pt}{\isacharquery}{\kern0pt}{\isasymphi}\ g\ n\ {\isacharminus}{\kern0pt}\ {\isadigit{1}}{\isacharparenright}{\kern0pt}{\isacharparenright}{\kern0pt}\ {\isachargreater}{\kern0pt}\ {\isadigit{0}}{\isachardoublequoteclose}\ {\isachardoublequoteopen}g\ {\isacharparenleft}{\kern0pt}{\isacharquery}{\kern0pt}{\isasymphi}\ g\ {\isacharparenleft}{\kern0pt}m\ {\isacharplus}{\kern0pt}\ n{\isacharparenright}{\kern0pt}\ {\isacharminus}{\kern0pt}\ {\isadigit{1}}{\isacharparenright}{\kern0pt}\ {\isacharminus}{\kern0pt}\ {\isacharparenleft}{\kern0pt}g\ {\isacharparenleft}{\kern0pt}{\isacharquery}{\kern0pt}{\isasymphi}\ g\ m{\isacharparenright}{\kern0pt}\ {\isacharplus}{\kern0pt}\ g\ {\isacharparenleft}{\kern0pt}{\isacharquery}{\kern0pt}{\isasymphi}\ g\ n{\isacharparenright}{\kern0pt}{\isacharparenright}{\kern0pt}\ {\isacharless}{\kern0pt}\ {\isadigit{0}}{\isachardoublequoteclose}\ \isacommand{using}\isamarkupfalse%
\ m{\isacharunderscore}{\kern0pt}n\ m\ n\ \isacommand{by}\isamarkupfalse%
\ simp{\isacharplus}{\kern0pt}\isanewline
\ \ \ \ \isanewline
\ \ \ \ \isacommand{have}\isamarkupfalse%
\ {\isachardoublequoteopen}g\ {\isacharparenleft}{\kern0pt}{\isacharquery}{\kern0pt}{\isasymphi}\ g\ {\isacharparenleft}{\kern0pt}m\ {\isacharplus}{\kern0pt}\ n{\isacharparenright}{\kern0pt}\ {\isacharminus}{\kern0pt}\ {\isacharparenleft}{\kern0pt}{\isacharquery}{\kern0pt}{\isasymphi}\ g\ m\ {\isacharplus}{\kern0pt}\ {\isacharquery}{\kern0pt}{\isasymphi}\ g\ n{\isacharparenright}{\kern0pt}{\isacharparenright}{\kern0pt}\ {\isasymle}\ g\ {\isacharparenleft}{\kern0pt}{\isacharquery}{\kern0pt}{\isasymphi}\ g\ {\isacharparenleft}{\kern0pt}m\ {\isacharplus}{\kern0pt}\ n{\isacharparenright}{\kern0pt}{\isacharparenright}{\kern0pt}\ {\isacharminus}{\kern0pt}\ {\isacharparenleft}{\kern0pt}g\ {\isacharparenleft}{\kern0pt}{\isacharquery}{\kern0pt}{\isasymphi}\ g\ m{\isacharparenright}{\kern0pt}\ {\isacharplus}{\kern0pt}\ g\ {\isacharparenleft}{\kern0pt}{\isacharquery}{\kern0pt}{\isasymphi}\ g\ n{\isacharparenright}{\kern0pt}{\isacharparenright}{\kern0pt}\ {\isacharplus}{\kern0pt}\ C{\isachardoublequoteclose}\ \isacommand{using}\isamarkupfalse%
\ C\ \isacommand{by}\isamarkupfalse%
\ blast\isanewline
\ \ \ \ \isacommand{also}\isamarkupfalse%
\ \isacommand{have}\isamarkupfalse%
\ {\isachardoublequoteopen}{\isachardot}{\kern0pt}{\isachardot}{\kern0pt}{\isachardot}{\kern0pt}\ {\isasymle}\ g\ {\isacharparenleft}{\kern0pt}{\isacharquery}{\kern0pt}{\isasymphi}\ g\ {\isacharparenleft}{\kern0pt}m\ {\isacharplus}{\kern0pt}\ n{\isacharparenright}{\kern0pt}\ {\isacharminus}{\kern0pt}\ {\isadigit{1}}{\isacharparenright}{\kern0pt}\ {\isacharminus}{\kern0pt}\ g\ {\isacharparenleft}{\kern0pt}{\isacharquery}{\kern0pt}{\isasymphi}\ g\ m{\isacharparenright}{\kern0pt}\ {\isacharminus}{\kern0pt}\ g\ {\isacharparenleft}{\kern0pt}{\isacharquery}{\kern0pt}{\isasymphi}\ g\ n{\isacharparenright}{\kern0pt}\ {\isacharplus}{\kern0pt}\ {\isacharparenleft}{\kern0pt}C\ {\isacharplus}{\kern0pt}\ L\ {\isacharplus}{\kern0pt}\ g\ {\isadigit{1}}{\isacharparenright}{\kern0pt}{\isachardoublequoteclose}\ \isacommand{using}\isamarkupfalse%
\ L\ \isacommand{by}\isamarkupfalse%
\ fastforce\isanewline
\ \ \ \ \isacommand{finally}\isamarkupfalse%
\ \isacommand{have}\isamarkupfalse%
\ upper{\isacharcolon}{\kern0pt}\ {\isachardoublequoteopen}g\ {\isacharparenleft}{\kern0pt}{\isacharquery}{\kern0pt}{\isasymphi}\ g\ {\isacharparenleft}{\kern0pt}m\ {\isacharplus}{\kern0pt}\ n{\isacharparenright}{\kern0pt}\ {\isacharminus}{\kern0pt}\ {\isacharparenleft}{\kern0pt}{\isacharquery}{\kern0pt}{\isasymphi}\ g\ m\ {\isacharplus}{\kern0pt}\ {\isacharquery}{\kern0pt}{\isasymphi}\ g\ n{\isacharparenright}{\kern0pt}{\isacharparenright}{\kern0pt}\ {\isasymle}\ C\ {\isacharplus}{\kern0pt}\ L\ {\isacharplus}{\kern0pt}\ g\ {\isadigit{1}}{\isachardoublequoteclose}\ \isacommand{using}\isamarkupfalse%
\ {\isacharasterisk}{\kern0pt}\ \isacommand{by}\isamarkupfalse%
\ fastforce\isanewline
\isanewline
\ \ \ \ \isacommand{have}\isamarkupfalse%
\ {\isachardoublequoteopen}{\isacharminus}{\kern0pt}\ {\isacharparenleft}{\kern0pt}C\ {\isacharplus}{\kern0pt}\ {\isacharparenleft}{\kern0pt}L\ {\isacharplus}{\kern0pt}\ g\ {\isadigit{1}}{\isacharparenright}{\kern0pt}\ {\isacharplus}{\kern0pt}\ {\isacharparenleft}{\kern0pt}L\ {\isacharplus}{\kern0pt}\ g\ {\isadigit{1}}{\isacharparenright}{\kern0pt}{\isacharparenright}{\kern0pt}\ {\isasymle}\ g\ {\isacharparenleft}{\kern0pt}{\isacharquery}{\kern0pt}{\isasymphi}\ g\ {\isacharparenleft}{\kern0pt}m\ {\isacharplus}{\kern0pt}\ n{\isacharparenright}{\kern0pt}{\isacharparenright}{\kern0pt}\ {\isacharminus}{\kern0pt}\ g\ {\isacharparenleft}{\kern0pt}{\isacharquery}{\kern0pt}{\isasymphi}\ g\ m\ {\isacharminus}{\kern0pt}\ {\isadigit{1}}{\isacharparenright}{\kern0pt}\ {\isacharminus}{\kern0pt}\ g\ {\isacharparenleft}{\kern0pt}{\isacharquery}{\kern0pt}{\isasymphi}\ g\ n\ {\isacharminus}{\kern0pt}\ {\isadigit{1}}{\isacharparenright}{\kern0pt}\ {\isacharminus}{\kern0pt}\ {\isacharparenleft}{\kern0pt}C\ {\isacharplus}{\kern0pt}\ {\isacharparenleft}{\kern0pt}L\ {\isacharplus}{\kern0pt}\ g\ {\isadigit{1}}{\isacharparenright}{\kern0pt}\ {\isacharplus}{\kern0pt}\ {\isacharparenleft}{\kern0pt}L\ {\isacharplus}{\kern0pt}\ g\ {\isadigit{1}}{\isacharparenright}{\kern0pt}{\isacharparenright}{\kern0pt}{\isachardoublequoteclose}\ \isacommand{using}\isamarkupfalse%
\ {\isacharasterisk}{\kern0pt}\ \isacommand{by}\isamarkupfalse%
\ linarith\isanewline
\ \ \ \ \isacommand{also}\isamarkupfalse%
\ \isacommand{have}\isamarkupfalse%
\ {\isachardoublequoteopen}{\isachardot}{\kern0pt}{\isachardot}{\kern0pt}{\isachardot}{\kern0pt}\ {\isasymle}\ g\ {\isacharparenleft}{\kern0pt}{\isacharquery}{\kern0pt}{\isasymphi}\ g\ {\isacharparenleft}{\kern0pt}m\ {\isacharplus}{\kern0pt}\ n{\isacharparenright}{\kern0pt}{\isacharparenright}{\kern0pt}\ {\isacharminus}{\kern0pt}\ {\isacharparenleft}{\kern0pt}g\ {\isacharparenleft}{\kern0pt}{\isacharquery}{\kern0pt}{\isasymphi}\ g\ m{\isacharparenright}{\kern0pt}\ {\isacharplus}{\kern0pt}\ g\ {\isacharparenleft}{\kern0pt}{\isacharquery}{\kern0pt}{\isasymphi}\ g\ n{\isacharparenright}{\kern0pt}{\isacharparenright}{\kern0pt}\ {\isacharplus}{\kern0pt}\ {\isacharparenleft}{\kern0pt}{\isacharminus}{\kern0pt}\ C{\isacharparenright}{\kern0pt}{\isachardoublequoteclose}\ \isacommand{using}\isamarkupfalse%
\ L{\isacharbrackleft}{\kern0pt}THEN\ le{\isacharunderscore}{\kern0pt}imp{\isacharunderscore}{\kern0pt}neg{\isacharunderscore}{\kern0pt}le{\isacharcomma}{\kern0pt}\ of\ {\isachardoublequoteopen}{\isacharquery}{\kern0pt}{\isasymphi}\ g\ m{\isachardoublequoteclose}{\isacharbrackright}{\kern0pt}\ L{\isacharbrackleft}{\kern0pt}THEN\ le{\isacharunderscore}{\kern0pt}imp{\isacharunderscore}{\kern0pt}neg{\isacharunderscore}{\kern0pt}le{\isacharcomma}{\kern0pt}\ of\ {\isachardoublequoteopen}{\isacharquery}{\kern0pt}{\isasymphi}\ g\ n{\isachardoublequoteclose}{\isacharbrackright}{\kern0pt}\ \isacommand{by}\isamarkupfalse%
\ linarith\isanewline
\ \ \ \ \isacommand{also}\isamarkupfalse%
\ \isacommand{have}\isamarkupfalse%
\ {\isachardoublequoteopen}{\isachardot}{\kern0pt}{\isachardot}{\kern0pt}{\isachardot}{\kern0pt}\ {\isasymle}\ g\ {\isacharparenleft}{\kern0pt}{\isacharquery}{\kern0pt}{\isasymphi}\ g\ {\isacharparenleft}{\kern0pt}m\ {\isacharplus}{\kern0pt}\ n{\isacharparenright}{\kern0pt}\ {\isacharminus}{\kern0pt}\ {\isacharparenleft}{\kern0pt}{\isacharquery}{\kern0pt}{\isasymphi}\ g\ m\ {\isacharplus}{\kern0pt}\ {\isacharquery}{\kern0pt}{\isasymphi}\ g\ n{\isacharparenright}{\kern0pt}{\isacharparenright}{\kern0pt}{\isachardoublequoteclose}\ \isacommand{using}\isamarkupfalse%
\ C\ \isacommand{by}\isamarkupfalse%
\ blast\isanewline
\ \ \ \ \isacommand{finally}\isamarkupfalse%
\ \isacommand{have}\isamarkupfalse%
\ lower{\isacharcolon}{\kern0pt}\ {\isachardoublequoteopen}{\isacharminus}{\kern0pt}\ {\isacharparenleft}{\kern0pt}C\ {\isacharplus}{\kern0pt}\ {\isacharparenleft}{\kern0pt}L\ {\isacharplus}{\kern0pt}\ g\ {\isadigit{1}}{\isacharparenright}{\kern0pt}\ {\isacharplus}{\kern0pt}\ {\isacharparenleft}{\kern0pt}L\ {\isacharplus}{\kern0pt}\ g\ {\isadigit{1}}{\isacharparenright}{\kern0pt}{\isacharparenright}{\kern0pt}\ {\isasymle}\ g\ {\isacharparenleft}{\kern0pt}{\isacharquery}{\kern0pt}{\isasymphi}\ g\ {\isacharparenleft}{\kern0pt}m\ {\isacharplus}{\kern0pt}\ n{\isacharparenright}{\kern0pt}\ {\isacharminus}{\kern0pt}\ {\isacharparenleft}{\kern0pt}{\isacharquery}{\kern0pt}{\isasymphi}\ g\ m\ {\isacharplus}{\kern0pt}\ {\isacharquery}{\kern0pt}{\isasymphi}\ g\ n{\isacharparenright}{\kern0pt}{\isacharparenright}{\kern0pt}{\isachardoublequoteclose}\ \isacommand{{\isachardot}{\kern0pt}}\isamarkupfalse%
\isanewline
\isanewline
\ \ \ \ \isacommand{have}\isamarkupfalse%
\ {\isachardoublequoteopen}{\isasymbar}g\ {\isacharparenleft}{\kern0pt}{\isacharquery}{\kern0pt}{\isasymphi}\ g\ {\isacharparenleft}{\kern0pt}m\ {\isacharplus}{\kern0pt}\ n{\isacharparenright}{\kern0pt}\ {\isacharminus}{\kern0pt}\ {\isacharparenleft}{\kern0pt}{\isacharquery}{\kern0pt}{\isasymphi}\ g\ m\ {\isacharplus}{\kern0pt}\ {\isacharquery}{\kern0pt}{\isasymphi}\ g\ n{\isacharparenright}{\kern0pt}{\isacharparenright}{\kern0pt}{\isasymbar}\ {\isasymle}\ D{\isachardoublequoteclose}\ \isacommand{using}\isamarkupfalse%
\ upper\ lower\ \isacommand{unfolding}\isamarkupfalse%
\ D{\isacharunderscore}{\kern0pt}def\ \isacommand{by}\isamarkupfalse%
\ simp\isanewline
\ \ \isacommand{{\isacharbraceright}{\kern0pt}}\isamarkupfalse%
\isanewline
\ \ \isacommand{hence}\isamarkupfalse%
\ {\isachardoublequoteopen}bounded\ {\isacharparenleft}{\kern0pt}g\ o\ {\isacharparenleft}{\kern0pt}{\isasymlambda}{\isacharparenleft}{\kern0pt}m{\isacharcomma}{\kern0pt}\ n{\isacharparenright}{\kern0pt}{\isachardot}{\kern0pt}\ {\isacharquery}{\kern0pt}{\isasymphi}\ g\ {\isacharparenleft}{\kern0pt}m\ {\isacharplus}{\kern0pt}\ n{\isacharparenright}{\kern0pt}\ {\isacharminus}{\kern0pt}\ {\isacharparenleft}{\kern0pt}{\isacharquery}{\kern0pt}{\isasymphi}\ g\ m\ {\isacharplus}{\kern0pt}\ {\isacharquery}{\kern0pt}{\isasymphi}\ g\ n{\isacharparenright}{\kern0pt}{\isacharparenright}{\kern0pt}\ o\ {\isacharparenleft}{\kern0pt}{\isasymlambda}{\isacharparenleft}{\kern0pt}m{\isacharcomma}{\kern0pt}\ n{\isacharparenright}{\kern0pt}{\isachardot}{\kern0pt}\ {\isacharparenleft}{\kern0pt}max\ {\isadigit{1}}\ m{\isacharcomma}{\kern0pt}\ max\ {\isadigit{1}}\ n{\isacharparenright}{\kern0pt}{\isacharparenright}{\kern0pt}{\isacharparenright}{\kern0pt}{\isachardoublequoteclose}\ \isacommand{by}\isamarkupfalse%
\ {\isacharparenleft}{\kern0pt}intro\ boundedI{\isacharbrackleft}{\kern0pt}of\ {\isacharunderscore}{\kern0pt}\ D{\isacharbrackright}{\kern0pt}{\isacharparenright}{\kern0pt}\ auto\isanewline
\ \ \isacommand{hence}\isamarkupfalse%
\ {\isachardoublequoteopen}bounded\ {\isacharparenleft}{\kern0pt}{\isacharparenleft}{\kern0pt}{\isasymlambda}{\isacharparenleft}{\kern0pt}m{\isacharcomma}{\kern0pt}\ n{\isacharparenright}{\kern0pt}{\isachardot}{\kern0pt}\ {\isacharquery}{\kern0pt}{\isasymphi}\ g\ {\isacharparenleft}{\kern0pt}m\ {\isacharplus}{\kern0pt}\ n{\isacharparenright}{\kern0pt}\ {\isacharminus}{\kern0pt}\ {\isacharparenleft}{\kern0pt}{\isacharquery}{\kern0pt}{\isasymphi}\ g\ m\ {\isacharplus}{\kern0pt}\ {\isacharquery}{\kern0pt}{\isasymphi}\ g\ n{\isacharparenright}{\kern0pt}{\isacharparenright}{\kern0pt}\ o\ {\isacharparenleft}{\kern0pt}{\isasymlambda}{\isacharparenleft}{\kern0pt}m{\isacharcomma}{\kern0pt}\ n{\isacharparenright}{\kern0pt}{\isachardot}{\kern0pt}\ {\isacharparenleft}{\kern0pt}max\ {\isadigit{1}}\ m{\isacharcomma}{\kern0pt}\ max\ {\isadigit{1}}\ n{\isacharparenright}{\kern0pt}{\isacharparenright}{\kern0pt}{\isacharparenright}{\kern0pt}{\isachardoublequoteclose}\ \isacommand{by}\isamarkupfalse%
\ {\isacharparenleft}{\kern0pt}metis\ {\isacharparenleft}{\kern0pt}mono{\isacharunderscore}{\kern0pt}tags{\isacharcomma}{\kern0pt}\ lifting{\isacharparenright}{\kern0pt}\ bounded\ comp{\isacharunderscore}{\kern0pt}assoc{\isacharparenright}{\kern0pt}\isanewline
\ \ \isacommand{then}\isamarkupfalse%
\ \isacommand{obtain}\isamarkupfalse%
\ C\ \isakeyword{where}\ {\isachardoublequoteopen}{\isasymbar}{\isacharparenleft}{\kern0pt}{\isacharparenleft}{\kern0pt}{\isasymlambda}{\isacharparenleft}{\kern0pt}m{\isacharcomma}{\kern0pt}\ n{\isacharparenright}{\kern0pt}{\isachardot}{\kern0pt}\ {\isacharquery}{\kern0pt}{\isasymphi}\ g\ {\isacharparenleft}{\kern0pt}m\ {\isacharplus}{\kern0pt}\ n{\isacharparenright}{\kern0pt}\ {\isacharminus}{\kern0pt}\ {\isacharparenleft}{\kern0pt}{\isacharquery}{\kern0pt}{\isasymphi}\ g\ m\ {\isacharplus}{\kern0pt}\ {\isacharquery}{\kern0pt}{\isasymphi}\ g\ n{\isacharparenright}{\kern0pt}{\isacharparenright}{\kern0pt}\ o\ {\isacharparenleft}{\kern0pt}{\isasymlambda}{\isacharparenleft}{\kern0pt}m{\isacharcomma}{\kern0pt}\ n{\isacharparenright}{\kern0pt}{\isachardot}{\kern0pt}\ {\isacharparenleft}{\kern0pt}max\ {\isadigit{1}}\ m{\isacharcomma}{\kern0pt}\ max\ {\isadigit{1}}\ n{\isacharparenright}{\kern0pt}{\isacharparenright}{\kern0pt}{\isacharparenright}{\kern0pt}\ {\isacharparenleft}{\kern0pt}m{\isacharcomma}{\kern0pt}\ n{\isacharparenright}{\kern0pt}{\isasymbar}\ {\isasymle}\ C{\isachardoublequoteclose}\ \isakeyword{for}\ m\ n\ \isacommand{by}\isamarkupfalse%
\ blast\isanewline
\ \ \isacommand{hence}\isamarkupfalse%
\ {\isachardoublequoteopen}{\isasymbar}{\isacharquery}{\kern0pt}{\isasymphi}\ g\ {\isacharparenleft}{\kern0pt}m\ {\isacharplus}{\kern0pt}\ n{\isacharparenright}{\kern0pt}\ {\isacharminus}{\kern0pt}\ {\isacharparenleft}{\kern0pt}{\isacharquery}{\kern0pt}{\isasymphi}\ g\ m\ {\isacharplus}{\kern0pt}\ {\isacharquery}{\kern0pt}{\isasymphi}\ g\ n{\isacharparenright}{\kern0pt}{\isasymbar}\ {\isasymle}\ C{\isachardoublequoteclose}\ \isakeyword{if}\ {\isachardoublequoteopen}m\ {\isasymge}\ {\isadigit{1}}{\isachardoublequoteclose}\ {\isachardoublequoteopen}n\ {\isasymge}\ {\isadigit{1}}{\isachardoublequoteclose}\ \isakeyword{for}\ m\ n\ \isacommand{using}\isamarkupfalse%
\ that{\isacharbrackleft}{\kern0pt}THEN\ max{\isacharunderscore}{\kern0pt}absorb{\isadigit{2}}{\isacharbrackright}{\kern0pt}\ \isacommand{by}\isamarkupfalse%
\ {\isacharparenleft}{\kern0pt}metis\ {\isacharparenleft}{\kern0pt}no{\isacharunderscore}{\kern0pt}types{\isacharcomma}{\kern0pt}\ lifting{\isacharparenright}{\kern0pt}\ comp{\isacharunderscore}{\kern0pt}apply\ prod{\isachardot}{\kern0pt}case{\isacharparenright}{\kern0pt}\isanewline
\ \ \isacommand{hence}\isamarkupfalse%
\ slope{\isacharcolon}{\kern0pt}\ {\isachardoublequoteopen}slope\ {\isacharparenleft}{\kern0pt}{\isacharquery}{\kern0pt}{\isasymphi}\ g{\isacharparenright}{\kern0pt}{\isachardoublequoteclose}\ \isacommand{by}\isamarkupfalse%
\ {\isacharparenleft}{\kern0pt}intro\ slope{\isacharunderscore}{\kern0pt}odd{\isacharbrackleft}{\kern0pt}of\ {\isacharunderscore}{\kern0pt}\ C{\isacharbrackright}{\kern0pt}{\isacharparenright}{\kern0pt}\ {\isacharparenleft}{\kern0pt}auto\ simp\ add{\isacharcolon}{\kern0pt}\ eudoxus{\isacharunderscore}{\kern0pt}pos{\isacharunderscore}{\kern0pt}inverse{\isacharunderscore}{\kern0pt}def{\isacharparenright}{\kern0pt}\isanewline
\ \ \isacommand{moreover}\isamarkupfalse%
\ \isanewline
\ \ \isacommand{{\isacharbraceleft}{\kern0pt}}\isamarkupfalse%
\isanewline
\ \ \ \ \isacommand{obtain}\isamarkupfalse%
\ C\ \isakeyword{where}\ C{\isacharcolon}{\kern0pt}\ {\isachardoublequoteopen}{\isasymbar}g\ {\isacharparenleft}{\kern0pt}{\isacharparenleft}{\kern0pt}{\isacharquery}{\kern0pt}{\isasymphi}\ g\ n\ {\isacharminus}{\kern0pt}\ {\isadigit{1}}{\isacharparenright}{\kern0pt}\ {\isacharplus}{\kern0pt}\ {\isadigit{1}}{\isacharparenright}{\kern0pt}\ {\isacharminus}{\kern0pt}\ {\isacharparenleft}{\kern0pt}g\ {\isacharparenleft}{\kern0pt}{\isacharquery}{\kern0pt}{\isasymphi}\ g\ n\ {\isacharminus}{\kern0pt}\ {\isadigit{1}}{\isacharparenright}{\kern0pt}\ {\isacharplus}{\kern0pt}\ g\ {\isadigit{1}}{\isacharparenright}{\kern0pt}{\isasymbar}\ {\isasymle}\ C{\isachardoublequoteclose}\ \isakeyword{for}\ n\ \isacommand{using}\isamarkupfalse%
\ slope{\isacharunderscore}{\kern0pt}g\ \isacommand{by}\isamarkupfalse%
\ fast\isanewline
\ \ \ \ \isacommand{have}\isamarkupfalse%
\ C{\isacharunderscore}{\kern0pt}bound{\isacharcolon}{\kern0pt}\ {\isachardoublequoteopen}g\ {\isacharparenleft}{\kern0pt}{\isacharquery}{\kern0pt}{\isasymphi}\ g\ n\ {\isacharminus}{\kern0pt}\ {\isadigit{1}}{\isacharparenright}{\kern0pt}\ {\isasymge}\ g\ {\isacharparenleft}{\kern0pt}{\isacharquery}{\kern0pt}{\isasymphi}\ g\ n{\isacharparenright}{\kern0pt}\ {\isacharminus}{\kern0pt}\ {\isacharparenleft}{\kern0pt}{\isasymbar}g\ {\isadigit{1}}{\isasymbar}\ {\isacharplus}{\kern0pt}\ C{\isacharparenright}{\kern0pt}{\isachardoublequoteclose}\ \isakeyword{for}\ n\ \isacommand{using}\isamarkupfalse%
\ C{\isacharbrackleft}{\kern0pt}of\ n{\isacharbrackright}{\kern0pt}\ \isacommand{by}\isamarkupfalse%
\ fastforce\isanewline
\ \ \ \ \isacommand{{\isacharbraceleft}{\kern0pt}}\isamarkupfalse%
\isanewline
\ \ \ \ \ \ \isacommand{fix}\isamarkupfalse%
\ n\ {\isacharcolon}{\kern0pt}{\isacharcolon}{\kern0pt}\ int\isanewline
\ \ \ \ \ \ \isacommand{assume}\isamarkupfalse%
\ asm{\isacharcolon}{\kern0pt}\ {\isachardoublequoteopen}n\ {\isachargreater}{\kern0pt}\ {\isadigit{0}}{\isachardoublequoteclose}\isanewline
\ \ \ \ \ \ \isacommand{have}\isamarkupfalse%
\ upper{\isacharcolon}{\kern0pt}\ {\isachardoublequoteopen}g\ {\isacharparenleft}{\kern0pt}{\isacharquery}{\kern0pt}{\isasymphi}\ g\ n{\isacharparenright}{\kern0pt}\ {\isasymge}\ n{\isachardoublequoteclose}\ \isacommand{using}\isamarkupfalse%
\ {\isasymphi}{\isacharunderscore}{\kern0pt}mem\ asm\ \isacommand{by}\isamarkupfalse%
\ simp\isanewline
\ \ \ \ \ \ \isacommand{moreover}\isamarkupfalse%
\ \isacommand{have}\isamarkupfalse%
\ {\isachardoublequoteopen}{\isacharquery}{\kern0pt}{\isasymphi}\ g\ n\ {\isachargreater}{\kern0pt}\ {\isadigit{1}}{\isachardoublequoteclose}\ \isacommand{using}\isamarkupfalse%
\ calculation\ g\ asm\ \isacommand{by}\isamarkupfalse%
\ simp\isanewline
\ \ \ \ \ \ \isacommand{moreover}\isamarkupfalse%
\ \isacommand{have}\isamarkupfalse%
\ {\isachardoublequoteopen}n\ {\isachargreater}{\kern0pt}\ g\ {\isacharparenleft}{\kern0pt}{\isacharquery}{\kern0pt}{\isasymphi}\ g\ n\ {\isacharminus}{\kern0pt}\ {\isadigit{1}}{\isacharparenright}{\kern0pt}{\isachardoublequoteclose}\ \isacommand{using}\isamarkupfalse%
\ calculation\ asm\ \isacommand{by}\isamarkupfalse%
\ {\isacharparenleft}{\kern0pt}intro\ {\isasymphi}{\isacharunderscore}{\kern0pt}bound{\isacharprime}{\kern0pt}{\isacharparenright}{\kern0pt}\ auto\isanewline
\ \ \ \ \ \ \isacommand{moreover}\isamarkupfalse%
\ \isacommand{have}\isamarkupfalse%
\ {\isachardoublequoteopen}n\ {\isasymge}\ g\ {\isacharparenleft}{\kern0pt}{\isacharquery}{\kern0pt}{\isasymphi}\ g\ n{\isacharparenright}{\kern0pt}\ {\isacharminus}{\kern0pt}\ {\isacharparenleft}{\kern0pt}{\isasymbar}g\ {\isadigit{1}}{\isasymbar}\ {\isacharplus}{\kern0pt}\ C{\isacharparenright}{\kern0pt}{\isachardoublequoteclose}\ \isacommand{using}\isamarkupfalse%
\ calculation\ C{\isacharunderscore}{\kern0pt}bound{\isacharbrackleft}{\kern0pt}of\ n{\isacharbrackright}{\kern0pt}\ \isacommand{by}\isamarkupfalse%
\ force\isanewline
\ \ \ \ \ \ \isacommand{ultimately}\isamarkupfalse%
\ \isacommand{have}\isamarkupfalse%
\ {\isachardoublequoteopen}{\isasymbar}g\ {\isacharparenleft}{\kern0pt}{\isacharquery}{\kern0pt}{\isasymphi}\ g\ n{\isacharparenright}{\kern0pt}\ {\isacharminus}{\kern0pt}\ n{\isasymbar}\ {\isasymle}\ {\isasymbar}g\ {\isadigit{1}}{\isasymbar}\ {\isacharplus}{\kern0pt}\ C{\isachardoublequoteclose}\ \isacommand{by}\isamarkupfalse%
\ simp\isanewline
\ \ \ \ \isacommand{{\isacharbraceright}{\kern0pt}}\isamarkupfalse%
\isanewline
\ \ \ \ \isacommand{hence}\isamarkupfalse%
\ id{\isacharcolon}{\kern0pt}\ {\isachardoublequoteopen}g\ {\isacharasterisk}{\kern0pt}\isactrlsub e\ {\isacharquery}{\kern0pt}{\isasymphi}\ g\ {\isasymsim}\isactrlsub e\ id{\isachardoublequoteclose}\ \isacommand{using}\isamarkupfalse%
\ slope{\isacharunderscore}{\kern0pt}g\ slope\ \isacommand{by}\isamarkupfalse%
\ {\isacharparenleft}{\kern0pt}intro\ eudoxus{\isacharunderscore}{\kern0pt}relI{\isacharbrackleft}{\kern0pt}of\ {\isacharunderscore}{\kern0pt}\ {\isacharunderscore}{\kern0pt}\ {\isadigit{1}}\ {\isachardoublequoteopen}{\isasymbar}g\ {\isadigit{1}}{\isasymbar}\ {\isacharplus}{\kern0pt}\ C{\isachardoublequoteclose}{\isacharbrackright}{\kern0pt}{\isacharparenright}{\kern0pt}\ {\isacharparenleft}{\kern0pt}auto\ simp\ add{\isacharcolon}{\kern0pt}\ eudoxus{\isacharunderscore}{\kern0pt}times{\isacharunderscore}{\kern0pt}def{\isacharparenright}{\kern0pt}\isanewline
\ \ \isacommand{{\isacharbraceright}{\kern0pt}}\isamarkupfalse%
\isanewline
\ \ \isacommand{ultimately}\isamarkupfalse%
\ \isacommand{show}\isamarkupfalse%
\ {\isacharquery}{\kern0pt}thesis\ \isacommand{using}\isamarkupfalse%
\ g\ that\ eudoxus{\isacharunderscore}{\kern0pt}rel{\isacharunderscore}{\kern0pt}trans\ eudoxus{\isacharunderscore}{\kern0pt}times{\isacharunderscore}{\kern0pt}cong\ slope{\isacharunderscore}{\kern0pt}reflI\ eudoxus{\isacharunderscore}{\kern0pt}times{\isacharunderscore}{\kern0pt}commute{\isacharbrackleft}{\kern0pt}OF\ slope\ slope{\isacharunderscore}{\kern0pt}g{\isacharbrackright}{\kern0pt}\ \isacommand{by}\isamarkupfalse%
\ metis\isanewline
\isacommand{qed}\isamarkupfalse%
%
\endisatagproof
{\isafoldproof}%
%
\isadelimproof
\ \ \isanewline
%
\endisadelimproof
\isanewline
\isacommand{definition}\isamarkupfalse%
\ eudoxus{\isacharunderscore}{\kern0pt}inverse\ {\isacharcolon}{\kern0pt}{\isacharcolon}{\kern0pt}\ {\isachardoublequoteopen}{\isacharparenleft}{\kern0pt}int\ {\isasymRightarrow}\ int{\isacharparenright}{\kern0pt}\ {\isasymRightarrow}\ {\isacharparenleft}{\kern0pt}int\ {\isasymRightarrow}\ int{\isacharparenright}{\kern0pt}{\isachardoublequoteclose}\ \isakeyword{where}\isanewline
\ \ {\isachardoublequoteopen}eudoxus{\isacharunderscore}{\kern0pt}inverse\ f\ {\isacharequal}{\kern0pt}\ {\isacharparenleft}{\kern0pt}if\ {\isasymnot}\ bounded\ f\ then\ SOME\ g{\isachardot}{\kern0pt}\ slope\ g\ {\isasymand}\ {\isacharparenleft}{\kern0pt}g\ {\isacharasterisk}{\kern0pt}\isactrlsub e\ f{\isacharparenright}{\kern0pt}\ {\isasymsim}\isactrlsub e\ id\ else\ {\isacharparenleft}{\kern0pt}{\isasymlambda}{\isacharunderscore}{\kern0pt}{\isachardot}{\kern0pt}\ {\isadigit{0}}{\isacharparenright}{\kern0pt}{\isacharparenright}{\kern0pt}{\isachardoublequoteclose}\isanewline
\isanewline
\isacommand{lemma}\isamarkupfalse%
\ \isanewline
\ \ \isakeyword{assumes}\ {\isachardoublequoteopen}slope\ f{\isachardoublequoteclose}\isanewline
\ \ \isakeyword{shows}\ slope{\isacharunderscore}{\kern0pt}eudoxus{\isacharunderscore}{\kern0pt}inverse{\isacharcolon}{\kern0pt}\ {\isachardoublequoteopen}slope\ {\isacharparenleft}{\kern0pt}eudoxus{\isacharunderscore}{\kern0pt}inverse\ f{\isacharparenright}{\kern0pt}{\isachardoublequoteclose}\ {\isacharparenleft}{\kern0pt}\isakeyword{is}\ {\isachardoublequoteopen}{\isacharquery}{\kern0pt}slope{\isachardoublequoteclose}{\isacharparenright}{\kern0pt}\ \isakeyword{and}\isanewline
\ \ \ \ \ \ \ \ eudoxus{\isacharunderscore}{\kern0pt}inverse{\isacharunderscore}{\kern0pt}id{\isacharcolon}{\kern0pt}\ {\isachardoublequoteopen}{\isasymnot}\ bounded\ f\ {\isasymLongrightarrow}\ eudoxus{\isacharunderscore}{\kern0pt}inverse\ f\ {\isacharasterisk}{\kern0pt}\isactrlsub e\ f\ {\isasymsim}\isactrlsub e\ id{\isachardoublequoteclose}\ {\isacharparenleft}{\kern0pt}\isakeyword{is}\ {\isachardoublequoteopen}{\isasymnot}\ bounded\ f\ {\isasymLongrightarrow}\ {\isacharquery}{\kern0pt}id{\isachardoublequoteclose}{\isacharparenright}{\kern0pt}\isanewline
%
\isadelimproof
%
\endisadelimproof
%
\isatagproof
\isacommand{proof}\isamarkupfalse%
\ {\isacharminus}{\kern0pt}\isanewline
\ \ \isacommand{have}\isamarkupfalse%
\ {\isacharasterisk}{\kern0pt}{\isacharcolon}{\kern0pt}\ {\isachardoublequoteopen}{\isasymlbrakk}slope\ g{\isacharsemicolon}{\kern0pt}\ {\isacharparenleft}{\kern0pt}g\ {\isacharasterisk}{\kern0pt}\isactrlsub e\ f{\isacharparenright}{\kern0pt}\ {\isasymsim}\isactrlsub e\ id{\isasymrbrakk}\ {\isasymLongrightarrow}\ {\isacharquery}{\kern0pt}slope{\isachardoublequoteclose}\ {\isachardoublequoteopen}{\isasymlbrakk}slope\ g{\isacharsemicolon}{\kern0pt}\ {\isacharparenleft}{\kern0pt}g\ {\isacharasterisk}{\kern0pt}\isactrlsub e\ f{\isacharparenright}{\kern0pt}\ {\isasymsim}\isactrlsub e\ id{\isacharsemicolon}{\kern0pt}\ {\isasymnot}\ bounded\ f{\isasymrbrakk}\ {\isasymLongrightarrow}\ {\isacharquery}{\kern0pt}id{\isachardoublequoteclose}\ \isakeyword{for}\ g\ \isanewline
\ \ \ \ \isacommand{unfolding}\isamarkupfalse%
\ eudoxus{\isacharunderscore}{\kern0pt}inverse{\isacharunderscore}{\kern0pt}def\ \isacommand{using}\isamarkupfalse%
\ someI{\isacharbrackleft}{\kern0pt}\isakeyword{where}\ {\isacharquery}{\kern0pt}P{\isacharequal}{\kern0pt}{\isachardoublequoteopen}{\isasymlambda}g{\isachardot}{\kern0pt}\ slope\ g\ {\isasymand}\ {\isacharparenleft}{\kern0pt}g\ {\isacharasterisk}{\kern0pt}\isactrlsub e\ f{\isacharparenright}{\kern0pt}\ {\isasymsim}\isactrlsub e\ id{\isachardoublequoteclose}{\isacharbrackright}{\kern0pt}\ \isacommand{by}\isamarkupfalse%
\ auto\isanewline
\ \ \isacommand{{\isacharbraceleft}{\kern0pt}}\isamarkupfalse%
\isanewline
\ \ \ \ \isacommand{assume}\isamarkupfalse%
\ pos{\isacharcolon}{\kern0pt}\ {\isachardoublequoteopen}pos\ f{\isachardoublequoteclose}\isanewline
\ \ \ \ \isacommand{then}\isamarkupfalse%
\ \isacommand{obtain}\isamarkupfalse%
\ g\ \isakeyword{where}\ {\isachardoublequoteopen}slope\ {\isacharparenleft}{\kern0pt}eudoxus{\isacharunderscore}{\kern0pt}pos{\isacharunderscore}{\kern0pt}inverse\ g{\isacharparenright}{\kern0pt}{\isachardoublequoteclose}\ {\isachardoublequoteopen}eudoxus{\isacharunderscore}{\kern0pt}pos{\isacharunderscore}{\kern0pt}inverse\ g\ {\isacharasterisk}{\kern0pt}\isactrlsub e\ f\ {\isasymsim}\isactrlsub e\ id{\isachardoublequoteclose}\ \isacommand{using}\isamarkupfalse%
\ eudoxus{\isacharunderscore}{\kern0pt}pos{\isacharunderscore}{\kern0pt}inverse{\isacharbrackleft}{\kern0pt}OF\ assms{\isacharbrackright}{\kern0pt}\ \isacommand{by}\isamarkupfalse%
\ blast\isanewline
\ \ \ \ \isacommand{hence}\isamarkupfalse%
\ {\isacharquery}{\kern0pt}slope\ {\isachardoublequoteopen}{\isasymnot}\ bounded\ f\ {\isasymLongrightarrow}\ {\isacharquery}{\kern0pt}id{\isachardoublequoteclose}\ \isacommand{using}\isamarkupfalse%
\ pos\ pos{\isacharunderscore}{\kern0pt}iff{\isacharunderscore}{\kern0pt}nonneg{\isacharunderscore}{\kern0pt}nonzero{\isacharbrackleft}{\kern0pt}OF\ assms{\isacharbrackright}{\kern0pt}\ {\isacharasterisk}{\kern0pt}\ \isacommand{by}\isamarkupfalse%
\ blast{\isacharplus}{\kern0pt}\isanewline
\ \ \isacommand{{\isacharbraceright}{\kern0pt}}\isamarkupfalse%
\isanewline
\ \ \isacommand{moreover}\isamarkupfalse%
\isanewline
\ \ \isacommand{{\isacharbraceleft}{\kern0pt}}\isamarkupfalse%
\isanewline
\ \ \ \ \isacommand{assume}\isamarkupfalse%
\ nonpos{\isacharcolon}{\kern0pt}\ {\isachardoublequoteopen}{\isasymnot}\ pos\ f{\isachardoublequoteclose}\isanewline
\ \ \ \ \isacommand{{\isacharbraceleft}{\kern0pt}}\isamarkupfalse%
\isanewline
\ \ \ \ \ \ \isacommand{assume}\isamarkupfalse%
\ nonzero{\isacharcolon}{\kern0pt}\ {\isachardoublequoteopen}{\isasymnot}\ bounded\ f{\isachardoublequoteclose}\isanewline
\ \ \ \ \ \ \isacommand{hence}\isamarkupfalse%
\ uminus{\isacharunderscore}{\kern0pt}f{\isacharcolon}{\kern0pt}\ {\isachardoublequoteopen}slope\ {\isacharparenleft}{\kern0pt}{\isacharminus}{\kern0pt}\isactrlsub e\ f{\isacharparenright}{\kern0pt}{\isachardoublequoteclose}\ {\isachardoublequoteopen}pos\ {\isacharparenleft}{\kern0pt}{\isacharminus}{\kern0pt}\isactrlsub e\ f{\isacharparenright}{\kern0pt}{\isachardoublequoteclose}\ \isacommand{using}\isamarkupfalse%
\ neg{\isacharunderscore}{\kern0pt}iff{\isacharunderscore}{\kern0pt}pos{\isacharunderscore}{\kern0pt}uminus\ neg{\isacharunderscore}{\kern0pt}iff{\isacharunderscore}{\kern0pt}nonpos{\isacharunderscore}{\kern0pt}nonzero\ assms\ slope{\isacharunderscore}{\kern0pt}refl\ nonpos\ \isacommand{by}\isamarkupfalse%
\ auto\isanewline
\ \ \ \ \ \ \isacommand{then}\isamarkupfalse%
\ \isacommand{obtain}\isamarkupfalse%
\ g\ \isakeyword{where}\ g{\isacharcolon}{\kern0pt}\ {\isachardoublequoteopen}slope\ {\isacharparenleft}{\kern0pt}eudoxus{\isacharunderscore}{\kern0pt}pos{\isacharunderscore}{\kern0pt}inverse\ g{\isacharparenright}{\kern0pt}{\isachardoublequoteclose}\ {\isachardoublequoteopen}eudoxus{\isacharunderscore}{\kern0pt}pos{\isacharunderscore}{\kern0pt}inverse\ g\ {\isacharasterisk}{\kern0pt}\isactrlsub e\ {\isacharparenleft}{\kern0pt}{\isacharminus}{\kern0pt}\isactrlsub e\ f{\isacharparenright}{\kern0pt}\ {\isasymsim}\isactrlsub e\ id{\isachardoublequoteclose}\ \isacommand{using}\isamarkupfalse%
\ eudoxus{\isacharunderscore}{\kern0pt}pos{\isacharunderscore}{\kern0pt}inverse\ \isacommand{by}\isamarkupfalse%
\ metis\isanewline
\ \ \ \ \ \ \isacommand{hence}\isamarkupfalse%
\ {\isachardoublequoteopen}{\isacharminus}{\kern0pt}\isactrlsub e\ {\isacharparenleft}{\kern0pt}eudoxus{\isacharunderscore}{\kern0pt}pos{\isacharunderscore}{\kern0pt}inverse\ g{\isacharparenright}{\kern0pt}\ {\isacharasterisk}{\kern0pt}\isactrlsub e\ f\ {\isasymsim}\isactrlsub e\ id{\isachardoublequoteclose}\ \isacommand{by}\isamarkupfalse%
\ {\isacharparenleft}{\kern0pt}metis\ {\isacharparenleft}{\kern0pt}full{\isacharunderscore}{\kern0pt}types{\isacharparenright}{\kern0pt}\ uminus{\isacharunderscore}{\kern0pt}f{\isacharparenleft}{\kern0pt}{\isadigit{1}}{\isacharparenright}{\kern0pt}\ abs{\isacharunderscore}{\kern0pt}real{\isacharunderscore}{\kern0pt}eq{\isacharunderscore}{\kern0pt}iff\ abs{\isacharunderscore}{\kern0pt}real{\isacharunderscore}{\kern0pt}times\ abs{\isacharunderscore}{\kern0pt}real{\isacharunderscore}{\kern0pt}uminus\ assms{\isacharparenleft}{\kern0pt}{\isadigit{1}}{\isacharparenright}{\kern0pt}\ eudoxus{\isacharunderscore}{\kern0pt}times{\isacharunderscore}{\kern0pt}commute\ minus{\isacharunderscore}{\kern0pt}mult{\isacharunderscore}{\kern0pt}commute\ rel{\isacharunderscore}{\kern0pt}funE\ uminus{\isacharunderscore}{\kern0pt}real{\isachardot}{\kern0pt}rsp{\isacharparenright}{\kern0pt}\isanewline
\ \ \ \ \ \ \isacommand{moreover}\isamarkupfalse%
\ \isacommand{have}\isamarkupfalse%
\ {\isachardoublequoteopen}slope\ {\isacharparenleft}{\kern0pt}{\isacharminus}{\kern0pt}\isactrlsub e\ {\isacharparenleft}{\kern0pt}eudoxus{\isacharunderscore}{\kern0pt}pos{\isacharunderscore}{\kern0pt}inverse\ g{\isacharparenright}{\kern0pt}{\isacharparenright}{\kern0pt}{\isachardoublequoteclose}\ \isacommand{using}\isamarkupfalse%
\ uminus{\isacharunderscore}{\kern0pt}f\ eudoxus{\isacharunderscore}{\kern0pt}uminus{\isacharunderscore}{\kern0pt}cong\ slope{\isacharunderscore}{\kern0pt}refl\ g\ \isacommand{by}\isamarkupfalse%
\ presburger\isanewline
\ \ \ \ \ \ \isacommand{ultimately}\isamarkupfalse%
\ \isacommand{have}\isamarkupfalse%
\ {\isacharquery}{\kern0pt}slope\ {\isacharquery}{\kern0pt}id\ \isacommand{using}\isamarkupfalse%
\ {\isacharasterisk}{\kern0pt}\ nonzero\ \isacommand{by}\isamarkupfalse%
\ blast{\isacharplus}{\kern0pt}\isanewline
\ \ \ \ \isacommand{{\isacharbraceright}{\kern0pt}}\isamarkupfalse%
\isanewline
\ \ \ \ \isacommand{moreover}\isamarkupfalse%
\ \isacommand{have}\isamarkupfalse%
\ {\isachardoublequoteopen}bounded\ f\ {\isasymLongrightarrow}\ {\isacharquery}{\kern0pt}slope{\isachardoublequoteclose}\ \isacommand{unfolding}\isamarkupfalse%
\ eudoxus{\isacharunderscore}{\kern0pt}inverse{\isacharunderscore}{\kern0pt}def\ \isacommand{by}\isamarkupfalse%
\ simp\isanewline
\ \ \ \ \isacommand{ultimately}\isamarkupfalse%
\ \isacommand{have}\isamarkupfalse%
\ {\isacharquery}{\kern0pt}slope\ {\isachardoublequoteopen}{\isasymnot}\ bounded\ f\ {\isasymLongrightarrow}\ {\isacharquery}{\kern0pt}id{\isachardoublequoteclose}\ \isacommand{by}\isamarkupfalse%
\ blast{\isacharplus}{\kern0pt}\isanewline
\ \ \isacommand{{\isacharbraceright}{\kern0pt}}\isamarkupfalse%
\isanewline
\ \ \isacommand{ultimately}\isamarkupfalse%
\ \isacommand{show}\isamarkupfalse%
\ {\isacharquery}{\kern0pt}slope\ {\isachardoublequoteopen}{\isasymnot}\ bounded\ f\ {\isasymLongrightarrow}\ {\isacharquery}{\kern0pt}id{\isachardoublequoteclose}\ \isacommand{by}\isamarkupfalse%
\ blast{\isacharplus}{\kern0pt}\isanewline
\isacommand{qed}\isamarkupfalse%
%
\endisatagproof
{\isafoldproof}%
%
\isadelimproof
\isanewline
%
\endisadelimproof
\isanewline
\isacommand{quotient{\isacharunderscore}{\kern0pt}definition}\isamarkupfalse%
\isanewline
\ \ {\isachardoublequoteopen}{\isacharparenleft}{\kern0pt}inverse\ {\isacharcolon}{\kern0pt}{\isacharcolon}{\kern0pt}\ real\ {\isasymRightarrow}\ real{\isacharparenright}{\kern0pt}{\isachardoublequoteclose}\ \isakeyword{is}\ eudoxus{\isacharunderscore}{\kern0pt}inverse\isanewline
%
\isadelimproof
%
\endisadelimproof
%
\isatagproof
\isacommand{proof}\isamarkupfalse%
\ {\isacharminus}{\kern0pt}\isanewline
\ \ \isacommand{fix}\isamarkupfalse%
\ x\ x{\isacharprime}{\kern0pt}\ \isacommand{assume}\isamarkupfalse%
\ asm{\isacharcolon}{\kern0pt}\ {\isachardoublequoteopen}x\ {\isasymsim}\isactrlsub e\ x{\isacharprime}{\kern0pt}{\isachardoublequoteclose}\isanewline
\ \ \isacommand{hence}\isamarkupfalse%
\ slopes{\isacharcolon}{\kern0pt}\ {\isachardoublequoteopen}slope\ x{\isachardoublequoteclose}\ {\isachardoublequoteopen}slope\ x{\isacharprime}{\kern0pt}{\isachardoublequoteclose}\ \isacommand{unfolding}\isamarkupfalse%
\ eudoxus{\isacharunderscore}{\kern0pt}rel{\isacharunderscore}{\kern0pt}def\ \isacommand{by}\isamarkupfalse%
\ blast{\isacharplus}{\kern0pt}\isanewline
\ \ \isacommand{show}\isamarkupfalse%
\ {\isachardoublequoteopen}eudoxus{\isacharunderscore}{\kern0pt}inverse\ x\ {\isasymsim}\isactrlsub e\ eudoxus{\isacharunderscore}{\kern0pt}inverse\ x{\isacharprime}{\kern0pt}{\isachardoublequoteclose}\isanewline
\ \ \isacommand{proof}\isamarkupfalse%
\ {\isacharparenleft}{\kern0pt}cases\ {\isachardoublequoteopen}bounded\ x{\isachardoublequoteclose}{\isacharparenright}{\kern0pt}\isanewline
\ \ \ \ \isacommand{case}\isamarkupfalse%
\ True\isanewline
\ \ \ \ \isacommand{hence}\isamarkupfalse%
\ {\isachardoublequoteopen}bounded\ x{\isacharprime}{\kern0pt}{\isachardoublequoteclose}\ \isacommand{by}\isamarkupfalse%
\ {\isacharparenleft}{\kern0pt}meson\ asm\ eudoxus{\isacharunderscore}{\kern0pt}rel{\isacharunderscore}{\kern0pt}sym\ eudoxus{\isacharunderscore}{\kern0pt}rel{\isacharunderscore}{\kern0pt}trans\ zero{\isacharunderscore}{\kern0pt}iff{\isacharunderscore}{\kern0pt}bounded{\isacharparenright}{\kern0pt}\isanewline
\ \ \ \ \isacommand{then}\isamarkupfalse%
\ \isacommand{show}\isamarkupfalse%
\ {\isacharquery}{\kern0pt}thesis\ \isacommand{unfolding}\isamarkupfalse%
\ eudoxus{\isacharunderscore}{\kern0pt}inverse{\isacharunderscore}{\kern0pt}def\ \isacommand{using}\isamarkupfalse%
\ True\ slope{\isacharunderscore}{\kern0pt}zero\ slope{\isacharunderscore}{\kern0pt}refl\ \isacommand{by}\isamarkupfalse%
\ auto\isanewline
\ \ \isacommand{next}\isamarkupfalse%
\isanewline
\ \ \ \ \isacommand{case}\isamarkupfalse%
\ False\isanewline
\ \ \ \ \isacommand{hence}\isamarkupfalse%
\ {\isachardoublequoteopen}{\isasymnot}\ bounded\ x{\isacharprime}{\kern0pt}{\isachardoublequoteclose}\ \isacommand{by}\isamarkupfalse%
\ {\isacharparenleft}{\kern0pt}meson\ asm\ eudoxus{\isacharunderscore}{\kern0pt}rel{\isacharunderscore}{\kern0pt}sym\ eudoxus{\isacharunderscore}{\kern0pt}rel{\isacharunderscore}{\kern0pt}trans\ zero{\isacharunderscore}{\kern0pt}iff{\isacharunderscore}{\kern0pt}bounded{\isacharparenright}{\kern0pt}\isanewline
\ \ \ \ \isacommand{hence}\isamarkupfalse%
\ inverses{\isacharcolon}{\kern0pt}\ {\isachardoublequoteopen}eudoxus{\isacharunderscore}{\kern0pt}inverse\ x\ {\isacharasterisk}{\kern0pt}\isactrlsub e\ x\ {\isasymsim}\isactrlsub e\ id{\isachardoublequoteclose}\ {\isachardoublequoteopen}eudoxus{\isacharunderscore}{\kern0pt}inverse\ x{\isacharprime}{\kern0pt}\ {\isacharasterisk}{\kern0pt}\isactrlsub e\ x{\isacharprime}{\kern0pt}\ {\isasymsim}\isactrlsub e\ id{\isachardoublequoteclose}\ \isacommand{using}\isamarkupfalse%
\ slopes\ eudoxus{\isacharunderscore}{\kern0pt}inverse{\isacharunderscore}{\kern0pt}id\ False\ \isacommand{by}\isamarkupfalse%
\ blast{\isacharplus}{\kern0pt}\isanewline
\isanewline
\ \ \ \ \isacommand{have}\isamarkupfalse%
\ alt{\isacharunderscore}{\kern0pt}inverse{\isacharcolon}{\kern0pt}\ {\isachardoublequoteopen}eudoxus{\isacharunderscore}{\kern0pt}inverse\ x\ {\isacharasterisk}{\kern0pt}\isactrlsub e\ x{\isacharprime}{\kern0pt}\ {\isasymsim}\isactrlsub e\ id{\isachardoublequoteclose}\ \isanewline
\ \ \ \ \ \ \isacommand{using}\isamarkupfalse%
\ inverses\ eudoxus{\isacharunderscore}{\kern0pt}times{\isacharunderscore}{\kern0pt}cong{\isacharbrackleft}{\kern0pt}OF\ slope{\isacharunderscore}{\kern0pt}reflI{\isacharcomma}{\kern0pt}\ OF\ slope{\isacharunderscore}{\kern0pt}eudoxus{\isacharunderscore}{\kern0pt}inverse\ asm{\isacharcomma}{\kern0pt}\ OF\ slopes{\isacharparenleft}{\kern0pt}{\isadigit{1}}{\isacharparenright}{\kern0pt}{\isacharbrackright}{\kern0pt}\isanewline
\ \ \ \ \ \ \ \ \ \ \ \ eudoxus{\isacharunderscore}{\kern0pt}rel{\isacharunderscore}{\kern0pt}sym\ eudoxus{\isacharunderscore}{\kern0pt}rel{\isacharunderscore}{\kern0pt}trans\ \isacommand{by}\isamarkupfalse%
\ blast\isanewline
\isanewline
\ \ \ \ \isacommand{have}\isamarkupfalse%
\ {\isachardoublequoteopen}eudoxus{\isacharunderscore}{\kern0pt}inverse\ x\ {\isasymsim}\isactrlsub e\ eudoxus{\isacharunderscore}{\kern0pt}inverse\ x\ {\isacharasterisk}{\kern0pt}\isactrlsub e\ {\isacharparenleft}{\kern0pt}eudoxus{\isacharunderscore}{\kern0pt}inverse\ x{\isacharprime}{\kern0pt}\ {\isacharasterisk}{\kern0pt}\isactrlsub e\ x{\isacharprime}{\kern0pt}{\isacharparenright}{\kern0pt}{\isachardoublequoteclose}\ \isanewline
\ \ \ \ \ \ \isacommand{using}\isamarkupfalse%
\ eudoxus{\isacharunderscore}{\kern0pt}times{\isacharunderscore}{\kern0pt}cong{\isacharbrackleft}{\kern0pt}OF\ slope{\isacharunderscore}{\kern0pt}reflI{\isacharcomma}{\kern0pt}\ OF\ slope{\isacharunderscore}{\kern0pt}eudoxus{\isacharunderscore}{\kern0pt}inverse\ inverses{\isacharparenleft}{\kern0pt}{\isadigit{2}}{\isacharparenright}{\kern0pt}{\isacharbrackleft}{\kern0pt}THEN\ eudoxus{\isacharunderscore}{\kern0pt}rel{\isacharunderscore}{\kern0pt}sym{\isacharbrackright}{\kern0pt}{\isacharcomma}{\kern0pt}\ OF\ slopes{\isacharparenleft}{\kern0pt}{\isadigit{1}}{\isacharparenright}{\kern0pt}{\isacharbrackright}{\kern0pt}\isanewline
\ \ \ \ \ \ \isacommand{by}\isamarkupfalse%
\ {\isacharparenleft}{\kern0pt}simp\ add{\isacharcolon}{\kern0pt}\ eudoxus{\isacharunderscore}{\kern0pt}times{\isacharunderscore}{\kern0pt}def{\isacharparenright}{\kern0pt}\isanewline
\ \ \ \ \isacommand{also}\isamarkupfalse%
\ \isacommand{have}\isamarkupfalse%
\ {\isachardoublequoteopen}{\isachardot}{\kern0pt}{\isachardot}{\kern0pt}{\isachardot}{\kern0pt}\ {\isasymsim}\isactrlsub e\ eudoxus{\isacharunderscore}{\kern0pt}inverse\ x{\isacharprime}{\kern0pt}\ {\isacharasterisk}{\kern0pt}\isactrlsub e\ {\isacharparenleft}{\kern0pt}eudoxus{\isacharunderscore}{\kern0pt}inverse\ x\ {\isacharasterisk}{\kern0pt}\isactrlsub e\ x{\isacharprime}{\kern0pt}{\isacharparenright}{\kern0pt}{\isachardoublequoteclose}\isanewline
\ \ \ \ \ \ \isacommand{using}\isamarkupfalse%
\ eudoxus{\isacharunderscore}{\kern0pt}times{\isacharunderscore}{\kern0pt}commute{\isacharbrackleft}{\kern0pt}OF\ slope{\isacharunderscore}{\kern0pt}eudoxus{\isacharunderscore}{\kern0pt}inverse{\isacharparenleft}{\kern0pt}{\isadigit{1}}{\isacharcomma}{\kern0pt}{\isadigit{1}}{\isacharparenright}{\kern0pt}{\isacharcomma}{\kern0pt}\ OF\ slopes{\isacharcomma}{\kern0pt}\ THEN\ eudoxus{\isacharunderscore}{\kern0pt}times{\isacharunderscore}{\kern0pt}cong{\isacharcomma}{\kern0pt}\ OF\ slope{\isacharunderscore}{\kern0pt}reflI{\isacharcomma}{\kern0pt}\ OF\ slopes{\isacharparenleft}{\kern0pt}{\isadigit{2}}{\isacharparenright}{\kern0pt}{\isacharbrackright}{\kern0pt}\ \isanewline
\ \ \ \ \ \ \isacommand{by}\isamarkupfalse%
\ {\isacharparenleft}{\kern0pt}simp\ add{\isacharcolon}{\kern0pt}\ eudoxus{\isacharunderscore}{\kern0pt}times{\isacharunderscore}{\kern0pt}def\ comp{\isacharunderscore}{\kern0pt}assoc{\isacharparenright}{\kern0pt}\ \ \ \ \ \ \isanewline
\ \ \ \ \isacommand{also}\isamarkupfalse%
\ \isacommand{have}\isamarkupfalse%
\ {\isachardoublequoteopen}{\isachardot}{\kern0pt}{\isachardot}{\kern0pt}{\isachardot}{\kern0pt}\ {\isasymsim}\isactrlsub e\ eudoxus{\isacharunderscore}{\kern0pt}inverse\ x{\isacharprime}{\kern0pt}\ {\isacharasterisk}{\kern0pt}\isactrlsub e\ id{\isachardoublequoteclose}\ \isacommand{using}\isamarkupfalse%
\ alt{\isacharunderscore}{\kern0pt}inverse\ eudoxus{\isacharunderscore}{\kern0pt}times{\isacharunderscore}{\kern0pt}cong{\isacharbrackleft}{\kern0pt}OF\ slope{\isacharunderscore}{\kern0pt}reflI{\isacharbrackright}{\kern0pt}\ slope{\isacharunderscore}{\kern0pt}eudoxus{\isacharunderscore}{\kern0pt}inverse\ slopes\ \isacommand{by}\isamarkupfalse%
\ blast\isanewline
\ \ \ \ \isacommand{also}\isamarkupfalse%
\ \isacommand{have}\isamarkupfalse%
\ {\isachardoublequoteopen}{\isachardot}{\kern0pt}{\isachardot}{\kern0pt}{\isachardot}{\kern0pt}\ {\isacharequal}{\kern0pt}\ eudoxus{\isacharunderscore}{\kern0pt}inverse\ x{\isacharprime}{\kern0pt}{\isachardoublequoteclose}\ \isacommand{unfolding}\isamarkupfalse%
\ eudoxus{\isacharunderscore}{\kern0pt}times{\isacharunderscore}{\kern0pt}def\ \isacommand{by}\isamarkupfalse%
\ simp\isanewline
\ \ \ \ \isacommand{finally}\isamarkupfalse%
\ \isacommand{show}\isamarkupfalse%
\ {\isacharquery}{\kern0pt}thesis\ \isacommand{{\isachardot}{\kern0pt}}\isamarkupfalse%
\isanewline
\ \ \isacommand{qed}\isamarkupfalse%
\isanewline
\isacommand{qed}\isamarkupfalse%
%
\endisatagproof
{\isafoldproof}%
%
\isadelimproof
\isanewline
%
\endisadelimproof
\isanewline
\isacommand{definition}\isamarkupfalse%
\ \isanewline
\ \ {\isachardoublequoteopen}x\ div\ {\isacharparenleft}{\kern0pt}y{\isacharcolon}{\kern0pt}{\isacharcolon}{\kern0pt}real{\isacharparenright}{\kern0pt}\ {\isacharequal}{\kern0pt}\ inverse\ y\ {\isacharasterisk}{\kern0pt}\ x{\isachardoublequoteclose}\isanewline
\isanewline
\isacommand{instance}\isamarkupfalse%
%
\isadelimproof
\ %
\endisadelimproof
%
\isatagproof
\isacommand{{\isachardot}{\kern0pt}{\isachardot}{\kern0pt}}\isamarkupfalse%
%
\endisatagproof
{\isafoldproof}%
%
\isadelimproof
%
\endisadelimproof
\isanewline
\isacommand{end}\isamarkupfalse%
\isanewline
\isanewline
\isacommand{lemmas}\isamarkupfalse%
\ eudoxus{\isacharunderscore}{\kern0pt}inverse{\isacharunderscore}{\kern0pt}cong\ {\isacharequal}{\kern0pt}\ apply{\isacharunderscore}{\kern0pt}rsp{\isacharprime}{\kern0pt}{\isacharbrackleft}{\kern0pt}OF\ inverse{\isacharunderscore}{\kern0pt}real{\isachardot}{\kern0pt}rsp{\isacharcomma}{\kern0pt}\ intro{\isacharbrackright}{\kern0pt}\isanewline
\isanewline
\isacommand{lemma}\isamarkupfalse%
\ eudoxus{\isacharunderscore}{\kern0pt}inverse{\isacharunderscore}{\kern0pt}abs{\isacharbrackleft}{\kern0pt}simp{\isacharbrackright}{\kern0pt}{\isacharcolon}{\kern0pt}\isanewline
\ \ \isakeyword{assumes}\ {\isachardoublequoteopen}slope\ f{\isachardoublequoteclose}\ {\isachardoublequoteopen}{\isasymnot}\ bounded\ f{\isachardoublequoteclose}\isanewline
\ \ \isakeyword{shows}\ {\isachardoublequoteopen}inverse\ {\isacharparenleft}{\kern0pt}abs{\isacharunderscore}{\kern0pt}real\ f{\isacharparenright}{\kern0pt}\ {\isacharasterisk}{\kern0pt}\ abs{\isacharunderscore}{\kern0pt}real\ f\ {\isacharequal}{\kern0pt}\ {\isadigit{1}}{\isachardoublequoteclose}\isanewline
%
\isadelimproof
\ \ %
\endisadelimproof
%
\isatagproof
\isacommand{unfolding}\isamarkupfalse%
\ inverse{\isacharunderscore}{\kern0pt}real{\isacharunderscore}{\kern0pt}def\ \isacommand{using}\isamarkupfalse%
\ eudoxus{\isacharunderscore}{\kern0pt}inverse{\isacharunderscore}{\kern0pt}id{\isacharbrackleft}{\kern0pt}OF\ assms{\isacharbrackright}{\kern0pt}\isanewline
\ \ \isacommand{by}\isamarkupfalse%
\ {\isacharparenleft}{\kern0pt}metis\ abs{\isacharunderscore}{\kern0pt}real{\isacharunderscore}{\kern0pt}eqI\ abs{\isacharunderscore}{\kern0pt}real{\isacharunderscore}{\kern0pt}times\ assms{\isacharparenleft}{\kern0pt}{\isadigit{1}}{\isacharparenright}{\kern0pt}\ eudoxus{\isacharunderscore}{\kern0pt}inverse{\isacharunderscore}{\kern0pt}cong\ map{\isacharunderscore}{\kern0pt}fun{\isacharunderscore}{\kern0pt}apply\ one{\isacharunderscore}{\kern0pt}def\ rep{\isacharunderscore}{\kern0pt}real{\isacharunderscore}{\kern0pt}abs{\isacharunderscore}{\kern0pt}real{\isacharunderscore}{\kern0pt}refl\ slope{\isacharunderscore}{\kern0pt}refl{\isacharparenright}{\kern0pt}%
\endisatagproof
{\isafoldproof}%
%
\isadelimproof
%
\endisadelimproof
%
\begin{isamarkuptext}%
The Eudoxus reals are a field, with inverses defined as above.%
\end{isamarkuptext}\isamarkuptrue%
\isacommand{instance}\isamarkupfalse%
\ real\ {\isacharcolon}{\kern0pt}{\isacharcolon}{\kern0pt}\ field\isanewline
%
\isadelimproof
%
\endisadelimproof
%
\isatagproof
\isacommand{proof}\isamarkupfalse%
\isanewline
\ \ \isacommand{fix}\isamarkupfalse%
\ x\ y\ {\isacharcolon}{\kern0pt}{\isacharcolon}{\kern0pt}\ real\isanewline
\ \ \isacommand{show}\isamarkupfalse%
\ {\isachardoublequoteopen}x\ {\isasymnoteq}\ {\isadigit{0}}\ {\isasymLongrightarrow}\ inverse\ x\ {\isacharasterisk}{\kern0pt}\ x\ {\isacharequal}{\kern0pt}\ {\isadigit{1}}{\isachardoublequoteclose}\ \isacommand{using}\isamarkupfalse%
\ eudoxus{\isacharunderscore}{\kern0pt}sgn{\isacharunderscore}{\kern0pt}iff{\isacharparenleft}{\kern0pt}{\isadigit{1}}{\isacharparenright}{\kern0pt}\ sgn{\isacharunderscore}{\kern0pt}abs{\isacharunderscore}{\kern0pt}real{\isacharunderscore}{\kern0pt}zero{\isacharunderscore}{\kern0pt}iff\ \isacommand{by}\isamarkupfalse%
\ {\isacharparenleft}{\kern0pt}induct\ x\ rule{\isacharcolon}{\kern0pt}\ slope{\isacharunderscore}{\kern0pt}induct{\isacharparenright}{\kern0pt}\ force\isanewline
\ \ \isacommand{show}\isamarkupfalse%
\ {\isachardoublequoteopen}x\ {\isacharslash}{\kern0pt}\ y\ {\isacharequal}{\kern0pt}\ x\ {\isacharasterisk}{\kern0pt}\ inverse\ y{\isachardoublequoteclose}\ \isacommand{unfolding}\isamarkupfalse%
\ divide{\isacharunderscore}{\kern0pt}real{\isacharunderscore}{\kern0pt}def\ \isacommand{by}\isamarkupfalse%
\ simp\isanewline
\ \ \isacommand{show}\isamarkupfalse%
\ {\isachardoublequoteopen}inverse\ {\isacharparenleft}{\kern0pt}{\isadigit{0}}\ {\isacharcolon}{\kern0pt}{\isacharcolon}{\kern0pt}\ real{\isacharparenright}{\kern0pt}\ {\isacharequal}{\kern0pt}\ {\isadigit{0}}{\isachardoublequoteclose}\ \isacommand{unfolding}\isamarkupfalse%
\ inverse{\isacharunderscore}{\kern0pt}real{\isacharunderscore}{\kern0pt}def\ eudoxus{\isacharunderscore}{\kern0pt}inverse{\isacharunderscore}{\kern0pt}def\ \isacommand{using}\isamarkupfalse%
\ zero{\isacharunderscore}{\kern0pt}def\ zero{\isacharunderscore}{\kern0pt}iff{\isacharunderscore}{\kern0pt}bounded{\isacharprime}{\kern0pt}\ \isacommand{by}\isamarkupfalse%
\ auto\ \isanewline
\isacommand{qed}\isamarkupfalse%
%
\endisatagproof
{\isafoldproof}%
%
\isadelimproof
\isanewline
%
\endisadelimproof
\isanewline
\isacommand{instantiation}\isamarkupfalse%
\ real\ {\isacharcolon}{\kern0pt}{\isacharcolon}{\kern0pt}\ distrib{\isacharunderscore}{\kern0pt}lattice\isanewline
\isakeyword{begin}\isanewline
\isanewline
\isacommand{definition}\isamarkupfalse%
\isanewline
\ \ {\isachardoublequoteopen}{\isacharparenleft}{\kern0pt}inf\ {\isacharcolon}{\kern0pt}{\isacharcolon}{\kern0pt}\ real\ {\isasymRightarrow}\ real\ {\isasymRightarrow}\ real{\isacharparenright}{\kern0pt}\ {\isacharequal}{\kern0pt}\ min{\isachardoublequoteclose}\isanewline
\ \ \ \ \ \ \ \ \ \ \ \ \ \ \ \ \ \ \ \ \ \ \ \ \ \ \ \ \ \ \ \ \ \ \ \isanewline
\isacommand{definition}\isamarkupfalse%
\isanewline
\ \ {\isachardoublequoteopen}{\isacharparenleft}{\kern0pt}sup\ {\isacharcolon}{\kern0pt}{\isacharcolon}{\kern0pt}\ real\ {\isasymRightarrow}\ real\ {\isasymRightarrow}\ real{\isacharparenright}{\kern0pt}\ {\isacharequal}{\kern0pt}\ max{\isachardoublequoteclose}\isanewline
\isanewline
\isacommand{instance}\isamarkupfalse%
%
\isadelimproof
\ %
\endisadelimproof
%
\isatagproof
\isacommand{by}\isamarkupfalse%
\ standard\ {\isacharparenleft}{\kern0pt}auto\ simp{\isacharcolon}{\kern0pt}\ inf{\isacharunderscore}{\kern0pt}real{\isacharunderscore}{\kern0pt}def\ sup{\isacharunderscore}{\kern0pt}real{\isacharunderscore}{\kern0pt}def\ max{\isacharunderscore}{\kern0pt}min{\isacharunderscore}{\kern0pt}distrib{\isadigit{2}}{\isacharparenright}{\kern0pt}%
\endisatagproof
{\isafoldproof}%
%
\isadelimproof
%
\endisadelimproof
\isanewline
\isanewline
\isacommand{end}\isamarkupfalse%
%
\begin{isamarkuptext}%
The ordering on the Eudoxus reals is linear.%
\end{isamarkuptext}\isamarkuptrue%
\isacommand{instance}\isamarkupfalse%
\ real\ {\isacharcolon}{\kern0pt}{\isacharcolon}{\kern0pt}\ linordered{\isacharunderscore}{\kern0pt}field\isanewline
%
\isadelimproof
%
\endisadelimproof
%
\isatagproof
\isacommand{proof}\isamarkupfalse%
\isanewline
\ \ \isacommand{fix}\isamarkupfalse%
\ x\ y\ z\ {\isacharcolon}{\kern0pt}{\isacharcolon}{\kern0pt}\ real\isanewline
\ \ \isacommand{show}\isamarkupfalse%
\ {\isachardoublequoteopen}z\ {\isacharplus}{\kern0pt}\ x\ {\isasymle}\ z\ {\isacharplus}{\kern0pt}\ y{\isachardoublequoteclose}\ \isakeyword{if}\ {\isachardoublequoteopen}x\ {\isasymle}\ y{\isachardoublequoteclose}\isanewline
\ \ \isacommand{proof}\isamarkupfalse%
\ {\isacharparenleft}{\kern0pt}cases\ {\isachardoublequoteopen}x\ {\isacharequal}{\kern0pt}\ y{\isachardoublequoteclose}{\isacharparenright}{\kern0pt}\isanewline
\ \ \ \ \isacommand{case}\isamarkupfalse%
\ False\isanewline
\ \ \ \ \isacommand{hence}\isamarkupfalse%
\ {\isachardoublequoteopen}x\ {\isacharless}{\kern0pt}\ y{\isachardoublequoteclose}\ \isacommand{using}\isamarkupfalse%
\ that\ \isacommand{by}\isamarkupfalse%
\ simp\isanewline
\ \ \ \ \isacommand{thus}\isamarkupfalse%
\ {\isacharquery}{\kern0pt}thesis\isanewline
\ \ \ \ \isacommand{proof}\isamarkupfalse%
\ {\isacharparenleft}{\kern0pt}induct\ x\ rule{\isacharcolon}{\kern0pt}\ slope{\isacharunderscore}{\kern0pt}induct{\isacharcomma}{\kern0pt}\ induct\ y\ rule{\isacharcolon}{\kern0pt}\ slope{\isacharunderscore}{\kern0pt}induct{\isacharcomma}{\kern0pt}\ induct\ z\ rule{\isacharcolon}{\kern0pt}\ slope{\isacharunderscore}{\kern0pt}induct{\isacharparenright}{\kern0pt}\isanewline
\ \ \ \ \ \ \isacommand{case}\isamarkupfalse%
\ {\isacharparenleft}{\kern0pt}slope\ h\ g\ f{\isacharparenright}{\kern0pt}\isanewline
\ \ \ \ \ \ \isacommand{hence}\isamarkupfalse%
\ {\isachardoublequoteopen}pos\ {\isacharparenleft}{\kern0pt}g\ {\isacharplus}{\kern0pt}\isactrlsub e\ {\isacharparenleft}{\kern0pt}{\isacharminus}{\kern0pt}\isactrlsub e\ f{\isacharparenright}{\kern0pt}{\isacharparenright}{\kern0pt}{\isachardoublequoteclose}\ \isacommand{unfolding}\isamarkupfalse%
\ less{\isacharunderscore}{\kern0pt}real{\isacharunderscore}{\kern0pt}def\ \isacommand{using}\isamarkupfalse%
\ sgn{\isacharunderscore}{\kern0pt}abs{\isacharunderscore}{\kern0pt}real{\isacharunderscore}{\kern0pt}one{\isacharunderscore}{\kern0pt}iff\ \isacommand{by}\isamarkupfalse%
\ {\isacharparenleft}{\kern0pt}force\ simp\ add{\isacharcolon}{\kern0pt}\ eudoxus{\isacharunderscore}{\kern0pt}plus{\isacharunderscore}{\kern0pt}def\ eudoxus{\isacharunderscore}{\kern0pt}uminus{\isacharunderscore}{\kern0pt}def{\isacharparenright}{\kern0pt}\ \isanewline
\ \ \ \ \ \ \isacommand{thus}\isamarkupfalse%
\ {\isacharquery}{\kern0pt}case\ \isacommand{by}\isamarkupfalse%
\ {\isacharparenleft}{\kern0pt}metis\ slope{\isacharparenleft}{\kern0pt}{\isadigit{4}}{\isacharparenright}{\kern0pt}\ less{\isacharunderscore}{\kern0pt}real{\isacharunderscore}{\kern0pt}def\ add{\isacharunderscore}{\kern0pt}diff{\isacharunderscore}{\kern0pt}cancel{\isacharunderscore}{\kern0pt}left\ nless{\isacharunderscore}{\kern0pt}le{\isacharparenright}{\kern0pt}\isanewline
\ \ \ \ \isacommand{qed}\isamarkupfalse%
\isanewline
\ \ \isacommand{qed}\isamarkupfalse%
\ {\isacharparenleft}{\kern0pt}force{\isacharparenright}{\kern0pt}\isanewline
\isanewline
\ \ \isacommand{show}\isamarkupfalse%
\ {\isachardoublequoteopen}{\isasymbar}x{\isasymbar}\ {\isacharequal}{\kern0pt}\ {\isacharparenleft}{\kern0pt}if\ x\ {\isacharless}{\kern0pt}\ {\isadigit{0}}\ then\ {\isacharminus}{\kern0pt}x\ else\ x{\isacharparenright}{\kern0pt}{\isachardoublequoteclose}\ \isacommand{by}\isamarkupfalse%
\ {\isacharparenleft}{\kern0pt}metis\ abs{\isacharunderscore}{\kern0pt}real\ less{\isacharunderscore}{\kern0pt}eq{\isacharunderscore}{\kern0pt}real{\isacharunderscore}{\kern0pt}def\ not{\isacharunderscore}{\kern0pt}less{\isacharunderscore}{\kern0pt}iff{\isacharunderscore}{\kern0pt}gr{\isacharunderscore}{\kern0pt}or{\isacharunderscore}{\kern0pt}eq{\isacharparenright}{\kern0pt}\isanewline
\ \ \isacommand{show}\isamarkupfalse%
\ {\isachardoublequoteopen}sgn\ x\ {\isacharequal}{\kern0pt}\ {\isacharparenleft}{\kern0pt}if\ x\ {\isacharequal}{\kern0pt}\ {\isadigit{0}}\ then\ {\isadigit{0}}\ else\ if\ {\isadigit{0}}\ {\isacharless}{\kern0pt}\ x\ then\ {\isadigit{1}}\ else\ {\isacharminus}{\kern0pt}\ {\isadigit{1}}{\isacharparenright}{\kern0pt}{\isachardoublequoteclose}\ \isacommand{using}\isamarkupfalse%
\ sgn{\isacharunderscore}{\kern0pt}range\ sgn{\isacharunderscore}{\kern0pt}zero{\isacharunderscore}{\kern0pt}iff\ \isacommand{by}\isamarkupfalse%
\ {\isacharparenleft}{\kern0pt}auto\ simp{\isacharcolon}{\kern0pt}\ less{\isacharunderscore}{\kern0pt}real{\isacharunderscore}{\kern0pt}def{\isacharparenright}{\kern0pt}\isanewline
\ \ \isacommand{show}\isamarkupfalse%
\ {\isachardoublequoteopen}{\isasymlbrakk}x\ {\isacharless}{\kern0pt}\ y{\isacharsemicolon}{\kern0pt}\ {\isadigit{0}}\ {\isacharless}{\kern0pt}\ z{\isasymrbrakk}\ {\isasymLongrightarrow}\ z\ {\isacharasterisk}{\kern0pt}\ x\ {\isacharless}{\kern0pt}\ z\ {\isacharasterisk}{\kern0pt}\ y{\isachardoublequoteclose}\ \isacommand{by}\isamarkupfalse%
\ {\isacharparenleft}{\kern0pt}metis\ {\isacharparenleft}{\kern0pt}no{\isacharunderscore}{\kern0pt}types{\isacharcomma}{\kern0pt}\ lifting{\isacharparenright}{\kern0pt}\ diff{\isacharunderscore}{\kern0pt}zero\ less{\isacharunderscore}{\kern0pt}real{\isacharunderscore}{\kern0pt}def\ mult{\isachardot}{\kern0pt}right{\isacharunderscore}{\kern0pt}neutral\ right{\isacharunderscore}{\kern0pt}diff{\isacharunderscore}{\kern0pt}distrib{\isacharprime}{\kern0pt}\ sgn{\isacharunderscore}{\kern0pt}times{\isacharparenright}{\kern0pt}\isanewline
\isacommand{qed}\isamarkupfalse%
%
\endisatagproof
{\isafoldproof}%
%
\isadelimproof
%
\endisadelimproof
%
\begin{isamarkuptext}%
The Eudoxus reals fulfill the Archimedean property.%
\end{isamarkuptext}\isamarkuptrue%
\isacommand{instance}\isamarkupfalse%
\ real\ {\isacharcolon}{\kern0pt}{\isacharcolon}{\kern0pt}\ archimedean{\isacharunderscore}{\kern0pt}field\isanewline
%
\isadelimproof
%
\endisadelimproof
%
\isatagproof
\isacommand{proof}\isamarkupfalse%
\ \ \ \ \ \ \ \ \ \ \ \ \ \ \ \ \ \ \ \ \ \ \ \ \ \isanewline
\ \ \isacommand{fix}\isamarkupfalse%
\ x\ {\isacharcolon}{\kern0pt}{\isacharcolon}{\kern0pt}\ real\isanewline
\ \ \isacommand{show}\isamarkupfalse%
\ {\isachardoublequoteopen}{\isasymexists}z{\isachardot}{\kern0pt}\ x\ {\isasymle}\ of{\isacharunderscore}{\kern0pt}int\ z{\isachardoublequoteclose}\isanewline
\ \ \isacommand{proof}\isamarkupfalse%
\ {\isacharparenleft}{\kern0pt}induct\ x\ rule{\isacharcolon}{\kern0pt}\ slope{\isacharunderscore}{\kern0pt}induct{\isacharparenright}{\kern0pt}\isanewline
\ \ \ \ \isacommand{case}\isamarkupfalse%
\ {\isacharparenleft}{\kern0pt}slope\ y{\isacharparenright}{\kern0pt}\isanewline
\ \ \ \ \isacommand{then}\isamarkupfalse%
\ \isacommand{obtain}\isamarkupfalse%
\ A\ B\ \isakeyword{where}\ linear{\isacharunderscore}{\kern0pt}bound{\isacharcolon}{\kern0pt}\ {\isachardoublequoteopen}{\isasymbar}y\ z{\isasymbar}\ {\isasymle}\ A\ {\isacharasterisk}{\kern0pt}\ {\isasymbar}z{\isasymbar}\ {\isacharplus}{\kern0pt}\ B{\isachardoublequoteclose}\ {\isachardoublequoteopen}{\isadigit{0}}\ {\isasymle}\ A{\isachardoublequoteclose}\ {\isachardoublequoteopen}{\isadigit{0}}\ {\isasymle}\ B{\isachardoublequoteclose}\ \isakeyword{for}\ z\ \isacommand{using}\isamarkupfalse%
\ slope{\isacharunderscore}{\kern0pt}linear{\isacharunderscore}{\kern0pt}bound\ \isacommand{by}\isamarkupfalse%
\ blast\isanewline
\ \ \ \ \isacommand{{\isacharbraceleft}{\kern0pt}}\isamarkupfalse%
\isanewline
\ \ \ \ \ \ \isacommand{fix}\isamarkupfalse%
\ C\ \isacommand{assume}\isamarkupfalse%
\ C{\isacharunderscore}{\kern0pt}nonneg{\isacharcolon}{\kern0pt}\ {\isachardoublequoteopen}{\isadigit{0}}\ {\isasymle}\ {\isacharparenleft}{\kern0pt}C\ {\isacharcolon}{\kern0pt}{\isacharcolon}{\kern0pt}\ int{\isacharparenright}{\kern0pt}{\isachardoublequoteclose}\isanewline
\ \ \ \ \ \ \isacommand{{\isacharbraceleft}{\kern0pt}}\isamarkupfalse%
\isanewline
\ \ \ \ \ \ \ \ \isacommand{fix}\isamarkupfalse%
\ z\ \isacommand{assume}\isamarkupfalse%
\ asm{\isacharcolon}{\kern0pt}\ {\isachardoublequoteopen}z\ {\isasymge}\ B\ {\isacharplus}{\kern0pt}\ C{\isachardoublequoteclose}\isanewline
\ \ \ \ \ \ \ \ \isacommand{have}\isamarkupfalse%
\ {\isachardoublequoteopen}y\ z\ {\isacharplus}{\kern0pt}\ C\ {\isasymle}\ A\ {\isacharasterisk}{\kern0pt}\ {\isasymbar}z{\isasymbar}\ {\isacharplus}{\kern0pt}\ B\ {\isacharplus}{\kern0pt}\ C{\isachardoublequoteclose}\ \isacommand{using}\isamarkupfalse%
\ abs{\isacharunderscore}{\kern0pt}le{\isacharunderscore}{\kern0pt}D{\isadigit{1}}\ linear{\isacharunderscore}{\kern0pt}bound\ \isacommand{by}\isamarkupfalse%
\ auto\isanewline
\ \ \ \ \ \ \ \ \isacommand{also}\isamarkupfalse%
\ \isacommand{have}\isamarkupfalse%
\ {\isachardoublequoteopen}{\isachardot}{\kern0pt}{\isachardot}{\kern0pt}{\isachardot}{\kern0pt}\ {\isasymle}\ {\isacharparenleft}{\kern0pt}A\ {\isacharplus}{\kern0pt}\ {\isadigit{1}}{\isacharparenright}{\kern0pt}\ {\isacharasterisk}{\kern0pt}\ {\isasymbar}z{\isasymbar}{\isachardoublequoteclose}\ \isacommand{using}\isamarkupfalse%
\ C{\isacharunderscore}{\kern0pt}nonneg\ linear{\isacharunderscore}{\kern0pt}bound{\isacharparenleft}{\kern0pt}{\isadigit{2}}{\isacharcomma}{\kern0pt}{\isadigit{3}}{\isacharparenright}{\kern0pt}\ asm\ \isacommand{by}\isamarkupfalse%
\ {\isacharparenleft}{\kern0pt}auto\ simp{\isacharcolon}{\kern0pt}\ distrib{\isacharunderscore}{\kern0pt}right{\isacharparenright}{\kern0pt}\isanewline
\ \ \ \ \ \ \ \ \isacommand{finally}\isamarkupfalse%
\ \isacommand{have}\isamarkupfalse%
\ {\isachardoublequoteopen}y\ z\ {\isacharplus}{\kern0pt}\ C\ {\isasymle}\ {\isacharparenleft}{\kern0pt}A\ {\isacharplus}{\kern0pt}\ {\isadigit{1}}{\isacharparenright}{\kern0pt}\ {\isacharasterisk}{\kern0pt}\ z{\isachardoublequoteclose}\ \isacommand{using}\isamarkupfalse%
\ add{\isacharunderscore}{\kern0pt}nonneg{\isacharunderscore}{\kern0pt}nonneg{\isacharbrackleft}{\kern0pt}OF\ C{\isacharunderscore}{\kern0pt}nonneg\ linear{\isacharunderscore}{\kern0pt}bound{\isacharparenleft}{\kern0pt}{\isadigit{3}}{\isacharparenright}{\kern0pt}{\isacharbrackright}{\kern0pt}\ abs{\isacharunderscore}{\kern0pt}of{\isacharunderscore}{\kern0pt}nonneg{\isacharbrackleft}{\kern0pt}of\ z{\isacharbrackright}{\kern0pt}\ asm\ \isacommand{by}\isamarkupfalse%
\ linarith\isanewline
\ \ \ \ \ \ \isacommand{{\isacharbraceright}{\kern0pt}}\isamarkupfalse%
\isanewline
\ \ \ \ \ \ \isacommand{hence}\isamarkupfalse%
\ {\isachardoublequoteopen}{\isasymexists}N{\isachardot}{\kern0pt}\ {\isasymforall}x\ {\isasymge}\ N{\isachardot}{\kern0pt}\ {\isacharparenleft}{\kern0pt}{\isacharparenleft}{\kern0pt}{\isacharparenleft}{\kern0pt}{\isacharasterisk}{\kern0pt}{\isacharparenright}{\kern0pt}\ {\isacharparenleft}{\kern0pt}A\ {\isacharplus}{\kern0pt}\ {\isadigit{1}}{\isacharparenright}{\kern0pt}{\isacharparenright}{\kern0pt}\ {\isacharplus}{\kern0pt}\isactrlsub e\ {\isacharminus}{\kern0pt}\isactrlsub e\ y{\isacharparenright}{\kern0pt}\ x\ {\isasymge}\ C{\isachardoublequoteclose}\ \isacommand{unfolding}\isamarkupfalse%
\ eudoxus{\isacharunderscore}{\kern0pt}plus{\isacharunderscore}{\kern0pt}def\ eudoxus{\isacharunderscore}{\kern0pt}uminus{\isacharunderscore}{\kern0pt}def\ \isacommand{by}\isamarkupfalse%
\ fastforce\isanewline
\ \ \ \ \isacommand{{\isacharbraceright}{\kern0pt}}\isamarkupfalse%
\isanewline
\ \ \ \ \isacommand{hence}\isamarkupfalse%
\ {\isachardoublequoteopen}pos\ {\isacharparenleft}{\kern0pt}{\isacharparenleft}{\kern0pt}{\isacharparenleft}{\kern0pt}{\isacharasterisk}{\kern0pt}{\isacharparenright}{\kern0pt}\ {\isacharparenleft}{\kern0pt}A\ {\isacharplus}{\kern0pt}\ {\isadigit{1}}{\isacharparenright}{\kern0pt}{\isacharparenright}{\kern0pt}\ {\isacharplus}{\kern0pt}\isactrlsub e\ {\isacharminus}{\kern0pt}\isactrlsub e\ y{\isacharparenright}{\kern0pt}{\isachardoublequoteclose}\ \isacommand{unfolding}\isamarkupfalse%
\ pos{\isacharunderscore}{\kern0pt}def\ \isacommand{by}\isamarkupfalse%
\ blast\isanewline
\ \ \ \ \isacommand{hence}\isamarkupfalse%
\ {\isachardoublequoteopen}pos\ {\isacharparenleft}{\kern0pt}rep{\isacharunderscore}{\kern0pt}real\ {\isacharparenleft}{\kern0pt}of{\isacharunderscore}{\kern0pt}int\ {\isacharparenleft}{\kern0pt}A\ {\isacharplus}{\kern0pt}\ {\isadigit{1}}{\isacharparenright}{\kern0pt}\ {\isacharminus}{\kern0pt}\ abs{\isacharunderscore}{\kern0pt}real\ y{\isacharparenright}{\kern0pt}{\isacharparenright}{\kern0pt}{\isachardoublequoteclose}\ \isacommand{unfolding}\isamarkupfalse%
\ real{\isacharunderscore}{\kern0pt}of{\isacharunderscore}{\kern0pt}int\ \isacommand{using}\isamarkupfalse%
\ slope\ \isacommand{by}\isamarkupfalse%
\ {\isacharparenleft}{\kern0pt}simp{\isacharcomma}{\kern0pt}\ subst\ pos{\isacharunderscore}{\kern0pt}cong{\isacharbrackleft}{\kern0pt}OF\ rep{\isacharunderscore}{\kern0pt}real{\isacharunderscore}{\kern0pt}abs{\isacharunderscore}{\kern0pt}real{\isacharunderscore}{\kern0pt}refl{\isacharbrackright}{\kern0pt}{\isacharparenright}{\kern0pt}\ {\isacharparenleft}{\kern0pt}auto\ simp\ add{\isacharcolon}{\kern0pt}\ eudoxus{\isacharunderscore}{\kern0pt}plus{\isacharunderscore}{\kern0pt}def\ eudoxus{\isacharunderscore}{\kern0pt}uminus{\isacharunderscore}{\kern0pt}def{\isacharparenright}{\kern0pt}\isanewline
\ \ \ \ \isacommand{hence}\isamarkupfalse%
\ {\isachardoublequoteopen}abs{\isacharunderscore}{\kern0pt}real\ y\ {\isacharless}{\kern0pt}\ of{\isacharunderscore}{\kern0pt}int\ {\isacharparenleft}{\kern0pt}A\ {\isacharplus}{\kern0pt}\ {\isadigit{1}}{\isacharparenright}{\kern0pt}{\isachardoublequoteclose}\ \isacommand{unfolding}\isamarkupfalse%
\ less{\isacharunderscore}{\kern0pt}real{\isacharunderscore}{\kern0pt}def\ \isacommand{by}\isamarkupfalse%
\ {\isacharparenleft}{\kern0pt}metis\ sgn{\isacharunderscore}{\kern0pt}pos\ rep{\isacharunderscore}{\kern0pt}real{\isacharunderscore}{\kern0pt}abs{\isacharunderscore}{\kern0pt}real{\isacharunderscore}{\kern0pt}refl\ rep{\isacharunderscore}{\kern0pt}real{\isacharunderscore}{\kern0pt}iff\ slope{\isacharunderscore}{\kern0pt}rep{\isacharunderscore}{\kern0pt}real{\isacharparenright}{\kern0pt}\isanewline
\ \ \ \ \isacommand{thus}\isamarkupfalse%
\ {\isacharquery}{\kern0pt}case\ \isacommand{unfolding}\isamarkupfalse%
\ less{\isacharunderscore}{\kern0pt}eq{\isacharunderscore}{\kern0pt}real{\isacharunderscore}{\kern0pt}def\ \isacommand{by}\isamarkupfalse%
\ blast\isanewline
\ \ \isacommand{qed}\isamarkupfalse%
\isanewline
\isacommand{qed}\isamarkupfalse%
%
\endisatagproof
{\isafoldproof}%
%
\isadelimproof
%
\endisadelimproof
%
\isadelimdocument
%
\endisadelimdocument
%
\isatagdocument
%
\isamarkupsubsection{Completeness%
}
\isamarkuptrue%
%
\endisatagdocument
{\isafolddocument}%
%
\isadelimdocument
%
\endisadelimdocument
%
\begin{isamarkuptext}%
To show that the Eudoxus reals are complete, we first introduce the floor function.%
\end{isamarkuptext}\isamarkuptrue%
\isacommand{instantiation}\isamarkupfalse%
\ real\ {\isacharcolon}{\kern0pt}{\isacharcolon}{\kern0pt}\ floor{\isacharunderscore}{\kern0pt}ceiling\isanewline
\isakeyword{begin}\isanewline
\isanewline
\isacommand{definition}\isamarkupfalse%
\ \isanewline
\ \ {\isachardoublequoteopen}{\isacharparenleft}{\kern0pt}floor\ {\isacharcolon}{\kern0pt}{\isacharcolon}{\kern0pt}\ {\isacharparenleft}{\kern0pt}real\ {\isasymRightarrow}\ int{\isacharparenright}{\kern0pt}{\isacharparenright}{\kern0pt}\ {\isacharequal}{\kern0pt}\ {\isacharparenleft}{\kern0pt}{\isasymlambda}x{\isachardot}{\kern0pt}\ {\isacharparenleft}{\kern0pt}SOME\ z{\isachardot}{\kern0pt}\ of{\isacharunderscore}{\kern0pt}int\ z\ {\isasymle}\ x\ {\isasymand}\ x\ {\isacharless}{\kern0pt}\ of{\isacharunderscore}{\kern0pt}int\ z\ {\isacharplus}{\kern0pt}\ {\isadigit{1}}{\isacharparenright}{\kern0pt}{\isacharparenright}{\kern0pt}{\isachardoublequoteclose}\isanewline
\isanewline
\isacommand{instance}\isamarkupfalse%
\isanewline
%
\isadelimproof
%
\endisadelimproof
%
\isatagproof
\isacommand{proof}\isamarkupfalse%
\isanewline
\ \ \isacommand{fix}\isamarkupfalse%
\ x\ {\isacharcolon}{\kern0pt}{\isacharcolon}{\kern0pt}\ real\isanewline
\ \ \isacommand{show}\isamarkupfalse%
\ {\isachardoublequoteopen}of{\isacharunderscore}{\kern0pt}int\ {\isasymlfloor}x{\isasymrfloor}\ {\isasymle}\ x\ {\isasymand}\ x\ {\isacharless}{\kern0pt}\ of{\isacharunderscore}{\kern0pt}int\ {\isacharparenleft}{\kern0pt}{\isasymlfloor}x{\isasymrfloor}\ {\isacharplus}{\kern0pt}\ {\isadigit{1}}{\isacharparenright}{\kern0pt}{\isachardoublequoteclose}\ \isacommand{using}\isamarkupfalse%
\ someI{\isacharbrackleft}{\kern0pt}of\ {\isachardoublequoteopen}{\isasymlambda}z{\isachardot}{\kern0pt}\ of{\isacharunderscore}{\kern0pt}int\ z\ {\isasymle}\ x\ {\isasymand}\ x\ {\isacharless}{\kern0pt}\ of{\isacharunderscore}{\kern0pt}int\ z\ {\isacharplus}{\kern0pt}\ {\isadigit{1}}{\isachardoublequoteclose}{\isacharbrackright}{\kern0pt}\ floor{\isacharunderscore}{\kern0pt}exists\ \isacommand{by}\isamarkupfalse%
\ {\isacharparenleft}{\kern0pt}fastforce\ simp\ add{\isacharcolon}{\kern0pt}\ floor{\isacharunderscore}{\kern0pt}real{\isacharunderscore}{\kern0pt}def{\isacharparenright}{\kern0pt}\isanewline
\isacommand{qed}\isamarkupfalse%
%
\endisatagproof
{\isafoldproof}%
%
\isadelimproof
\isanewline
%
\endisadelimproof
\isacommand{end}\isamarkupfalse%
\isanewline
\isanewline
\isacommand{lemma}\isamarkupfalse%
\ eudoxus{\isacharunderscore}{\kern0pt}dense{\isacharunderscore}{\kern0pt}rational{\isacharcolon}{\kern0pt}\isanewline
\ \ \isakeyword{fixes}\ x\ y\ {\isacharcolon}{\kern0pt}{\isacharcolon}{\kern0pt}\ real\isanewline
\ \ \isakeyword{assumes}\ {\isachardoublequoteopen}x\ {\isacharless}{\kern0pt}\ y{\isachardoublequoteclose}\isanewline
\ \ \isakeyword{obtains}\ m\ n\ \isakeyword{where}\ {\isachardoublequoteopen}x\ {\isacharless}{\kern0pt}\ {\isacharparenleft}{\kern0pt}of{\isacharunderscore}{\kern0pt}int\ m\ {\isacharslash}{\kern0pt}\ of{\isacharunderscore}{\kern0pt}int\ n{\isacharparenright}{\kern0pt}{\isachardoublequoteclose}\ {\isachardoublequoteopen}{\isacharparenleft}{\kern0pt}of{\isacharunderscore}{\kern0pt}int\ m\ {\isacharslash}{\kern0pt}\ of{\isacharunderscore}{\kern0pt}int\ n{\isacharparenright}{\kern0pt}\ {\isacharless}{\kern0pt}\ y{\isachardoublequoteclose}\ {\isachardoublequoteopen}n\ {\isachargreater}{\kern0pt}\ {\isadigit{0}}{\isachardoublequoteclose}\isanewline
%
\isadelimproof
%
\endisadelimproof
%
\isatagproof
\isacommand{proof}\isamarkupfalse%
\ {\isacharminus}{\kern0pt}\isanewline
\ \ \isacommand{obtain}\isamarkupfalse%
\ n\ {\isacharcolon}{\kern0pt}{\isacharcolon}{\kern0pt}\ int\ \isakeyword{where}\ n{\isacharcolon}{\kern0pt}\ {\isachardoublequoteopen}inverse\ {\isacharparenleft}{\kern0pt}y\ {\isacharminus}{\kern0pt}\ x{\isacharparenright}{\kern0pt}\ {\isacharless}{\kern0pt}\ of{\isacharunderscore}{\kern0pt}int\ n{\isachardoublequoteclose}\ {\isachardoublequoteopen}n\ {\isachargreater}{\kern0pt}\ {\isadigit{0}}{\isachardoublequoteclose}\ \isacommand{by}\isamarkupfalse%
\ {\isacharparenleft}{\kern0pt}metis\ ex{\isacharunderscore}{\kern0pt}less{\isacharunderscore}{\kern0pt}of{\isacharunderscore}{\kern0pt}int\ antisym{\isacharunderscore}{\kern0pt}conv{\isadigit{3}}\ dual{\isacharunderscore}{\kern0pt}order{\isachardot}{\kern0pt}strict{\isacharunderscore}{\kern0pt}trans\ of{\isacharunderscore}{\kern0pt}int{\isacharunderscore}{\kern0pt}less{\isacharunderscore}{\kern0pt}iff{\isacharparenright}{\kern0pt}\isanewline
\ \ \isacommand{hence}\isamarkupfalse%
\ {\isacharasterisk}{\kern0pt}{\isacharcolon}{\kern0pt}\ {\isachardoublequoteopen}inverse\ {\isacharparenleft}{\kern0pt}of{\isacharunderscore}{\kern0pt}int\ n{\isacharparenright}{\kern0pt}\ {\isacharless}{\kern0pt}\ y\ {\isacharminus}{\kern0pt}\ x{\isachardoublequoteclose}\ \isacommand{by}\isamarkupfalse%
\ {\isacharparenleft}{\kern0pt}metis\ assms\ diff{\isacharunderscore}{\kern0pt}gt{\isacharunderscore}{\kern0pt}{\isadigit{0}}{\isacharunderscore}{\kern0pt}iff{\isacharunderscore}{\kern0pt}gt\ inverse{\isacharunderscore}{\kern0pt}inverse{\isacharunderscore}{\kern0pt}eq\ inverse{\isacharunderscore}{\kern0pt}less{\isacharunderscore}{\kern0pt}iff{\isacharunderscore}{\kern0pt}less\ inverse{\isacharunderscore}{\kern0pt}positive{\isacharunderscore}{\kern0pt}iff{\isacharunderscore}{\kern0pt}positive\ of{\isacharunderscore}{\kern0pt}int{\isacharunderscore}{\kern0pt}{\isadigit{0}}{\isacharunderscore}{\kern0pt}less{\isacharunderscore}{\kern0pt}iff{\isacharparenright}{\kern0pt}\isanewline
\ \ \isacommand{define}\isamarkupfalse%
\ m\ \isakeyword{where}\ {\isachardoublequoteopen}m\ {\isacharequal}{\kern0pt}\ floor\ {\isacharparenleft}{\kern0pt}x\ {\isacharasterisk}{\kern0pt}\ of{\isacharunderscore}{\kern0pt}int\ n{\isacharparenright}{\kern0pt}\ {\isacharplus}{\kern0pt}\ {\isadigit{1}}{\isachardoublequoteclose}\isanewline
\ \ \isacommand{{\isacharbraceleft}{\kern0pt}}\isamarkupfalse%
\isanewline
\ \ \ \ \isacommand{assume}\isamarkupfalse%
\ {\isachardoublequoteopen}y\ {\isasymle}\ of{\isacharunderscore}{\kern0pt}int\ m\ {\isacharslash}{\kern0pt}\ of{\isacharunderscore}{\kern0pt}int\ n{\isachardoublequoteclose}\isanewline
\ \ \ \ \isacommand{hence}\isamarkupfalse%
\ {\isachardoublequoteopen}inverse\ {\isacharparenleft}{\kern0pt}of{\isacharunderscore}{\kern0pt}int\ n{\isacharparenright}{\kern0pt}\ {\isacharless}{\kern0pt}\ of{\isacharunderscore}{\kern0pt}int\ m\ {\isacharslash}{\kern0pt}\ of{\isacharunderscore}{\kern0pt}int\ n\ {\isacharminus}{\kern0pt}\ x{\isachardoublequoteclose}\ \isacommand{using}\isamarkupfalse%
\ {\isacharasterisk}{\kern0pt}\ \isacommand{by}\isamarkupfalse%
\ linarith\isanewline
\ \ \ \ \isacommand{hence}\isamarkupfalse%
\ {\isachardoublequoteopen}x\ {\isacharless}{\kern0pt}\ {\isacharparenleft}{\kern0pt}of{\isacharunderscore}{\kern0pt}int\ m\ {\isacharminus}{\kern0pt}\ {\isadigit{1}}{\isacharparenright}{\kern0pt}\ {\isacharslash}{\kern0pt}\ of{\isacharunderscore}{\kern0pt}int\ n{\isachardoublequoteclose}\ \isacommand{by}\isamarkupfalse%
\ {\isacharparenleft}{\kern0pt}simp\ add{\isacharcolon}{\kern0pt}\ diff{\isacharunderscore}{\kern0pt}divide{\isacharunderscore}{\kern0pt}distrib\ inverse{\isacharunderscore}{\kern0pt}eq{\isacharunderscore}{\kern0pt}divide{\isacharparenright}{\kern0pt}\isanewline
\ \ \ \ \isacommand{hence}\isamarkupfalse%
\ False\ \isacommand{unfolding}\isamarkupfalse%
\ m{\isacharunderscore}{\kern0pt}def\ \isacommand{using}\isamarkupfalse%
\ n{\isacharparenleft}{\kern0pt}{\isadigit{2}}{\isacharparenright}{\kern0pt}\ divide{\isacharunderscore}{\kern0pt}le{\isacharunderscore}{\kern0pt}eq\ linorder{\isacharunderscore}{\kern0pt}not{\isacharunderscore}{\kern0pt}less\ \isacommand{by}\isamarkupfalse%
\ fastforce\isanewline
\ \ \isacommand{{\isacharbraceright}{\kern0pt}}\isamarkupfalse%
\isanewline
\ \ \isacommand{moreover}\isamarkupfalse%
\ \isacommand{have}\isamarkupfalse%
\ {\isachardoublequoteopen}x\ {\isacharless}{\kern0pt}\ of{\isacharunderscore}{\kern0pt}int\ m\ {\isacharslash}{\kern0pt}\ of{\isacharunderscore}{\kern0pt}int\ n{\isachardoublequoteclose}\ \isacommand{unfolding}\isamarkupfalse%
\ m{\isacharunderscore}{\kern0pt}def\ \isacommand{by}\isamarkupfalse%
\ {\isacharparenleft}{\kern0pt}meson\ n{\isacharparenleft}{\kern0pt}{\isadigit{2}}{\isacharparenright}{\kern0pt}\ floor{\isacharunderscore}{\kern0pt}correct\ mult{\isacharunderscore}{\kern0pt}imp{\isacharunderscore}{\kern0pt}less{\isacharunderscore}{\kern0pt}div{\isacharunderscore}{\kern0pt}pos\ of{\isacharunderscore}{\kern0pt}int{\isacharunderscore}{\kern0pt}pos{\isacharparenright}{\kern0pt}\isanewline
\ \ \isacommand{ultimately}\isamarkupfalse%
\ \isacommand{show}\isamarkupfalse%
\ {\isacharquery}{\kern0pt}thesis\ \isacommand{using}\isamarkupfalse%
\ that\ n\ \isacommand{by}\isamarkupfalse%
\ fastforce\isanewline
\isacommand{qed}\isamarkupfalse%
%
\endisatagproof
{\isafoldproof}%
%
\isadelimproof
%
\endisadelimproof
%
\begin{isamarkuptext}%
The Eudoxus reals are a complete field.%
\end{isamarkuptext}\isamarkuptrue%
\isacommand{lemma}\isamarkupfalse%
\ eudoxus{\isacharunderscore}{\kern0pt}complete{\isacharcolon}{\kern0pt}\isanewline
\ \ \isakeyword{assumes}\ {\isachardoublequoteopen}S\ {\isasymnoteq}\ {\isacharbraceleft}{\kern0pt}{\isacharbraceright}{\kern0pt}{\isachardoublequoteclose}\ {\isachardoublequoteopen}bdd{\isacharunderscore}{\kern0pt}above\ S{\isachardoublequoteclose}\isanewline
\ \ \isakeyword{obtains}\ u\ {\isacharcolon}{\kern0pt}{\isacharcolon}{\kern0pt}\ real\ \isakeyword{where}\ {\isachardoublequoteopen}{\isasymAnd}s{\isachardot}{\kern0pt}\ s\ {\isasymin}\ S\ {\isasymLongrightarrow}\ s\ {\isasymle}\ u{\isachardoublequoteclose}\ {\isachardoublequoteopen}{\isasymAnd}y{\isachardot}{\kern0pt}\ {\isacharparenleft}{\kern0pt}{\isasymAnd}s{\isachardot}{\kern0pt}\ s\ {\isasymin}\ S\ {\isasymLongrightarrow}\ s\ {\isasymle}\ y{\isacharparenright}{\kern0pt}\ {\isasymLongrightarrow}\ u\ {\isasymle}\ y{\isachardoublequoteclose}\isanewline
%
\isadelimproof
%
\endisadelimproof
%
\isatagproof
\isacommand{proof}\isamarkupfalse%
\ {\isacharparenleft}{\kern0pt}cases\ {\isachardoublequoteopen}{\isasymexists}u\ {\isasymin}\ S{\isachardot}{\kern0pt}\ {\isasymforall}s\ {\isasymin}\ S{\isachardot}{\kern0pt}\ s\ {\isasymle}\ u{\isachardoublequoteclose}{\isacharparenright}{\kern0pt}\isanewline
\ \ \isacommand{case}\isamarkupfalse%
\ False\isanewline
\ \ \isacommand{hence}\isamarkupfalse%
\ no{\isacharunderscore}{\kern0pt}greatest{\isacharunderscore}{\kern0pt}element{\isacharcolon}{\kern0pt}\ {\isachardoublequoteopen}{\isasymexists}y\ {\isasymin}\ S{\isachardot}{\kern0pt}\ x\ {\isacharless}{\kern0pt}\ y{\isachardoublequoteclose}\ \isakeyword{if}\ {\isachardoublequoteopen}x\ {\isasymin}\ S{\isachardoublequoteclose}\ \isakeyword{for}\ x\ \isacommand{using}\isamarkupfalse%
\ that\ \isacommand{by}\isamarkupfalse%
\ force\isanewline
\ \ \isacommand{define}\isamarkupfalse%
\ u\ {\isacharcolon}{\kern0pt}{\isacharcolon}{\kern0pt}\ {\isachardoublequoteopen}int\ {\isasymRightarrow}\ int{\isachardoublequoteclose}\ \isakeyword{where}\ {\isachardoublequoteopen}u\ {\isacharequal}{\kern0pt}\ {\isacharparenleft}{\kern0pt}{\isasymlambda}z{\isachardot}{\kern0pt}\ sgn\ z\ {\isacharasterisk}{\kern0pt}\ Sup\ {\isacharparenleft}{\kern0pt}{\isacharparenleft}{\kern0pt}{\isasymlambda}x{\isachardot}{\kern0pt}\ {\isasymlfloor}of{\isacharunderscore}{\kern0pt}int\ {\isasymbar}z{\isasymbar}\ {\isacharasterisk}{\kern0pt}\ x{\isasymrfloor}{\isacharparenright}{\kern0pt}\ {\isacharbackquote}{\kern0pt}\ S{\isacharparenright}{\kern0pt}{\isacharparenright}{\kern0pt}{\isachardoublequoteclose}\isanewline
\ \ \isanewline
\ \ \isacommand{have}\isamarkupfalse%
\ bdd{\isacharunderscore}{\kern0pt}above{\isacharunderscore}{\kern0pt}u{\isacharcolon}{\kern0pt}\ {\isachardoublequoteopen}bdd{\isacharunderscore}{\kern0pt}above\ {\isacharparenleft}{\kern0pt}{\isacharparenleft}{\kern0pt}{\isasymlambda}x{\isachardot}{\kern0pt}\ {\isasymlfloor}of{\isacharunderscore}{\kern0pt}int\ {\isasymbar}z{\isasymbar}\ {\isacharasterisk}{\kern0pt}\ x{\isasymrfloor}{\isacharparenright}{\kern0pt}\ {\isacharbackquote}{\kern0pt}\ S{\isacharparenright}{\kern0pt}{\isachardoublequoteclose}\ \isakeyword{for}\ z\ \isacommand{by}\isamarkupfalse%
\ {\isacharparenleft}{\kern0pt}intro\ bdd{\isacharunderscore}{\kern0pt}above{\isacharunderscore}{\kern0pt}image{\isacharunderscore}{\kern0pt}mono{\isacharbrackleft}{\kern0pt}OF\ {\isacharunderscore}{\kern0pt}\ assms{\isacharparenleft}{\kern0pt}{\isadigit{2}}{\isacharparenright}{\kern0pt}{\isacharbrackright}{\kern0pt}\ monoI{\isacharparenright}{\kern0pt}\ {\isacharparenleft}{\kern0pt}simp\ add{\isacharcolon}{\kern0pt}\ floor{\isacharunderscore}{\kern0pt}mono\ mult{\isachardot}{\kern0pt}commute\ mult{\isacharunderscore}{\kern0pt}right{\isacharunderscore}{\kern0pt}mono{\isacharparenright}{\kern0pt}\isanewline
\ \ \isanewline
\ \ \isacommand{have}\isamarkupfalse%
\ u{\isacharunderscore}{\kern0pt}Sup{\isacharunderscore}{\kern0pt}nonneg{\isacharcolon}{\kern0pt}\ {\isachardoublequoteopen}z\ {\isasymge}\ {\isadigit{0}}\ {\isasymLongrightarrow}\ {\isasymlfloor}of{\isacharunderscore}{\kern0pt}int\ z\ {\isacharasterisk}{\kern0pt}\ s{\isasymrfloor}\ {\isasymle}\ u\ z{\isachardoublequoteclose}\ \isakeyword{and}\ \isanewline
\ \ \ \ \ \ \ u{\isacharunderscore}{\kern0pt}Sup{\isacharunderscore}{\kern0pt}nonpos{\isacharcolon}{\kern0pt}\ {\isachardoublequoteopen}z\ {\isasymle}\ {\isadigit{0}}\ {\isasymLongrightarrow}\ {\isacharminus}{\kern0pt}\ {\isasymlfloor}of{\isacharunderscore}{\kern0pt}int\ {\isacharparenleft}{\kern0pt}{\isacharminus}{\kern0pt}\ z{\isacharparenright}{\kern0pt}\ {\isacharasterisk}{\kern0pt}\ s{\isasymrfloor}\ {\isasymge}\ u\ z{\isachardoublequoteclose}\ \isakeyword{if}\ {\isachardoublequoteopen}s\ {\isasymin}\ S{\isachardoublequoteclose}\ \isakeyword{for}\ s\ z\ \isanewline
\ \ \ \ \isacommand{unfolding}\isamarkupfalse%
\ u{\isacharunderscore}{\kern0pt}def\ \isacommand{using}\isamarkupfalse%
\ cSup{\isacharunderscore}{\kern0pt}upper{\isacharbrackleft}{\kern0pt}OF\ {\isacharunderscore}{\kern0pt}\ bdd{\isacharunderscore}{\kern0pt}above{\isacharunderscore}{\kern0pt}u{\isacharcomma}{\kern0pt}\ of\ {\isachardoublequoteopen}{\isasymlfloor}of{\isacharunderscore}{\kern0pt}int\ {\isasymbar}z{\isasymbar}\ {\isacharasterisk}{\kern0pt}\ s{\isasymrfloor}{\isachardoublequoteclose}\ z{\isacharbrackright}{\kern0pt}\ that\ abs{\isacharunderscore}{\kern0pt}of{\isacharunderscore}{\kern0pt}nonpos\ zsgn{\isacharunderscore}{\kern0pt}def\ \isacommand{by}\isamarkupfalse%
\ force{\isacharplus}{\kern0pt}\isanewline
\isanewline
\ \ \isacommand{have}\isamarkupfalse%
\ u{\isacharunderscore}{\kern0pt}mem{\isacharcolon}{\kern0pt}\ {\isachardoublequoteopen}u\ z\ {\isasymin}\ {\isacharparenleft}{\kern0pt}{\isasymlambda}x{\isachardot}{\kern0pt}\ sgn\ z\ {\isacharasterisk}{\kern0pt}\ {\isasymlfloor}of{\isacharunderscore}{\kern0pt}int\ {\isasymbar}z{\isasymbar}\ {\isacharasterisk}{\kern0pt}\ x{\isasymrfloor}{\isacharparenright}{\kern0pt}\ {\isacharbackquote}{\kern0pt}\ S{\isachardoublequoteclose}\ \isakeyword{for}\ z\ \isacommand{unfolding}\isamarkupfalse%
\ u{\isacharunderscore}{\kern0pt}def\ \ \isacommand{using}\isamarkupfalse%
\ int{\isacharunderscore}{\kern0pt}Sup{\isacharunderscore}{\kern0pt}mem{\isacharbrackleft}{\kern0pt}OF\ {\isacharunderscore}{\kern0pt}\ bdd{\isacharunderscore}{\kern0pt}above{\isacharunderscore}{\kern0pt}u{\isacharcomma}{\kern0pt}\ of\ z{\isacharbrackright}{\kern0pt}\ assms\ \isacommand{by}\isamarkupfalse%
\ auto\isanewline
\isanewline
\ \ \isacommand{have}\isamarkupfalse%
\ slope{\isacharcolon}{\kern0pt}\ {\isachardoublequoteopen}slope\ u{\isachardoublequoteclose}\isanewline
\ \ \isacommand{proof}\isamarkupfalse%
\ {\isacharminus}{\kern0pt}\isanewline
\ \ \ \ \isacommand{{\isacharbraceleft}{\kern0pt}}\isamarkupfalse%
\isanewline
\ \ \ \ \ \ \isacommand{fix}\isamarkupfalse%
\ m\ n\ {\isacharcolon}{\kern0pt}{\isacharcolon}{\kern0pt}\ int\ \isacommand{assume}\isamarkupfalse%
\ asm{\isacharcolon}{\kern0pt}\ {\isachardoublequoteopen}m\ {\isachargreater}{\kern0pt}\ {\isadigit{0}}{\isachardoublequoteclose}\ {\isachardoublequoteopen}n\ {\isachargreater}{\kern0pt}\ {\isadigit{0}}{\isachardoublequoteclose}\isanewline
\ \ \ \ \ \ \isacommand{obtain}\isamarkupfalse%
\ x{\isacharunderscore}{\kern0pt}m\ \isakeyword{where}\ x{\isacharunderscore}{\kern0pt}m{\isacharcolon}{\kern0pt}\ {\isachardoublequoteopen}x{\isacharunderscore}{\kern0pt}m\ {\isasymin}\ S{\isachardoublequoteclose}\ {\isachardoublequoteopen}u\ m\ {\isacharequal}{\kern0pt}\ {\isasymlfloor}of{\isacharunderscore}{\kern0pt}int\ m\ {\isacharasterisk}{\kern0pt}\ x{\isacharunderscore}{\kern0pt}m{\isasymrfloor}{\isachardoublequoteclose}\ \isacommand{using}\isamarkupfalse%
\ u{\isacharunderscore}{\kern0pt}mem{\isacharbrackleft}{\kern0pt}of\ m{\isacharbrackright}{\kern0pt}\ asm\ zsgn{\isacharunderscore}{\kern0pt}def\ \isacommand{by}\isamarkupfalse%
\ auto\isanewline
\ \ \ \ \ \ \isacommand{obtain}\isamarkupfalse%
\ x{\isacharunderscore}{\kern0pt}n\ \isakeyword{where}\ x{\isacharunderscore}{\kern0pt}n{\isacharcolon}{\kern0pt}\ {\isachardoublequoteopen}x{\isacharunderscore}{\kern0pt}n\ {\isasymin}\ S{\isachardoublequoteclose}\ {\isachardoublequoteopen}u\ n\ {\isacharequal}{\kern0pt}\ {\isasymlfloor}of{\isacharunderscore}{\kern0pt}int\ n\ {\isacharasterisk}{\kern0pt}\ x{\isacharunderscore}{\kern0pt}n{\isasymrfloor}{\isachardoublequoteclose}\ \isacommand{using}\isamarkupfalse%
\ u{\isacharunderscore}{\kern0pt}mem{\isacharbrackleft}{\kern0pt}of\ n{\isacharbrackright}{\kern0pt}\ asm\ zsgn{\isacharunderscore}{\kern0pt}def\ \isacommand{by}\isamarkupfalse%
\ auto\isanewline
\ \ \ \ \ \ \isacommand{obtain}\isamarkupfalse%
\ x{\isacharunderscore}{\kern0pt}m{\isacharunderscore}{\kern0pt}n\ \isakeyword{where}\ x{\isacharunderscore}{\kern0pt}m{\isacharunderscore}{\kern0pt}n{\isacharcolon}{\kern0pt}\ {\isachardoublequoteopen}x{\isacharunderscore}{\kern0pt}m{\isacharunderscore}{\kern0pt}n\ {\isasymin}\ S{\isachardoublequoteclose}\ {\isachardoublequoteopen}u\ {\isacharparenleft}{\kern0pt}m\ {\isacharplus}{\kern0pt}\ n{\isacharparenright}{\kern0pt}\ {\isacharequal}{\kern0pt}\ {\isasymlfloor}of{\isacharunderscore}{\kern0pt}int\ {\isacharparenleft}{\kern0pt}m\ {\isacharplus}{\kern0pt}\ n{\isacharparenright}{\kern0pt}\ {\isacharasterisk}{\kern0pt}\ x{\isacharunderscore}{\kern0pt}m{\isacharunderscore}{\kern0pt}n{\isasymrfloor}{\isachardoublequoteclose}\ \isacommand{using}\isamarkupfalse%
\ u{\isacharunderscore}{\kern0pt}mem{\isacharbrackleft}{\kern0pt}of\ {\isachardoublequoteopen}m\ {\isacharplus}{\kern0pt}\ n{\isachardoublequoteclose}{\isacharbrackright}{\kern0pt}\ asm\ zsgn{\isacharunderscore}{\kern0pt}def\ \isacommand{by}\isamarkupfalse%
\ auto\isanewline
\isanewline
\ \ \ \ \ \ \isacommand{define}\isamarkupfalse%
\ x\ \isakeyword{where}\ {\isachardoublequoteopen}x\ {\isacharequal}{\kern0pt}\ max\ {\isacharparenleft}{\kern0pt}max\ x{\isacharunderscore}{\kern0pt}m\ x{\isacharunderscore}{\kern0pt}n{\isacharparenright}{\kern0pt}\ x{\isacharunderscore}{\kern0pt}m{\isacharunderscore}{\kern0pt}n{\isachardoublequoteclose}\isanewline
\ \ \ \ \ \ \isacommand{have}\isamarkupfalse%
\ x{\isacharcolon}{\kern0pt}\ {\isachardoublequoteopen}x\ {\isasymin}\ S{\isachardoublequoteclose}\ \isacommand{unfolding}\isamarkupfalse%
\ x{\isacharunderscore}{\kern0pt}def\ \isacommand{using}\isamarkupfalse%
\ x{\isacharunderscore}{\kern0pt}m\ x{\isacharunderscore}{\kern0pt}n\ x{\isacharunderscore}{\kern0pt}m{\isacharunderscore}{\kern0pt}n\ \isacommand{by}\isamarkupfalse%
\ linarith\isanewline
\isanewline
\ \ \ \ \ \ \isacommand{have}\isamarkupfalse%
\ {\isachardoublequoteopen}x\ {\isasymge}\ x{\isacharunderscore}{\kern0pt}m{\isachardoublequoteclose}\ {\isachardoublequoteopen}x\ {\isasymge}\ x{\isacharunderscore}{\kern0pt}n{\isachardoublequoteclose}\ {\isachardoublequoteopen}x\ {\isasymge}\ x{\isacharunderscore}{\kern0pt}m{\isacharunderscore}{\kern0pt}n{\isachardoublequoteclose}\ \isacommand{unfolding}\isamarkupfalse%
\ x{\isacharunderscore}{\kern0pt}def\ \isacommand{by}\isamarkupfalse%
\ linarith{\isacharplus}{\kern0pt}\isanewline
\ \ \ \ \ \ \isacommand{hence}\isamarkupfalse%
\ {\isachardoublequoteopen}u\ m\ {\isasymle}\ {\isasymlfloor}of{\isacharunderscore}{\kern0pt}int\ m\ {\isacharasterisk}{\kern0pt}\ x{\isasymrfloor}{\isachardoublequoteclose}\ {\isachardoublequoteopen}u\ n\ {\isasymle}\ {\isasymlfloor}of{\isacharunderscore}{\kern0pt}int\ n\ {\isacharasterisk}{\kern0pt}\ x{\isasymrfloor}{\isachardoublequoteclose}\ {\isachardoublequoteopen}u\ {\isacharparenleft}{\kern0pt}m\ {\isacharplus}{\kern0pt}\ n{\isacharparenright}{\kern0pt}\ {\isasymle}\ {\isasymlfloor}of{\isacharunderscore}{\kern0pt}int\ {\isacharparenleft}{\kern0pt}m\ {\isacharplus}{\kern0pt}\ n{\isacharparenright}{\kern0pt}\ {\isacharasterisk}{\kern0pt}\ x{\isasymrfloor}{\isachardoublequoteclose}\ \isanewline
\ \ \ \ \ \ \ \ \isacommand{unfolding}\isamarkupfalse%
\ x{\isacharunderscore}{\kern0pt}m\ x{\isacharunderscore}{\kern0pt}n\ x{\isacharunderscore}{\kern0pt}m{\isacharunderscore}{\kern0pt}n\ \isacommand{by}\isamarkupfalse%
\ {\isacharparenleft}{\kern0pt}meson\ asm\ floor{\isacharunderscore}{\kern0pt}less{\isacharunderscore}{\kern0pt}cancel\ linorder{\isacharunderscore}{\kern0pt}not{\isacharunderscore}{\kern0pt}less\ mult{\isacharunderscore}{\kern0pt}le{\isacharunderscore}{\kern0pt}cancel{\isacharunderscore}{\kern0pt}iff{\isadigit{2}}\ of{\isacharunderscore}{\kern0pt}int{\isacharunderscore}{\kern0pt}{\isadigit{0}}{\isacharunderscore}{\kern0pt}less{\isacharunderscore}{\kern0pt}iff\ add{\isacharunderscore}{\kern0pt}pos{\isacharunderscore}{\kern0pt}pos{\isacharparenright}{\kern0pt}{\isacharplus}{\kern0pt}\ \ \ \ \ \ \ \isanewline
\ \ \ \ \ \ \isacommand{hence}\isamarkupfalse%
\ {\isachardoublequoteopen}u\ m\ {\isacharequal}{\kern0pt}\ {\isasymlfloor}of{\isacharunderscore}{\kern0pt}int\ m\ {\isacharasterisk}{\kern0pt}\ x{\isasymrfloor}{\isachardoublequoteclose}\ {\isachardoublequoteopen}u\ n\ {\isacharequal}{\kern0pt}\ {\isasymlfloor}of{\isacharunderscore}{\kern0pt}int\ n\ {\isacharasterisk}{\kern0pt}\ x{\isasymrfloor}{\isachardoublequoteclose}\ {\isachardoublequoteopen}u\ {\isacharparenleft}{\kern0pt}m\ {\isacharplus}{\kern0pt}\ n{\isacharparenright}{\kern0pt}\ {\isacharequal}{\kern0pt}\ {\isasymlfloor}of{\isacharunderscore}{\kern0pt}int\ m\ {\isacharasterisk}{\kern0pt}\ x\ {\isacharplus}{\kern0pt}\ of{\isacharunderscore}{\kern0pt}int\ n\ {\isacharasterisk}{\kern0pt}\ x{\isasymrfloor}{\isachardoublequoteclose}\ \isanewline
\ \ \ \ \ \ \ \ \isacommand{using}\isamarkupfalse%
\ u{\isacharunderscore}{\kern0pt}Sup{\isacharunderscore}{\kern0pt}nonneg{\isacharbrackleft}{\kern0pt}OF\ x{\isacharparenleft}{\kern0pt}{\isadigit{1}}{\isacharparenright}{\kern0pt}{\isacharcomma}{\kern0pt}\ of\ m{\isacharbrackright}{\kern0pt}\ u{\isacharunderscore}{\kern0pt}Sup{\isacharunderscore}{\kern0pt}nonneg{\isacharbrackleft}{\kern0pt}OF\ x{\isacharparenleft}{\kern0pt}{\isadigit{1}}{\isacharparenright}{\kern0pt}{\isacharcomma}{\kern0pt}\ of\ n{\isacharbrackright}{\kern0pt}\ u{\isacharunderscore}{\kern0pt}Sup{\isacharunderscore}{\kern0pt}nonneg{\isacharbrackleft}{\kern0pt}OF\ x{\isacharparenleft}{\kern0pt}{\isadigit{1}}{\isacharparenright}{\kern0pt}{\isacharcomma}{\kern0pt}\ of\ {\isachardoublequoteopen}m\ {\isacharplus}{\kern0pt}\ n{\isachardoublequoteclose}{\isacharbrackright}{\kern0pt}\ asm\ add{\isacharunderscore}{\kern0pt}pos{\isacharunderscore}{\kern0pt}pos{\isacharbrackleft}{\kern0pt}OF\ asm{\isacharbrackright}{\kern0pt}\ \isacommand{by}\isamarkupfalse%
\ {\isacharparenleft}{\kern0pt}force\ simp\ add{\isacharcolon}{\kern0pt}\ distrib{\isacharunderscore}{\kern0pt}right{\isacharparenright}{\kern0pt}{\isacharplus}{\kern0pt}\isanewline
\ \ \ \ \ \ \isacommand{moreover}\isamarkupfalse%
\ \isanewline
\ \ \ \ \ \ \isacommand{{\isacharbraceleft}{\kern0pt}}\isamarkupfalse%
\isanewline
\ \ \ \ \ \ \ \ \isacommand{fix}\isamarkupfalse%
\ a\ b\ {\isacharcolon}{\kern0pt}{\isacharcolon}{\kern0pt}\ real\isanewline
\ \ \ \ \ \ \ \ \isacommand{have}\isamarkupfalse%
\ {\isachardoublequoteopen}a\ {\isacharminus}{\kern0pt}\ of{\isacharunderscore}{\kern0pt}int\ {\isasymlfloor}a{\isasymrfloor}\ {\isasymin}\ {\isacharbraceleft}{\kern0pt}{\isadigit{0}}{\isachardot}{\kern0pt}{\isachardot}{\kern0pt}{\isacharless}{\kern0pt}{\isadigit{1}}{\isacharbraceright}{\kern0pt}{\isachardoublequoteclose}\ \isacommand{using}\isamarkupfalse%
\ floor{\isacharunderscore}{\kern0pt}less{\isacharunderscore}{\kern0pt}one\ \isacommand{by}\isamarkupfalse%
\ fastforce\isanewline
\ \ \ \ \ \ \ \ \isacommand{moreover}\isamarkupfalse%
\ \isacommand{have}\isamarkupfalse%
\ {\isachardoublequoteopen}b\ {\isacharminus}{\kern0pt}\ of{\isacharunderscore}{\kern0pt}int\ {\isasymlfloor}b{\isasymrfloor}\ {\isasymin}\ {\isacharbraceleft}{\kern0pt}{\isadigit{0}}{\isachardot}{\kern0pt}{\isachardot}{\kern0pt}{\isacharless}{\kern0pt}{\isadigit{1}}{\isacharbraceright}{\kern0pt}{\isachardoublequoteclose}\ \isacommand{using}\isamarkupfalse%
\ floor{\isacharunderscore}{\kern0pt}less{\isacharunderscore}{\kern0pt}one\ \isacommand{by}\isamarkupfalse%
\ fastforce\isanewline
\ \ \ \ \ \ \ \ \isacommand{ultimately}\isamarkupfalse%
\ \isacommand{have}\isamarkupfalse%
\ {\isachardoublequoteopen}{\isacharparenleft}{\kern0pt}a\ {\isacharminus}{\kern0pt}\ of{\isacharunderscore}{\kern0pt}int\ {\isasymlfloor}a{\isasymrfloor}{\isacharparenright}{\kern0pt}\ {\isacharplus}{\kern0pt}\ {\isacharparenleft}{\kern0pt}b\ {\isacharminus}{\kern0pt}\ of{\isacharunderscore}{\kern0pt}int\ {\isasymlfloor}b{\isasymrfloor}{\isacharparenright}{\kern0pt}\ {\isasymin}\ {\isacharbraceleft}{\kern0pt}{\isadigit{0}}{\isachardot}{\kern0pt}{\isachardot}{\kern0pt}{\isacharless}{\kern0pt}{\isadigit{2}}{\isacharbraceright}{\kern0pt}{\isachardoublequoteclose}\ \isacommand{unfolding}\isamarkupfalse%
\ atLeastLessThan{\isacharunderscore}{\kern0pt}def\ \isacommand{by}\isamarkupfalse%
\ simp\isanewline
\ \ \ \ \ \ \ \ \isacommand{hence}\isamarkupfalse%
\ {\isachardoublequoteopen}{\isacharparenleft}{\kern0pt}a\ {\isacharplus}{\kern0pt}\ b{\isacharparenright}{\kern0pt}\ {\isacharminus}{\kern0pt}\ {\isacharparenleft}{\kern0pt}of{\isacharunderscore}{\kern0pt}int\ {\isasymlfloor}a{\isasymrfloor}\ {\isacharplus}{\kern0pt}\ of{\isacharunderscore}{\kern0pt}int\ {\isasymlfloor}b{\isasymrfloor}{\isacharparenright}{\kern0pt}\ {\isasymin}\ {\isacharbraceleft}{\kern0pt}{\isadigit{0}}{\isachardot}{\kern0pt}{\isachardot}{\kern0pt}{\isacharless}{\kern0pt}{\isadigit{2}}{\isacharbraceright}{\kern0pt}{\isachardoublequoteclose}\ \isacommand{by}\isamarkupfalse%
\ {\isacharparenleft}{\kern0pt}simp\ add{\isacharcolon}{\kern0pt}\ diff{\isacharunderscore}{\kern0pt}add{\isacharunderscore}{\kern0pt}eq{\isacharparenright}{\kern0pt}\isanewline
\ \ \ \ \ \ \ \ \isacommand{hence}\isamarkupfalse%
\ {\isachardoublequoteopen}{\isasymlfloor}a\ {\isacharplus}{\kern0pt}\ b\ {\isacharminus}{\kern0pt}\ {\isacharparenleft}{\kern0pt}of{\isacharunderscore}{\kern0pt}int\ {\isasymlfloor}a{\isasymrfloor}\ {\isacharplus}{\kern0pt}\ of{\isacharunderscore}{\kern0pt}int\ {\isasymlfloor}b{\isasymrfloor}{\isacharparenright}{\kern0pt}{\isasymrfloor}\ {\isasymin}\ {\isacharbraceleft}{\kern0pt}{\isadigit{0}}{\isachardot}{\kern0pt}{\isachardot}{\kern0pt}{\isacharless}{\kern0pt}{\isadigit{2}}{\isacharbraceright}{\kern0pt}{\isachardoublequoteclose}\ \isacommand{by}\isamarkupfalse%
\ simp\isanewline
\ \ \ \ \ \ \ \ \isacommand{hence}\isamarkupfalse%
\ {\isachardoublequoteopen}{\isasymlfloor}a\ {\isacharplus}{\kern0pt}\ b{\isasymrfloor}\ {\isacharminus}{\kern0pt}\ {\isacharparenleft}{\kern0pt}{\isasymlfloor}a{\isasymrfloor}\ {\isacharplus}{\kern0pt}\ {\isasymlfloor}b{\isasymrfloor}{\isacharparenright}{\kern0pt}\ {\isasymin}\ {\isacharbraceleft}{\kern0pt}{\isadigit{0}}{\isachardot}{\kern0pt}{\isachardot}{\kern0pt}{\isacharless}{\kern0pt}{\isadigit{2}}{\isacharbraceright}{\kern0pt}{\isachardoublequoteclose}\ \isacommand{by}\isamarkupfalse%
\ {\isacharparenleft}{\kern0pt}metis\ floor{\isacharunderscore}{\kern0pt}diff{\isacharunderscore}{\kern0pt}of{\isacharunderscore}{\kern0pt}int\ of{\isacharunderscore}{\kern0pt}int{\isacharunderscore}{\kern0pt}add{\isacharparenright}{\kern0pt}\isanewline
\ \ \ \ \ \ \isacommand{{\isacharbraceright}{\kern0pt}}\isamarkupfalse%
\isanewline
\ \ \ \ \ \ \isacommand{ultimately}\isamarkupfalse%
\ \isacommand{have}\isamarkupfalse%
\ {\isachardoublequoteopen}{\isasymbar}u\ {\isacharparenleft}{\kern0pt}m\ {\isacharplus}{\kern0pt}\ n{\isacharparenright}{\kern0pt}\ {\isacharminus}{\kern0pt}\ {\isacharparenleft}{\kern0pt}u\ m\ {\isacharplus}{\kern0pt}\ u\ n{\isacharparenright}{\kern0pt}{\isasymbar}\ {\isasymle}\ {\isadigit{2}}{\isachardoublequoteclose}\ \isacommand{by}\isamarkupfalse%
\ {\isacharparenleft}{\kern0pt}metis\ abs{\isacharunderscore}{\kern0pt}of{\isacharunderscore}{\kern0pt}nonneg\ atLeastLessThan{\isacharunderscore}{\kern0pt}iff\ nless{\isacharunderscore}{\kern0pt}le{\isacharparenright}{\kern0pt}\isanewline
\ \ \ \ \isacommand{{\isacharbraceright}{\kern0pt}}\isamarkupfalse%
\isanewline
\ \ \ \ \isacommand{moreover}\isamarkupfalse%
\ \isacommand{have}\isamarkupfalse%
\ {\isachardoublequoteopen}u\ z\ {\isacharequal}{\kern0pt}\ {\isacharminus}{\kern0pt}\ u\ {\isacharparenleft}{\kern0pt}{\isacharminus}{\kern0pt}\ z{\isacharparenright}{\kern0pt}{\isachardoublequoteclose}\ \isakeyword{for}\ z\ \isacommand{unfolding}\isamarkupfalse%
\ u{\isacharunderscore}{\kern0pt}def\ \isacommand{by}\isamarkupfalse%
\ simp\isanewline
\ \ \ \ \isacommand{ultimately}\isamarkupfalse%
\ \isacommand{show}\isamarkupfalse%
\ {\isacharquery}{\kern0pt}thesis\ \isacommand{using}\isamarkupfalse%
\ slope{\isacharunderscore}{\kern0pt}odd\ \isacommand{by}\isamarkupfalse%
\ blast\isanewline
\ \ \isacommand{qed}\isamarkupfalse%
\isanewline
\ \ \isacommand{{\isacharbraceleft}{\kern0pt}}\isamarkupfalse%
\isanewline
\ \ \ \ \isacommand{fix}\isamarkupfalse%
\ s\ \isacommand{assume}\isamarkupfalse%
\ {\isachardoublequoteopen}s\ {\isasymin}\ S{\isachardoublequoteclose}\isanewline
\ \ \ \ \isacommand{then}\isamarkupfalse%
\ \isacommand{obtain}\isamarkupfalse%
\ y\ \isakeyword{where}\ y{\isacharcolon}{\kern0pt}\ {\isachardoublequoteopen}s\ {\isacharless}{\kern0pt}\ y{\isachardoublequoteclose}\ {\isachardoublequoteopen}y\ {\isasymin}\ S{\isachardoublequoteclose}\ \isacommand{using}\isamarkupfalse%
\ no{\isacharunderscore}{\kern0pt}greatest{\isacharunderscore}{\kern0pt}element\ \isacommand{by}\isamarkupfalse%
\ blast\isanewline
\ \ \ \ \isacommand{then}\isamarkupfalse%
\ \isacommand{obtain}\isamarkupfalse%
\ m\ n\ {\isacharcolon}{\kern0pt}{\isacharcolon}{\kern0pt}\ int\ \isakeyword{where}\ {\isacharasterisk}{\kern0pt}{\isacharcolon}{\kern0pt}\ {\isachardoublequoteopen}s\ {\isacharless}{\kern0pt}\ {\isacharparenleft}{\kern0pt}of{\isacharunderscore}{\kern0pt}int\ m\ {\isacharslash}{\kern0pt}\ of{\isacharunderscore}{\kern0pt}int\ n{\isacharparenright}{\kern0pt}{\isachardoublequoteclose}\ {\isachardoublequoteopen}{\isacharparenleft}{\kern0pt}of{\isacharunderscore}{\kern0pt}int\ m\ {\isacharslash}{\kern0pt}\ of{\isacharunderscore}{\kern0pt}int\ n{\isacharparenright}{\kern0pt}\ {\isacharless}{\kern0pt}\ y{\isachardoublequoteclose}\ {\isachardoublequoteopen}n\ {\isachargreater}{\kern0pt}\ {\isadigit{0}}{\isachardoublequoteclose}\ \isacommand{using}\isamarkupfalse%
\ eudoxus{\isacharunderscore}{\kern0pt}dense{\isacharunderscore}{\kern0pt}rational\ \isacommand{by}\isamarkupfalse%
\ blast\isanewline
\ \ \ \ \isacommand{hence}\isamarkupfalse%
\ n{\isacharunderscore}{\kern0pt}nonneg{\isacharcolon}{\kern0pt}\ {\isachardoublequoteopen}n\ {\isasymge}\ {\isadigit{0}}{\isachardoublequoteclose}\ \isacommand{by}\isamarkupfalse%
\ simp\isanewline
\ \ \ \ \isacommand{{\isacharbraceleft}{\kern0pt}}\isamarkupfalse%
\isanewline
\ \ \ \ \ \ \isacommand{fix}\isamarkupfalse%
\ z\ {\isacharcolon}{\kern0pt}{\isacharcolon}{\kern0pt}\ int\ \isacommand{assume}\isamarkupfalse%
\ z{\isacharunderscore}{\kern0pt}nonneg{\isacharcolon}{\kern0pt}\ {\isachardoublequoteopen}z\ {\isasymge}\ {\isadigit{0}}{\isachardoublequoteclose}\isanewline
\ \ \ \ \ \ \isacommand{have}\isamarkupfalse%
\ {\isachardoublequoteopen}z\ {\isacharasterisk}{\kern0pt}\ m\ {\isacharequal}{\kern0pt}\ {\isasymlfloor}of{\isacharunderscore}{\kern0pt}int\ {\isacharparenleft}{\kern0pt}z\ {\isacharasterisk}{\kern0pt}\ n{\isacharparenright}{\kern0pt}\ {\isacharasterisk}{\kern0pt}\ {\isacharparenleft}{\kern0pt}of{\isacharunderscore}{\kern0pt}int\ m\ {\isacharslash}{\kern0pt}\ of{\isacharunderscore}{\kern0pt}int\ n{\isacharparenright}{\kern0pt}\ {\isacharcolon}{\kern0pt}{\isacharcolon}{\kern0pt}\ real{\isasymrfloor}{\isachardoublequoteclose}\ \isacommand{using}\isamarkupfalse%
\ {\isacharasterisk}{\kern0pt}{\isacharparenleft}{\kern0pt}{\isadigit{3}}{\isacharparenright}{\kern0pt}\ \isacommand{by}\isamarkupfalse%
\ simp\ {\isacharparenleft}{\kern0pt}auto\ simp\ only{\isacharcolon}{\kern0pt}\ of{\isacharunderscore}{\kern0pt}int{\isacharunderscore}{\kern0pt}mult{\isacharbrackleft}{\kern0pt}symmetric{\isacharbrackright}{\kern0pt}\ floor{\isacharunderscore}{\kern0pt}of{\isacharunderscore}{\kern0pt}int{\isacharparenright}{\kern0pt}\isanewline
\ \ \ \ \ \ \isacommand{also}\isamarkupfalse%
\ \isacommand{have}\isamarkupfalse%
\ {\isachardoublequoteopen}{\isachardot}{\kern0pt}{\isachardot}{\kern0pt}{\isachardot}{\kern0pt}\ {\isasymle}\ {\isasymlfloor}of{\isacharunderscore}{\kern0pt}int\ {\isacharparenleft}{\kern0pt}z\ {\isacharasterisk}{\kern0pt}\ n{\isacharparenright}{\kern0pt}\ {\isacharasterisk}{\kern0pt}\ y{\isasymrfloor}{\isachardoublequoteclose}\ \isacommand{using}\isamarkupfalse%
\ {\isacharasterisk}{\kern0pt}{\isacharparenleft}{\kern0pt}{\isadigit{2}}{\isacharparenright}{\kern0pt}\ \isacommand{by}\isamarkupfalse%
\ {\isacharparenleft}{\kern0pt}meson\ floor{\isacharunderscore}{\kern0pt}mono\ mult{\isacharunderscore}{\kern0pt}left{\isacharunderscore}{\kern0pt}mono\ n{\isacharunderscore}{\kern0pt}nonneg\ nless{\isacharunderscore}{\kern0pt}le\ of{\isacharunderscore}{\kern0pt}int{\isacharunderscore}{\kern0pt}nonneg\ z{\isacharunderscore}{\kern0pt}nonneg\ zero{\isacharunderscore}{\kern0pt}le{\isacharunderscore}{\kern0pt}mult{\isacharunderscore}{\kern0pt}iff{\isacharparenright}{\kern0pt}\isanewline
\ \ \ \ \ \ \isacommand{also}\isamarkupfalse%
\ \isacommand{have}\isamarkupfalse%
\ {\isachardoublequoteopen}{\isachardot}{\kern0pt}{\isachardot}{\kern0pt}{\isachardot}{\kern0pt}\ {\isasymle}\ u\ {\isacharparenleft}{\kern0pt}z\ {\isacharasterisk}{\kern0pt}\ n{\isacharparenright}{\kern0pt}{\isachardoublequoteclose}\ \isacommand{using}\isamarkupfalse%
\ u{\isacharunderscore}{\kern0pt}Sup{\isacharunderscore}{\kern0pt}nonneg{\isacharbrackleft}{\kern0pt}OF\ y{\isacharparenleft}{\kern0pt}{\isadigit{2}}{\isacharparenright}{\kern0pt}{\isacharbrackright}{\kern0pt}\ mult{\isacharunderscore}{\kern0pt}nonneg{\isacharunderscore}{\kern0pt}nonneg{\isacharbrackleft}{\kern0pt}OF\ z{\isacharunderscore}{\kern0pt}nonneg\ n{\isacharunderscore}{\kern0pt}nonneg{\isacharbrackright}{\kern0pt}\ \isacommand{by}\isamarkupfalse%
\ blast\isanewline
\ \ \ \ \ \ \isacommand{finally}\isamarkupfalse%
\ \isacommand{have}\isamarkupfalse%
\ {\isachardoublequoteopen}u\ {\isacharparenleft}{\kern0pt}z\ {\isacharasterisk}{\kern0pt}\ n{\isacharparenright}{\kern0pt}\ {\isasymge}\ z\ {\isacharasterisk}{\kern0pt}\ m{\isachardoublequoteclose}\ \isacommand{{\isachardot}{\kern0pt}}\isamarkupfalse%
\isanewline
\ \ \ \ \isacommand{{\isacharbraceright}{\kern0pt}}\isamarkupfalse%
\isanewline
\ \ \ \ \isacommand{hence}\isamarkupfalse%
\ {\isachardoublequoteopen}abs{\isacharunderscore}{\kern0pt}real\ {\isacharparenleft}{\kern0pt}u\ {\isacharasterisk}{\kern0pt}\isactrlsub e\ {\isacharparenleft}{\kern0pt}{\isacharasterisk}{\kern0pt}{\isacharparenright}{\kern0pt}\ n{\isacharparenright}{\kern0pt}\ {\isasymge}\ of{\isacharunderscore}{\kern0pt}int\ m{\isachardoublequoteclose}\ \isacommand{using}\isamarkupfalse%
\ slope\ \isacommand{unfolding}\isamarkupfalse%
\ real{\isacharunderscore}{\kern0pt}of{\isacharunderscore}{\kern0pt}int\ eudoxus{\isacharunderscore}{\kern0pt}times{\isacharunderscore}{\kern0pt}def\ \isacommand{by}\isamarkupfalse%
\ {\isacharparenleft}{\kern0pt}intro\ abs{\isacharunderscore}{\kern0pt}real{\isacharunderscore}{\kern0pt}leI{\isacharbrackleft}{\kern0pt}\isakeyword{where}\ {\isacharquery}{\kern0pt}N{\isacharequal}{\kern0pt}{\isadigit{0}}{\isacharbrackright}{\kern0pt}{\isacharparenright}{\kern0pt}\ {\isacharparenleft}{\kern0pt}auto\ simp\ add{\isacharcolon}{\kern0pt}\ mult{\isachardot}{\kern0pt}commute{\isacharparenright}{\kern0pt}\ \isanewline
\ \ \ \ \isacommand{moreover}\isamarkupfalse%
\ \isacommand{have}\isamarkupfalse%
\ {\isachardoublequoteopen}abs{\isacharunderscore}{\kern0pt}real\ u\ {\isacharasterisk}{\kern0pt}\ of{\isacharunderscore}{\kern0pt}int\ n\ {\isacharequal}{\kern0pt}\ abs{\isacharunderscore}{\kern0pt}real\ {\isacharparenleft}{\kern0pt}u\ {\isacharasterisk}{\kern0pt}\isactrlsub e\ {\isacharparenleft}{\kern0pt}{\isacharasterisk}{\kern0pt}{\isacharparenright}{\kern0pt}\ n{\isacharparenright}{\kern0pt}{\isachardoublequoteclose}\ \isacommand{unfolding}\isamarkupfalse%
\ real{\isacharunderscore}{\kern0pt}of{\isacharunderscore}{\kern0pt}int\ \isacommand{using}\isamarkupfalse%
\ slope\ \isacommand{by}\isamarkupfalse%
\ {\isacharparenleft}{\kern0pt}simp\ add{\isacharcolon}{\kern0pt}\ eudoxus{\isacharunderscore}{\kern0pt}times{\isacharunderscore}{\kern0pt}def\ comp{\isacharunderscore}{\kern0pt}def{\isacharparenright}{\kern0pt}\isanewline
\ \ \ \ \isacommand{ultimately}\isamarkupfalse%
\ \isacommand{have}\isamarkupfalse%
\ {\isachardoublequoteopen}s\ {\isasymle}\ abs{\isacharunderscore}{\kern0pt}real\ u{\isachardoublequoteclose}\ \isacommand{using}\isamarkupfalse%
\ {\isacharasterisk}{\kern0pt}\ \isacommand{by}\isamarkupfalse%
\ {\isacharparenleft}{\kern0pt}metis\ leI\ mult{\isacharunderscore}{\kern0pt}imp{\isacharunderscore}{\kern0pt}div{\isacharunderscore}{\kern0pt}pos{\isacharunderscore}{\kern0pt}le\ of{\isacharunderscore}{\kern0pt}int{\isacharunderscore}{\kern0pt}{\isadigit{0}}{\isacharunderscore}{\kern0pt}less{\isacharunderscore}{\kern0pt}iff\ order{\isacharunderscore}{\kern0pt}le{\isacharunderscore}{\kern0pt}less{\isacharunderscore}{\kern0pt}trans\ order{\isacharunderscore}{\kern0pt}less{\isacharunderscore}{\kern0pt}asym{\isacharparenright}{\kern0pt}\isanewline
\ \ \isacommand{{\isacharbraceright}{\kern0pt}}\isamarkupfalse%
\isanewline
\ \ \isacommand{moreover}\isamarkupfalse%
\isanewline
\ \ \isacommand{{\isacharbraceleft}{\kern0pt}}\isamarkupfalse%
\isanewline
\ \ \ \ \isacommand{fix}\isamarkupfalse%
\ y\ \isacommand{assume}\isamarkupfalse%
\ asm{\isacharcolon}{\kern0pt}\ {\isachardoublequoteopen}s\ {\isasymle}\ y{\isachardoublequoteclose}\ \isakeyword{if}\ {\isachardoublequoteopen}s\ {\isasymin}\ S{\isachardoublequoteclose}\ \isakeyword{for}\ s\isanewline
\ \ \ \ \isacommand{assume}\isamarkupfalse%
\ {\isachardoublequoteopen}abs{\isacharunderscore}{\kern0pt}real\ u\ {\isachargreater}{\kern0pt}\ y{\isachardoublequoteclose}\isanewline
\ \ \ \ \isacommand{then}\isamarkupfalse%
\ \isacommand{obtain}\isamarkupfalse%
\ m\ n\ {\isacharcolon}{\kern0pt}{\isacharcolon}{\kern0pt}\ int\ \isakeyword{where}\ {\isacharasterisk}{\kern0pt}{\isacharcolon}{\kern0pt}\ {\isachardoublequoteopen}y\ {\isacharless}{\kern0pt}\ {\isacharparenleft}{\kern0pt}of{\isacharunderscore}{\kern0pt}int\ m\ {\isacharslash}{\kern0pt}\ of{\isacharunderscore}{\kern0pt}int\ n{\isacharparenright}{\kern0pt}{\isachardoublequoteclose}\ {\isachardoublequoteopen}{\isacharparenleft}{\kern0pt}of{\isacharunderscore}{\kern0pt}int\ m\ {\isacharslash}{\kern0pt}\ of{\isacharunderscore}{\kern0pt}int\ n{\isacharparenright}{\kern0pt}\ {\isacharless}{\kern0pt}\ abs{\isacharunderscore}{\kern0pt}real\ u{\isachardoublequoteclose}\ {\isachardoublequoteopen}n\ {\isachargreater}{\kern0pt}\ {\isadigit{0}}{\isachardoublequoteclose}\ \isacommand{using}\isamarkupfalse%
\ eudoxus{\isacharunderscore}{\kern0pt}dense{\isacharunderscore}{\kern0pt}rational\ \isacommand{by}\isamarkupfalse%
\ blast\isanewline
\ \ \ \ \isacommand{hence}\isamarkupfalse%
\ {\isachardoublequoteopen}of{\isacharunderscore}{\kern0pt}int\ m\ {\isacharless}{\kern0pt}\ abs{\isacharunderscore}{\kern0pt}real\ u\ {\isacharasterisk}{\kern0pt}\ of{\isacharunderscore}{\kern0pt}int\ n{\isachardoublequoteclose}\ \isacommand{by}\isamarkupfalse%
\ {\isacharparenleft}{\kern0pt}simp\ add{\isacharcolon}{\kern0pt}\ pos{\isacharunderscore}{\kern0pt}divide{\isacharunderscore}{\kern0pt}less{\isacharunderscore}{\kern0pt}eq{\isacharparenright}{\kern0pt}\isanewline
\ \ \ \ \isacommand{hence}\isamarkupfalse%
\ {\isachardoublequoteopen}of{\isacharunderscore}{\kern0pt}int\ m\ {\isacharless}{\kern0pt}\ abs{\isacharunderscore}{\kern0pt}real\ {\isacharparenleft}{\kern0pt}u\ {\isacharasterisk}{\kern0pt}\isactrlsub e\ {\isacharparenleft}{\kern0pt}{\isacharasterisk}{\kern0pt}{\isacharparenright}{\kern0pt}\ n{\isacharparenright}{\kern0pt}{\isachardoublequoteclose}\ \isacommand{unfolding}\isamarkupfalse%
\ real{\isacharunderscore}{\kern0pt}of{\isacharunderscore}{\kern0pt}int\ \isacommand{using}\isamarkupfalse%
\ slope\ \isacommand{by}\isamarkupfalse%
\ {\isacharparenleft}{\kern0pt}simp\ add{\isacharcolon}{\kern0pt}\ eudoxus{\isacharunderscore}{\kern0pt}times{\isacharunderscore}{\kern0pt}def\ comp{\isacharunderscore}{\kern0pt}def{\isacharparenright}{\kern0pt}\isanewline
\ \ \ \ \isacommand{moreover}\isamarkupfalse%
\ \isacommand{have}\isamarkupfalse%
\ {\isachardoublequoteopen}slope\ {\isacharparenleft}{\kern0pt}u\ {\isacharasterisk}{\kern0pt}\isactrlsub e\ {\isacharparenleft}{\kern0pt}{\isacharasterisk}{\kern0pt}{\isacharparenright}{\kern0pt}\ n{\isacharparenright}{\kern0pt}{\isachardoublequoteclose}\ \isacommand{using}\isamarkupfalse%
\ slope\ \isacommand{by}\isamarkupfalse%
\ {\isacharparenleft}{\kern0pt}simp\ add{\isacharcolon}{\kern0pt}\ eudoxus{\isacharunderscore}{\kern0pt}times{\isacharunderscore}{\kern0pt}def{\isacharparenright}{\kern0pt}\isanewline
\ \ \ \ \isacommand{ultimately}\isamarkupfalse%
\ \isacommand{obtain}\isamarkupfalse%
\ z\ \isakeyword{where}\ z{\isacharcolon}{\kern0pt}\ {\isachardoublequoteopen}{\isacharparenleft}{\kern0pt}u\ {\isacharasterisk}{\kern0pt}\isactrlsub e\ {\isacharparenleft}{\kern0pt}{\isacharasterisk}{\kern0pt}{\isacharparenright}{\kern0pt}\ n{\isacharparenright}{\kern0pt}\ z\ {\isachargreater}{\kern0pt}\ m\ {\isacharasterisk}{\kern0pt}\ z{\isachardoublequoteclose}\ {\isachardoublequoteopen}z\ {\isasymge}\ {\isadigit{1}}{\isachardoublequoteclose}\ \isacommand{unfolding}\isamarkupfalse%
\ real{\isacharunderscore}{\kern0pt}of{\isacharunderscore}{\kern0pt}int\ \isacommand{using}\isamarkupfalse%
\ abs{\isacharunderscore}{\kern0pt}real{\isacharunderscore}{\kern0pt}lessD\ \isacommand{by}\isamarkupfalse%
\ blast\isanewline
\ \ \ \ \isacommand{hence}\isamarkupfalse%
\ {\isacharasterisk}{\kern0pt}{\isacharasterisk}{\kern0pt}{\isacharcolon}{\kern0pt}\ {\isachardoublequoteopen}u\ {\isacharparenleft}{\kern0pt}n\ {\isacharasterisk}{\kern0pt}\ z{\isacharparenright}{\kern0pt}\ {\isachargreater}{\kern0pt}\ m\ {\isacharasterisk}{\kern0pt}\ z{\isachardoublequoteclose}\ \isacommand{by}\isamarkupfalse%
\ {\isacharparenleft}{\kern0pt}simp\ add{\isacharcolon}{\kern0pt}\ eudoxus{\isacharunderscore}{\kern0pt}times{\isacharunderscore}{\kern0pt}def\ comp{\isacharunderscore}{\kern0pt}def{\isacharparenright}{\kern0pt}\isanewline
\isanewline
\ \ \ \ \isacommand{obtain}\isamarkupfalse%
\ x\ \isakeyword{where}\ x{\isacharcolon}{\kern0pt}\ {\isachardoublequoteopen}x\ {\isasymin}\ S{\isachardoublequoteclose}\ {\isachardoublequoteopen}u\ {\isacharparenleft}{\kern0pt}n\ {\isacharasterisk}{\kern0pt}\ z{\isacharparenright}{\kern0pt}\ {\isacharequal}{\kern0pt}\ {\isasymlfloor}of{\isacharunderscore}{\kern0pt}int\ {\isacharparenleft}{\kern0pt}n\ {\isacharasterisk}{\kern0pt}\ z{\isacharparenright}{\kern0pt}\ {\isacharasterisk}{\kern0pt}\ x{\isasymrfloor}{\isachardoublequoteclose}\ \isacommand{using}\isamarkupfalse%
\ u{\isacharunderscore}{\kern0pt}mem{\isacharbrackleft}{\kern0pt}of\ {\isachardoublequoteopen}n\ {\isacharasterisk}{\kern0pt}\ z{\isachardoublequoteclose}{\isacharbrackright}{\kern0pt}\ zsgn{\isacharunderscore}{\kern0pt}def{\isacharbrackleft}{\kern0pt}of\ {\isachardoublequoteopen}n\ {\isacharasterisk}{\kern0pt}\ z{\isachardoublequoteclose}{\isacharbrackright}{\kern0pt}\ mult{\isacharunderscore}{\kern0pt}pos{\isacharunderscore}{\kern0pt}pos{\isacharbrackleft}{\kern0pt}OF\ {\isacharasterisk}{\kern0pt}{\isacharparenleft}{\kern0pt}{\isadigit{3}}{\isacharparenright}{\kern0pt}{\isacharcomma}{\kern0pt}\ of\ z{\isacharbrackright}{\kern0pt}\ z{\isacharparenleft}{\kern0pt}{\isadigit{2}}{\isacharparenright}{\kern0pt}\ \isacommand{by}\isamarkupfalse%
\ fastforce\isanewline
\ \ \ \ \isanewline
\ \ \ \ \isacommand{have}\isamarkupfalse%
\ {\isachardoublequoteopen}of{\isacharunderscore}{\kern0pt}int\ {\isacharparenleft}{\kern0pt}n\ {\isacharasterisk}{\kern0pt}\ z{\isacharparenright}{\kern0pt}\ {\isacharasterisk}{\kern0pt}\ x\ {\isasymle}\ of{\isacharunderscore}{\kern0pt}int\ z\ {\isacharasterisk}{\kern0pt}\ of{\isacharunderscore}{\kern0pt}int\ n\ {\isacharasterisk}{\kern0pt}\ y{\isachardoublequoteclose}\ \isacommand{using}\isamarkupfalse%
\ asm{\isacharbrackleft}{\kern0pt}OF\ x{\isacharparenleft}{\kern0pt}{\isadigit{1}}{\isacharparenright}{\kern0pt}{\isacharbrackright}{\kern0pt}\ \isacommand{using}\isamarkupfalse%
\ {\isacharasterisk}{\kern0pt}\ z\ \isacommand{by}\isamarkupfalse%
\ auto\isanewline
\ \ \ \ \isacommand{also}\isamarkupfalse%
\ \isacommand{have}\isamarkupfalse%
\ {\isachardoublequoteopen}{\isachardot}{\kern0pt}{\isachardot}{\kern0pt}{\isachardot}{\kern0pt}\ {\isacharless}{\kern0pt}\ of{\isacharunderscore}{\kern0pt}int\ z\ {\isacharasterisk}{\kern0pt}\ of{\isacharunderscore}{\kern0pt}int\ m{\isachardoublequoteclose}\ \isacommand{using}\isamarkupfalse%
\ {\isacharasterisk}{\kern0pt}\ z\ \isacommand{by}\isamarkupfalse%
\ {\isacharparenleft}{\kern0pt}simp\ add{\isacharcolon}{\kern0pt}\ mult{\isachardot}{\kern0pt}commute\ pos{\isacharunderscore}{\kern0pt}less{\isacharunderscore}{\kern0pt}divide{\isacharunderscore}{\kern0pt}eq{\isacharparenright}{\kern0pt}\isanewline
\ \ \ \ \isacommand{finally}\isamarkupfalse%
\ \isacommand{have}\isamarkupfalse%
\ {\isachardoublequoteopen}of{\isacharunderscore}{\kern0pt}int\ {\isacharparenleft}{\kern0pt}n\ {\isacharasterisk}{\kern0pt}\ z{\isacharparenright}{\kern0pt}\ {\isacharasterisk}{\kern0pt}\ x\ {\isacharless}{\kern0pt}\ of{\isacharunderscore}{\kern0pt}int\ {\isacharparenleft}{\kern0pt}m\ {\isacharasterisk}{\kern0pt}\ z{\isacharparenright}{\kern0pt}{\isachardoublequoteclose}\ \isacommand{by}\isamarkupfalse%
\ {\isacharparenleft}{\kern0pt}simp\ add{\isacharcolon}{\kern0pt}\ mult{\isachardot}{\kern0pt}commute{\isacharparenright}{\kern0pt}\isanewline
\ \ \ \ \isacommand{hence}\isamarkupfalse%
\ False\ \isacommand{using}\isamarkupfalse%
\ {\isacharasterisk}{\kern0pt}{\isacharasterisk}{\kern0pt}\ \isacommand{by}\isamarkupfalse%
\ {\isacharparenleft}{\kern0pt}metis\ floor{\isacharunderscore}{\kern0pt}less{\isacharunderscore}{\kern0pt}iff\ less{\isacharunderscore}{\kern0pt}le{\isacharunderscore}{\kern0pt}not{\isacharunderscore}{\kern0pt}le\ x{\isacharparenleft}{\kern0pt}{\isadigit{2}}{\isacharparenright}{\kern0pt}{\isacharparenright}{\kern0pt}\isanewline
\ \ \isacommand{{\isacharbraceright}{\kern0pt}}\isamarkupfalse%
\isanewline
\ \ \isacommand{ultimately}\isamarkupfalse%
\ \isacommand{show}\isamarkupfalse%
\ {\isacharquery}{\kern0pt}thesis\ \isacommand{using}\isamarkupfalse%
\ that\ \isacommand{by}\isamarkupfalse%
\ force\isanewline
\isacommand{qed}\isamarkupfalse%
\ blast%
\endisatagproof
{\isafoldproof}%
%
\isadelimproof
\isanewline
%
\endisadelimproof
%
\isadelimtheory
\isanewline
%
\endisadelimtheory
%
\isatagtheory
\isacommand{end}\isamarkupfalse%
%
\endisatagtheory
{\isafoldtheory}%
%
\isadelimtheory
%
\endisadelimtheory
%
\end{isabellebody}%
\endinput
%:%file=Eudoxus.tex%:%
%:%6=2%:%
%:%7=3%:%
%:%12=4%:%
%:%13=4%:%
%:%14=5%:%
%:%15=6%:%
%:%29=8%:%
%:%33=10%:%
%:%45=12%:%
%:%47=13%:%
%:%48=13%:%
%:%49=14%:%
%:%50=15%:%
%:%51=16%:%
%:%52=16%:%
%:%53=17%:%
%:%60=18%:%
%:%61=18%:%
%:%62=19%:%
%:%63=19%:%
%:%64=19%:%
%:%65=19%:%
%:%66=20%:%
%:%67=20%:%
%:%68=20%:%
%:%69=20%:%
%:%70=21%:%
%:%71=21%:%
%:%72=21%:%
%:%73=21%:%
%:%74=22%:%
%:%84=24%:%
%:%86=25%:%
%:%87=25%:%
%:%90=26%:%
%:%94=26%:%
%:%95=26%:%
%:%100=26%:%
%:%103=27%:%
%:%104=28%:%
%:%105=28%:%
%:%108=29%:%
%:%112=29%:%
%:%113=29%:%
%:%114=29%:%
%:%119=29%:%
%:%122=30%:%
%:%123=31%:%
%:%124=31%:%
%:%127=32%:%
%:%131=32%:%
%:%132=32%:%
%:%133=32%:%
%:%138=32%:%
%:%141=33%:%
%:%142=34%:%
%:%143=34%:%
%:%144=35%:%
%:%145=36%:%
%:%146=36%:%
%:%147=37%:%
%:%148=38%:%
%:%149=38%:%
%:%150=39%:%
%:%151=40%:%
%:%154=41%:%
%:%158=41%:%
%:%159=41%:%
%:%160=41%:%
%:%165=41%:%
%:%168=42%:%
%:%169=43%:%
%:%170=43%:%
%:%173=44%:%
%:%177=44%:%
%:%178=44%:%
%:%183=44%:%
%:%186=45%:%
%:%187=46%:%
%:%188=46%:%
%:%190=46%:%
%:%194=46%:%
%:%195=46%:%
%:%196=46%:%
%:%203=46%:%
%:%204=47%:%
%:%205=48%:%
%:%206=48%:%
%:%207=49%:%
%:%208=49%:%
%:%209=50%:%
%:%210=51%:%
%:%211=51%:%
%:%212=52%:%
%:%213=52%:%
%:%214=53%:%
%:%215=54%:%
%:%216=54%:%
%:%217=55%:%
%:%218=56%:%
%:%219=56%:%
%:%221=56%:%
%:%225=56%:%
%:%226=56%:%
%:%233=56%:%
%:%234=57%:%
%:%235=58%:%
%:%236=58%:%
%:%237=59%:%
%:%238=60%:%
%:%245=61%:%
%:%246=61%:%
%:%247=62%:%
%:%248=62%:%
%:%249=62%:%
%:%250=62%:%
%:%251=63%:%
%:%252=64%:%
%:%253=64%:%
%:%254=64%:%
%:%255=64%:%
%:%256=65%:%
%:%257=66%:%
%:%258=66%:%
%:%259=66%:%
%:%260=66%:%
%:%261=67%:%
%:%262=68%:%
%:%263=68%:%
%:%264=68%:%
%:%265=68%:%
%:%266=69%:%
%:%267=70%:%
%:%268=70%:%
%:%269=71%:%
%:%270=71%:%
%:%271=71%:%
%:%272=72%:%
%:%273=72%:%
%:%274=72%:%
%:%275=72%:%
%:%276=73%:%
%:%277=73%:%
%:%278=73%:%
%:%279=74%:%
%:%280=75%:%
%:%281=75%:%
%:%282=75%:%
%:%283=75%:%
%:%284=75%:%
%:%285=76%:%
%:%286=76%:%
%:%287=77%:%
%:%288=77%:%
%:%289=78%:%
%:%290=78%:%
%:%291=79%:%
%:%292=79%:%
%:%293=80%:%
%:%294=80%:%
%:%295=81%:%
%:%296=81%:%
%:%297=81%:%
%:%298=81%:%
%:%299=81%:%
%:%300=82%:%
%:%301=82%:%
%:%302=83%:%
%:%303=83%:%
%:%304=84%:%
%:%305=84%:%
%:%306=84%:%
%:%307=85%:%
%:%308=85%:%
%:%309=85%:%
%:%310=85%:%
%:%311=86%:%
%:%312=86%:%
%:%313=86%:%
%:%314=86%:%
%:%315=86%:%
%:%316=87%:%
%:%317=87%:%
%:%318=87%:%
%:%319=88%:%
%:%320=88%:%
%:%321=89%:%
%:%322=89%:%
%:%323=90%:%
%:%324=90%:%
%:%325=90%:%
%:%326=90%:%
%:%327=90%:%
%:%328=91%:%
%:%343=93%:%
%:%355=95%:%
%:%357=96%:%
%:%358=96%:%
%:%359=97%:%
%:%360=98%:%
%:%361=99%:%
%:%362=99%:%
%:%363=100%:%
%:%365=100%:%
%:%369=100%:%
%:%377=100%:%
%:%378=101%:%
%:%379=102%:%
%:%380=102%:%
%:%381=103%:%
%:%382=104%:%
%:%383=104%:%
%:%385=104%:%
%:%389=104%:%
%:%390=104%:%
%:%397=104%:%
%:%398=105%:%
%:%399=105%:%
%:%401=105%:%
%:%405=105%:%
%:%406=105%:%
%:%413=105%:%
%:%414=106%:%
%:%415=107%:%
%:%416=107%:%
%:%418=107%:%
%:%422=107%:%
%:%423=107%:%
%:%424=107%:%
%:%431=107%:%
%:%432=108%:%
%:%433=109%:%
%:%434=109%:%
%:%435=110%:%
%:%436=111%:%
%:%437=112%:%
%:%438=112%:%
%:%439=113%:%
%:%440=114%:%
%:%441=114%:%
%:%442=115%:%
%:%449=116%:%
%:%450=116%:%
%:%451=117%:%
%:%452=117%:%
%:%453=117%:%
%:%454=118%:%
%:%455=118%:%
%:%456=118%:%
%:%457=118%:%
%:%458=119%:%
%:%459=119%:%
%:%460=119%:%
%:%461=119%:%
%:%462=120%:%
%:%468=120%:%
%:%471=121%:%
%:%472=122%:%
%:%473=122%:%
%:%474=123%:%
%:%475=124%:%
%:%476=124%:%
%:%477=125%:%
%:%478=126%:%
%:%481=127%:%
%:%485=127%:%
%:%486=127%:%
%:%487=127%:%
%:%488=127%:%
%:%493=127%:%
%:%496=128%:%
%:%497=129%:%
%:%498=129%:%
%:%499=130%:%
%:%500=131%:%
%:%501=132%:%
%:%502=132%:%
%:%503=133%:%
%:%504=134%:%
%:%505=134%:%
%:%506=135%:%
%:%513=136%:%
%:%514=136%:%
%:%515=137%:%
%:%516=137%:%
%:%517=137%:%
%:%518=138%:%
%:%519=138%:%
%:%520=138%:%
%:%521=138%:%
%:%522=139%:%
%:%523=139%:%
%:%524=139%:%
%:%525=139%:%
%:%526=140%:%
%:%532=140%:%
%:%535=141%:%
%:%536=142%:%
%:%537=142%:%
%:%538=143%:%
%:%539=144%:%
%:%540=144%:%
%:%541=145%:%
%:%542=146%:%
%:%545=147%:%
%:%549=147%:%
%:%550=147%:%
%:%551=147%:%
%:%552=147%:%
%:%557=147%:%
%:%560=148%:%
%:%561=149%:%
%:%562=149%:%
%:%563=150%:%
%:%564=151%:%
%:%565=151%:%
%:%566=152%:%
%:%567=153%:%
%:%568=153%:%
%:%569=154%:%
%:%570=155%:%
%:%573=156%:%
%:%577=156%:%
%:%578=156%:%
%:%579=156%:%
%:%584=156%:%
%:%587=157%:%
%:%588=158%:%
%:%591=158%:%
%:%595=158%:%
%:%603=158%:%
%:%604=159%:%
%:%607=161%:%
%:%609=162%:%
%:%610=162%:%
%:%617=163%:%
%:%618=163%:%
%:%619=164%:%
%:%620=164%:%
%:%621=165%:%
%:%622=165%:%
%:%623=165%:%
%:%624=166%:%
%:%625=166%:%
%:%626=166%:%
%:%627=167%:%
%:%628=167%:%
%:%629=167%:%
%:%630=168%:%
%:%631=168%:%
%:%632=168%:%
%:%633=169%:%
%:%634=169%:%
%:%648=171%:%
%:%660=173%:%
%:%662=174%:%
%:%663=174%:%
%:%664=175%:%
%:%665=176%:%
%:%666=177%:%
%:%667=177%:%
%:%668=178%:%
%:%670=178%:%
%:%674=178%:%
%:%682=178%:%
%:%683=179%:%
%:%684=180%:%
%:%685=180%:%
%:%686=181%:%
%:%687=182%:%
%:%688=182%:%
%:%690=182%:%
%:%694=182%:%
%:%695=182%:%
%:%696=182%:%
%:%703=182%:%
%:%704=183%:%
%:%705=184%:%
%:%706=184%:%
%:%707=185%:%
%:%708=186%:%
%:%709=187%:%
%:%710=187%:%
%:%711=188%:%
%:%712=188%:%
%:%713=189%:%
%:%714=190%:%
%:%715=190%:%
%:%716=191%:%
%:%723=192%:%
%:%724=192%:%
%:%725=193%:%
%:%726=193%:%
%:%727=193%:%
%:%728=194%:%
%:%729=194%:%
%:%730=194%:%
%:%731=194%:%
%:%732=195%:%
%:%733=196%:%
%:%734=196%:%
%:%735=196%:%
%:%736=196%:%
%:%737=196%:%
%:%738=197%:%
%:%739=198%:%
%:%740=198%:%
%:%741=198%:%
%:%742=198%:%
%:%743=199%:%
%:%744=200%:%
%:%745=200%:%
%:%746=200%:%
%:%747=200%:%
%:%748=200%:%
%:%749=201%:%
%:%750=202%:%
%:%751=202%:%
%:%752=203%:%
%:%753=203%:%
%:%754=204%:%
%:%755=204%:%
%:%756=205%:%
%:%757=205%:%
%:%758=205%:%
%:%759=205%:%
%:%760=206%:%
%:%761=206%:%
%:%762=206%:%
%:%763=206%:%
%:%764=206%:%
%:%765=207%:%
%:%766=207%:%
%:%767=207%:%
%:%768=207%:%
%:%769=207%:%
%:%770=208%:%
%:%771=208%:%
%:%772=208%:%
%:%773=208%:%
%:%774=209%:%
%:%775=209%:%
%:%776=210%:%
%:%777=210%:%
%:%778=210%:%
%:%779=210%:%
%:%780=211%:%
%:%781=211%:%
%:%782=211%:%
%:%783=211%:%
%:%784=211%:%
%:%785=212%:%
%:%791=212%:%
%:%794=213%:%
%:%795=214%:%
%:%796=214%:%
%:%797=215%:%
%:%798=215%:%
%:%799=216%:%
%:%800=217%:%
%:%801=217%:%
%:%802=218%:%
%:%803=219%:%
%:%806=220%:%
%:%810=220%:%
%:%811=220%:%
%:%812=221%:%
%:%813=221%:%
%:%814=221%:%
%:%819=221%:%
%:%822=222%:%
%:%823=223%:%
%:%824=223%:%
%:%825=224%:%
%:%826=225%:%
%:%829=226%:%
%:%833=226%:%
%:%834=226%:%
%:%835=226%:%
%:%836=226%:%
%:%841=226%:%
%:%844=227%:%
%:%845=228%:%
%:%848=228%:%
%:%852=228%:%
%:%860=228%:%
%:%861=229%:%
%:%862=229%:%
%:%863=230%:%
%:%864=231%:%
%:%865=231%:%
%:%867=231%:%
%:%871=231%:%
%:%872=231%:%
%:%873=231%:%
%:%880=231%:%
%:%881=232%:%
%:%882=232%:%
%:%884=232%:%
%:%888=232%:%
%:%889=232%:%
%:%890=232%:%
%:%899=234%:%
%:%901=235%:%
%:%902=235%:%
%:%909=236%:%
%:%910=236%:%
%:%911=237%:%
%:%912=237%:%
%:%913=238%:%
%:%914=238%:%
%:%915=238%:%
%:%916=239%:%
%:%917=239%:%
%:%918=239%:%
%:%919=240%:%
%:%920=240%:%
%:%921=240%:%
%:%922=241%:%
%:%923=241%:%
%:%924=241%:%
%:%925=242%:%
%:%926=242%:%
%:%927=242%:%
%:%928=243%:%
%:%929=243%:%
%:%930=243%:%
%:%931=243%:%
%:%932=243%:%
%:%933=244%:%
%:%939=244%:%
%:%942=245%:%
%:%943=246%:%
%:%944=246%:%
%:%945=247%:%
%:%952=248%:%
%:%953=248%:%
%:%954=249%:%
%:%955=249%:%
%:%956=250%:%
%:%957=250%:%
%:%958=250%:%
%:%959=250%:%
%:%960=251%:%
%:%961=251%:%
%:%962=252%:%
%:%963=252%:%
%:%964=253%:%
%:%965=253%:%
%:%966=253%:%
%:%967=253%:%
%:%968=254%:%
%:%974=254%:%
%:%977=255%:%
%:%978=256%:%
%:%979=256%:%
%:%980=257%:%
%:%987=258%:%
%:%988=258%:%
%:%989=259%:%
%:%990=259%:%
%:%991=260%:%
%:%992=260%:%
%:%993=260%:%
%:%994=260%:%
%:%995=261%:%
%:%996=261%:%
%:%997=262%:%
%:%998=262%:%
%:%999=263%:%
%:%1000=263%:%
%:%1001=263%:%
%:%1002=263%:%
%:%1003=264%:%
%:%1004=264%:%
%:%1005=265%:%
%:%1006=265%:%
%:%1007=266%:%
%:%1008=266%:%
%:%1009=266%:%
%:%1010=266%:%
%:%1011=267%:%
%:%1021=269%:%
%:%1023=270%:%
%:%1024=270%:%
%:%1031=271%:%
%:%1032=271%:%
%:%1033=272%:%
%:%1034=272%:%
%:%1035=273%:%
%:%1036=273%:%
%:%1037=274%:%
%:%1038=274%:%
%:%1039=274%:%
%:%1040=275%:%
%:%1041=275%:%
%:%1042=275%:%
%:%1043=275%:%
%:%1044=275%:%
%:%1045=276%:%
%:%1046=276%:%
%:%1047=276%:%
%:%1048=276%:%
%:%1049=277%:%
%:%1050=277%:%
%:%1051=277%:%
%:%1052=277%:%
%:%1053=278%:%
%:%1054=278%:%
%:%1055=278%:%
%:%1056=278%:%
%:%1057=279%:%
%:%1058=279%:%
%:%1059=279%:%
%:%1060=279%:%
%:%1061=280%:%
%:%1062=280%:%
%:%1063=280%:%
%:%1064=280%:%
%:%1065=281%:%
%:%1066=281%:%
%:%1067=282%:%
%:%1082=284%:%
%:%1094=286%:%
%:%1096=287%:%
%:%1097=287%:%
%:%1098=288%:%
%:%1099=289%:%
%:%1100=290%:%
%:%1101=290%:%
%:%1102=291%:%
%:%1103=292%:%
%:%1104=293%:%
%:%1105=293%:%
%:%1106=294%:%
%:%1107=295%:%
%:%1108=296%:%
%:%1109=296%:%
%:%1111=296%:%
%:%1115=296%:%
%:%1116=296%:%
%:%1117=296%:%
%:%1124=296%:%
%:%1125=297%:%
%:%1126=298%:%
%:%1127=298%:%
%:%1129=298%:%
%:%1133=298%:%
%:%1134=298%:%
%:%1135=298%:%
%:%1142=298%:%
%:%1143=299%:%
%:%1144=300%:%
%:%1145=300%:%
%:%1147=300%:%
%:%1151=300%:%
%:%1152=300%:%
%:%1153=300%:%
%:%1160=300%:%
%:%1161=301%:%
%:%1162=302%:%
%:%1163=302%:%
%:%1164=303%:%
%:%1165=304%:%
%:%1172=305%:%
%:%1173=305%:%
%:%1174=306%:%
%:%1175=306%:%
%:%1176=307%:%
%:%1177=307%:%
%:%1178=308%:%
%:%1179=308%:%
%:%1180=308%:%
%:%1181=309%:%
%:%1182=309%:%
%:%1183=309%:%
%:%1184=310%:%
%:%1185=310%:%
%:%1186=310%:%
%:%1187=310%:%
%:%1188=311%:%
%:%1189=311%:%
%:%1190=311%:%
%:%1191=312%:%
%:%1192=312%:%
%:%1193=313%:%
%:%1194=313%:%
%:%1195=313%:%
%:%1196=314%:%
%:%1197=314%:%
%:%1198=315%:%
%:%1199=315%:%
%:%1200=316%:%
%:%1201=316%:%
%:%1202=316%:%
%:%1203=316%:%
%:%1204=316%:%
%:%1205=317%:%
%:%1206=318%:%
%:%1207=318%:%
%:%1208=318%:%
%:%1209=318%:%
%:%1210=319%:%
%:%1211=320%:%
%:%1212=320%:%
%:%1213=321%:%
%:%1214=321%:%
%:%1215=322%:%
%:%1216=323%:%
%:%1217=323%:%
%:%1218=324%:%
%:%1219=324%:%
%:%1220=325%:%
%:%1221=325%:%
%:%1222=325%:%
%:%1223=326%:%
%:%1224=326%:%
%:%1225=326%:%
%:%1226=327%:%
%:%1227=327%:%
%:%1228=327%:%
%:%1229=327%:%
%:%1230=328%:%
%:%1231=328%:%
%:%1232=329%:%
%:%1233=329%:%
%:%1234=329%:%
%:%1235=330%:%
%:%1236=331%:%
%:%1237=331%:%
%:%1238=332%:%
%:%1239=332%:%
%:%1240=333%:%
%:%1241=333%:%
%:%1242=334%:%
%:%1243=334%:%
%:%1244=334%:%
%:%1245=335%:%
%:%1246=335%:%
%:%1247=335%:%
%:%1248=336%:%
%:%1249=336%:%
%:%1250=336%:%
%:%1251=336%:%
%:%1252=337%:%
%:%1253=337%:%
%:%1254=337%:%
%:%1255=337%:%
%:%1256=337%:%
%:%1257=338%:%
%:%1258=338%:%
%:%1259=338%:%
%:%1260=338%:%
%:%1261=338%:%
%:%1262=339%:%
%:%1263=339%:%
%:%1264=339%:%
%:%1265=339%:%
%:%1266=339%:%
%:%1267=340%:%
%:%1268=340%:%
%:%1269=340%:%
%:%1270=340%:%
%:%1271=341%:%
%:%1272=341%:%
%:%1273=342%:%
%:%1274=342%:%
%:%1275=343%:%
%:%1276=343%:%
%:%1277=343%:%
%:%1278=344%:%
%:%1279=345%:%
%:%1280=345%:%
%:%1281=346%:%
%:%1282=346%:%
%:%1283=346%:%
%:%1284=346%:%
%:%1285=346%:%
%:%1286=347%:%
%:%1287=348%:%
%:%1288=348%:%
%:%1289=348%:%
%:%1290=348%:%
%:%1291=349%:%
%:%1292=349%:%
%:%1293=349%:%
%:%1294=349%:%
%:%1295=349%:%
%:%1296=349%:%
%:%1297=350%:%
%:%1298=350%:%
%:%1299=350%:%
%:%1300=350%:%
%:%1301=351%:%
%:%1302=352%:%
%:%1303=352%:%
%:%1304=353%:%
%:%1305=353%:%
%:%1306=354%:%
%:%1307=354%:%
%:%1308=355%:%
%:%1309=355%:%
%:%1310=355%:%
%:%1311=355%:%
%:%1312=356%:%
%:%1313=356%:%
%:%1314=356%:%
%:%1315=356%:%
%:%1316=357%:%
%:%1317=357%:%
%:%1318=357%:%
%:%1319=357%:%
%:%1320=358%:%
%:%1321=358%:%
%:%1322=358%:%
%:%1323=358%:%
%:%1324=358%:%
%:%1325=358%:%
%:%1326=359%:%
%:%1327=359%:%
%:%1328=359%:%
%:%1329=359%:%
%:%1330=360%:%
%:%1331=360%:%
%:%1332=360%:%
%:%1333=360%:%
%:%1334=361%:%
%:%1335=361%:%
%:%1336=361%:%
%:%1337=362%:%
%:%1338=362%:%
%:%1339=363%:%
%:%1340=363%:%
%:%1341=363%:%
%:%1342=363%:%
%:%1343=364%:%
%:%1344=364%:%
%:%1345=365%:%
%:%1346=365%:%
%:%1347=365%:%
%:%1348=365%:%
%:%1349=366%:%
%:%1355=366%:%
%:%1358=367%:%
%:%1359=368%:%
%:%1360=368%:%
%:%1361=369%:%
%:%1362=370%:%
%:%1369=371%:%
%:%1370=371%:%
%:%1371=372%:%
%:%1372=372%:%
%:%1373=373%:%
%:%1374=373%:%
%:%1375=373%:%
%:%1376=373%:%
%:%1377=373%:%
%:%1378=374%:%
%:%1379=374%:%
%:%1380=374%:%
%:%1381=374%:%
%:%1382=375%:%
%:%1383=375%:%
%:%1384=375%:%
%:%1385=375%:%
%:%1386=376%:%
%:%1387=376%:%
%:%1388=376%:%
%:%1389=376%:%
%:%1390=376%:%
%:%1391=377%:%
%:%1392=377%:%
%:%1393=378%:%
%:%1394=378%:%
%:%1395=379%:%
%:%1396=379%:%
%:%1397=379%:%
%:%1398=379%:%
%:%1399=380%:%
%:%1400=380%:%
%:%1401=380%:%
%:%1402=380%:%
%:%1403=381%:%
%:%1404=381%:%
%:%1405=381%:%
%:%1406=381%:%
%:%1407=381%:%
%:%1408=382%:%
%:%1409=382%:%
%:%1410=382%:%
%:%1411=382%:%
%:%1412=382%:%
%:%1413=383%:%
%:%1419=383%:%
%:%1422=384%:%
%:%1423=385%:%
%:%1424=385%:%
%:%1425=386%:%
%:%1426=387%:%
%:%1433=388%:%
%:%1434=388%:%
%:%1435=389%:%
%:%1436=389%:%
%:%1437=390%:%
%:%1438=390%:%
%:%1439=390%:%
%:%1440=391%:%
%:%1441=391%:%
%:%1442=391%:%
%:%1443=392%:%
%:%1444=392%:%
%:%1445=392%:%
%:%1446=392%:%
%:%1447=392%:%
%:%1448=393%:%
%:%1449=393%:%
%:%1450=393%:%
%:%1451=393%:%
%:%1452=394%:%
%:%1453=394%:%
%:%1454=394%:%
%:%1455=395%:%
%:%1456=395%:%
%:%1457=395%:%
%:%1458=396%:%
%:%1459=396%:%
%:%1460=396%:%
%:%1461=396%:%
%:%1462=397%:%
%:%1463=397%:%
%:%1464=398%:%
%:%1465=398%:%
%:%1466=398%:%
%:%1467=398%:%
%:%1468=398%:%
%:%1469=399%:%
%:%1470=399%:%
%:%1471=399%:%
%:%1472=400%:%
%:%1478=400%:%
%:%1481=401%:%
%:%1482=402%:%
%:%1483=402%:%
%:%1484=403%:%
%:%1485=404%:%
%:%1492=405%:%
%:%1493=405%:%
%:%1494=406%:%
%:%1495=406%:%
%:%1496=407%:%
%:%1497=407%:%
%:%1498=407%:%
%:%1499=408%:%
%:%1500=408%:%
%:%1501=408%:%
%:%1502=409%:%
%:%1503=409%:%
%:%1504=409%:%
%:%1505=409%:%
%:%1506=409%:%
%:%1507=410%:%
%:%1508=410%:%
%:%1509=410%:%
%:%1510=410%:%
%:%1511=410%:%
%:%1512=411%:%
%:%1513=411%:%
%:%1514=411%:%
%:%1515=411%:%
%:%1516=412%:%
%:%1517=412%:%
%:%1518=412%:%
%:%1519=413%:%
%:%1520=413%:%
%:%1521=413%:%
%:%1522=413%:%
%:%1523=414%:%
%:%1524=414%:%
%:%1525=415%:%
%:%1526=415%:%
%:%1527=415%:%
%:%1528=415%:%
%:%1529=415%:%
%:%1530=416%:%
%:%1531=416%:%
%:%1532=416%:%
%:%1533=417%:%
%:%1539=417%:%
%:%1542=418%:%
%:%1543=419%:%
%:%1544=419%:%
%:%1545=420%:%
%:%1546=421%:%
%:%1553=422%:%
%:%1554=422%:%
%:%1555=423%:%
%:%1556=423%:%
%:%1557=424%:%
%:%1558=424%:%
%:%1559=424%:%
%:%1560=424%:%
%:%1561=424%:%
%:%1562=425%:%
%:%1563=425%:%
%:%1564=425%:%
%:%1565=426%:%
%:%1566=426%:%
%:%1567=426%:%
%:%1568=426%:%
%:%1569=427%:%
%:%1570=427%:%
%:%1571=427%:%
%:%1572=427%:%
%:%1573=427%:%
%:%1574=428%:%
%:%1575=428%:%
%:%1576=429%:%
%:%1577=429%:%
%:%1578=430%:%
%:%1579=430%:%
%:%1580=430%:%
%:%1581=430%:%
%:%1582=431%:%
%:%1583=431%:%
%:%1584=431%:%
%:%1585=431%:%
%:%1586=431%:%
%:%1587=432%:%
%:%1588=432%:%
%:%1589=432%:%
%:%1590=432%:%
%:%1591=433%:%
%:%1592=433%:%
%:%1593=433%:%
%:%1594=433%:%
%:%1595=434%:%
%:%1596=434%:%
%:%1597=434%:%
%:%1598=434%:%
%:%1599=434%:%
%:%1600=435%:%
%:%1606=435%:%
%:%1609=436%:%
%:%1610=437%:%
%:%1611=437%:%
%:%1612=438%:%
%:%1613=439%:%
%:%1616=440%:%
%:%1620=440%:%
%:%1621=440%:%
%:%1622=441%:%
%:%1623=442%:%
%:%1624=442%:%
%:%1633=444%:%
%:%1635=445%:%
%:%1636=445%:%
%:%1637=446%:%
%:%1638=447%:%
%:%1639=448%:%
%:%1640=448%:%
%:%1641=449%:%
%:%1642=450%:%
%:%1643=451%:%
%:%1644=452%:%
%:%1647=453%:%
%:%1651=453%:%
%:%1652=453%:%
%:%1653=453%:%
%:%1658=453%:%
%:%1661=454%:%
%:%1662=455%:%
%:%1663=455%:%
%:%1664=456%:%
%:%1667=457%:%
%:%1671=457%:%
%:%1672=457%:%
%:%1673=458%:%
%:%1674=458%:%
%:%1675=458%:%
%:%1680=458%:%
%:%1683=459%:%
%:%1684=460%:%
%:%1685=460%:%
%:%1686=461%:%
%:%1687=462%:%
%:%1688=462%:%
%:%1689=463%:%
%:%1690=464%:%
%:%1693=465%:%
%:%1697=465%:%
%:%1698=465%:%
%:%1699=465%:%
%:%1704=465%:%
%:%1707=466%:%
%:%1708=467%:%
%:%1709=467%:%
%:%1711=467%:%
%:%1715=467%:%
%:%1716=467%:%
%:%1717=467%:%
%:%1724=467%:%
%:%1725=468%:%
%:%1726=469%:%
%:%1727=469%:%
%:%1728=470%:%
%:%1729=471%:%
%:%1732=472%:%
%:%1736=472%:%
%:%1737=472%:%
%:%1738=473%:%
%:%1739=473%:%
%:%1744=473%:%
%:%1747=474%:%
%:%1748=475%:%
%:%1749=475%:%
%:%1752=476%:%
%:%1756=476%:%
%:%1757=476%:%
%:%1758=477%:%
%:%1759=477%:%
%:%1764=477%:%
%:%1767=478%:%
%:%1768=479%:%
%:%1769=479%:%
%:%1771=479%:%
%:%1775=479%:%
%:%1776=479%:%
%:%1783=479%:%
%:%1784=480%:%
%:%1785=481%:%
%:%1786=481%:%
%:%1787=482%:%
%:%1788=483%:%
%:%1791=484%:%
%:%1795=484%:%
%:%1796=484%:%
%:%1797=485%:%
%:%1798=485%:%
%:%1803=485%:%
%:%1806=486%:%
%:%1807=487%:%
%:%1808=487%:%
%:%1809=488%:%
%:%1810=489%:%
%:%1811=489%:%
%:%1813=489%:%
%:%1817=489%:%
%:%1818=489%:%
%:%1825=489%:%
%:%1826=490%:%
%:%1827=491%:%
%:%1828=491%:%
%:%1829=492%:%
%:%1830=493%:%
%:%1833=494%:%
%:%1837=494%:%
%:%1838=494%:%
%:%1839=495%:%
%:%1840=495%:%
%:%1845=495%:%
%:%1848=496%:%
%:%1849=497%:%
%:%1850=497%:%
%:%1851=498%:%
%:%1852=499%:%
%:%1853=499%:%
%:%1855=499%:%
%:%1859=499%:%
%:%1860=499%:%
%:%1867=499%:%
%:%1868=500%:%
%:%1869=501%:%
%:%1870=501%:%
%:%1871=502%:%
%:%1872=503%:%
%:%1879=504%:%
%:%1880=504%:%
%:%1881=505%:%
%:%1882=505%:%
%:%1883=505%:%
%:%1884=505%:%
%:%1885=506%:%
%:%1886=506%:%
%:%1887=507%:%
%:%1888=507%:%
%:%1889=507%:%
%:%1890=508%:%
%:%1891=508%:%
%:%1892=508%:%
%:%1893=508%:%
%:%1894=508%:%
%:%1895=508%:%
%:%1896=509%:%
%:%1897=509%:%
%:%1898=509%:%
%:%1899=509%:%
%:%1900=509%:%
%:%1901=510%:%
%:%1902=510%:%
%:%1903=510%:%
%:%1904=511%:%
%:%1905=511%:%
%:%1906=512%:%
%:%1907=512%:%
%:%1908=512%:%
%:%1909=512%:%
%:%1910=513%:%
%:%1916=513%:%
%:%1919=514%:%
%:%1920=515%:%
%:%1921=515%:%
%:%1928=516%:%
%:%1929=516%:%
%:%1930=517%:%
%:%1931=517%:%
%:%1932=518%:%
%:%1933=518%:%
%:%1934=519%:%
%:%1935=519%:%
%:%1936=520%:%
%:%1937=520%:%
%:%1938=521%:%
%:%1939=521%:%
%:%1940=521%:%
%:%1941=521%:%
%:%1942=522%:%
%:%1943=522%:%
%:%1944=523%:%
%:%1945=523%:%
%:%1946=524%:%
%:%1947=524%:%
%:%1948=525%:%
%:%1949=525%:%
%:%1950=526%:%
%:%1951=526%:%
%:%1952=526%:%
%:%1953=527%:%
%:%1954=527%:%
%:%1955=527%:%
%:%1956=527%:%
%:%1957=527%:%
%:%1958=527%:%
%:%1959=528%:%
%:%1960=528%:%
%:%1961=528%:%
%:%1962=528%:%
%:%1963=528%:%
%:%1964=528%:%
%:%1965=529%:%
%:%1966=529%:%
%:%1967=529%:%
%:%1968=529%:%
%:%1969=530%:%
%:%1970=530%:%
%:%1971=531%:%
%:%1972=531%:%
%:%1973=531%:%
%:%1974=531%:%
%:%1975=532%:%
%:%1976=532%:%
%:%1977=532%:%
%:%1978=532%:%
%:%1979=533%:%
%:%1980=533%:%
%:%1981=534%:%
%:%1982=534%:%
%:%1983=535%:%
%:%1984=535%:%
%:%1985=535%:%
%:%1986=535%:%
%:%1987=536%:%
%:%1988=536%:%
%:%1989=537%:%
%:%1990=537%:%
%:%1991=537%:%
%:%1992=538%:%
%:%1993=538%:%
%:%1994=538%:%
%:%1995=538%:%
%:%1996=538%:%
%:%1997=538%:%
%:%1998=539%:%
%:%1999=539%:%
%:%2000=539%:%
%:%2001=539%:%
%:%2002=539%:%
%:%2003=539%:%
%:%2004=540%:%
%:%2005=540%:%
%:%2006=540%:%
%:%2007=540%:%
%:%2008=541%:%
%:%2009=541%:%
%:%2010=542%:%
%:%2011=542%:%
%:%2012=542%:%
%:%2013=542%:%
%:%2014=543%:%
%:%2015=543%:%
%:%2016=543%:%
%:%2017=543%:%
%:%2018=544%:%
%:%2019=544%:%
%:%2020=545%:%
%:%2021=545%:%
%:%2022=546%:%
%:%2023=546%:%
%:%2024=546%:%
%:%2025=547%:%
%:%2026=547%:%
%:%2027=548%:%
%:%2028=548%:%
%:%2029=548%:%
%:%2030=549%:%
%:%2031=549%:%
%:%2032=549%:%
%:%2033=549%:%
%:%2034=549%:%
%:%2035=550%:%
%:%2036=550%:%
%:%2037=550%:%
%:%2038=550%:%
%:%2039=550%:%
%:%2040=551%:%
%:%2041=551%:%
%:%2042=551%:%
%:%2043=552%:%
%:%2044=552%:%
%:%2045=552%:%
%:%2046=552%:%
%:%2047=553%:%
%:%2048=553%:%
%:%2049=554%:%
%:%2050=554%:%
%:%2051=554%:%
%:%2052=554%:%
%:%2053=555%:%
%:%2054=555%:%
%:%2055=556%:%
%:%2056=556%:%
%:%2061=556%:%
%:%2064=557%:%
%:%2065=558%:%
%:%2066=558%:%
%:%2068=558%:%
%:%2072=558%:%
%:%2073=558%:%
%:%2080=558%:%
%:%2081=559%:%
%:%2082=560%:%
%:%2083=560%:%
%:%2084=561%:%
%:%2085=562%:%
%:%2088=563%:%
%:%2092=563%:%
%:%2093=563%:%
%:%2094=564%:%
%:%2095=564%:%
%:%2100=564%:%
%:%2103=565%:%
%:%2104=566%:%
%:%2105=566%:%
%:%2106=567%:%
%:%2107=568%:%
%:%2114=569%:%
%:%2115=569%:%
%:%2116=570%:%
%:%2117=570%:%
%:%2118=570%:%
%:%2119=571%:%
%:%2120=571%:%
%:%2121=571%:%
%:%2122=571%:%
%:%2123=571%:%
%:%2124=572%:%
%:%2125=572%:%
%:%2126=572%:%
%:%2127=572%:%
%:%2128=573%:%
%:%2129=573%:%
%:%2130=573%:%
%:%2131=573%:%
%:%2132=574%:%
%:%2138=574%:%
%:%2141=575%:%
%:%2142=576%:%
%:%2143=576%:%
%:%2144=577%:%
%:%2145=578%:%
%:%2148=579%:%
%:%2152=579%:%
%:%2153=579%:%
%:%2154=579%:%
%:%2155=579%:%
%:%2160=579%:%
%:%2163=580%:%
%:%2164=581%:%
%:%2165=581%:%
%:%2166=582%:%
%:%2167=583%:%
%:%2174=584%:%
%:%2175=584%:%
%:%2176=585%:%
%:%2177=585%:%
%:%2178=585%:%
%:%2179=585%:%
%:%2180=585%:%
%:%2181=586%:%
%:%2182=586%:%
%:%2183=586%:%
%:%2184=586%:%
%:%2185=587%:%
%:%2186=587%:%
%:%2187=587%:%
%:%2188=587%:%
%:%2189=588%:%
%:%2190=588%:%
%:%2191=588%:%
%:%2192=588%:%
%:%2193=589%:%
%:%2194=589%:%
%:%2195=589%:%
%:%2196=589%:%
%:%2197=589%:%
%:%2198=589%:%
%:%2199=590%:%
%:%2200=590%:%
%:%2201=590%:%
%:%2202=590%:%
%:%2203=590%:%
%:%2204=591%:%
%:%2210=591%:%
%:%2213=592%:%
%:%2214=593%:%
%:%2215=593%:%
%:%2216=594%:%
%:%2217=595%:%
%:%2224=596%:%
%:%2225=596%:%
%:%2226=597%:%
%:%2227=597%:%
%:%2228=597%:%
%:%2229=597%:%
%:%2230=597%:%
%:%2231=598%:%
%:%2232=598%:%
%:%2233=598%:%
%:%2234=599%:%
%:%2235=599%:%
%:%2236=599%:%
%:%2237=599%:%
%:%2238=600%:%
%:%2239=600%:%
%:%2240=600%:%
%:%2241=600%:%
%:%2242=601%:%
%:%2243=601%:%
%:%2244=601%:%
%:%2245=601%:%
%:%2246=602%:%
%:%2247=602%:%
%:%2248=602%:%
%:%2249=602%:%
%:%2250=602%:%
%:%2251=602%:%
%:%2252=603%:%
%:%2253=603%:%
%:%2254=603%:%
%:%2255=603%:%
%:%2256=603%:%
%:%2257=604%:%
%:%2263=604%:%
%:%2266=605%:%
%:%2267=606%:%
%:%2268=606%:%
%:%2269=607%:%
%:%2270=608%:%
%:%2277=609%:%
%:%2278=609%:%
%:%2279=610%:%
%:%2280=610%:%
%:%2281=610%:%
%:%2282=610%:%
%:%2283=610%:%
%:%2284=611%:%
%:%2285=611%:%
%:%2286=611%:%
%:%2287=612%:%
%:%2288=612%:%
%:%2289=612%:%
%:%2290=612%:%
%:%2291=613%:%
%:%2292=613%:%
%:%2293=613%:%
%:%2294=613%:%
%:%2295=614%:%
%:%2296=614%:%
%:%2297=614%:%
%:%2298=614%:%
%:%2299=615%:%
%:%2300=615%:%
%:%2301=615%:%
%:%2302=615%:%
%:%2303=615%:%
%:%2304=615%:%
%:%2305=616%:%
%:%2306=616%:%
%:%2307=616%:%
%:%2308=616%:%
%:%2309=616%:%
%:%2310=617%:%
%:%2316=617%:%
%:%2319=618%:%
%:%2320=619%:%
%:%2321=619%:%
%:%2322=620%:%
%:%2323=621%:%
%:%2330=622%:%
%:%2331=622%:%
%:%2332=623%:%
%:%2333=623%:%
%:%2334=623%:%
%:%2335=623%:%
%:%2336=623%:%
%:%2337=624%:%
%:%2338=624%:%
%:%2339=624%:%
%:%2340=625%:%
%:%2341=625%:%
%:%2342=625%:%
%:%2343=625%:%
%:%2344=626%:%
%:%2345=626%:%
%:%2346=626%:%
%:%2347=626%:%
%:%2348=627%:%
%:%2349=627%:%
%:%2350=627%:%
%:%2351=627%:%
%:%2352=628%:%
%:%2353=628%:%
%:%2354=628%:%
%:%2355=628%:%
%:%2356=628%:%
%:%2357=628%:%
%:%2358=629%:%
%:%2359=629%:%
%:%2360=629%:%
%:%2361=629%:%
%:%2362=629%:%
%:%2363=630%:%
%:%2373=632%:%
%:%2375=633%:%
%:%2376=633%:%
%:%2377=634%:%
%:%2378=635%:%
%:%2379=636%:%
%:%2380=636%:%
%:%2381=637%:%
%:%2382=638%:%
%:%2383=639%:%
%:%2384=639%:%
%:%2385=640%:%
%:%2386=641%:%
%:%2387=642%:%
%:%2390=642%:%
%:%2394=642%:%
%:%2402=642%:%
%:%2403=643%:%
%:%2404=643%:%
%:%2405=644%:%
%:%2406=645%:%
%:%2407=645%:%
%:%2414=646%:%
%:%2415=646%:%
%:%2416=647%:%
%:%2417=647%:%
%:%2418=648%:%
%:%2419=648%:%
%:%2420=648%:%
%:%2421=648%:%
%:%2422=648%:%
%:%2423=649%:%
%:%2424=649%:%
%:%2425=649%:%
%:%2426=649%:%
%:%2427=650%:%
%:%2428=650%:%
%:%2429=650%:%
%:%2430=650%:%
%:%2431=650%:%
%:%2432=651%:%
%:%2433=651%:%
%:%2434=651%:%
%:%2435=651%:%
%:%2436=651%:%
%:%2437=652%:%
%:%2438=652%:%
%:%2439=652%:%
%:%2440=652%:%
%:%2441=652%:%
%:%2442=653%:%
%:%2448=653%:%
%:%2451=654%:%
%:%2452=655%:%
%:%2453=656%:%
%:%2454=656%:%
%:%2455=657%:%
%:%2456=658%:%
%:%2459=659%:%
%:%2463=659%:%
%:%2464=659%:%
%:%2465=659%:%
%:%2466=659%:%
%:%2471=659%:%
%:%2474=660%:%
%:%2475=661%:%
%:%2476=661%:%
%:%2477=662%:%
%:%2478=663%:%
%:%2481=664%:%
%:%2485=664%:%
%:%2486=664%:%
%:%2487=664%:%
%:%2488=664%:%
%:%2493=664%:%
%:%2496=665%:%
%:%2497=666%:%
%:%2498=666%:%
%:%2499=667%:%
%:%2500=668%:%
%:%2507=669%:%
%:%2508=669%:%
%:%2509=670%:%
%:%2510=670%:%
%:%2511=671%:%
%:%2512=671%:%
%:%2513=672%:%
%:%2514=672%:%
%:%2515=672%:%
%:%2516=673%:%
%:%2517=673%:%
%:%2518=673%:%
%:%2519=674%:%
%:%2520=674%:%
%:%2521=674%:%
%:%2522=674%:%
%:%2523=675%:%
%:%2524=675%:%
%:%2525=676%:%
%:%2526=676%:%
%:%2527=677%:%
%:%2528=677%:%
%:%2529=678%:%
%:%2530=678%:%
%:%2531=678%:%
%:%2532=678%:%
%:%2533=678%:%
%:%2534=679%:%
%:%2535=679%:%
%:%2536=679%:%
%:%2537=679%:%
%:%2538=679%:%
%:%2539=680%:%
%:%2540=680%:%
%:%2541=680%:%
%:%2542=680%:%
%:%2543=681%:%
%:%2544=681%:%
%:%2545=682%:%
%:%2546=682%:%
%:%2547=682%:%
%:%2548=682%:%
%:%2549=682%:%
%:%2550=683%:%
%:%2551=683%:%
%:%2552=684%:%
%:%2553=684%:%
%:%2554=684%:%
%:%2555=684%:%
%:%2556=685%:%
%:%2562=685%:%
%:%2565=686%:%
%:%2566=687%:%
%:%2567=687%:%
%:%2568=688%:%
%:%2569=689%:%
%:%2576=690%:%
%:%2577=690%:%
%:%2578=691%:%
%:%2579=691%:%
%:%2580=692%:%
%:%2581=692%:%
%:%2582=693%:%
%:%2583=693%:%
%:%2584=693%:%
%:%2585=693%:%
%:%2586=693%:%
%:%2587=694%:%
%:%2588=694%:%
%:%2589=694%:%
%:%2590=694%:%
%:%2591=694%:%
%:%2592=695%:%
%:%2593=695%:%
%:%2594=695%:%
%:%2595=695%:%
%:%2596=696%:%
%:%2597=696%:%
%:%2598=697%:%
%:%2599=697%:%
%:%2600=697%:%
%:%2601=697%:%
%:%2602=697%:%
%:%2603=698%:%
%:%2604=698%:%
%:%2605=698%:%
%:%2606=698%:%
%:%2607=699%:%
%:%2613=699%:%
%:%2616=700%:%
%:%2617=701%:%
%:%2618=701%:%
%:%2619=702%:%
%:%2620=703%:%
%:%2627=704%:%
%:%2628=704%:%
%:%2629=705%:%
%:%2630=705%:%
%:%2631=706%:%
%:%2632=706%:%
%:%2633=707%:%
%:%2634=707%:%
%:%2635=707%:%
%:%2636=707%:%
%:%2637=708%:%
%:%2638=708%:%
%:%2639=708%:%
%:%2640=708%:%
%:%2641=709%:%
%:%2642=709%:%
%:%2643=710%:%
%:%2644=710%:%
%:%2645=710%:%
%:%2646=710%:%
%:%2647=711%:%
%:%2662=713%:%
%:%2674=715%:%
%:%2676=716%:%
%:%2677=716%:%
%:%2678=717%:%
%:%2679=718%:%
%:%2680=719%:%
%:%2681=719%:%
%:%2682=720%:%
%:%2683=721%:%
%:%2684=722%:%
%:%2685=722%:%
%:%2686=723%:%
%:%2687=724%:%
%:%2694=725%:%
%:%2695=725%:%
%:%2696=726%:%
%:%2697=726%:%
%:%2698=727%:%
%:%2699=727%:%
%:%2700=727%:%
%:%2701=727%:%
%:%2702=728%:%
%:%2703=728%:%
%:%2704=728%:%
%:%2705=728%:%
%:%2706=729%:%
%:%2707=729%:%
%:%2708=729%:%
%:%2709=729%:%
%:%2710=729%:%
%:%2711=730%:%
%:%2712=731%:%
%:%2713=731%:%
%:%2714=731%:%
%:%2715=731%:%
%:%2716=731%:%
%:%2717=732%:%
%:%2718=732%:%
%:%2719=732%:%
%:%2720=733%:%
%:%2721=733%:%
%:%2722=733%:%
%:%2723=734%:%
%:%2724=734%:%
%:%2725=734%:%
%:%2726=734%:%
%:%2727=734%:%
%:%2728=735%:%
%:%2729=735%:%
%:%2730=735%:%
%:%2731=735%:%
%:%2732=736%:%
%:%2733=737%:%
%:%2734=737%:%
%:%2735=737%:%
%:%2736=737%:%
%:%2737=737%:%
%:%2738=738%:%
%:%2739=739%:%
%:%2740=739%:%
%:%2741=739%:%
%:%2742=739%:%
%:%2743=740%:%
%:%2744=740%:%
%:%2745=740%:%
%:%2746=741%:%
%:%2747=741%:%
%:%2748=741%:%
%:%2749=741%:%
%:%2750=742%:%
%:%2751=743%:%
%:%2752=743%:%
%:%2753=744%:%
%:%2754=745%:%
%:%2755=745%:%
%:%2756=745%:%
%:%2757=745%:%
%:%2758=746%:%
%:%2759=746%:%
%:%2760=746%:%
%:%2761=746%:%
%:%2762=747%:%
%:%2763=747%:%
%:%2764=747%:%
%:%2765=748%:%
%:%2766=748%:%
%:%2767=748%:%
%:%2768=748%:%
%:%2769=749%:%
%:%2770=750%:%
%:%2771=750%:%
%:%2772=751%:%
%:%2773=751%:%
%:%2774=751%:%
%:%2775=752%:%
%:%2776=752%:%
%:%2777=752%:%
%:%2778=752%:%
%:%2779=753%:%
%:%2780=753%:%
%:%2781=753%:%
%:%2782=753%:%
%:%2783=754%:%
%:%2784=754%:%
%:%2785=754%:%
%:%2786=754%:%
%:%2787=755%:%
%:%2788=756%:%
%:%2789=756%:%
%:%2790=757%:%
%:%2791=757%:%
%:%2792=758%:%
%:%2793=758%:%
%:%2794=759%:%
%:%2795=759%:%
%:%2796=759%:%
%:%2797=759%:%
%:%2798=760%:%
%:%2799=760%:%
%:%2800=760%:%
%:%2801=760%:%
%:%2802=760%:%
%:%2803=761%:%
%:%2804=761%:%
%:%2805=761%:%
%:%2806=761%:%
%:%2807=761%:%
%:%2808=762%:%
%:%2809=762%:%
%:%2810=762%:%
%:%2811=762%:%
%:%2812=762%:%
%:%2813=763%:%
%:%2814=763%:%
%:%2815=763%:%
%:%2816=764%:%
%:%2817=764%:%
%:%2818=764%:%
%:%2819=764%:%
%:%2820=765%:%
%:%2821=765%:%
%:%2822=765%:%
%:%2823=766%:%
%:%2824=766%:%
%:%2825=766%:%
%:%2826=766%:%
%:%2827=767%:%
%:%2828=767%:%
%:%2829=768%:%
%:%2830=769%:%
%:%2831=769%:%
%:%2832=770%:%
%:%2833=770%:%
%:%2834=771%:%
%:%2835=771%:%
%:%2836=772%:%
%:%2837=772%:%
%:%2838=773%:%
%:%2839=774%:%
%:%2840=774%:%
%:%2841=774%:%
%:%2842=774%:%
%:%2843=775%:%
%:%2844=775%:%
%:%2845=775%:%
%:%2846=775%:%
%:%2847=775%:%
%:%2848=776%:%
%:%2849=776%:%
%:%2850=776%:%
%:%2851=776%:%
%:%2852=776%:%
%:%2853=777%:%
%:%2854=777%:%
%:%2855=777%:%
%:%2856=777%:%
%:%2857=778%:%
%:%2858=779%:%
%:%2859=779%:%
%:%2860=779%:%
%:%2861=779%:%
%:%2862=780%:%
%:%2863=780%:%
%:%2864=780%:%
%:%2865=780%:%
%:%2866=780%:%
%:%2867=781%:%
%:%2868=781%:%
%:%2869=781%:%
%:%2870=781%:%
%:%2871=781%:%
%:%2872=782%:%
%:%2873=782%:%
%:%2874=782%:%
%:%2875=782%:%
%:%2876=783%:%
%:%2877=784%:%
%:%2878=784%:%
%:%2879=784%:%
%:%2880=784%:%
%:%2881=785%:%
%:%2882=785%:%
%:%2883=785%:%
%:%2884=785%:%
%:%2885=785%:%
%:%2886=786%:%
%:%2887=786%:%
%:%2888=786%:%
%:%2889=786%:%
%:%2890=786%:%
%:%2891=787%:%
%:%2892=787%:%
%:%2893=787%:%
%:%2894=787%:%
%:%2895=788%:%
%:%2896=789%:%
%:%2897=789%:%
%:%2898=789%:%
%:%2899=789%:%
%:%2900=790%:%
%:%2901=791%:%
%:%2902=791%:%
%:%2903=791%:%
%:%2904=791%:%
%:%2905=792%:%
%:%2906=792%:%
%:%2907=792%:%
%:%2908=792%:%
%:%2909=792%:%
%:%2910=793%:%
%:%2911=793%:%
%:%2912=793%:%
%:%2913=793%:%
%:%2914=793%:%
%:%2915=794%:%
%:%2916=795%:%
%:%2917=795%:%
%:%2918=795%:%
%:%2919=795%:%
%:%2920=796%:%
%:%2921=796%:%
%:%2922=796%:%
%:%2923=796%:%
%:%2924=796%:%
%:%2925=797%:%
%:%2926=797%:%
%:%2927=797%:%
%:%2928=797%:%
%:%2929=797%:%
%:%2930=798%:%
%:%2931=798%:%
%:%2932=798%:%
%:%2933=798%:%
%:%2934=799%:%
%:%2935=800%:%
%:%2936=800%:%
%:%2937=800%:%
%:%2938=800%:%
%:%2939=800%:%
%:%2940=801%:%
%:%2941=801%:%
%:%2942=802%:%
%:%2943=802%:%
%:%2944=802%:%
%:%2945=803%:%
%:%2946=803%:%
%:%2947=803%:%
%:%2948=804%:%
%:%2949=804%:%
%:%2950=804%:%
%:%2951=804%:%
%:%2952=805%:%
%:%2953=805%:%
%:%2954=805%:%
%:%2955=805%:%
%:%2956=806%:%
%:%2957=806%:%
%:%2958=806%:%
%:%2959=807%:%
%:%2960=807%:%
%:%2961=808%:%
%:%2962=808%:%
%:%2963=809%:%
%:%2964=809%:%
%:%2965=809%:%
%:%2966=809%:%
%:%2967=810%:%
%:%2968=810%:%
%:%2969=810%:%
%:%2970=810%:%
%:%2971=811%:%
%:%2972=811%:%
%:%2973=812%:%
%:%2974=812%:%
%:%2975=813%:%
%:%2976=813%:%
%:%2977=814%:%
%:%2978=814%:%
%:%2979=814%:%
%:%2980=814%:%
%:%2981=815%:%
%:%2982=815%:%
%:%2983=815%:%
%:%2984=815%:%
%:%2985=815%:%
%:%2986=816%:%
%:%2987=816%:%
%:%2988=816%:%
%:%2989=816%:%
%:%2990=816%:%
%:%2991=817%:%
%:%2992=817%:%
%:%2993=817%:%
%:%2994=817%:%
%:%2995=817%:%
%:%2996=818%:%
%:%2997=818%:%
%:%2998=818%:%
%:%2999=818%:%
%:%3000=819%:%
%:%3001=819%:%
%:%3002=820%:%
%:%3003=820%:%
%:%3004=820%:%
%:%3005=820%:%
%:%3006=821%:%
%:%3007=821%:%
%:%3008=822%:%
%:%3009=822%:%
%:%3010=822%:%
%:%3011=822%:%
%:%3012=822%:%
%:%3013=823%:%
%:%3019=823%:%
%:%3022=824%:%
%:%3023=825%:%
%:%3024=825%:%
%:%3025=826%:%
%:%3026=827%:%
%:%3027=828%:%
%:%3028=828%:%
%:%3029=829%:%
%:%3030=830%:%
%:%3031=831%:%
%:%3038=832%:%
%:%3039=832%:%
%:%3040=833%:%
%:%3041=833%:%
%:%3042=834%:%
%:%3043=834%:%
%:%3044=834%:%
%:%3045=834%:%
%:%3046=835%:%
%:%3047=835%:%
%:%3048=836%:%
%:%3049=836%:%
%:%3050=837%:%
%:%3051=837%:%
%:%3052=837%:%
%:%3053=837%:%
%:%3054=837%:%
%:%3055=838%:%
%:%3056=838%:%
%:%3057=838%:%
%:%3058=838%:%
%:%3059=839%:%
%:%3060=839%:%
%:%3061=840%:%
%:%3062=840%:%
%:%3063=841%:%
%:%3064=841%:%
%:%3065=842%:%
%:%3066=842%:%
%:%3067=843%:%
%:%3068=843%:%
%:%3069=844%:%
%:%3070=844%:%
%:%3071=845%:%
%:%3072=845%:%
%:%3073=845%:%
%:%3074=845%:%
%:%3075=846%:%
%:%3076=846%:%
%:%3077=846%:%
%:%3078=846%:%
%:%3079=846%:%
%:%3080=847%:%
%:%3081=847%:%
%:%3082=847%:%
%:%3083=848%:%
%:%3084=848%:%
%:%3085=848%:%
%:%3086=848%:%
%:%3087=848%:%
%:%3088=849%:%
%:%3089=849%:%
%:%3090=849%:%
%:%3091=849%:%
%:%3092=849%:%
%:%3093=850%:%
%:%3094=850%:%
%:%3095=851%:%
%:%3096=851%:%
%:%3097=851%:%
%:%3098=851%:%
%:%3099=851%:%
%:%3100=852%:%
%:%3101=852%:%
%:%3102=852%:%
%:%3103=852%:%
%:%3104=853%:%
%:%3105=853%:%
%:%3106=854%:%
%:%3107=854%:%
%:%3108=854%:%
%:%3109=854%:%
%:%3110=855%:%
%:%3116=855%:%
%:%3119=856%:%
%:%3120=857%:%
%:%3121=857%:%
%:%3122=858%:%
%:%3129=859%:%
%:%3130=859%:%
%:%3131=860%:%
%:%3132=860%:%
%:%3133=860%:%
%:%3134=861%:%
%:%3135=861%:%
%:%3136=861%:%
%:%3137=861%:%
%:%3138=862%:%
%:%3139=862%:%
%:%3140=863%:%
%:%3141=863%:%
%:%3142=864%:%
%:%3143=864%:%
%:%3144=865%:%
%:%3145=865%:%
%:%3146=865%:%
%:%3147=866%:%
%:%3148=866%:%
%:%3149=866%:%
%:%3150=866%:%
%:%3151=866%:%
%:%3152=866%:%
%:%3153=867%:%
%:%3154=867%:%
%:%3155=868%:%
%:%3156=868%:%
%:%3157=869%:%
%:%3158=869%:%
%:%3159=869%:%
%:%3160=870%:%
%:%3161=870%:%
%:%3162=870%:%
%:%3163=870%:%
%:%3164=871%:%
%:%3165=872%:%
%:%3166=872%:%
%:%3167=873%:%
%:%3168=873%:%
%:%3169=874%:%
%:%3170=874%:%
%:%3171=875%:%
%:%3172=876%:%
%:%3173=876%:%
%:%3174=877%:%
%:%3175=877%:%
%:%3176=878%:%
%:%3177=878%:%
%:%3178=879%:%
%:%3179=879%:%
%:%3180=879%:%
%:%3181=880%:%
%:%3182=880%:%
%:%3183=881%:%
%:%3184=881%:%
%:%3185=882%:%
%:%3186=882%:%
%:%3187=882%:%
%:%3188=882%:%
%:%3189=882%:%
%:%3190=883%:%
%:%3191=883%:%
%:%3192=883%:%
%:%3193=883%:%
%:%3194=883%:%
%:%3195=884%:%
%:%3196=884%:%
%:%3197=884%:%
%:%3198=884%:%
%:%3199=885%:%
%:%3200=885%:%
%:%3201=886%:%
%:%3207=886%:%
%:%3210=887%:%
%:%3211=888%:%
%:%3212=888%:%
%:%3213=889%:%
%:%3214=890%:%
%:%3215=891%:%
%:%3218=891%:%
%:%3222=891%:%
%:%3230=891%:%
%:%3231=892%:%
%:%3232=892%:%
%:%3233=893%:%
%:%3234=894%:%
%:%3235=894%:%
%:%3236=895%:%
%:%3237=896%:%
%:%3238=896%:%
%:%3239=897%:%
%:%3240=898%:%
%:%3243=899%:%
%:%3247=899%:%
%:%3248=899%:%
%:%3249=899%:%
%:%3250=900%:%
%:%3251=900%:%
%:%3260=902%:%
%:%3262=903%:%
%:%3263=903%:%
%:%3270=904%:%
%:%3271=904%:%
%:%3272=905%:%
%:%3273=905%:%
%:%3274=906%:%
%:%3275=906%:%
%:%3276=906%:%
%:%3277=906%:%
%:%3278=907%:%
%:%3279=907%:%
%:%3280=907%:%
%:%3281=907%:%
%:%3282=908%:%
%:%3283=908%:%
%:%3284=908%:%
%:%3285=908%:%
%:%3286=908%:%
%:%3287=909%:%
%:%3293=909%:%
%:%3296=910%:%
%:%3297=911%:%
%:%3298=911%:%
%:%3299=912%:%
%:%3300=913%:%
%:%3301=914%:%
%:%3302=914%:%
%:%3303=915%:%
%:%3304=916%:%
%:%3305=917%:%
%:%3306=917%:%
%:%3307=918%:%
%:%3308=919%:%
%:%3309=920%:%
%:%3312=920%:%
%:%3316=920%:%
%:%3317=920%:%
%:%3324=920%:%
%:%3325=921%:%
%:%3326=922%:%
%:%3329=924%:%
%:%3331=925%:%
%:%3332=925%:%
%:%3339=926%:%
%:%3340=926%:%
%:%3341=927%:%
%:%3342=927%:%
%:%3343=928%:%
%:%3344=928%:%
%:%3345=929%:%
%:%3346=929%:%
%:%3347=930%:%
%:%3348=930%:%
%:%3349=931%:%
%:%3350=931%:%
%:%3351=931%:%
%:%3352=931%:%
%:%3353=932%:%
%:%3354=932%:%
%:%3355=933%:%
%:%3356=933%:%
%:%3357=934%:%
%:%3358=934%:%
%:%3359=935%:%
%:%3360=935%:%
%:%3361=935%:%
%:%3362=935%:%
%:%3363=935%:%
%:%3364=936%:%
%:%3365=936%:%
%:%3366=936%:%
%:%3367=937%:%
%:%3368=937%:%
%:%3369=938%:%
%:%3370=938%:%
%:%3371=939%:%
%:%3372=940%:%
%:%3373=940%:%
%:%3374=940%:%
%:%3375=941%:%
%:%3376=941%:%
%:%3377=941%:%
%:%3378=941%:%
%:%3379=942%:%
%:%3380=942%:%
%:%3381=942%:%
%:%3382=943%:%
%:%3392=945%:%
%:%3394=946%:%
%:%3395=946%:%
%:%3402=947%:%
%:%3403=947%:%
%:%3404=948%:%
%:%3405=948%:%
%:%3406=949%:%
%:%3407=949%:%
%:%3408=950%:%
%:%3409=950%:%
%:%3410=951%:%
%:%3411=951%:%
%:%3412=952%:%
%:%3413=952%:%
%:%3414=952%:%
%:%3415=952%:%
%:%3416=952%:%
%:%3417=953%:%
%:%3418=953%:%
%:%3419=954%:%
%:%3420=954%:%
%:%3421=954%:%
%:%3422=955%:%
%:%3423=955%:%
%:%3424=956%:%
%:%3425=956%:%
%:%3426=956%:%
%:%3427=957%:%
%:%3428=957%:%
%:%3429=957%:%
%:%3430=957%:%
%:%3431=958%:%
%:%3432=958%:%
%:%3433=958%:%
%:%3434=958%:%
%:%3435=958%:%
%:%3436=959%:%
%:%3437=959%:%
%:%3438=959%:%
%:%3439=959%:%
%:%3440=959%:%
%:%3441=960%:%
%:%3442=960%:%
%:%3443=961%:%
%:%3444=961%:%
%:%3445=961%:%
%:%3446=961%:%
%:%3447=962%:%
%:%3448=962%:%
%:%3449=963%:%
%:%3450=963%:%
%:%3451=963%:%
%:%3452=963%:%
%:%3453=964%:%
%:%3454=964%:%
%:%3455=964%:%
%:%3456=964%:%
%:%3457=964%:%
%:%3458=965%:%
%:%3459=965%:%
%:%3460=965%:%
%:%3461=965%:%
%:%3462=966%:%
%:%3463=966%:%
%:%3464=966%:%
%:%3465=966%:%
%:%3466=967%:%
%:%3467=967%:%
%:%3468=968%:%
%:%3483=970%:%
%:%3495=972%:%
%:%3497=973%:%
%:%3498=973%:%
%:%3499=974%:%
%:%3500=975%:%
%:%3501=976%:%
%:%3502=976%:%
%:%3503=977%:%
%:%3504=978%:%
%:%3505=979%:%
%:%3506=979%:%
%:%3513=980%:%
%:%3514=980%:%
%:%3515=981%:%
%:%3516=981%:%
%:%3517=982%:%
%:%3518=982%:%
%:%3519=982%:%
%:%3520=982%:%
%:%3521=983%:%
%:%3527=983%:%
%:%3530=984%:%
%:%3531=984%:%
%:%3532=985%:%
%:%3533=986%:%
%:%3534=986%:%
%:%3535=987%:%
%:%3536=988%:%
%:%3537=989%:%
%:%3544=990%:%
%:%3545=990%:%
%:%3546=991%:%
%:%3547=991%:%
%:%3548=991%:%
%:%3549=992%:%
%:%3550=992%:%
%:%3551=992%:%
%:%3552=993%:%
%:%3553=993%:%
%:%3554=994%:%
%:%3555=994%:%
%:%3556=995%:%
%:%3557=995%:%
%:%3558=996%:%
%:%3559=996%:%
%:%3560=996%:%
%:%3561=996%:%
%:%3562=997%:%
%:%3563=997%:%
%:%3564=997%:%
%:%3565=998%:%
%:%3566=998%:%
%:%3567=998%:%
%:%3568=998%:%
%:%3569=998%:%
%:%3570=999%:%
%:%3571=999%:%
%:%3572=1000%:%
%:%3573=1000%:%
%:%3574=1000%:%
%:%3575=1000%:%
%:%3576=1000%:%
%:%3577=1001%:%
%:%3578=1001%:%
%:%3579=1001%:%
%:%3580=1001%:%
%:%3581=1001%:%
%:%3582=1002%:%
%:%3592=1004%:%
%:%3594=1005%:%
%:%3595=1005%:%
%:%3596=1006%:%
%:%3597=1007%:%
%:%3604=1008%:%
%:%3605=1008%:%
%:%3606=1009%:%
%:%3607=1009%:%
%:%3608=1010%:%
%:%3609=1010%:%
%:%3610=1010%:%
%:%3611=1010%:%
%:%3612=1011%:%
%:%3613=1011%:%
%:%3614=1012%:%
%:%3615=1013%:%
%:%3616=1013%:%
%:%3617=1013%:%
%:%3618=1014%:%
%:%3619=1015%:%
%:%3620=1015%:%
%:%3621=1016%:%
%:%3622=1017%:%
%:%3623=1017%:%
%:%3624=1017%:%
%:%3625=1017%:%
%:%3626=1018%:%
%:%3627=1019%:%
%:%3628=1019%:%
%:%3629=1019%:%
%:%3630=1019%:%
%:%3631=1019%:%
%:%3632=1020%:%
%:%3633=1021%:%
%:%3634=1021%:%
%:%3635=1022%:%
%:%3636=1022%:%
%:%3637=1023%:%
%:%3638=1023%:%
%:%3639=1024%:%
%:%3640=1024%:%
%:%3641=1024%:%
%:%3642=1025%:%
%:%3643=1025%:%
%:%3644=1025%:%
%:%3645=1025%:%
%:%3646=1026%:%
%:%3647=1026%:%
%:%3648=1026%:%
%:%3649=1026%:%
%:%3650=1027%:%
%:%3651=1027%:%
%:%3652=1027%:%
%:%3653=1027%:%
%:%3654=1028%:%
%:%3655=1029%:%
%:%3656=1029%:%
%:%3657=1030%:%
%:%3658=1030%:%
%:%3659=1030%:%
%:%3660=1030%:%
%:%3661=1030%:%
%:%3662=1031%:%
%:%3663=1032%:%
%:%3664=1032%:%
%:%3665=1032%:%
%:%3666=1032%:%
%:%3667=1033%:%
%:%3668=1033%:%
%:%3669=1034%:%
%:%3670=1034%:%
%:%3671=1034%:%
%:%3672=1035%:%
%:%3673=1035%:%
%:%3674=1036%:%
%:%3675=1036%:%
%:%3676=1036%:%
%:%3677=1037%:%
%:%3678=1037%:%
%:%3679=1038%:%
%:%3680=1038%:%
%:%3681=1039%:%
%:%3682=1039%:%
%:%3683=1040%:%
%:%3684=1040%:%
%:%3685=1040%:%
%:%3686=1040%:%
%:%3687=1041%:%
%:%3688=1041%:%
%:%3689=1041%:%
%:%3690=1041%:%
%:%3691=1041%:%
%:%3692=1042%:%
%:%3693=1042%:%
%:%3694=1042%:%
%:%3695=1042%:%
%:%3696=1042%:%
%:%3697=1043%:%
%:%3698=1043%:%
%:%3699=1043%:%
%:%3700=1044%:%
%:%3701=1044%:%
%:%3702=1044%:%
%:%3703=1045%:%
%:%3704=1045%:%
%:%3705=1045%:%
%:%3706=1046%:%
%:%3707=1046%:%
%:%3708=1047%:%
%:%3709=1047%:%
%:%3710=1047%:%
%:%3711=1047%:%
%:%3712=1048%:%
%:%3713=1048%:%
%:%3714=1049%:%
%:%3715=1049%:%
%:%3716=1049%:%
%:%3717=1049%:%
%:%3718=1049%:%
%:%3719=1050%:%
%:%3720=1050%:%
%:%3721=1050%:%
%:%3722=1050%:%
%:%3723=1050%:%
%:%3724=1051%:%
%:%3725=1051%:%
%:%3726=1052%:%
%:%3727=1052%:%
%:%3728=1053%:%
%:%3729=1053%:%
%:%3730=1053%:%
%:%3731=1054%:%
%:%3732=1054%:%
%:%3733=1054%:%
%:%3734=1054%:%
%:%3735=1054%:%
%:%3736=1055%:%
%:%3737=1055%:%
%:%3738=1055%:%
%:%3739=1055%:%
%:%3740=1055%:%
%:%3741=1056%:%
%:%3742=1056%:%
%:%3743=1056%:%
%:%3744=1057%:%
%:%3745=1057%:%
%:%3746=1058%:%
%:%3747=1058%:%
%:%3748=1058%:%
%:%3749=1059%:%
%:%3750=1059%:%
%:%3751=1059%:%
%:%3752=1059%:%
%:%3753=1060%:%
%:%3754=1060%:%
%:%3755=1060%:%
%:%3756=1060%:%
%:%3757=1060%:%
%:%3758=1061%:%
%:%3759=1061%:%
%:%3760=1061%:%
%:%3761=1061%:%
%:%3762=1061%:%
%:%3763=1062%:%
%:%3764=1062%:%
%:%3765=1062%:%
%:%3766=1062%:%
%:%3767=1063%:%
%:%3768=1063%:%
%:%3769=1064%:%
%:%3770=1064%:%
%:%3771=1064%:%
%:%3772=1064%:%
%:%3773=1064%:%
%:%3774=1065%:%
%:%3775=1065%:%
%:%3776=1065%:%
%:%3777=1065%:%
%:%3778=1065%:%
%:%3779=1065%:%
%:%3780=1066%:%
%:%3781=1066%:%
%:%3782=1066%:%
%:%3783=1066%:%
%:%3784=1066%:%
%:%3785=1067%:%
%:%3786=1067%:%
%:%3787=1068%:%
%:%3788=1068%:%
%:%3789=1069%:%
%:%3790=1069%:%
%:%3791=1070%:%
%:%3792=1070%:%
%:%3793=1070%:%
%:%3794=1071%:%
%:%3795=1071%:%
%:%3796=1072%:%
%:%3797=1072%:%
%:%3798=1072%:%
%:%3799=1072%:%
%:%3800=1072%:%
%:%3801=1073%:%
%:%3802=1073%:%
%:%3803=1073%:%
%:%3804=1074%:%
%:%3805=1074%:%
%:%3806=1074%:%
%:%3807=1074%:%
%:%3808=1074%:%
%:%3809=1075%:%
%:%3810=1075%:%
%:%3811=1075%:%
%:%3812=1075%:%
%:%3813=1075%:%
%:%3814=1076%:%
%:%3815=1076%:%
%:%3816=1076%:%
%:%3817=1076%:%
%:%3818=1076%:%
%:%3819=1076%:%
%:%3820=1077%:%
%:%3821=1077%:%
%:%3822=1077%:%
%:%3823=1078%:%
%:%3824=1079%:%
%:%3825=1079%:%
%:%3826=1079%:%
%:%3827=1079%:%
%:%3828=1080%:%
%:%3829=1081%:%
%:%3830=1081%:%
%:%3831=1081%:%
%:%3832=1081%:%
%:%3833=1081%:%
%:%3834=1082%:%
%:%3835=1082%:%
%:%3836=1082%:%
%:%3837=1082%:%
%:%3838=1082%:%
%:%3839=1083%:%
%:%3840=1083%:%
%:%3841=1083%:%
%:%3842=1083%:%
%:%3843=1084%:%
%:%3844=1084%:%
%:%3845=1084%:%
%:%3846=1084%:%
%:%3847=1085%:%
%:%3848=1085%:%
%:%3849=1086%:%
%:%3850=1086%:%
%:%3851=1086%:%
%:%3852=1086%:%
%:%3853=1086%:%
%:%3854=1087%:%
%:%3855=1087%:%
%:%3860=1087%:%
%:%3865=1088%:%
%:%3870=1089%:%



% optional bibliography
\bibliographystyle{abbrv}
\bibliography{root}

\end{document}

%%% Local Variables:
%%% mode: latex
%%% TeX-master: t
%%% End:
